\chapter{Typechecking as a Semantics}


\section{An Algebraic Approach to Typechecking}\label{section:typechecking}

Following Atkey~\citeyear{atkey2015algebraic}, we can consider type checking
and type inference as a possible semantics for a bi-directional~\cite{pierce2000local}
language. We represent the raw syntax of a simply typed bi-directional calculus
as a bi-sorted language using a notion of \AD{Mode} to distinguish between terms
for which we will be able to \AIC{Infer} the type and the ones for which we will
have to \AIC{Check} a type candidate.

Following traditional presentations, eliminators give rise to \AIC{Infer}rable
terms under the condition that the term they are eliminating is also \AIC{Infer}rable
and the other arguments are \AIC{Check}able whilst constructors are always \AIC{Check}able.
Two extra constructors allow changes of direction: \AIC{Cut} annotates a \AIC{Check}able
term with its type thus making it \AIC{Infer}rable whilst \AIC{Emb} embeds
\AIC{Infer}rables into \AIC{Check}ables.

\begin{figure}[h]
\ExecuteMetaData[generic-syntax.agda/Generic/Examples/TypeChecking.tex]{type}
\ExecuteMetaData[generic-syntax.agda/Generic/Examples/TypeChecking.tex]{constructors}
\ExecuteMetaData[generic-syntax.agda/Generic/Examples/TypeChecking.tex]{mode}
\ExecuteMetaData[generic-syntax.agda/Generic/Examples/TypeChecking.tex]{bidirectional}
\caption{A Bidirectional Simply Typed Language}
\end{figure}

Both the values and computations will be constant in the scope. The values stored
in the environment will be \AD{Type} information for bound variables. Instead of
considering that we get a type no matter what the \AD{Mode} of the variable is,
we enforce the fact that all variables need to be \AIC{Infer}rable.

\ExecuteMetaData[generic-syntax.agda/Generic/Examples/TypeChecking.tex]{varmode}

In contrast, the generated computations will, depending on the mode, either take a
type candidate and \AIC{Check} it is valid or \AIC{Infer} a type for their argument.
These computations are always potentially failing so we use the \AD{Maybe} monad.

\ExecuteMetaData[generic-syntax.agda/Generic/Examples/TypeChecking.tex]{typemode}

Before defining typechecking as a \AR{Semantics}.

\ExecuteMetaData[generic-syntax.agda/Generic/Examples/TypeChecking.tex]{arrow}
\ExecuteMetaData[generic-syntax.agda/Generic/Examples/TypeChecking.tex]{equal}

Equipped with these combinators, we can define the two most interesting cases
as top-level combinators: application and $\lambda$-abstraction.

When facing an application: infer the type of the function, make sure it is an
arrow type, check the argument at the domain's type and return the codomain.

\ExecuteMetaData[generic-syntax.agda/Generic/Examples/TypeChecking.tex]{app}

For a $\lambda$-abstraction: check the input type is an arrow type and check
the body at the codomain type in the extended environment where the newly-bound
variable has the domain's type.

\ExecuteMetaData[generic-syntax.agda/Generic/Examples/TypeChecking.tex]{lam}

We can now define typechecking itself. Because values are constant in the scope,
the \ARF{th\textasciicircum{}𝓥} constraint is trivial. The \ARF{var} constraint
is also easy: values correspond to \AIC{Infer}rable terms so we can simply return
the type looked up in the environment. The algebra describes the algorithm by pieces.
We have already handled application and $\lambda$-abstraction, a cut always comes with
a type candidate against which to check the term  and to be returned in case of
success (\AF{\_<\$\_} takes an \AB{A} and a {\AD{Maybe} \AB{B}} and returns
a {\AD{Maybe} \AB{A}} which has the same structure as its second argument).
Finally, the change of direction from \AIC{Infer}rable to \AIC{Check}able is successful
when the inferred type is equal to the expected one.

\begin{figure}[h]
\ExecuteMetaData[generic-syntax.agda/Generic/Examples/TypeChecking.tex]{typecheck}
\caption{Typechecking as a \AR{Semantics}}
\end{figure}

From this we can derive our \AF{type-}(\AIC{Infer}/\AIC{Check}) function which takes
a closed term and computers either an inferred type or a validation function for a
type candidate. We make use of the special case of \AF{semantics} for closed term
introduced in \cref{fig:closedsem}.

\begin{figure}[h]
\ExecuteMetaData[generic-syntax.agda/Generic/Examples/TypeChecking.tex]{typefun}
\caption{Type Inference and Type Checking as Mode-indexed Semantics\label{fig:typeinfersem}}
\end{figure}

We can run this typechecking function on an example

\begin{figure}[h]
\ExecuteMetaData[generic-syntax.agda/Generic/Examples/TypeChecking.tex]{example}
\caption{Running the Type Inference}
\end{figure}

We have demonstrated how to define a bidirectional typechecker for this simple language
by leveraging the \AF{Semantics} framework. However the output of this function is not
very informative. We can do better: instead of simply generating a type or checking that
a candidate will do, we can use our \AD{Desc}riptions to describe a language of evidence
and generate not only an expression's type but also a well scoped and well typed term of
that type.

\section{An Algebraic Approach to Elaboration}


\ExecuteMetaData[generic-syntax.agda/Generic/Examples/Elaboration.tex]{typing}
\ExecuteMetaData[generic-syntax.agda/Generic/Examples/Elaboration.tex]{fromtyping}
\ExecuteMetaData[generic-syntax.agda/Generic/Examples/Elaboration.tex]{elab}
\ExecuteMetaData[generic-syntax.agda/Generic/Examples/Elaboration.tex]{varmode}
\ExecuteMetaData[generic-syntax.agda/Generic/Examples/Elaboration.tex]{typemode}
\ExecuteMetaData[generic-syntax.agda/Generic/Examples/Elaboration.tex]{equal}
\ExecuteMetaData[generic-syntax.agda/Generic/Examples/Elaboration.tex]{arrow}
\ExecuteMetaData[generic-syntax.agda/Generic/Examples/Elaboration.tex]{app}
\ExecuteMetaData[generic-syntax.agda/Generic/Examples/Elaboration.tex]{lam}
\ExecuteMetaData[generic-syntax.agda/Generic/Examples/Elaboration.tex]{emb}
\ExecuteMetaData[generic-syntax.agda/Generic/Examples/Elaboration.tex]{elaborate}
