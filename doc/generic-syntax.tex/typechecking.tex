\chapter{Typechecking as a Semantics}


\section{An Algebraic Approach to Typechecking}\label{section:typechecking}

Following Atkey~\citeyear{atkey2015algebraic}, we can consider type checking
and type inference as a possible semantics for a bi-directional~\cite{pierce2000local}
language. We represent the raw syntax of a simply typed bi-directional calculus
as a bi-sorted language using a notion of \AD{Mode} to distinguish between terms
for which we will be able to \AIC{Infer} the type and the ones for which we will
have to \AIC{Check} a type candidate.

Following traditional presentations, eliminators give rise to \AIC{Infer}rable
terms under the condition that the term they are eliminating is also \AIC{Infer}rable
and the other arguments are \AIC{Check}able whilst constructors are always \AIC{Check}able.
Two extra constructors allow changes of direction: \AIC{Cut} annotates a \AIC{Check}able
term with its type thus making it \AIC{Infer}rable whilst \AIC{Emb} embeds \AIC{Infer}rables
into \AIC{Check}ables.



\begin{figure}[h]
\ExecuteMetaData[generic-syntax.agda/Generic/Examples/TypeChecking.tex]{type}
\ExecuteMetaData[generic-syntax.agda/Generic/Examples/TypeChecking.tex]{constructors}
\ExecuteMetaData[generic-syntax.agda/Generic/Examples/TypeChecking.tex]{mode}
\ExecuteMetaData[generic-syntax.agda/Generic/Examples/TypeChecking.tex]{bidirectional}
\caption{A Bidirectional Simply Typed Language}
\end{figure}

The values stored in the environment will be \AD{Type} information for bound
variables. Instead of considering that we get a type no matter what the \AD{Mode}
of the variable is, we enforce the fact that all variables need to be \AIC{Infer}rable.

\ExecuteMetaData[generic-syntax.agda/Generic/Examples/TypeChecking.tex]{varmode}

In contrast, the generated computations will, depending on the mode, either take a
type candidate and \AIC{Check} it is valid or \AIC{Infer} a type for their argument.
These computations are always potentially failing so we use the \AD{Maybe} monad.

\ExecuteMetaData[generic-syntax.agda/Generic/Examples/TypeChecking.tex]{typemode}


\ExecuteMetaData[generic-syntax.agda/Generic/Examples/TypeChecking.tex]{arrow}
\ExecuteMetaData[generic-syntax.agda/Generic/Examples/TypeChecking.tex]{app}
\ExecuteMetaData[generic-syntax.agda/Generic/Examples/TypeChecking.tex]{lam}

\ExecuteMetaData[generic-syntax.agda/Generic/Examples/TypeChecking.tex]{equal}
\ExecuteMetaData[generic-syntax.agda/Generic/Examples/TypeChecking.tex]{typecheck}


\section{An Algebraic Approach to Elaboration}
