\chapter{Plea For a Universe of Syntaxes with Binding}

Now that we have a way to structure our traversals and proofs about them, we can
tackle a practical example. Let us look at the formalisation of an apparently
straightforward program transformation: the inlining of let-bound variables by
substitution and the proof of a simple correctness lemma. We have two languages:
the source (\AD{S}), which has let-bindings, and the target (\AD{T}), which only
differs in that it does not. We want to write a function elaborating source term
into target ones and then prove that each reduction step on the source term can
be simulated by zero or more reduction steps on its elaboration.

Breaking the task down, we need to start by defining the two languages. We already
now how to do this from \cref{sec:scopedtypedterms}. The only downside is that we
need to write down the same constructor types twice for \AIC{`var}, \AIC{`lam},
and \AIC{`app}.

\begin{figure}[h]
\ExecuteMetaData[generic-syntax.agda/Motivation/Problem/Naive.tex]{source}
\ExecuteMetaData[generic-syntax.agda/Motivation/Problem/Naive.tex]{target}
\caption{Source and Target Languages}
\end{figure}

Ignoring for now the \AR{Semantics} framework, we jump straight to defining the
program transformation we are interested in. We notice immediately that we need
to prove \AD{T} to be \AF{Thinnable} first so that we may push the environment
of inlined terms under binders. We also notice that the only interesting case
is the one dealing with \AIC{`let}: all the other ones are purely structural.

\begin{figure}[h]
\ExecuteMetaData[generic-syntax.agda/Motivation/Problem/Naive.tex]{unlet}
\caption{Let-Inlining Traversal}
\end{figure}

We now want to state our correctness lemma: each reduction step on a source term
can be simulated by zero or more reduction steps on its elaboration. We need to
define an operational semantics for each language. We only show the one for \AD{S}
in \cref{fig:opersem}: the one for \AD{T} is exactly the same minus the
\AIC{`let}-related rules. We immediately notice that to write down the type of
\AIC{β} we need to define substitution (and thus renaming) for each of the languages.

\begin{figure}[h]
\ExecuteMetaData[generic-syntax.agda/Motivation/Problem/Naive.tex]{redS}
\caption{Operational Semantics for the Source Language\label{fig:opersem}}
\end{figure}

In the course of simply stating our problem, we have already had to define two
eerily similar languages, spell out all the purely structural cases when defining
the transformation we are interested in and implement four auxiliary traversals
which are essentially the same.

In the course of proving the correctness lemma (which we abstain from doin here),
we discover that we need to prove eight lemmas about the interactions of renaming,
substitution, and let-inlining. They are all remarkably similar, but must be stated
and proved separately (e.g, as in \cite{benton2012strongly}).

Even after doing all of this work, we have only a result for a single pair of
source and target languages. If you were to change our languages \AD{S} or
\AD{T}, we would have to repeat the same work all over again or at least do a
lot of cutting, pasting, and editing. And if we add more constructs to both
languages, we will have to extend our transformation with more and more code
that essentially does nothing of interest.

This state of things is not inevitable. After having implemented numerous
semantics in \cref{type-scope-semantics}, we have gained an important insight:
the structure of the constraints telling us how to define a \AR{Semantics} is
tightly coupled to the definition of the language. So much so that we should
in fact be able to \emph{derive} them directly from the definition of the
language.

This is what we set out to do in this part and in particular in \cref{section:letbinding}
where we define a \emph{generic} notion of let-binding to extend any language
with together with the corresponding generic let-inlining transformation.
