\chapter{A Universe of Scope Safe and Well Kinded Syntaxes}
\label{chapter:universe}

Our universe of scope safe and well kinded syntaxes follows the same
principle as CDMM's universe of datatypes, except that we are not
building endofunctors on \AS{Setᴵ} any more but rather on
{\AB{I} \AF{─Scoped}} (defined in \cref{fig:scoped}).

We now think of the index type \AB{I} as the sorts used to distinguish
terms in our embedded language. The \AIC{`$\sigma$} and \AIC{`∎}
constructors are as in the CDMM \AD{Desc} type, and are used to represent
data and index constraints respectively.

\section{Descriptions and Their Meaning as Functors}

Following our approach in the previous chapter, we are going to interleave
the definition of the indexed datatype of descriptions and the meaning
function \AF{⟦\_⟧} associated to it. They are essentially identical to
their counterparts defining a universe of datatype in \cref{section:data}
except for two aspects.

First, the constructor marking the presence of a recursive substructure
will take an extra {(\AD{List} \AB{I})} argument specifying the sorts of
the newly bound variables that will be present in the subterm.

Second, the type of the meaning function is more complex. It does not
translate descriptions into an endofunctor on {(\AB{I} \AF{─Scoped})};
it is more general than that. The function takes an \AB{X} of type
{\AD{List} \AB{I} → \AB{I} \AD{─Scoped}} to interpret demand for
recursive substructures. This first list argument corresponds to the
newly bound variables that will be present in the subterm and having
it as a separate argument will play a crucial role when defining the
description's semantics as a binding structure in
\cref{figure:debruijnscope,figure:freemonad}.

Without further ado, let us look at the definition. The declaration
of both the universe of descriptions and the meaning function associated
to them first.

\begin{AgdaSuppressSpace}
\ExecuteMetaData[generic-syntax.agda/Generic/Syntax.tex]{desc:type}
\ExecuteMetaData[generic-syntax.agda/Generic/Syntax.tex]{interp:type}
\end{AgdaSuppressSpace}

The \AIC{`σ} is exactly identical to that of the universe of data. It
also serves the same purpose: either provide the ability to store a
payload, or a tag upon which the shape of the rest of the description
will depend.

\begin{AgdaSuppressSpace}
\ExecuteMetaData[generic-syntax.agda/Generic/Syntax.tex]{desc:sigma}
\ExecuteMetaData[generic-syntax.agda/Generic/Syntax.tex]{interp:sigma}
\end{AgdaSuppressSpace}

The case of interest is the one for recursive substructures. As explained
above, it packs an extra argument specifying the number and sorts of the
newly bound variables. For a λ-abstraction, this list would be a singleton,
meaning that a single extra variable is in scope in the body of the lambda;
for both of the arguments of an application node the list would be empty.

\begin{AgdaSuppressSpace}
\ExecuteMetaData[generic-syntax.agda/Generic/Syntax.tex]{desc:rec}
\ExecuteMetaData[generic-syntax.agda/Generic/Syntax.tex]{interp:rec}
\end{AgdaSuppressSpace}

Finally, the stop case is identical to that of the universe of data.
And its role is the same: to enforce that the branch selected by the
user has the sort that was demanded.

\begin{AgdaSuppressSpace}
\ExecuteMetaData[generic-syntax.agda/Generic/Syntax.tex]{desc:stop}
\ExecuteMetaData[generic-syntax.agda/Generic/Syntax.tex]{interp:stop}
\end{AgdaSuppressSpace}

The astute reader may have noticed that \AF{⟦\_⟧} is uniform in $X$
and Γ; however refactoring \AF{⟦\_⟧} to use the partially applied
$X\,\_\,\_\,\Gamma$ following this observation would lead to
a definition harder to use with the combinators for indexed sets
described in \cref{sec:indexed-combinators} which make our types
much more readable.


\subsection{A Syntactic Meaning: De Bruijn Scopes}

If we pre-compose (using \AF{\_⊢\_} defined in \cref{sec:indexed-combinators})
the meaning function \AF{⟦\_⟧} with a notion of
`de Bruijn scopes' (denoted \AF{Scope} here) that turns any
{(\AB{I} \AF{─Scoped})} family into a function
of type {(\AD{List} \AB{I} \AS{→} \AB{I} \AF{─Scoped})} by appending the two
\AD{List} indices, we recover a meaning function producing an endofunctor on
\AB{I} \AF{─Scoped}. It corresponds to the syntactic interpretation of
the description we expect when building terms: the newly bound variables
are simply used to extend the ambient context.

\begin{figure}[h]
\ExecuteMetaData[generic-syntax.agda/Generic/Syntax.tex]{scope}
\caption{De Bruijn Scopes}\label{figure:debruijnscope}
\end{figure}

So far we have only shown the action of the functor on objects; its action
on morphisms is given by a function \AF{fmap} defined by induction over the
description just like in Section~\ref{section:data}. We give \AF{fmap} the
most general type we can, the action of functors is then a specialized
version of it.

\begin{figure}[h]
\ExecuteMetaData[generic-syntax.agda/Generic/Syntax.tex]{fmap}
\caption{Action of Syntax Functors on Morphism\label{figure:descfmap}}
\end{figure}

\section{Terms as Free Relative Monads}

The endofunctors thus defined are strictly positive and we can take
their fixpoints.
As we want to define the terms of a language with variables, instead of
considering the initial algebra, this time we opt for the free relative
monad (\cite{Altenkirch2010, JFR4389}) with respect to the functor \AF{Var}.
The \AIC{`var} constructor corresponds to return, and we will define bind (also
known as the parallel substitution \AF{sub}) in the next section.
We have once more a \AD{Size} index to get all the benefits of type based
termination checking when defining traversals over terms.

\begin{figure}[h]
\ExecuteMetaData[generic-syntax.agda/Generic/Syntax.tex]{mu}
\caption{Term Trees: The Free \AF{Var}-Relative Monads on Descriptions\label{figure:freemonad}}
\end{figure}

Because we often use closed terms of size \AF{∞} (that is to say fully-defined)
in concrete examples, we name this notion.

\begin{figure}[h]
\ExecuteMetaData[generic-syntax.agda/Generic/Syntax.tex]{closed}
\caption{Type of Closed Terms\label{fig:closedtm}}
\end{figure}

\subsection{Three Small Examples of Syntaxes With Binding}

Rather than immediately coming back to our original example which has quite
a few constructors, we are going to give codes for 3 variations on the
λ-calculus to show the whole spectrum of syntaxes: untyped, well sorted,
and well typed by construction.
%
We start with the simplest example to lay down the foundations: the well
scoped untyped λ-calculus.
%
We then itnroduce a well scoped and well sorted but not well typed
bidirectional language.
%
We finally consider the code for the intrinsically typed STλC.
%
In all examples, the variable case will be added by the free monad
construction so we only have to describe the other constructors.

\paragraph{Untyped} languages are, as Harper would say, uni-typed syntaxes and
can thus be modelled using descriptions whose kind parameter is the unit type
(\AR{⊤}).
%
We take the disjoint sum of the respective descriptions for the application
and λ-abstraction constructors by using the classic construction in type
theory we have already deployed once for lists and once for vectors: a
dependent pair of a \AF{Bool} picking one of the two branches and a second
component whose type is either that of application or λ-abstraction depending
on that boolean.

Application has two substructures (\AIC{`X}) which do not bind any extra
argument and λ-abstraction has exactly one substructure with precisely
one extra bound variable. Both constructors' descriptions end with
(\AIC{`∎} \AIC{tt}), the only inhabitant of the trivial sort.

\begin{figure}[h]
\ExecuteMetaData[generic-syntax.agda/Generic/Examples/UntypedLC.tex]{ulc}
\caption{\AF{Desc}ription of The Untyped Lambda Calculus
\label{fig:desculc}\label{fig:descUTLC}}
\end{figure}

\paragraph{Bidirectional STLC}\label{par:bidirectional} Our second example
is a bidirectional (\cite{pierce2000local}) language hence the introduction
of a notion of \AD{Mode}. Each term is either part of the \AIC{Synth} (the
terms whose type can be synthesised) or \AIC{Check} (the terms whose type
can be checked) fraction of the language.
%
This language has four constructors
which we list in the ad-hoc \AD{`Bidi} type of constructor tags, its
decoding \AD{Bidi} is defined by a pattern-matching λ-expression in Agda.
Application and λ-abstraction behave as expected, with the important
observation that λ-abstraction binds a \AIC{Synth}esisable term. The two
remaining constructors correspond to changes of direction: one can freely
\AIC{Emb}bed inferrable terms as checkable ones whereas we require a type
annotation when forming a \AIC{Cut}.

\begin{figure}[h]
\noindent\begin{minipage}[t]{0.35\textwidth}
  \ExecuteMetaData[generic-syntax.agda/Generic/Syntax/Bidirectional.tex]{tagmode}
\end{minipage}
\begin{minipage}[t]{0.65\textwidth}
  \ExecuteMetaData[generic-syntax.agda/Generic/Syntax/Bidirectional.tex]{desc}
\end{minipage}
  \caption{Description for the bidirectional STLC\label{fig:descBidiSTLC}}
\end{figure}

\paragraph{Intrinsically typed STLC}\label{par:intrinsicSTLC}
In the typed case, we are back to two
constructors: the terms are fully annotated and therefore it is not necessary
to distinguish between \AD{Mode}s anymore. We need our tags to carry extra
information about the types involved so we use once more and ad-hoc datatype
\AD{`STLC}, and define its decoding \AD{STLC} by a pattern-matching
λ-expression.

Application has two substructures none of which bind extra variables. The first
has a function type and the second the type of its domain. The overall type of
the branch is enforced to be that of the function's codomain by the \AIC{`∎}
constructor.

λ-abstraction has exactly one substructure of type \AB{τ} with a newly bound
variable of type \AB{σ}. The overall type of the branch is once more enforced
by \AIC{`∎}: it is (\AB{σ} \AIC{`→} \AB{τ}).

\begin{figure}[h]
\ExecuteMetaData[generic-syntax.agda/Generic/Examples/STLC.tex]{stlc}
\caption{\AF{Desc}ription of the Simply Typed Lambda Calculus\label{fig:descstlc}\label{fig:descSTLC}}
\end{figure}

For convenience we use Agda's pattern synonyms corresponding to the
original constructors in~\cref{sec:stlccalculus}:
\AIC{`app} for application and \AIC{`lam} for λ-abstraction. These
synonyms can be used when pattern-matching on a term and Agda resugars
them when displaying a goal. This means that the end user can
seamlessly work with encoded terms without dealing with the gnarly
details of the encoding.  These pattern definitions can omit some
arguments by using ``\AS{\_}'', in which case they will be filled in
by unification just like any other implicit argument: there is no
extra cost to using an encoding!  The only downside is that the
language currently does not allow the user to specify type annotations
for pattern synonyms. We only include examples of pattern synonyms
for the two extreme examples, the definition for \AF{Bidi} are similar.

\begin{figure}[h]
  \begin{AgdaSuppressSpace}
  \ExecuteMetaData[generic-syntax.agda/Generic/Examples/UntypedLC.tex]{patterns}
  \ExecuteMetaData[generic-syntax.agda/Generic/Examples/STLC.tex]{patterns}
  \end{AgdaSuppressSpace}
  \caption{Respective Pattern Synonyms for \AF{UTLC}~and \AF{STLC}%
    \label{fig:patsLC}}
\end{figure}

As a usage example of these pattern synonyms, we define the identity
function in all three languages in Figure~\ref{fig:identity}, using the
same caret-based naming convention we introduced earlier. The code
is virtually the same except for \AF{Bidi} which explicitly records
the change of direction from \AIC{Check} to \AIC{Synth}.

\begin{figure}[h]
\begin{minipage}{0.25\textwidth}
  \ExecuteMetaData[generic-syntax.agda/Generic/Examples/UntypedLC.tex]{identity}
\end{minipage}\hfill
\begin{minipage}{0.35\textwidth}
  \ExecuteMetaData[generic-syntax.agda/Generic/Syntax/Bidirectional.tex]{BDid}
\end{minipage}\hfill
\begin{minipage}{0.30\textwidth}
  \ExecuteMetaData[generic-syntax.agda/Generic/Examples/STLC.tex]{identity}
\end{minipage}
\caption{Identity function in all three languages}\label{fig:identity}
\end{figure}

\subsection{Porting our Earlier Running Example}

To contrast and compare the two approaches, let us write down side by side
the original definition of the intrinsincally typed STLC we introduced in
\cref{fig:term} and its encoding as a description.

We are going to take this definition apart, considering the different
classes of constructors that it introduces and how to adequatly represent
them using our universe of description. Our representation will once more
rely on a type \AD{`Term} of tags listing the available term constructors
together with a description \AD{Term} using a dependent pair to offer users
the ability to pick the constructor of their choosing. That is to say that
we are considering the following three definitions.

\noindent\begin{minipage}[t]{0.5\textwidth}
    \ExecuteMetaData[type-scope-semantics.agda/Syntax/Calculus.tex]{termcompact:decl}
\end{minipage}\begin{minipage}[t]{0.5\textwidth}
  \begin{AgdaSuppressSpace}
  \ExecuteMetaData[generic-syntax.agda/Generic/Syntax/Extended.tex]{tag:decl}
  \ExecuteMetaData[generic-syntax.agda/Generic/Syntax/Extended.tex]{term:decl}
  \end{AgdaSuppressSpace}
\end{minipage}

The most basic of constructors are the one with no subterms. The tag
associated to them is a simple constructor and their decoding just
uses the stop description constructor (\AIC{`∎}) to enforce that
their return type matches the one specified in the original definition.

\noindent\begin{minipage}[t]{0.4\textwidth}
    \ExecuteMetaData[type-scope-semantics.agda/Syntax/Calculus.tex]{termcompact:base}
\end{minipage}\begin{minipage}[t]{0.6\textwidth}
  \begin{AgdaSuppressSpace}
  \ExecuteMetaData[generic-syntax.agda/Generic/Syntax/Extended.tex]{tag:base}
  \ExecuteMetaData[generic-syntax.agda/Generic/Syntax/Extended.tex]{term:base}
  \end{AgdaSuppressSpace}
\end{minipage}

Next we have the constructors that have subterms in which no extra variable
is bound. Their encoding will use the constructor for recursive substructures
and apply it to the empty list to reflect that fact. Additionally, both
\AIC{`app} and \AIC{`ifte} store some additional information: the type of
the subterms at hand. We will push this information into the tags just like
we did in the previous section. We once more use the stop constructor to
enforce that the constructors' respective return types are faithfully
reproduced.

\noindent\begin{minipage}[t]{0.4\textwidth}
    \ExecuteMetaData[type-scope-semantics.agda/Syntax/Calculus.tex]{termcompact:struct}
\end{minipage}\begin{minipage}[t]{0.6\textwidth}
  \begin{AgdaSuppressSpace}
  \ExecuteMetaData[generic-syntax.agda/Generic/Syntax/Extended.tex]{tag:struct}
  \ExecuteMetaData[generic-syntax.agda/Generic/Syntax/Extended.tex]{term:struct}
  \end{AgdaSuppressSpace}
\end{minipage}

Finally, we have the λ-abstraction constructor. Its body is defined in
a context extended with exactly one newly bound variable whose type is
the domain of the overall term's function type. This is modelled by
applying the constructor for recursive substructure to a singleton list.

\noindent\begin{minipage}[t]{0.4\textwidth}
    \ExecuteMetaData[type-scope-semantics.agda/Syntax/Calculus.tex]{termcompact:lam}
\end{minipage}\begin{minipage}[t]{0.6\textwidth}
  \begin{AgdaSuppressSpace}
  \ExecuteMetaData[generic-syntax.agda/Generic/Syntax/Extended.tex]{tag:lam}
  \ExecuteMetaData[generic-syntax.agda/Generic/Syntax/Extended.tex]{term:lam}
  \end{AgdaSuppressSpace}
\end{minipage}

This concludes the translation. It clearly is a very mechanical process
and thus could be automated using reflection
(\cite{DBLP:conf/ifl/WaltS12,thesis:christiansen}). Such automation is however
outside the scope of this thesis.

\section{Common Combinators and Their Properties}
\label{desccomb}

In order to avoid repeatedly re-encoding the same logic, we introduce
a combinator demonstrating that descriptions are closed under finite
sums.

As we wrote, the construction used in \cref{fig:desculc} to define the
syntax for the untyped λ-calculus is classic. It is actually the third
time (the first and second times being the definition of \AF{listD} and
\AF{vecD} in \cref{figure:listD,figure:vecD}) that we use a \AF{Bool}
to distinguish between two constructors. We define once and for all
the disjoint union of two descriptions thanks to the \AF{\_`+\_} combinator
in \cref{figure:descsum}.

\begin{figure}[h]
  \ExecuteMetaData[generic-syntax.agda/Generic/Syntax.tex]{descsum}
  \caption{Descriptions are Closed Under Disjoint Sums\label{figure:descsum}}
\end{figure}

It comes together with an appropriate eliminator \AF{case} defined in
\cref{figure:descsumelim} which, given two continuations, picks the one
corresponding to the chosen branch.

\begin{figure}[h]
\ExecuteMetaData[generic-syntax.agda/Generic/Syntax.tex]{case}
\caption{Eliminator for \AF{\_`+\_}\label{figure:descsumelim}}
\end{figure}

A concrete use case for the disjoint union combinator and its eliminator
will be given in \cref{section:letbinding} where we explain how to seamlessly
enrich any existing syntax with let-bindings and how to use the \AR{Semantics}
framework to elaborate them away.

%% \subsection{Closure Under Finite Product of Recursive Positions}

%% Closure under product does not hold in general. Indeed, the equality constraints
%% introduced by the two end tokens of two descriptions may be incompatible. So far,
%% a limited form of closure (closure under finite product of recursive positions)
%% has been sufficient for all of our use cases. Provided a list of pairs of context
%% extensions and kinds, we can add to an existing description that many recursive
%% substructures.

%% \begin{figure}[h]
%% \ExecuteMetaData[generic-syntax.agda/Generic/Syntax.tex]{xs}
%% \caption{Descriptions are Closed Under Finite Product of Recursive Positions\label{figure:descprod}}
%% \end{figure}

%% As with coproducts, we can define an appropriate eliminator. The function \AF{unXs}
%% takes a value in the encoding and extracts its constituents ({\AD{All} \AB{P} \AB{xs}}
%% is defined in Agda's standard library and makes sure that the predicate \AB{P} holds
%% true of all the elements in the list \AB{xs}).

%% \begin{figure}[h]
%% \ExecuteMetaData[generic-syntax.agda/Generic/Syntax.tex]{unxs}
%% \caption{Breaking Down a Finite Product of Recursive Positions\label{figure:descprodelim}}
%% \end{figure}

%% We will see in \cref{section:letbinding} how to define let-bindings as a generic
%% language extension and their inlining as a generic semantics over the extended
%% syntax and into the base one. Closure under a finite product of recursive positions
%% demonstrates that we could extend this construction to parallel (or even mutually-recursive)
%% let-bindings where the number and the types of the bound expressions can be arbitrary.
%% We will not go into the details of this construction as it is essentially a combination
%% of \AF{Xs}, \AF{unXs} and the techniques used when defining single let-bindings.
