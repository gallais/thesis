\chapter{A Universe of Scope Safe and Well Kinded Syntaxes}\label{chapter:universe}

Our universe of scope safe and well kinded syntaxes follows the same principle
as CDMM's universe of datatypes, except that we are not building endofunctors on
\AS{Setᴵ} any more but rather on {\AB{I} \AF{─Scoped}}. We now think of the index
type \AB{I} as the sorts used to distinguish terms in our embedded language.
The \AIC{`$\sigma$} and \AIC{`∎} constructors are as in the CDMM \AD{Desc} type,
and are used to represent data and index constraints respectively.

\section{Descriptions and Their Meaning as Functors}

What distinguishes this new universe \AD{Desc} from that of Section~\ref{section:data}
is that the \AIC{`X} constructor is now augmented with an additional {\AD{List} \AB{I}}
argument that describes the new binders that are brought into scope at this recursive
position. This list of the kinds of the newly-bound variables will play a crucial role when
defining the description's semantics as a binding structure in
\cref{figure:syntaxmeaning,figure:debruijnscope,figure:freemonad}.

\begin{figure}[h]
\ExecuteMetaData[generic-syntax.agda/Generic/Syntax.tex]{desc}
\caption{Syntax Descriptions}
\end{figure}

The meaning function \AF{⟦\_⟧} we associate to a description follows closely
its CDMM equivalent. It only departs from it in the \AIC{`X} case and the fact
it is not an endofunctor on \AB{I} \AF{─Scoped}; it is more general than that.
The function takes an \AB{X} of type {\AD{List} \AB{I} $\rightarrow$ \AB{I} \AD{─Scoped}}
to interpret {\AIC{`X} \AB{Δ} \AB{j}} (i.e. substructures of sort \AB{j} with
newly-bound variables in \AB{Δ}) in an ambient scope \AB{Γ} as {\AB{X} \AB{Δ} \AB{j} \AB{Γ}}.

\begin{figure}[h]
\ExecuteMetaData[generic-syntax.agda/Generic/Syntax.tex]{interp}
\caption{Descriptions' Meanings}\label{figure:syntaxmeaning}
\end{figure}

The astute reader may have noticed that \AF{⟦\_⟧} is uniform in $X$ and $\Gamma$; however
refactoring \AF{⟦\_⟧} to use the partially applied $X\,\_\,\_\,\Gamma$ following
this observation would lead to a definition harder to use with the
combinators for indexed sets described in \cref{sec:indexed-combinators}
which make our types much more readable.

If we pre-compose the meaning function \AF{⟦\_⟧} with a notion of `de Bruijn scopes'
(denoted \AF{Scope} here) which turns any \AB{I} \AF{─Scoped} family into a function
of type \AD{List} \AB{I} \AS{→} \AB{I} \AF{─Scoped} by appending the two
\AD{List} indices, we recover a meaning function producing an endofunctor on
\AB{I} \AF{─Scoped}.

\begin{figure}[h]
\ExecuteMetaData[generic-syntax.agda/Generic/Syntax.tex]{scope}
\caption{De Bruijn Scopes}\label{figure:debruijnscope}
\end{figure}

So far we have only shown the action of the functor on objects; its action on
morphisms is given by a function \AF{fmap} defined by induction over the
description just like in Section~\ref{section:data}. We give \AF{fmap} the most
general type we can, the action of functors is then a specialized version of it.

\begin{figure}[h]
\ExecuteMetaData[generic-syntax.agda/Generic/Syntax.tex]{fmap}
\caption{Action of Syntax Functors on Morphism\label{figure:descfmap}}
\end{figure}

\section{Terms as Free Relative Monads}

The endofunctors thus defined are strictly positive and we can take their fixpoints.
As we want to define the terms of a language with variables, instead of
considering the initial algebra, this time we opt for the free relative
monad (\cite{Altenkirch2010, JFR4389}) with respect to the functor \AF{Var}.
The \AIC{`var} constructor corresponds to return, and we will define bind (also
known as the parallel substitution \AF{sub}) in the next section.
We have once more a \AD{Size} index to get all the benefits of type based
termination checking when defining traversals over terms.

\begin{figure}[h]
\ExecuteMetaData[generic-syntax.agda/Generic/Syntax.tex]{mu}
\caption{Term Trees: The Free \AF{Var}-Relative Monads on Descriptions\label{figure:freemonad}}
\end{figure}

Because we often use closed terms of size \AF{∞} (that is to say fully-defined)
in concrete examples, we name this notion.

\begin{figure}[h]
\ExecuteMetaData[generic-syntax.agda/Generic/Syntax.tex]{closed}
\caption{Type of Closed Terms\label{fig:closedtm}}
\end{figure}

\subsection{Examples of Syntaxes With Binding}

Coming back to our original example, we now have the ability to give a code for
the intrinsically typed STλC. But we start with a simpler example to lay
down the foundations: the well scoped untyped λ-calculus. In both cases, the
variable case will be added by the free monad construction so we only have to
describe two constructors: application and λ-abstraction.

\paragraph{Untyped} languages are, as Harper would say, uni-typed syntaxes and
can thus be modelled using descriptions whose kind parameter is the unit type.
We take the disjoint sum of the respective descriptions for the application and
λ-abstraction constructors by using the classic construction in type theory: a
dependent pair of a \AF{Bool} picking one of the two branches and a second
component whose type is either that of application or λ-abstraction depending
on that boolean.

Application has two substructures (\AIC{`X}) which do not bind any extra argument
and λ-abstraction has exactly one substructure with precisely one extra bound variable.
Both constructors' descriptions end with (\AIC{`∎} \AIC{tt}), the only inhabitant
of the trivial kind.

\begin{figure}[h]
\ExecuteMetaData[generic-syntax.agda/Generic/Examples/UntypedLC.tex]{ulc}
\caption{\AF{Desc}ription of The Untyped Lambda Calculus\label{fig:desculc}}
\end{figure}


\paragraph{Typed} syntax comes with extra constraints: our tags need to carry
extra information about the types involved so we use the ad-hoc \AD{`STLC} type.
The description is then the dependent pairing of an \AD{`STLC} tag together with
its decoding defined by a pattern-matching λ-expression in Agda.

Application has two substructures none of which bind extra variables. The first
has a function type and the second the type of its domain. The overall type of
the branch is enforced to be that of the function's codomain by the \AIC{`∎}
constructor.

λ-abstraction has exactly one substructure of type \AB{τ} with a newly-bound
variable of type \AB{σ}. The overall type of the branch is once more enforced
by \AIC{`∎}: it is (\AB{σ} \AIC{`→} \AB{τ}).

\begin{figure}[h]
\ExecuteMetaData[generic-syntax.agda/Generic/Examples/STLC.tex]{stlc}
\caption{\AF{Desc}ription of the Simply Typed Lambda Calculus\label{fig:descstlc}}
\end{figure}

For convenience we use Agda's pattern synonyms corresponding to the
original constructors in~\cref{sec:stlccalculus}:
\AIC{`app} for application and \AIC{`lam} for λ-abstraction. These
synonyms can be used when pattern-matching on a term and Agda resugars
them when displaying a goal. This means that the end user can
seamlessly work with encoded terms without dealing with the gnarly
details of the encoding.  These pattern definitions can omit some
arguments by using ``\AS{\_}'', in which case they will be filled in
by unification just like any other implicit argument: there is no
extra cost to using an encoding!  The only downside is that the
language currently does not allow the user to specify type annotations
for pattern synonyms.

\begin{figure}[h]
\ExecuteMetaData[generic-syntax.agda/Generic/Examples/STLC.tex]{patterns}
\ExecuteMetaData[generic-syntax.agda/Generic/Examples/STLC.tex]{identity}
\caption{Recovering Readable Syntax via Pattern Synonyms}
\end{figure}

\section{Common Combinators and Their Properties}\label{desccomb}

In order to avoid repeatedly re-encoding the same logic, we introduce
combinators demonstrating that descriptions are closed under finite
sums and finite products of recursive positions.

\subsection{Closure under Disjoint Union}

As we wrote, the construction used in \cref{fig:desculc} to define the
syntax for the untyped λ-calculus is classic. It is actually the third
time (the first and second times being the definition of \AF{listD} and
\AF{vecD} in \cref{figure:listD,figure:vecD}) that we use a \AF{Bool}
to distinguish between two constructors.

We define once and for all the disjoint union of two descriptions thanks
to the \AF{\_`+\_} combinator. It comes together with an appropriate
eliminator \AF{case} which, given two continuations, picks the one
corresponding to the chosen branch.

\begin{figure}[h]
\ExecuteMetaData[generic-syntax.agda/Generic/Syntax.tex]{descsum}
\ExecuteMetaData[generic-syntax.agda/Generic/Syntax.tex]{case}
\caption{Descriptions are Closed Under Disjoint Sums\label{figure:descsum}}
\end{figure}

A concrete use case for the disjoint union combinator and its eliminator
will be given in \cref{section:letbinding} where we explain how to seamlessly
enrich any existing syntax with let-bindings and how to use the \AR{Semantics}
framework to elaborate them away.

\subsection{Closure Under Finite Product of Recursive Positions}

Closure under product does not hold in general. Indeed, the equality constraints
introduced by the two end tokens of two descriptions may be incompatible. So far,
a limited form of closure (closure under finite product of recursive positions)
has been sufficient for all of our use cases. Provided a list of pairs of context
extensions and kinds, we can add to an existing description that many recursive
substructures.

\begin{figure}[h]
\ExecuteMetaData[generic-syntax.agda/Generic/Syntax.tex]{xs}
\caption{Descriptions are Closed Under Finite Product of Recursive Positions\label{figure:descprod}}
\end{figure}

As with coproducts, we can define an appropriate eliminator. The function \AF{unXs}
takes a value in the encoding and extracts its constituents ({\AD{All} \AB{P} \AB{xs}}
is defined in Agda's standard library and makes sure that the predicate \AB{P} holds
true of all the elements in the list \AB{xs}).

\begin{figure}[h]
\ExecuteMetaData[generic-syntax.agda/Generic/Syntax.tex]{unxs}
\caption{Breaking Down a Finite Product of Recursive Positions\label{figure:descprodelim}}
\end{figure}

We will see in \cref{section:letbinding} how to define let-bindings as a generic
language extension and their inlining as a generic semantics over the extended
syntax and into the base one. Closure under a finite product of recursive positions
demonstrates that we could extend this construction to parallel (or even mutually-recursive)
let-bindings where the number and the types of the bound expressions can be arbitrary.
We will not go into the details of this construction as it is essentially a combination
of \AF{Xs}, \AF{unXs} and the techniques used when defining single let-bindings.
