\chapter{Generic Scope Safe and Well Kinded Programs for Syntaxes}\label{chapter:generic-semantics}

The set of constraints we called a \AR{Semantics} in
\cref{section:generic-semantics} for the specific example of the
simply typed λ-calculus could be divided in two groups: the one
arising from the fact that we need to be able to push environment
values under binders and the ones in one-to-one correspondence
with constructors for the language.

Based on this observation, we can define a generic notion of semantics
for all syntax descriptions. It is once more parametrised by two
{(\AB{I}\AF{─Scoped})} families \AB{𝓥} and \AB{𝓒} corresponding
respectively to values associated to bound variables and computations
delivered by evaluating terms.

\ExecuteMetaData[generic-syntax.agda/Generic/Semantics.tex]{semrec}

These two families have to abide by three constraints. First, values
should be thinnable so that we can push the evaluation environment
under binders.

\ExecuteMetaData[generic-syntax.agda/Generic/Semantics.tex]{thv}

Second, values should embed into computations for us to be able to
return the value associated to a variable in the environment as the
result of its evaluation.

\ExecuteMetaData[generic-syntax.agda/Generic/Semantics.tex]{var}

Third, we have a constraint similar to the one needed to define \AF{fold} in
\cref{section:data} (\cref{figure:datamu}). We should have an algebra taking
a term whose substructures have already been evaluated and returning a
computation for the overall term.

\ExecuteMetaData[generic-syntax.agda/Generic/Semantics.tex]{alg}

To make formal this idea of ``hav[ing] already been evaluated'' we crucially use
the fact that the meaning of a description is defined in terms of a function
interpreting substructures which has the type (\AF{List} \AB{I} \AS{→} \AB{I}\AF{─Scoped}),
i.e. that gets access to the current scope but also the exact list of the newly
bound variables' kinds.

We define a function \AF{Kripke} by case analysis on the number of newly bound
variables. It is essentially a subcomputation waiting for a value associated to
each one of the fresh variables. If it's $0$ we expect the substructure to be a
computation corresponding to the result of the evaluation function's recursive
call; but if there are newly bound variables then we expect to have a function
space. In any context extension, it will take an environment of values for the
newly-bound variables and produce a computation corresponding to the evaluation
of the body of the binder.

\ExecuteMetaData[shared/Data/Environment.tex]{kripke}

%The name \AF{Kripke} comes from the connection with the interpretation of
%implication in Kripke models where if something holds in one world, it holds
%in all successive ones.

It is once more the case that the abstract notion of Semantics comes with a
fundamental lemma: all \AB{I} \AF{─Scoped} families \AB{𝓥} and \AB{𝓒} satisfying
the three criteria we have put forward give rise to an evaluation function.
We introduce a notion of computation \AF{\_─Comp} analogous to that of environments:
instead of associating values to variables, it associates computations to terms.

\begin{figure}[h]
\ExecuteMetaData[generic-syntax.agda/Generic/Semantics.tex]{comp}
\caption{\AF{\_─Comp}: Associating Computations to Terms}
\end{figure}

We can now define the type of the fundamental lemma (called \AF{semantics}) which
takes a semantics and returns a function from environments to computations. It is
defined mutually with a function \AF{body} turning syntactic binders into
semantics binders: to each de Bruijn \AF{Scope} (i.e. a substructure in a potentially
extended context) it associates a \AF{Kripke} (i.e. a subcomputation expecting a
value for each newly bound variable).

\begin{figure}[h]
\ExecuteMetaData[generic-syntax.agda/Generic/Semantics.tex]{semtype}
\caption{Statement of the Fundamental Lemma of \AR{Semantics}}
\end{figure}

The proof of \AF{semantics} is straightforward now that we have clearly identified the
problem's structure and the constraints we need to enforce. If the term considered
is a variable, we lookup the associated value in the evaluation environment and
turn it into a computation using \ARF{var}. If it is a non variable constructor
then we call \AF{fmap} to evaluate the substructures using \AF{body} and then
call the \ARF{alg}ebra to combine these results.

\begin{figure}[h]
\ExecuteMetaData[generic-syntax.agda/Generic/Semantics.tex]{semproof}
\caption{Proof of the Fundamental Lemma of \AR{Semantics} -- \AF{semantics}}
\end{figure}

The auxiliary lemma \AF{body} distinguishes two cases. If no new variable has
been bound in the recursive substructure, it is a matter of calling \AF{semantics}
recursively. Otherwise we are provided with a \AF{Thinning}, some additional
values and evaluate the substructure in the thinned and extended evaluation
environment thanks to a auxiliary function \AF{\_>>\_} which given two
environments {(\AB{Γ} \AR{─Env}) \AB{𝓥} \AB{Θ}} and {(\AB{Δ} \AR{─Env}) \AB{𝓥} \AB{Θ}}
produces an environment {((\AB{Γ} \AF{++} \AB{Δ}) \AR{─Env}) \AB{𝓥} \AB{Θ})}.

\begin{figure}[h]
\ExecuteMetaData[generic-syntax.agda/Generic/Semantics.tex]{bodyproof}
\caption{Proof of the Fundamental Lemma of \AR{Semantics} -- \AF{body}\label{fig:genbody}}
\end{figure}

Given that \AF{fmap} introduces one level of indirection between the recursive
calls and the subterms they are acting upon, the fact that our terms are indexed
by a \AF{Size} is once more crucial in getting the termination checker to see
that our proof is indeed well founded.

Because most of our examples involve closed terms (for which we have
introduced a special notation in \cref{fig:closedtm}), we immediately
introduce \AF{closed}, a corollary of the fundamental lemma of semantics
for the special cases of closed terms in Figure~\ref{fig:closedsem}.
Given a \AR{Semantics} with value type \AB{𝓥} and computation type \AB{𝓒},
we can evaluate a closed term of type \AB{σ} and obtain a computation of
type {(\AB{𝓒} \AB{σ} \AIC{[]})} by kickstarting the evaluation with an
empty environment.

\begin{figure}[h]
\ExecuteMetaData[generic-syntax.agda/Generic/Semantics.tex]{closed}
\caption{Special Case: Fundamental Lemma of \AR{Semantics} for Closed Terms\label{fig:closedsem}}
\end{figure}

\section{Our First Generic Programs: Renaming and Substitution}\label{section:renandsub}

Similarly to \cref{sec:syntactic} renaming and substitutions can be defined generically
for all syntax descriptions.

\paragraph{Renaming} is a semantics with \AF{Var} as values and \AD{Tm} as computations.
The first two constraints on \AF{Var} described earlier are trivially satisfied. Observing
that renaming strictly respects the structure of the term it goes through, it makes
sense for the algebra to be implemented using \AF{fmap}. When dealing with the body
of a binder, we `reify' the \AF{Kripke} function by evaluating it in an extended
context and feeding it placeholder values corresponding to the extra variables
introduced by that context. This is reminiscent both of what we did in
\cref{sec:syntactic} and the definition of reification in the setting of normalisation
by evaluation (see e.g. Coquand's work~\citeyear{coquand2002formalised}).

\begin{figure}[h]
\ExecuteMetaData[generic-syntax.agda/Generic/Semantics/Syntactic.tex]{renaming}
\caption{Renaming: A Generic Semantics for Syntaxes with Binding\label{fig:genrensem}}
\end{figure}

From this instance, we can derive the proof that all terms are \AF{Thinnable} as
a corollary of the fundamental lemma of \AR{Semantics}.

\begin{figure}[h]
\ExecuteMetaData[generic-syntax.agda/Generic/Semantics/Syntactic.tex]{thTm}
\caption{Corollary: Generic Thinning\label{fig:genren}}
\end{figure}

\paragraph{Substitution} is defined in a similar manner with \AD{Tm}
as both values and computations. Of the two constraints applying to
terms as values, the first one corresponds to renaming and the second
one is trivial. The algebra is once more defined by using \AF{fmap} and
reifying the bodies of binders. We can, once more, obtain parallel
substitution as a corollary of the fundamental lemma of \AR{Semantics}.

\begin{figure}[h]
\ExecuteMetaData[generic-syntax.agda/Generic/Semantics/Syntactic.tex]{substitution}
\ExecuteMetaData[generic-syntax.agda/Generic/Semantics/Syntactic.tex]{sub}
\caption{Generic Parallel Substitution for All Syntaxes with Binding\label{fig:gensub}}
\end{figure}

\paragraph{The reification process} mentioned in the definition of renaming
and substitution can be implemented generically for \semrec{} families
which have \AR{VarLike} values, i.e.  values which are thinnable and
such that we can craft placeholder values in non-empty contexts. It is
almost immediate that both \AD{Var} and \AD{Tm} are \AR{VarLike} (with
proofs \AF{vl\textasciicircum{}Var} and \AF{vl\textasciicircum{}Tm},
respectively).

\begin{figure}[h]
\ExecuteMetaData[shared/Data/Var/Varlike.tex]{varlike}
\caption{\AR{VarLike}: \AF{Thinnable} and with placeholder values}
\end{figure}

\label{sec:varlike:base}
Given a proof that \AB{𝓥} is \AR{VarLike}, we can manufacture
several useful \AB{𝓥}-environments. We provide users with
\AF{base} of type {(\AB{Γ} \AR{─Env}) \AB{𝓥} \AB{Γ}},
\AF{fresh\textsuperscript{r}} of type {(\AB{Γ} \AR{─Env}) \AB{𝓥} (\AB{Δ} \AF{++} \AB{Γ})}
and \AF{fresh\textsuperscript{l}} of type {(\AB{Γ} \AR{─Env}) \AB{𝓥} (\AB{Γ} \AF{++} \AB{Δ})}
by combining the use of placeholder values and thinnings.
In the \AD{Var} case these very general definitions respectively specialise
to the identity renaming for a context \AB{Γ} and the injection of \AB{Γ}
fresh variables to the right or the left of an ambient context \AB{Δ}.

For any \AR{VarLike} \AB{𝓥}, we can define \AF{fresh\textsuperscript{r}} of
type {((\AB{Γ} \AR{─Env}) \AB{𝓥} (\AB{Δ} \AF{++} \AB{Γ}))} and \AF{fresh\textsuperscript{l}} of
type {((\AB{Γ} \AR{─Env}) \AB{𝓥} (\AB{Γ} \AF{++} \AB{Δ}))} by combining the use
of placeholder values and thinnings.

Using these definitions, we can then implement \AF{reify} as in
Figure~\ref{fig:kripkereify} turning \AF{Kripke} function spaces
from \AB{𝓥} to \AB{𝓒} into \AF{Scope}s of \AB{𝓒} computations.

\begin{figure}[h]
\ExecuteMetaData[shared/Data/Var/Varlike.tex]{reify}
\caption{Generic Reification thanks to \AR{VarLike} Values\label{fig:kripkereify}}
\end{figure}

We can now showcase other usages by providing a catalogue of generic programs
for syntaxes with binding.

\section{Printing with Names}\label{section:genericprinting}

Coming back to our work on (rudimentary) printing with names in
\cref{prettyprint}, we can now give a generic account of it. This is a
particularly interesting example because it demonstrates that we may
sometimes want to give \AD{Desc} a different semantics to accommodate
a specific use-case: we do not want our users to deal explicitly with
name generation, explicit variable binding, etc.

Unlike renaming or substitution, this generic program will require user
guidance: there is no way for us to guess how an encoded term should be
printed. We can however take care of the name generation, deal with variable
binding, and implement the traversal generically. We are going to reuse
some of the components defined in \cref{prettyprint}: we can rely on the
same state monad (\AF{Fresh}) for name generation, the same \AF{fresh}
function and the same notions of \AF{Name} and \AF{Printer} for the
semantics' values and computations.
%
We want our printer to have type:
\begin{agdasnippet}
\ExecuteMetaData[generic-syntax.agda/Generic/Semantics/Printing.tex]{printtype}
\end{agdasnippet}
%
where \AF{Display} explains how to print one `layer' of term provided that
we are handed the \AF{Pieces} corresponding to the printed subterm and
names for the bound variables.
%
\begin{agdasnippet}
  \ExecuteMetaData[generic-syntax.agda/Generic/Semantics/Printing.tex]{display}
\end{agdasnippet}
%
Reusing the notion of \AR{Name} introduced in Section~\ref{section:printing},
we can make \AF{Pieces} formal.
%
The structure of \AR{Semantics} would suggest giving our users an interface
where sub-structures are interpreted as \AF{Kripke} function spaces expecting
fresh names for the fresh variables and returning a printer i.e. a monadic
computation returning a \AD{String}. However we can do better: we can
preemptively generate a set of fresh names for the newly-bound variables and
hand them to the user together with the result of printing the body with
these names. As usual we have a special case for the substructures without
any newly-bound variable. Note that the specific target context of the
environment of \AF{Name}s is only picked for convenience as \AF{Name}
ignores the scope: ({\AB{Δ} \AF{++} \AB{Γ}}) is what \AF{freshˡ} gives us.
%
In other words: \AF{Pieces} states that a subterm has already been printed
if we have a string representation of it together with an environment of
\AR{Name}s we have attached to the newly-bound variables this structure
contains.
%
\begin{agdasnippet}
\ExecuteMetaData[generic-syntax.agda/Generic/Semantics/Printing.tex]{pieces}
\end{agdasnippet}
%
The key observation that will help us define a generic printer is that
\AF{Fresh} composed with \AR{Name} is \AR{VarLike}. Indeed, as the
composition of a functor and a trivially thinnable \AR{Wrap}per,
\AF{Fresh} is \AF{Thinnable}, and \AF{fresh} (defined in
Figure~\ref{fig:fresh}) is the proof that we can generate
placeholder values thanks to the name supply.

\begin{agdasnippet}
\ExecuteMetaData[generic-syntax.agda/Generic/Semantics/Printing.tex]{vlmname}
\end{agdasnippet}

This \AR{VarLike} instance empowers us to reify in an effectful manner
a \AF{Kripke} function space taking \AF{Name}s and returning a \AF{Printer}
to a set of \AF{Pieces}.

\begin{agdasnippet}
\ExecuteMetaData[generic-syntax.agda/Generic/Semantics/Printing.tex]{reifytype}
\end{agdasnippet}

In case there are no newly bound variables, the \AF{Kripke} function space
collapses to a mere \AR{Printer} which is precisely the wrapped version of
the type we expect.

\begin{agdasnippet}
\ExecuteMetaData[generic-syntax.agda/Generic/Semantics/Printing.tex]{reifybase}
\end{agdasnippet}

Otherwise we proceed in a manner reminiscent of the pure reification function
defined in Figure~\ref{fig:genericreify}. We start by generating an environment
of names for the newly-bound variables by using the fact that \AF{Fresh} composed
with \AF{Name} is \AR{VarLike} together with the fact that environments are
Traversable~\cite{mcbride_paterson_2008}, %%%
and thus admit the standard Haskell-like \AF{mapA} and \AF{sequenceA}
traversals. %%%
We then run the \AF{Kripke} function
on these names to obtain the string representation of the subterm. We finally
return the names we used together with this string.

\begin{agdasnippet}
\ExecuteMetaData[generic-syntax.agda/Generic/Semantics/Printing.tex]{reifypieces}
\end{agdasnippet}

We can put all of these pieces together to obtain the \AF{Printing} semantics
presented in Figure~\ref{fig:genericprinting}.
The first two constraints can be trivially discharged. When defining the
algebra we start by reifying the subterms, then use the fact that  one ``layer''
of term of our syntaxes with binding is always traversable to combine all of
these results into a value we can apply our display function to.

\begin{figure}[h]
\ExecuteMetaData[generic-syntax.agda/Generic/Semantics/Printing.tex]{printing}
\caption{Printing with \AF{Name}s as a \AR{Semantics}}\label{fig:genericprinting}
\end{figure}

This allows us to write a \AF{printer} for open terms as demonstrated in
Figure~\ref{fig:genericprint}.
We start by using \AF{base} (defined in Section~\ref{sec:varlike:base})
to generate an environment of \AR{Name}s for the free variables, then use
our semantics to get a \AF{printer} which we can run using a stream \AF{names} of distinct
strings as our name supply.

\begin{figure}[h]
\ExecuteMetaData[generic-syntax.agda/Generic/Semantics/Printing.tex]{print}
\caption{Generic Printer for Open Terms}\label{fig:genericprint}
\end{figure}


\paragraph{Untyped λ-calculus} Defining a printer for the untyped
λ-calculus is now very easy: we define a \AF{Display} by case analysis.
In the application case, we combine the string representation of the
function, wrap its argument's representation between parentheses and
concatenate the two together. In the lambda abstraction case, we are
handed the name the bound variable was assigned together with the body's
representation; it is once more a matter of putting the \AF{Pieces}
together.

\begin{agdasnippet}
\ExecuteMetaData[generic-syntax.agda/Generic/Examples/Printing.tex]{printUTLC}
\end{agdasnippet}

As always, these functions are readily executable and we can check
their behaviour by writing tests. First, we print the identity function
defined in Figure~\ref{fig:identity}
in an empty context and verify that we do obtain the string \AStr{"λa. a"}.
Next, we print an open term in a context of size two and can immediately
observe that names are generated for the free variables first, and then the
expression itself is printed.

\begin{minipage}[t]{0.45\textwidth}
  \begin{agdasnippet}
  \ExecuteMetaData[generic-syntax.agda/Generic/Examples/Printing.tex]{printid}
  \end{agdasnippet}
\end{minipage}
\begin{minipage}[t]{0.45\textwidth}
  \begin{agdasnippet}
  \ExecuteMetaData[generic-syntax.agda/Generic/Examples/Printing.tex]{printopen}
  \end{agdasnippet}
\end{minipage}

\section{(Unsafe) Normalisation by Evaluation}\label{section:unsafenbyeval}

A key type of traversal we have not studied yet is a language's
evaluator. Our universe of syntaxes with binding does not impose
any typing discipline on the user-defined languages and as such
cannot guarantee their totality. This is embodied by one of our running
examples: the untyped λ-calculus. As a consequence there
is no hope for a safe generic framework to define normalisation
functions.

The clear connection between the \AF{Kripke} functional space
characteristic of our semantics and the one that shows up in
normalisation by evaluation suggests we ought to manage to
give an unsafe generic framework for normalisation by evaluation.
By temporarily \textbf{disabling Agda's positivity checker},
we can define a generic reflexive domain \AD{Dm}
(cf. \cref{fig:reflexivedomain}) in which to
interpret our syntaxes. It has three constructors corresponding
respectively to a free variable, a constructor's counterpart where
scopes have become \AF{Kripke} functional spaces on \AD{Dm} and
an error token because the evaluation of untyped programs may go wrong
(a user may for instance try to add a function and a number).

\begin{figure}[h]
\ExecuteMetaData[generic-syntax.agda/Generic/Semantics/NbyE.tex]{domain}
\caption{Generic Reflexive Domain}\label{fig:reflexivedomain}
\end{figure}

This datatype definition is utterly unsafe. The more conservative
user will happily restrict themselves to particular syntaxes where
the typed settings allows for domain to be defined as a logical
predicate or opt instead for a step-indexed approach. We did develop
a step-indexed model construction but it was unusable: we could not
get Agda to normalise even the simplest of terms.

But this domain does make it possible to define a generic \AF{nbe}
semantics which, given a term, produces a value in the reflexive
domain. Thanks to the fact we have picked a universe of finitary syntaxes, we
can \emph{traverse}~(\cite{mcbride_paterson_2008,DBLP:journals/jfp/GibbonsO09})
the functor to define
a (potentially failing) reification function turning elements of the
reflexive domain into terms. By composing them, we obtain the
normalisation function which gives its name to normalisation by
evaluation.

The user still has to explicitly pass an interpretation of
the various constructors because there is no way for us to
know what the binders are supposed to represent: they may
stand for λ-abstractions, $\Sigma$-types, fixpoints, or
anything else.

\begin{figure}[h]
\ExecuteMetaData[generic-syntax.agda/Generic/Semantics/NbyE.tex]{nbe-setup}
\caption{Generic Normalisation by Evaluation Framework\label{defn:NbE}}
\end{figure}

\subsection{Example: Evaluator for the Untyped Lambda-Calculus}

Using this setup, we can write a normaliser for the untyped λ-calculus
by providing an algebra. The key observation that allows us to implement
this algebra is that we can turn a Kripke function, \AB{f}, mapping values
of type \AB{σ} to computations of type \AB{τ} into an Agda function from
values of type \AB{σ} to computations of type \AB{τ}. This is witnessed
by the application function (\AF{\_\$\$\_}) defined in Figure~\ref{fig:kripkeapp}:
we first use \AF{extract} (defined in Figure~\ref{fig:Thinnable}) to obtain
a function taking environments of values to computations. We then use the
combinators defined in Figure~\ref{fig:baseenv} to manufacture the singleton
environment {(\AF{ε} \AB{∙} \AB{t})} containing the value \AB{t} of type
\AB{σ}.

\begin{figure}[h]
  \ExecuteMetaData[generic-syntax.agda/Generic/Examples/NbyE.tex]{app}
\caption{Applying a Kripke Function to an argument}\label{fig:kripkeapp}
\end{figure}

We now define two patterns for semantical values: one for application and
the other for lambda abstraction. This should make the case of interest of
our algebra (a function applied to an argument) fairly readable.

\begin{figure}[h]
\ExecuteMetaData[generic-syntax.agda/Generic/Examples/NbyE.tex]{nbepatterns}
\caption{Pattern synonyms for UTLC-specific \AD{Dm}~values}
\end{figure}

We finally define the algebra by case analysis: if the node at hand is an
application and its first component evaluates to a lambda, we can apply
the function to its argument using \AF{\_\$\$\_}. Otherwise we have either a
stuck application or a lambda, in other words we already have a value and can
simply return it using \AIC{C}.

\begin{figure}[h]
  \ExecuteMetaData[generic-syntax.agda/Generic/Examples/NbyE.tex]{nbelc}
\caption{Normalisation by Evaluation for the Untyped λ-Calculus}
\end{figure}

We have not used the \AIC{⊥} constructor so \emph{if} the evaluation terminates
(by disabling totality checking we have lost all guarantees of the sort) we know
we will get a term in normal form. See for instance in
Figure~\ref{fig:normuntyped} the evaluation of an untyped yet normalising
term: {\app{(\lam{x}{x})}{(\app{(\lam{x}{x})}{(\lam{x}{x})})}}
normalises to {(\lam{x}{x})}.

\begin{figure}[h]
\ExecuteMetaData[generic-syntax.agda/Generic/Examples/NbyE.tex]{example}
\caption{Example of a normalising untyped term}
\label{fig:normuntyped}
\end{figure}

