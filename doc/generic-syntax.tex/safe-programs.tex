\chapter{Generic Scope Safe and Well Kinded Programs for Syntaxes}\label{chapter:generic-semantics}

The set of constraints we called a \AR{Semantics} in \cref{section:generic-semantics}
for the specific example of the simply typed λ-calculus could be divided in two groups:
the one arising from the fact that we need to be able to push environment values under
binders and the ones in one-to-one correspondence with constructors for the language.

Based on this observation, we can define a generic notion of semantics for all syntax
descriptions. It is once more parametrised by two (\AB{I}\AF{─Scoped}) families \AB{𝓥}
and \AB{𝓒} corresponding respectively to values associated to bound variables and
computations delivered by evaluating terms.

\ExecuteMetaData[generic-syntax.agda/Generic/Semantics.tex]{semrec}

These two families have to abide by three constraints. First, values should be thinnable
so that we can push the evaluation environment under binders.

\ExecuteMetaData[generic-syntax.agda/Generic/Semantics.tex]{thv}

Second, values should embed into computations for us to be able to return the value
associated to a variable in the environment as the result of its evaluation.

\ExecuteMetaData[generic-syntax.agda/Generic/Semantics.tex]{var}

Third, we have a constraint similar to the one needed to define \AF{fold} in
\cref{section:data} (\cref{figure:datamu}). We should have an algebra taking
a term whose substructures have already been evaluated and returning a
computation for the overall term.

\ExecuteMetaData[generic-syntax.agda/Generic/Semantics.tex]{alg}

To make formal this idea of ``hav[ing] already been evaluated'' we crucially use
the fact that the meaning of a description is defined in terms of a function
interpreting substructures which has the type (\AF{List} \AB{I} \AS{→} \AB{I}\AF{─Scoped}),
i.e. that gets access to the current scope but also the exact list of the newly
bound variables' kinds.

We define a function \AF{Kripke} by case analysis on the number of newly bound
variables. It is essentially a subcomputation waiting for a value associated to
each one of the fresh variables. If it's $0$ we expect the substructure to be a
computation corresponding to the result of the evaluation function's recursive
call; but if there are newly bound variables then we expect to have a function
space. In any context extension, it will take an environment of values for the
newly-bound variables and produce a computation corresponding to the evaluation
of the body of the binder.

\ExecuteMetaData[shared/Data/Environment.tex]{kripke}

%The name \AF{Kripke} comes from the connection with the interpretation of
%implication in Kripke models where if something holds in one world, it holds
%in all successive ones.

It is once more the case that the abstract notion of Semantics comes with a
fundamental lemma: all \AB{I} \AF{─Scoped} families \AB{𝓥} and \AB{𝓒} satisfying
the three criteria we have put forward give rise to an evaluation function.
We introduce a notion of computation \AF{\_─Comp} analogous to that of environments:
instead of associating values to variables, it associates computations to terms.

\begin{figure}[h]
\ExecuteMetaData[generic-syntax.agda/Generic/Semantics.tex]{comp}
\caption{\AF{\_─Comp}: Associating Computations to Terms}
\end{figure}

We can now define the type of the fundamental lemma (called \AF{semantics}) which
takes a semantics and returns a function from environments to computations. It is
defined mutually with a function \AF{body} turning syntactic binders into
semantics binders: to each de Bruijn \AF{Scope} (i.e. a substructure in a potentially
extended context) it associates a \AF{Kripke} (i.e. a subcomputation expecting a
value for each newly bound variable).

\begin{figure}[h]
\ExecuteMetaData[generic-syntax.agda/Generic/Semantics.tex]{semtype}
\caption{Statement of the Fundamental Lemma of \AR{Semantics}}
\end{figure}

The proof of \AF{semantics} is straightforward now that we have clearly identified the
problem's structure and the constraints we need to enforce. If the term considered
is a variable, we lookup the associated value in the evaluation environment and
turn it into a computation using \ARF{var}. If it is a non variable constructor
then we call \AF{fmap} to evaluate the substructures using \AF{body} and then
call the \ARF{alg}ebra to combine these results.

\begin{figure}[h]
\ExecuteMetaData[generic-syntax.agda/Generic/Semantics.tex]{semproof}
\caption{Proof of the Fundamental Lemma of \AR{Semantics} -- \AF{semantics}}
\end{figure}

The auxiliary lemma \AF{body} distinguishes two cases. If no new variable has
been bound in the recursive substructure, it is a matter of calling \AF{semantics}
recursively. Otherwise we are provided with a \AF{Thinning}, some additional
values and evaluate the substructure in the thinned and extended evaluation
environment thanks to a auxiliary function \AF{\_>>\_} which given two
environments {(\AB{Γ} \AR{─Env}) \AB{𝓥} \AB{Θ}} and {(\AB{Δ} \AR{─Env}) \AB{𝓥} \AB{Θ}}
produces an environment {((\AB{Γ} \AF{++} \AB{Δ}) \AR{─Env}) \AB{𝓥} \AB{Θ})}.

\begin{figure}[h]
\ExecuteMetaData[generic-syntax.agda/Generic/Semantics.tex]{bodyproof}
\caption{Proof of the Fundamental Lemma of \AR{Semantics} -- \AF{body}\label{fig:genbody}}
\end{figure}

Given that \AF{fmap} introduces one level of indirection between the recursive
calls and the subterms they are acting upon, the fact that our terms are indexed
by a \AF{Size} is once more crucial in getting the termination checker to see
that our proof is indeed well founded.


Because most of our examples involve closed terms (for which we have introduced a special
notation in \cref{fig:closedtm}), we define a specialised of the fundamental lemma of semantics
for closed terms and apply it to the empty environment.

\begin{figure}[h]
\ExecuteMetaData[generic-syntax.agda/Generic/Semantics.tex]{closed}
\caption{Special Case: Fundamental Lemma of \AR{Semantics} for Closed Terms\label{fig:closedsem}}
\end{figure}

\section{Our First Generic Programs: Renaming and Substitution}\label{section:renandsub}

Similarly to \cref{sec:syntactic} renaming and substitutions can be defined generically
for all syntax descriptions.

\paragraph{Renaming} is a semantics with \AF{Var} as values and \AD{Tm} as computations.
The first two constraints on \AF{Var} described earlier are trivially satisfied. Observing
that renaming strictly respects the structure of the term it goes through, it makes
sense for the algebra to be implemented using \AF{fmap}. When dealing with the body
of a binder, we `reify' the \AF{Kripke} function by evaluating it in an extended
context and feeding it placeholder values corresponding to the extra variables
introduced by that context. This is reminiscent both of what we did in
\cref{sec:syntactic} and the definition of reification in the setting of normalisation
by evaluation (see e.g. Coquand's work~\citeyear{coquand2002formalised}).

\begin{figure}[h]
\ExecuteMetaData[generic-syntax.agda/Generic/Semantics/Syntactic.tex]{renaming}
\caption{Renaming: A Generic Semantics for Syntaxes with Binding\label{fig:genrensem}}
\end{figure}

From this instance, we can derive the proof that all terms are \AF{Thinnable} as
a corollary of the fundamental lemma of \AR{Semantics}.

\begin{figure}[h]
\ExecuteMetaData[generic-syntax.agda/Generic/Semantics/Syntactic.tex]{thTm}
\caption{Corollary: Generic Thinning\label{fig:genren}}
\end{figure}

\paragraph{Substitution} is defined in a similar manner with \AD{Tm} as both values and computations.
Of the two constraints applying to terms as values, the first one corresponds to renaming
and the second one is trivial. The algebra is once more defined by using \AF{fmap} and
reifying the bodies of binders. We can, once more, obtain parallel substitution as a
corollary of the fundamental lemma of \AR{Semantics}.

\begin{figure}[h]
\ExecuteMetaData[generic-syntax.agda/Generic/Semantics/Syntactic.tex]{substitution}
\ExecuteMetaData[generic-syntax.agda/Generic/Semantics/Syntactic.tex]{sub}
\caption{Generic Parallel Substitution for All Syntaxes with Binding\label{fig:gensub}}
\end{figure}

\paragraph{The reification process} mentioned in the definition of renaming and substitution
can be implemented generically for \AR{Semantics} families which have \AR{VarLike}
values (\AF{vl\textasciicircum{}Var} and \AF{vl\textasciicircum{}Tm} are proofs of
\AR{VarLike} for \AD{Var} and \AD{Tm} respectively) i.e. values which are thinnable
and such that we can craft placeholder values in non-empty contexts.

\begin{figure}[h]
\ExecuteMetaData[shared/Data/Var/Varlike.tex]{varlike}
\caption{\AR{VarLike}: \AF{Thinnable} and with placeholder values}
\end{figure}

For any \AR{VarLike} \AB{𝓥}, we can define \AF{fresh\textsuperscript{r}} of
type {((\AB{Γ} \AR{─Env}) \AB{𝓥} (\AB{Δ} \AF{++} \AB{Γ}))} and \AF{fresh\textsuperscript{l}} of
type {((\AB{Γ} \AR{─Env}) \AB{𝓥} (\AB{Γ} \AF{++} \AB{Δ}))} by combining the use
of placeholder values and thinnings. Hence, we can then write a generic \AF{reify}
(\cref{fig:kripkereify}) turning \AF{Kripke} function spaces from \AB{𝓥} to \AB{𝓒}
into \AF{Scope}s of \AB{𝓒} computations.

\begin{figure}[h]
\ExecuteMetaData[shared/Data/Var/Varlike.tex]{reify}
\caption{Generic Reification thanks to \AR{VarLike} Values\label{fig:kripkereify}}
\end{figure}

We can now showcase other usages by providing a catalogue of generic programs
for syntaxes with binding.

\subsection{Sugar and Desugaring as a Semantics}\label{section:letbinding}

One of the advantages of having a universe of programming language
descriptions is the ability to concisely define an \emph{extension}
of an existing language by using \AD{Desc}ription transformers
grafting extra constructors à la Swiestra~\citeyear{swierstra_2008}.
This is made extremely simple by the disjoint sum combinator
\AF{\_`+\_} which we defined in Figure~\ref{figure:descsum}.
An example of such an extension is the addition of let-bindings to
an existing language.

Let bindings allow the user to avoid repeating themselves by naming
sub-expressions and then using these names to refer to the associated
terms. Preprocessors adding these types of mechanisms to existing
languages (from C to CSS) are rather popular. In Figure~\ref{defn:letD},
we introduce a description \AD{Let} which can be used to extend any
language description \AB{d} to a language with let-bindings
(\AB{d} \AF{`+} \AF{Let}).

\begin{figure}[h]
\begin{minipage}[t]{0.45\textwidth}
  \ExecuteMetaData[generic-syntax.agda/Generic/Syntax/LetBinder.tex]{letcode}
\end{minipage}
\begin{minipage}[t]{0.45\textwidth}
  \ExecuteMetaData[generic-syntax.agda/Generic/Syntax/LetBinder.tex]{letpattern}
\end{minipage}
\caption{Description of a single let binding, associated pattern synonyms
  \label{defn:letD}}
\end{figure}

This description states that a let-binding node stores a pair of types
\AB{σ} and \AB{τ} and two subterms. First comes the let-bound
expression of type \AB{σ} and second comes the body of the let which
has type \AB{τ} in a context extended with a fresh variable of type
\AB{σ}. This defines a term of type \AB{τ}.

In a dependently typed language, a type may depend on a value which
in the presence of let bindings may be a variable standing for an
expression. The user naturally does not want it to make any difference
whether they used a variable referring to a let-bound expression or
the expression itself. Various typechecking strategies can accommodate
this expectation: in Coq~\cite{Coq:manual} let bindings are primitive
constructs of the language and have their own typing and reduction
rules whereas in Agda they are elaborated away to the core language
by inlining.

This latter approach to extending a language \AB{d} with let bindings
by inlining them before typechecking can be implemented generically as
a semantics over (\AB{d} \AF{`+} \AF{Let}). For this semantics values
in the environment and computations are both let-free terms. The algebra
of the semantics can be defined by parts thanks to \AF{case}, the eliminator
for \AF{\_`+\_} defined in Figure~\ref{figure:descsum}:
the old constructors are kept the same by
interpreting them using the generic substitution algebra (\AF{Substitution});
whilst the let-binder precisely provides the extra value to be added to the
environment.

\begin{figure}[h]
  \ExecuteMetaData[generic-syntax.agda/Generic/Semantics/Elaboration/LetBinder.tex]{unletcode}
\caption{Desugaring as a \AR{Semantics}\label{defn:UnLet}}
\end{figure}

The process of removing let binders is then kickstarted with the placeholder
environment \AF{id\textasciicircum{}Tm}~=~\AIC{pack}~\AIC{`var}
of type {(\AB{Γ} \AR{─Env}) (\AD{Tm} \AB{d} ∞) \AB{Γ}}.

\begin{figure}[h]
  \ExecuteMetaData[generic-syntax.agda/Generic/Semantics/Elaboration/LetBinder.tex]{unlet}
\caption{Specialising \AF{semantics} with an environment of placeholder values\label{defn:unlet}}
\end{figure}

In less than 10 lines of code we have defined a generic extension of
syntaxes with binding together with a semantics which corresponds
to an elaborator translating away this new construct.
In \cref{cps-transformation} we had focused on STLC only
and showed that it is similarly possible to implement a Continuation
Passing Style transformation as the composition of two semantics
à la Hatcliff and Danvy~\citeyear{hatcliff1994generic}.
The first semantics embeds STLC into Moggi's
Meta-Language~\citeyear{DBLP:journals/iandc/Moggi91} and thus fixes
an evaluation order. The second one translates Moggi's ML back into
STLC in terms of explicit continuations with a fixed return type.

We have demonstrated how easily one can define extensions and combine
them on top of a base language without having to reimplement common
traversals for each one of the intermediate representations. Moreover,
it is possible to define \emph{generic} transformations elaborating
these added features in terms of lower-level ones. This suggests that
this setup could be a good candidate to implement generic compilation
passes and could deal with a framework using a wealth of slightly
different intermediate languages à la Nanopass~\cite{Keep:2013:NFC:2544174.2500618}.

\section{Reference Counting and Inlining as a Semantics}\label{section:inlining}

Although useful in its own right, desugaring all let bindings can lead
to an exponential blow-up in code size. Compiler passes typically try
to maintain sharing by only inlining let-bound expressions which appear
at most one time. Unused expressions are eliminated as dead code whilst
expressions used exactly one time can be inlined: this transformation is
size preserving and opens up opportunities for additional optimisations.

As we will see shortly, we can implement reference counting and size
respecting let-inlining as a generic transformation over all syntaxes
with binding equipped with let binders. This two-pass simple transformation
takes linear time which may seem surprising given the results due to Appel and
Jim~(\citeyear{DBLP:journals/jfp/AppelJ97}). Our optimisation only inlines
let-bound variables whereas theirs also encompasses the reduction of static
β-redexes of (potentially) recursive function. While we can easily count how
often a variable is used in the body of a let binder, the interaction between
inlining and β-reduction in theirs creates cascading simplification opportunities
thus making the problem much harder.

But first, we need to look at an example demonstrating that this is a
slightly subtle matter. Assuming that \AB{expensive} takes a long time
to evaluate, inlining all of the lets in the first expression is a really
good idea whilst we only want to inline the one binding \AB{y} in the
second one to avoid duplicating work. That is to say that the contribution
of the expression bound to \AB{y} in the overall count depends directly
on whether \AB{y} itself appears free in the body of the let which binds it.

\begin{figure}[h]
\begin{minipage}{0.45\textwidth}
  \ExecuteMetaData[generic-syntax.agda/Generic/Syntax/LetCounter.tex]{cheap}
\end{minipage}
\begin{minipage}{0.45\textwidth}
  \ExecuteMetaData[generic-syntax.agda/Generic/Syntax/LetCounter.tex]{expensive}
\end{minipage}
\end{figure}

Our transformation will consist of two passes: the first one will annotate
the tree with accurate count information precisely recording whether
let-bound variables are used \AIC{zero}, \AIC{one}, or \AIC{many} times.
The second one will inline precisely the let-binders whose variable is
used at most once.

During the counting phase we need to be particularly careful not to overestimate
the contribution of a let-bound expression. If the let-bound variable is not used
then we can naturally safely ignore the associated count. But if it used \AIC{many}
times then we know we will not inline this let-binding and the count should
therefore only contribute once to the running total. We define the \AF{control}
combinator in Figure~\ref{fig:combinatorscount} precisely to explicitly handle this
subtle case.

The first step is to introduce the \AD{Counter} additive monoid
(cf. Figure~\ref{fig:countermonoid}). Addition will
allow us to combine counts coming from different subterms: if any of the two
counters is \AIC{zero} then we return the other, otherwise we know we have
\AIC{many} occurences.

\begin{figure}[h]
\begin{minipage}{0.45\textwidth}
  \ExecuteMetaData[generic-syntax.agda/Generic/Syntax/LetCounter.tex]{counter}
\end{minipage}
\begin{minipage}{0.45\textwidth}
  \ExecuteMetaData[generic-syntax.agda/Generic/Syntax/LetCounter.tex]{addition}
\end{minipage}
\caption{The (\AD{Counter}, \AIC{zero}, \AF{\_+\_}) additive monoid}
\label{fig:countermonoid}
\end{figure}

The syntax extension \AF{CLet} defined in Figure~\ref{fig:cletdef} is
a variation on the \AF{Let} syntax extension of Section~\ref{section:letbinding},
attaching a \AD{Counter} to each \AF{Let} node. The annotation process
can then be described as a function computing a
{(\AB{d} \AF{`+} \AF{CLet})} term from a {(\AB{d} \AF{`+} \AF{Let})} one.

\begin{figure}[h]
  \ExecuteMetaData[generic-syntax.agda/Generic/Syntax/LetCounter.tex]{clet}
  \caption{Counted Lets}
  \label{fig:cletdef}
\end{figure}

We keep a tally of the usage information for the variables in scope. This
allows us to know which \AD{Counter} to attach to each \AF{Let} node.
Following the same strategy as in Section~\ref{section:genericscoping},
we use the standard library's \AD{All} to represent this mapping. We say
that a scoped value has been \AF{Counted} if it is paired with a \AD{Count}.

\begin{figure}[h]
\begin{minipage}{0.45\textwidth}
  \ExecuteMetaData[generic-syntax.agda/Generic/Syntax/LetCounter.tex]{count}
\end{minipage}
\begin{minipage}{0.45\textwidth}
  \ExecuteMetaData[generic-syntax.agda/Generic/Semantics/Elaboration/LetCounter.tex]{counted}
\end{minipage}
\caption{Counting i.e. Associating a \AD{Counter}~to each \AD{Var}~in scope.}
\end{figure}

The two most basic counts are described in Figure~\ref{fig:basiccount}: the
empty one is \AIC{zero} everywhere and the one corresponding to a single use
of a single variable \AB{v} which is \AIC{zero} everywhere except for \AB{v}
where it's \AIC{one}.

\begin{figure}[h]
\begin{minipage}{0.45\textwidth}
  \ExecuteMetaData[generic-syntax.agda/Generic/Syntax/LetCounter.tex]{zeros}
\end{minipage}
\begin{minipage}{0.45\textwidth}
  \ExecuteMetaData[generic-syntax.agda/Generic/Syntax/LetCounter.tex]{fromVar}
\end{minipage}
\caption{Zero Count and Count of One for a Specific Variable}\label{fig:basiccount}
\end{figure}

When we collect usage information from different subterms, we need to put the
various counts together. The combinators in Figure~\ref{fig:combinatorscount}
allow us to easily do so: \AF{merge} adds up two counts in a pointwise manner
while \AF{control} uses one \AD{Counter} to decide whether to erase an existing
\AD{Count}. This is particularly convenient when computing the contribution of
a let-bound expression to the total tally: the contribution of the let-bound
expression will only matter if the corresponding variable is actually used.

\begin{figure}[h]
\begin{minipage}{0.5\textwidth}
  \ExecuteMetaData[generic-syntax.agda/Generic/Syntax/LetCounter.tex]{merge}
\end{minipage}
\begin{minipage}{0.4\textwidth}
  \ExecuteMetaData[generic-syntax.agda/Generic/Syntax/LetCounter.tex]{control}
\end{minipage}
\caption{Combinators to Compute \AD{Count}s}\label{fig:combinatorscount}
\end{figure}

We can now focus on the core of the annotation phase. We define a
\AR{Semantics} whose values are variables themselves and whose computations
are the pairing of a term in {(\AB{d} \AF{`+} \AF{CLet})} together with
a \AF{Count}. The variable case is trivial: provided a variable \AB{v},
we return {(\AIC{`var} \AB{v})} together with the count {(\AF{fromVar} \AB{v})}.

The non-let case is purely structural: we reify the \AF{Kripke} function
space and obtain a scope together with the corresponding \AF{Count}. We
unceremoniously \AF{drop} the \AD{Counter}s associated to the variables
bound in this subterm and return the scope together with the tally for
the ambient context.

\begin{figure}[h]
  \ExecuteMetaData[generic-syntax.agda/Generic/Semantics/Elaboration/LetCounter.tex]{reifycount}
  \caption{Purely Structural Case}
\end{figure}

The \AF{Let}-to-\AF{CLet} case in Figure~\ref{fig:lettoclet} is the most
interesting one. We start by reifying the \AB{body} of the let binder which
gives us a tally \AB{cx} for the bound variable and \AB{ct} for the body's
contribution to the ambient environment's \AD{Count}. We annotate the node
with \AB{cx} and use it as a \AF{control} to decide whether we are going to
merge any of the let-bound's expression contribution \AB{ce} to form the
overall tally.

\begin{figure}[h]
  \ExecuteMetaData[generic-syntax.agda/Generic/Semantics/Elaboration/LetCounter.tex]{letcount}
  \caption{Annotating Let Binders}\label{fig:lettoclet}
\end{figure}

Putting all of these things together we obtain the \AR{Semantics} \AF{Annotate}.
We promptly specialise it using an environment of placeholder values to obtain
the traversal \AF{annotate} elaborating raw let-binders into counted ones.

\begin{figure}[h]
  \ExecuteMetaData[generic-syntax.agda/Generic/Semantics/Elaboration/LetCounter.tex]{annotate}
\caption{Specialising \AR{semantics}~to obtain an annotation function}
\end{figure}

Using techniques similar to the ones described in Section~\ref{section:letbinding},
we can write an \AF{Inline} semantics working on {(\AB{d} \AF{`+} \AF{CLet})} terms
and producing {(\AB{d} \AF{`+} \AF{Let})} ones. We make sure to preserve all the
let-binders annotated with \AIC{many} and to inline all the other ones. By composing
\AF{Annotate} with \AF{Inline} we obtain a size-preserving generic optimisation pass.

\section{(Unsafe) Normalisation by Evaluation}\label{section:unsafenbyeval}

A key type of traversal we have not studied yet is a language's
evaluator. Our universe of syntaxes with binding does not impose
any typing discipline on the user-defined languages and as such
cannot guarantee their totality. This is embodied by one of our running
examples: the untyped λ-calculus. As a consequence there
is no hope for a safe generic framework to define normalisation
functions.

The clear connection between the \AF{Kripke} functional space
characteristic of our semantics and the one that shows up in
normalisation by evaluation suggests we ought to manage to
give an unsafe generic framework for normalisation by evaluation.
By temporarily \textbf{disabling Agda's positivity checker},
we can define a generic reflexive domain \AD{Dm}
(cf. \cref{fig:reflexivedomain}) in which to
interpret our syntaxes. It has three constructors corresponding
respectively to a free variable, a constructor's counterpart where
scopes have become \AF{Kripke} functional spaces on \AD{Dm} and
an error token because the evaluation of untyped programs may go wrong
(a user may for instance try to add a function and a number).

\begin{figure}[h]
\ExecuteMetaData[generic-syntax.agda/Generic/Semantics/NbyE.tex]{domain}
\caption{Generic Reflexive Domain}\label{fig:reflexivedomain}
\end{figure}

This datatype definition is utterly unsafe. The more conservative
user will happily restrict themselves to particular syntaxes where
the typed settings allows for domain to be defined as a logical
predicate or opt instead for a step-indexed approach. We did develop
a step-indexed model construction but it was unusable: we could not
get Agda to normalise even the simplest of terms.

But this domain does make it possible to define a generic \AF{nbe}
semantics which, given a term, produces a value in the reflexive
domain. Thanks to the fact we have picked a universe of finitary syntaxes, we
can \emph{traverse}~(\cite{mcbride_paterson_2008,DBLP:journals/jfp/GibbonsO09})
the functor to define
a (potentially failing) reification function turning elements of the
reflexive domain into terms. By composing them, we obtain the
normalisation function which gives its name to normalisation by
evaluation.

The user still has to explicitly pass an interpretation of
the various constructors because there is no way for us to
know what the binders are supposed to represent: they may
stand for λ-abstractions, $\Sigma$-types, fixpoints, or
anything else.

\begin{figure}[h]
\ExecuteMetaData[generic-syntax.agda/Generic/Semantics/NbyE.tex]{nbe-setup}
\caption{Generic Normalisation by Evaluation Framework\label{defn:NbE}}
\end{figure}

\subsection{Example: Evaluator for the Untyped Lambda-Calculus}

Using this setup, we can write a normaliser for the untyped λ-calculus
by providing an algebra. The key observation that allows us to implement
this algebra is that we can turn a Kripke function, \AB{f}, mapping values
of type \AB{σ} to computations of type \AB{τ} into an Agda function from
values of type \AB{σ} to computations of type \AB{τ}. This is witnessed
by the application function (\AF{\_\$\$\_}) defined in Figure~\ref{fig:kripkeapp}:
we first use \AF{extract} (defined in Figure~\ref{fig:Thinnable}) to obtain
a function taking environments of values to computations. We then use the
combinators defined in Figure~\ref{fig:baseenv} to manufacture the singleton
environment {(\AF{ε} \AB{∙} \AB{t})} containing the value \AB{t} of type
\AB{σ}.

\begin{figure}[h]
  \ExecuteMetaData[generic-syntax.agda/Generic/Examples/NbyE.tex]{app}
\caption{Applying a Kripke Function to an argument}\label{fig:kripkeapp}
\end{figure}

We now define two patterns for semantical values: one for application and
the other for lambda abstraction. This should make the case of interest of
our algebra (a function applied to an argument) fairly readable.

\begin{figure}[h]
\ExecuteMetaData[generic-syntax.agda/Generic/Examples/NbyE.tex]{nbepatterns}
\caption{Pattern synonyms for UTLC-specific \AD{Dm}~values}
\end{figure}

We finally define the algebra by case analysis: if the node at hand is an
application and its first component evaluates to a lambda, we can apply
the function to its argument using \AF{\_\$\$\_}. Otherwise we have either a
stuck application or a lambda, in other words we already have a value and can
simply return it using \AIC{C}.

\begin{figure}[h]
  \ExecuteMetaData[generic-syntax.agda/Generic/Examples/NbyE.tex]{nbelc}
\caption{Normalisation by Evaluation for the Untyped λ-Calculus}
\end{figure}

We have not used the \AIC{⊥} constructor so \emph{if} the evaluation terminates
(by disabling totality checking we have lost all guarantees of the sort) we know
we will get a term in normal form. See for instance in
Figure~\ref{fig:normuntyped} the evaluation of an untyped yet normalising
term: {\app{(\lam{x}{x})}{(\app{(\lam{x}{x})}{(\lam{x}{x})})}}
normalises to {(\lam{x}{x})}.

\begin{figure}[h]
\ExecuteMetaData[generic-syntax.agda/Generic/Examples/NbyE.tex]{example}
\caption{Example of a normalising untyped term}
\label{fig:normuntyped}
\end{figure}

\section{Printing with Names}\label{section:genericprinting}

Coming back to our work on (rudimentary) printing with names in
\cref{prettyprint}, we can now give a generic account of it. This is a
particularly interesting example because it demonstrates that we may
sometimes want to give \AD{Desc} a different semantics to accommodate
a specific use-case: we do not want our users to deal explicitly with
name generation, explicit variable binding, etc.

Unlike renaming or substitution, this generic program will require user
guidance: there is no way for us to guess how an encoded term should be
printed. We can however take care of the name generation, deal with variable
binding, and implement the traversal generically. We are going to reuse
some of the components defined in \cref{prettyprint}: we can rely on the
same state monad (\AF{Fresh}) for name generation, the same \AF{fresh}
function and the same notions of \AF{Name} and \AF{Printer} for the
semantics' values and computations.
%
We want our printer to have type:
\begin{agdasnippet}
\ExecuteMetaData[generic-syntax.agda/Generic/Semantics/Printing.tex]{printtype}
\end{agdasnippet}
%
where \AF{Display} explains how to print one `layer' of term provided that
we are handed the \AF{Pieces} corresponding to the printed subterm and
names for the bound variables.
%
\begin{agdasnippet}
  \ExecuteMetaData[generic-syntax.agda/Generic/Semantics/Printing.tex]{display}
\end{agdasnippet}
%
Reusing the notion of \AR{Name} introduced in Section~\ref{section:printing},
we can make \AF{Pieces} formal.
%
The structure of \AR{Semantics} would suggest giving our users an interface
where sub-structures are interpreted as \AF{Kripke} function spaces expecting
fresh names for the fresh variables and returning a printer i.e. a monadic
computation returning a \AD{String}. However we can do better: we can
preemptively generate a set of fresh names for the newly-bound variables and
hand them to the user together with the result of printing the body with
these names. As usual we have a special case for the substructures without
any newly-bound variable. Note that the specific target context of the
environment of \AF{Name}s is only picked for convenience as \AF{Name}
ignores the scope: ({\AB{Δ} \AF{++} \AB{Γ}}) is what \AF{freshˡ} gives us.
%
In other words: \AF{Pieces} states that a subterm has already been printed
if we have a string representation of it together with an environment of
\AR{Name}s we have attached to the newly-bound variables this structure
contains.
%
\begin{agdasnippet}
\ExecuteMetaData[generic-syntax.agda/Generic/Semantics/Printing.tex]{pieces}
\end{agdasnippet}
%
The key observation that will help us define a generic printer is that
\AF{Fresh} composed with \AR{Name} is \AR{VarLike}. Indeed, as the
composition of a functor and a trivially thinnable \AR{Wrap}per,
\AF{Fresh} is \AF{Thinnable}, and \AF{fresh} (defined in
Figure~\ref{fig:fresh}) is the proof that we can generate
placeholder values thanks to the name supply.

\begin{agdasnippet}
\ExecuteMetaData[generic-syntax.agda/Generic/Semantics/Printing.tex]{vlmname}
\end{agdasnippet}

This \AR{VarLike} instance empowers us to reify in an effectful manner
a \AF{Kripke} function space taking \AF{Name}s and returning a \AF{Printer}
to a set of \AF{Pieces}.

\begin{agdasnippet}
\ExecuteMetaData[generic-syntax.agda/Generic/Semantics/Printing.tex]{reifytype}
\end{agdasnippet}

In case there are no newly bound variables, the \AF{Kripke} function space
collapses to a mere \AR{Printer} which is precisely the wrapped version of
the type we expect.

\begin{agdasnippet}
\ExecuteMetaData[generic-syntax.agda/Generic/Semantics/Printing.tex]{reifybase}
\end{agdasnippet}

Otherwise we proceed in a manner reminiscent of the pure reification function
defined in Figure~\ref{fig:genericreify}. We start by generating an environment
of names for the newly-bound variables by using the fact that \AF{Fresh} composed
with \AF{Name} is \AR{VarLike} together with the fact that environments are
Traversable~\cite{mcbride_paterson_2008}, %%%
and thus admit the standard Haskell-like \AF{mapA} and \AF{sequenceA}
traversals. %%%
We then run the \AF{Kripke} function
on these names to obtain the string representation of the subterm. We finally
return the names we used together with this string.

\begin{agdasnippet}
\ExecuteMetaData[generic-syntax.agda/Generic/Semantics/Printing.tex]{reifypieces}
\end{agdasnippet}

We can put all of these pieces together to obtain the \AF{Printing} semantics
presented in Figure~\ref{fig:genericprinting}.
The first two constraints can be trivially discharged. When defining the
algebra we start by reifying the subterms, then use the fact that  one ``layer''
of term of our syntaxes with binding is always traversable to combine all of
these results into a value we can apply our display function to.

\begin{figure}[h]
\ExecuteMetaData[generic-syntax.agda/Generic/Semantics/Printing.tex]{printing}
\caption{Printing with \AF{Name}s as a \AR{Semantics}}\label{fig:genericprinting}
\end{figure}

This allows us to write a \AF{printer} for open terms as demonstrated in
Figure~\ref{fig:genericprint}.
We start by using \AF{base} (defined in Section~\ref{sec:varlike:base})
to generate an environment of \AR{Name}s for the free variables, then use
our semantics to get a \AF{printer} which we can run using a stream \AF{names} of distinct
strings as our name supply.

\begin{figure}[h]
\ExecuteMetaData[generic-syntax.agda/Generic/Semantics/Printing.tex]{print}
\caption{Generic Printer for Open Terms}\label{fig:genericprint}
\end{figure}


\paragraph{Untyped λ-calculus} Defining a printer for the untyped
λ-calculus is now very easy: we define a \AF{Display} by case analysis.
In the application case, we combine the string representation of the
function, wrap its argument's representation between parentheses and
concatenate the two together. In the lambda abstraction case, we are
handed the name the bound variable was assigned together with the body's
representation; it is once more a matter of putting the \AF{Pieces}
together.

\begin{agdasnippet}
\ExecuteMetaData[generic-syntax.agda/Generic/Examples/Printing.tex]{printUTLC}
\end{agdasnippet}

As always, these functions are readily executable and we can check
their behaviour by writing tests. First, we print the identity function
defined in Figure~\ref{fig:identity}
in an empty context and verify that we do obtain the string \AStr{"λa. a"}.
Next, we print an open term in a context of size two and can immediately
observe that names are generated for the free variables first, and then the
expression itself is printed.

\begin{minipage}[t]{0.45\textwidth}
  \begin{agdasnippet}
  \ExecuteMetaData[generic-syntax.agda/Generic/Examples/Printing.tex]{printid}
  \end{agdasnippet}
\end{minipage}
\begin{minipage}[t]{0.45\textwidth}
  \begin{agdasnippet}
  \ExecuteMetaData[generic-syntax.agda/Generic/Examples/Printing.tex]{printopen}
  \end{agdasnippet}
\end{minipage}

