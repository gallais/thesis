\chapter{A Primer on Universes of Data Types}\label{section:data}

Chapman, Dagand, McBride and Morris
(CDMM, \citeyear{Chapman:2010:GAL:1863543.1863547})
defined a universe of data types inspired by Dybjer and Setzer's
finite axiomatisation of inductive-recursive definitions (\citeyear{Dybjer1999}),
Altenkirch and McBride's observation that ``generic programming is just
programming'' within dependent types
(\citeyear{DBLP:conf/ifip2-1/AltenkirchM02})
Benke, Dybjer and Jansson's universes for generic programs and proofs
(\citeyear{benke-ugpp}).
%
This explicit definition of \emph{codes} for data types empowers the
user to write generic programs tackling \emph{all} of the data types
one can obtain this way. In this section we recall the main aspects
of this construction we are interested in to build up our generic
representation of syntaxes with binding.

\section{Datatypes as Fixpoints of Strictly Positive Functors}

The first component of CDMM's universe's definition is an inductive
type of \AD{Desc}riptions of strictly positive functors and their
fixpoints.

\subsection{Descriptions and Their Meaning as Functors}
\label{fig:datades}
\label{fig:datadescmeaning}

Our type of descriptions represents functors from $\Set{}^J$ to $\Set{}^I$.
These functors correspond to \AB{I}-indexed containers of \AB{J}-indexed
payloads. Keeping these index types distinct prevents mistaking one for the
other when constructing the interpretation of descriptions. Later of course
we can use these containers as the nodes of recursive datastructures by
interpreting some payloads sorts as requests for
subnodes~(\cite{DBLP:journals/jfp/AltenkirchGHMM15}).

We will interleave the definition of the type of description and the
recursive function \AF{⟦\_⟧} assigning a meaning to them provided it
is handed the meaning to attach to the substructures. Interpretation
of descriptions gives rise to right-nested tuples terminated by equality
constraints.

\noindent\begin{minipage}{0.5\textwidth}
  \ExecuteMetaData[generic-syntax.agda/Generic/Data.tex]{desc:type}
\end{minipage}\begin{minipage}{0.5\textwidth}
  \ExecuteMetaData[generic-syntax.agda/Generic/Data.tex]{interp:type}
\end{minipage}

First, \AIC{`σ} stores data. This can be used to either attach a payload
to a node in which case the rest of the description will be constant in
the value stored, or offer a choice of possible descriptions computed
from the stored value.

\noindent\begin{minipage}[t]{0.5\textwidth}
  \ExecuteMetaData[generic-syntax.agda/Generic/Data.tex]{desc:sigma}
\end{minipage}\begin{minipage}[t]{0.5\textwidth}
  \ExecuteMetaData[generic-syntax.agda/Generic/Data.tex]{interp:sigma}
\end{minipage}

Then \AIC{`X} attaches a recursive substructure indexed by $J$.

\noindent\begin{minipage}{0.5\textwidth}
  \ExecuteMetaData[generic-syntax.agda/Generic/Data.tex]{desc:rec}
\end{minipage}\begin{minipage}{0.5\textwidth}
  \ExecuteMetaData[generic-syntax.agda/Generic/Data.tex]{interp:rec}
\end{minipage}

Finally, \AIC{`$\blacksquare$} to stop with a particular index value.
Its interpretation makes sure that the index we demanded corresponds
to the one we were handed.

\noindent\begin{minipage}{0.5\textwidth}
  \ExecuteMetaData[generic-syntax.agda/Generic/Data.tex]{desc:stop}
\end{minipage}\begin{minipage}{0.5\textwidth}
  \ExecuteMetaData[generic-syntax.agda/Generic/Data.tex]{interp:stop}
\end{minipage}

Note that our universe definition precludes higher-order branching
\emph{by choice}: we are interested in terms and these are first-order
objects.

These constructors give the programmer the ability to build up the data
types they are used to. For instance, the functor corresponding
to lists of elements in $A$ stores a \AD{Bool}ean which stands for whether
the current node is the empty list or not. Depending on its value, the
rest of the description is either the ``stop'' token or a pair of an element
in $A$ and a recursive substructure i.e. the tail of the list.
The \AD{List} type is unindexed, we represent the lack of an index with
the unit type \AD{$\top$}.

\begin{figure}[h]
\ExecuteMetaData[generic-syntax.agda/Generic/Data.tex]{listD}
\caption{The Description of the base functor for \AD{List} \AB{A}}
\label{figure:listD}
\end{figure}

Indices can be used to enforce invariants. For example, the type
({\AD{Vec} \AB{A} \AB{n}}) of length-indexed lists. It has the
same structure as the definition of \AF{listD}.
We start with a \AF{Bool}ean distinguishing the two constructors: either
the empty list (in which case the branch's index is enforced to be $0$) or a
non-empty one in which case we store a natural number \AB{n}, the head of type
\AB{A} and a tail of size \AB{n} (and the branch's index is enforced to be
(\AIC{suc} \AB{n}).

\begin{figure}[h]
\ExecuteMetaData[generic-syntax.agda/Generic/Data.tex]{vecD}
\caption{The Description of the base functor for \AD{Vec} \AB{A} \AB{n}}\label{figure:vecD}
\end{figure}

\subsection{Datatypes as Least Fixpoints}

All the functors obtained as meanings of \AD{Desc}riptions are strictly
positive. So we can build the least fixpoint of the ones that are
endofunctors i.e. the ones for which \AB{I} equals \AB{J}. This fixpoint
is called \AD{μ} and we provide its definition in \cref{figure:datamu}.

\begin{figure}[h]
\ExecuteMetaData[generic-syntax.agda/Generic/Data.tex]{mu}
\caption{Least Fixpoint of an Endofunctor}\label{figure:datamu}
\end{figure}

Equipped with this fixpoint operator, we can go back to our examples
redefining lists and vectors as instances of this framework. We can
see in Figure~\ref{fig:listpat} that we can recover the types we are
used to thanks to this least fixpoint (we only show the list example).
%
Pattern synonyms let us hide away the encoding: users can use them
to pattern-match on lists and Agda conveniently resugars them when
displaying a goal.

\begin{figure}[h]
\begin{minipage}{0.5\textwidth}
  \ExecuteMetaData[generic-syntax.agda/Generic/Data.tex]{listcons}
\end{minipage}\begin{minipage}{0.5\textwidth}
  \ExecuteMetaData[generic-syntax.agda/Generic/Data.tex]{list}
\end{minipage}

\begin{minipage}{0.5\textwidth}
  \ExecuteMetaData[generic-syntax.agda/Generic/Data.tex]{listnil}
\end{minipage}\begin{minipage}{0.5\textwidth}
  \ExecuteMetaData[generic-syntax.agda/Generic/Data.tex]{examplelist}
\end{minipage}
\caption{List Type, Patterns and Example}\label{figure:listpat}
\end{figure}

Now that we have the ability to represent the data we are interested it,
we are only missing the ability to write programs manipulating it.

\section{Generic Programming over Datatypes}

The payoff for encoding our datatypes as descriptions is that we can
define generic programs for whole classes of data types. The decoding
function \AF{⟦\_⟧} acted on the objects of $\Set{}^J$, and we will now
define the function \AF{fmap} by recursion over a code \AB{d}.
It describes the action of the functor corresponding to \AB{d} over
morphisms in $\Set{}^J$. This is the first example of generic
programming over all the functors one can obtain as the meaning
of a description.

\begin{figure}[h]
\ExecuteMetaData[generic-syntax.agda/Generic/Data.tex]{fmap}
\caption{Action on Morphisms of the Functor corresponding to a \AF{Desc}ription\label{fig:fmapD}}
\end{figure}

Once we take the fixpoint of such functors, we expect to be able to define
an iterator. It is given by the definition of \AF{fold} \AB{d}, our second
generic program over datatypes. In \cref{figure:datamu}, we skipped over
the \AD{Size} (\cite{DBLP:journals/corr/abs-1012-4896}) index added to the
inductive definition of \AD{μ}. It plays a crucial role in getting the
termination checker to see that \AF{fold} is a total function, just like
sizes played a crucial role in proving that \AF{map\textasciicircum{}Rose}
was total in~\cref{sec:sizetermination}.

\begin{figure}[h]
\ExecuteMetaData[generic-syntax.agda/Generic/Data.tex]{fold}
\caption{Eliminator for the Least Fixpoint}\label{figure:datafold}
\end{figure}

Finally, \cref{fig:listfold} demonstrates that we can get our hands
on the types' eliminators we are used to by instantiating the generic
\AF{fold} we have just defined.

\begin{figure}
  \ExecuteMetaData[generic-syntax.agda/Generic/Data.tex]{foldr}
  \caption{The Eliminator for \AD{List}\label{fig:listfold}}
\end{figure}

We can readily use this eliminator to define e.g. append, and then
flatten for lists and check that it behaves as we expect by running
it on the example introduced in \cref{figure:listpat}.

\begin{figure}[h]
\begin{minipage}{0.5\textwidth}
    \ExecuteMetaData[generic-syntax.agda/Generic/Data.tex]{append}
  \end{minipage}\begin{minipage}{0.5\textwidth}
    \ExecuteMetaData[generic-syntax.agda/Generic/Data.tex]{flatten}
  \end{minipage}
  \ExecuteMetaData[generic-syntax.agda/Generic/Data.tex]{test}
\caption{Applications\label{fig:appendflatten}}
\end{figure}

The CDMM approach therefore allows us to generically define iteration
principles for all data types that can be described. These are exactly
the features we desire for a universe of syntaxes with binding, so in
the next section we are going to see whether we could use CDMM's approach
to represent abstract syntax trees equipped with a notion of binding.

\section{A Free Monad Construction}

The natural candidate to represent trees with binding and thus variables
is to take not the least fixpoint of the strictly positive functor but
rather the corresponding free monad. It can be easily defined by reusing
the interpretation function, adding a special case corresponding to the
notion of variable we may want to use.

\begin{figure}[h]
  \ExecuteMetaData[generic-syntax.agda/Generic/Data.tex]{free}
  \caption{Free Indexed Monad of a Description\label{fig:datafreemonad}}
\end{figure}

The \AIC{pure} constructor already gives us a unit for the free monad,
all we have left to do is to define the Kleisli extension of a morphism.
The definition goes by analysis over the term in the free monad. The
\AIC{pure} case is trivial and, once again, the \AIC{node} case
essentially amounts to using \AF{fmap} to recursively call \AF{kleisli}
on all of the subterms of a \AIC{node}.

\begin{figure}[h]
  \ExecuteMetaData[generic-syntax.agda/Generic/Data.tex]{kleisli}
  \caption{Free Indexed Monad of a Description\label{fig:datafreemonad}}
\end{figure}

This gives us a notion of trees with variables in them and a substitution
operation replacing these variables with entire subtrees.
%
The functor underlying any well scoped and sorted syntax can be coded as
some {\AD{Desc} (I \AR{$\times$} \AD{List} I) (I \AR{$\times$} \AD{List} I)},
with the free monad construction from CDMM uniformly adding the variable
case. Whilst a good start, \AD{Desc} treats its index types as unstructured,
so this construction is blind to what makes the {\AD{List} I} index a
\emph{scope}. The resulting `bind' operator demands a function which maps
variables in \emph{any} sort and scope to terms in the \emph{same} sort
and scope. However, the behaviour we need is to preserve sort while mapping
between specific source and target scopes which may differ. We need to
account for the fact that scopes change only by extension, and hence that our
specifically scoped operations can be pushed under binders by weakening.

We will see in the next chapter how to modify CDMM's universe construction
to account for variable binding and finally obtain a scope-aware universe.
