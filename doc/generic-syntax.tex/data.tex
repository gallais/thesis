\chapter{A Primer on the Universe of Data Types}\label{section:data}

Chapman, Dagand, McBride and Morris (CDMM, \citeyear{Chapman:2010:GAL:1863543.1863547})
defined a universe of data types inspired by Dybjer and Setzer's
finite axiomatisation of inductive-recursive definitions (\citeyear{Dybjer1999})
and Benke, Dybjer and Jansson's universes for generic programs and proofs
(\citeyear{benke-ugpp}).
This explicit definition of \emph{codes} for data types empowers the
user to write generic programs tackling \emph{all} of the data types
one can obtain this way. In this section we recall the main aspects
of this construction we are interested in to build up our generic
representation of syntaxes with binding.

\section{Descriptions and Their Meaning as Functors}

The first component of CDMM's universe's definition is an inductive type of
\AD{Desc}riptions of strictly positive functors from $\Set{}^J$ to $\Set{}^I$.
%
These functors correspond to \AB{I}-indexed containers of \AB{J}-indexed
payloads. Keeping these index types distinct prevents mistaking one for the
other when constructing the interpretation of descriptions. Later of course
we can use these containers as the nodes of recursive datastructures by
interpreting some payloads sorts as requests for
subnodes~(\cite{DBLP:journals/jfp/AltenkirchGHMM15}).

The inductive type of descriptions has three constructors:
\AIC{`σ} to store data (the rest of the description can depend upon
this stored value),
%
\AIC{`X} to attach a recursive substructure indexed by $J$
%
and \AIC{`$\blacksquare$} to stop with a particular index value.

\begin{figure}[h]
\ExecuteMetaData[generic-syntax.agda/Generic/Data.tex]{desc}
\caption{Datatype Descriptions\label{fig:datades}}
\end{figure}

The recursive function \AF{⟦\_⟧} makes the interpretation of the descriptions
formal. Interpretation of descriptions give rise to right-nested tuples
terminated by equality constraints.

\begin{figure}[h]
\ExecuteMetaData[generic-syntax.agda/Generic/Data.tex]{interp}
\caption{Descriptions' meanings as Functors\label{fig:datadescmeaning}}
\end{figure}

These constructors give the programmer the ability to build up the data
types they are used to. For instance, the functor corresponding
to lists of elements in $A$ stores a \AD{Bool}ean which stands for whether
the current node is the empty list or not. Depending on its value, the
rest of the description is either the ``stop'' token or a pair of an element
in $A$ and a recursive substructure i.e. the tail of the list. The \AD{List} type
is unindexed, we represent the lack of an index with the unit type \AD{$\top$}.

\begin{figure}[h]
\ExecuteMetaData[generic-syntax.agda/Generic/Data.tex]{listD}
\caption{The Description of the base functor for \AD{List} \AB{A}}\label{figure:listD}
\end{figure}

Indices can be used to enforce invariants. For example, the type ({\AD{Vec} \AB{A} \AB{n}})
of length-indexed lists. It has the same structure as the definition of \AF{listD}.
We start with a \AF{Bool}ean distinguishing the two constructors: either
the empty list (in which case the branch's index is enforced to be $0$) or a
non-empty one in which case we store a natural number \AB{n}, the head of type
\AB{A} and a tail of size \AB{n} (and the branch's index is enforced to be
(\AIC{suc} \AB{n}).

\begin{figure}[h]
\ExecuteMetaData[generic-syntax.agda/Generic/Data.tex]{vecD}
\caption{The Description of the base functor for \AD{Vec} \AB{A} \AB{n}}\label{figure:vecD}
\end{figure}

The payoff for encoding our datatypes as descriptions is that we can define
generic programs for whole classes of data types. The decoding function \AF{⟦\_⟧}
acted on the objects of $\Set{}^J$, and we will now define the function \AF{fmap} by
recursion over a code \AB{d}. It describes the action of the functor corresponding
to \AB{d} over morphisms in $\Set{}^J$. This is the first example of generic
programming over all the functors one can obtain as the meaning of a description.

\begin{figure}[h]
\ExecuteMetaData[generic-syntax.agda/Generic/Data.tex]{fmap}
\caption{Action on Morphisms of the Functor corresponding to a \AF{Desc}ription}
\end{figure}

\section{Datatypes as Least Fixpoints}

All the functors obtained as meanings of \AD{Desc}riptions are strictly positive.
So we can build the least fixpoint of the ones that are endofunctors i.e. the ones
for which \AB{I} equals \AB{J}. This fixpoint is called \AD{μ} and its iterator is
given by the definition of \AF{fold} \AB{d}. In \cref{figure:datamu} the
\AD{Size} (\cite{DBLP:journals/corr/abs-1012-4896}) index added to the inductive
definition of \AD{μ} plays a crucial role in getting the termination checker to
see that \AF{fold} is a total function, just like sizes played a crucial role in
proving that \AF{map\textasciicircum{}Rose} was total in~\cref{sec:sizetermination}.

\begin{figure}[h]
\ExecuteMetaData[generic-syntax.agda/Generic/Data.tex]{mu}
\ExecuteMetaData[generic-syntax.agda/Generic/Data.tex]{fold}
\caption{Least Fixpoint of an Endofunctor and Corresponding Generic Fold}\label{figure:datamu}
\end{figure}

We can see in Figure~\ref{fig:listpat} that we can recover the types we are
used to thanks to this least fixpoint. Pattern synonyms let us hide away
the encoding: users can use them to pattern-match on lists and Agda
conveniently resugars them when displaying a goal.

\begin{figure}[h]
\begin{center}
  \ExecuteMetaData[generic-syntax.agda/Generic/Data.tex]{list}
\end{center}

\begin{minipage}[t]{0.45\textwidth}
  \ExecuteMetaData[generic-syntax.agda/Generic/Data.tex]{nil}
\end{minipage}
\begin{minipage}[t]{0.45\textwidth}
\ExecuteMetaData[generic-syntax.agda/Generic/Data.tex]{cons}
\end{minipage}
\caption{The list type and its usual constructors\label{fig:listpat}}
\end{figure}

Finally, Figure~\ref{fig:listfold} demonstrates that we can get our
hands on the types' eliminators by instantiating the generic \AF{fold}.

\begin{figure}
  \ExecuteMetaData[generic-syntax.agda/Generic/Data.tex]{foldr}
\caption{The eliminator for \AD{List}\label{fig:listfold}}
\end{figure}

The CDMM approach therefore allows us to generically define iteration principles
for all data types that can be described. These are exactly the features we desire
for a universe of syntaxes with binding, so in the next section we will see how
to extend CDMM's approach to include binding.

The functor underlying any well scoped and sorted syntax can be coded as some
{\AD{Desc} (I \AR{$\times$} \AD{List} I) (I \AR{$\times$} \AD{List} I)}, with the
free monad construction from CDMM uniformly adding the variable case. Whilst a
good start, \AD{Desc} treats its index types as unstructured, so this construction
is blind to what makes the {\AD{List} I} index a \emph{scope}. The resulting
`bind' operator demands a function which maps variables in \emph{any} sort and
scope to terms in the \emph{same} sort and scope. However, the behaviour we need
is to preserve sort while mapping between specific source and target scopes which
may differ. We need to account for the fact that scopes change only by extension,
and hence that our specifically scoped operations can be pushed under binders by
weakening.
