\section{(Unsafe) Normalisation by Evaluation}\label{section:unsafenbyeval}

A key type of traversal we have not studied yet is a language's
evaluator. Our universe of syntaxes with binding does not impose
any typing discipline on the user-defined languages and as such
cannot guarantee their totality. This is embodied by one of our running
examples: the untyped λ-calculus. As a consequence there
is no hope for a safe generic framework to define normalisation
functions.

The clear connection between the \AF{Kripke} functional space
characteristic of our semantics and the one that shows up in
normalisation by evaluation suggests we ought to manage to
give an unsafe generic framework for normalisation by evaluation.
By temporarily \textbf{disabling Agda's positivity checker},
we can define a generic reflexive domain \AD{Dm}
(cf. \cref{fig:reflexivedomain}) in which to
interpret our syntaxes. It has three constructors corresponding
respectively to a free variable, a constructor's counterpart where
scopes have become \AF{Kripke} functional spaces on \AD{Dm} and
an error token because the evaluation of untyped programs may go wrong
(a user may for instance try to add a function and a number).

\begin{figure}[h]
\ExecuteMetaData[generic-syntax.agda/Generic/Semantics/NbyE.tex]{domain}
\caption{Generic Reflexive Domain}\label{fig:reflexivedomain}
\end{figure}

This datatype definition is utterly unsafe. The more conservative
user will happily restrict themselves to particular syntaxes where
the typed settings allows for domain to be defined as a logical
predicate or opt instead for a step-indexed approach. We did develop
a step-indexed model construction but it was unusable: we could not
get Agda to normalise even the simplest of terms.

But this domain does make it possible to define a generic \AF{nbe}
semantics which, given a term, produces a value in the reflexive
domain. Thanks to the fact we have picked a universe of finitary syntaxes, we
can \emph{traverse}~\cite{mcbride_paterson_2008,DBLP:journals/jfp/GibbonsO09}
the functor to define
a (potentially failing) reification function turning elements of the
reflexive domain into terms. By composing them, we obtain the
normalisation function which gives its name to normalisation by
evaluation.

The user still has to explicitly pass an interpretation of
the various constructors because there is no way for us to
know what the binders are supposed to represent: they may
stand for λ-abstractions, $\Sigma$-types, fixpoints, or
anything else.

\begin{figure}[h]
\ExecuteMetaData[generic-syntax.agda/Generic/Semantics/NbyE.tex]{nbe-setup}
\caption{Generic Normalisation by Evaluation Framework\label{defn:NbE}}
\end{figure}

\subsection{Example: Evaluator for the Untyped Lambda-Calculus}

Using this setup, we can write a normaliser for the untyped λ-calculus
by providing an algebra. The key observation that allows us to implement
this algebra is that we can turn a Kripke function, \AB{f}, mapping values
of type \AB{σ} to computations of type \AB{τ} into an Agda function from
values of type \AB{σ} to computations of type \AB{τ}. This is witnessed
by the application function (\AF{\_\$\$\_}) defined in Figure~\ref{fig:kripkeapp}:
we first use \AF{extract} (defined in Figure~\ref{fig:Thinnable}) to obtain
a function taking environments of values to computations. We then use the
combinators defined in Figure~\ref{fig:baseenv} to manufacture the singleton
environment {(\AF{ε} \AB{∙} \AB{t})} containing the value \AB{t} of type
\AB{σ}.

\begin{figure}[h]
  \ExecuteMetaData[generic-syntax.agda/Generic/Examples/NbyE.tex]{app}
\caption{Applying a Kripke Function to an argument}\label{fig:kripkeapp}
\end{figure}

We now define two patterns for semantical values: one for application and
the other for lambda abstraction. This should make the case of interest of
our algebra (a function applied to an argument) fairly readable.

\begin{figure}[h]
\ExecuteMetaData[generic-syntax.agda/Generic/Examples/NbyE.tex]{nbepatterns}
\caption{Pattern synonyms for UTLC-specific \AD{Dm} values}
\end{figure}

We finally define the algebra by case analysis: if the node at hand is an
application and its first component evaluates to a lambda, we can apply
the function to its argument using \AF{\_\$\$\_}. Otherwise we have either a
stuck application or a lambda, in other words we already have a value and can
simply return it using \AIC{C}.

\begin{figure}[h]
  \ExecuteMetaData[generic-syntax.agda/Generic/Examples/NbyE.tex]{nbelc}
\caption{Normalisation by Evaluation for the Untyped λ-Calculus}
\end{figure}

We have not used the \AIC{⊥} constructor so \emph{if} the evaluation terminates
(by disabling totality checking we have lost all guarantees of the sort) we know
we will get a term in normal form. See for instance in
Figure~\ref{fig:normuntyped} the evaluation of an untyped yet normalising
term: {(λx. x) ((λx. x) (λx. x))} normalises to {(λx. x)}.

\begin{figure}[h]
\ExecuteMetaData[generic-syntax.agda/Generic/Examples/NbyE.tex]{example}
\caption{Example of a normalising untyped term}
\label{fig:normuntyped}
\end{figure}
