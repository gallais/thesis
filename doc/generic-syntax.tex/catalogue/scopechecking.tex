\section{Writing a Generic Scope Checker}\label{section:genericscoping}

Converting terms in the internal syntax to strings which can in turn be
displayed in a terminal or an editor window is only part of a compiler's
interaction loop. The other direction takes strings as inputs and attempts to
produce terms in the internal syntax. The first step is to parse the input
strings into structured data, the second is to perform scope checking,
and the third step consists of type checking.

Parsing is currently out of scope for our library; users can write safe
ad-hoc parsers for their object language by either using a library of total
parser combinators~(\cite{DBLP:conf/icfp/Danielsson10,allais2018agdarsec})
or invoking a parser generator oracle whose target is a total
language~(\cite{Stump:2016:VFP:2841316}). As we will see shortly, we can
write a generic scope checker transforming terms in a raw syntax where
variables are represented as strings into a well scoped syntax. We will
come back to typechecking with a concrete example in section~\ref{section:typechecking}
and then discuss related future work in the conclusion.

Our scope checker will be a function taking two explicit arguments: a name for
each variable in scope \AB{Γ} and a raw term for a syntax description \AB{d}.
It will either fail (the Monad \AF{Fail} granting us the ability to fail is made
explicit in Figure~\ref{fig:scopemonad}) or return a well scoped and sorted
term for that description.

\begin{agdasnippet}
\ExecuteMetaData[generic-syntax.agda/Generic/Scopecheck.tex]{totmtype}
\end{agdasnippet}

\paragraph{Scope} We can obtain \AF{Names}, the datastructure associating to
each variable in scope its raw name as a string by reusing the standard library's
\AD{All}. The inductive family \AD{All} is a predicate transformer making sure a
predicate holds of all the element of a list. It is defined in a style common in
Agda: because \AD{All}'s constructors are in one to one correspondence with that
of its index type (\AD{List} \AB{A}), the same name are reused: \AIC{[]} is the
name of the proof that \AB{P} trivially holds of all the elements in the empty
list \AIC{[]}; similarly \AIC{\_∷\_} is the proof that provided that \AB{P} holds
of the element \AB{a} on the one hand and of the elements of the list \AB{as}
on the other then it holds of all the elements of the list (\AB{a} \AIC{∷} \AB{as}).

\begin{figure}[h]
\begin{minipage}[t]{0.6\textwidth}
\ExecuteMetaData[shared/Stdlib.tex]{all}
\end{minipage}
\begin{minipage}[t]{0.3\textwidth}
  \ExecuteMetaData[generic-syntax.agda/Generic/Scopecheck.tex]{names}
\end{minipage}
\caption{Associating a raw string to each variable in scope}
\end{figure}

\paragraph{Raw terms}
The definition of \AF{WithNames} is analogous to \AF{Pieces} in the
previous section: we expect \AF{Names} for the newly bound
variables. Terms in the raw syntax then leverage these
definitions. They are either a variables or another ``layer'' of raw
terms. Variables \AIC{'var} carry a \AD{String} and potentially some
extra information \AB{E} (typically a position in a file). The other
constructor \AIC{'con} carries a layer of raw terms where subterms are
raw terms equipped with names for any newly bound variables.

\begin{figure}[h]
\ExecuteMetaData[generic-syntax.agda/Generic/Scopecheck.tex]{withnames}
\ExecuteMetaData[generic-syntax.agda/Generic/Scopecheck.tex]{raw}
\caption{Names and Raw Terms}
\end{figure}

\paragraph{Error Handling} Various things can go wrong during scope checking:
evidently a name can be out of scope but it is also possible that it may be
associated to a variable of the wrong sort. We define an enumerating type
covering these two cases. The scope checker will return a computation in the
Monad \AF{Fail} thus allowing us to fail and return an error, the string that
caused the failure and the extra data of type \AB{E} that accompanied it.

\begin{figure}[h]
\begin{minipage}[t]{0.5\textwidth}
  \ExecuteMetaData[generic-syntax.agda/Generic/Scopecheck.tex]{error}
\end{minipage}
\begin{minipage}[t]{0.4\textwidth}
  \ExecuteMetaData[generic-syntax.agda/Generic/Scopecheck.tex]{monad}
  \ExecuteMetaData[generic-syntax.agda/Generic/Scopecheck.tex]{fail}
\end{minipage}
\caption{Error Type and Scope Checking Monad}\label{fig:scopemonad}
\end{figure}

Equipped with these notions, we can write down the type of \AF{toVar} which
tackles the core of the problem: variable resolution. The function takes a
string and a sort as well the names and sorts of the variables in the ambient
scope. Provided that we have a function \AB{\_≟I\_} to decide equality on sorts,
we can check whether the string corresponds to an existing variable and whether
that binding is of the right sort. Thus we either fail or return a well scoped
and well sorted \AD{Var}.

If the ambient scope is empty then we can only fail with an \AIC{OutOfScope} error.
Alternatively, if the variable's name corresponds to that of the first one
in scope we check that the sorts match up and either return \AIC{z} or fail
with a \AIC{WrongSort} error. Otherwise we look for the variable further
down the scope and use \AIC{s} to lift the result to the full scope.

\begin{figure}[h]
\ExecuteMetaData[generic-syntax.agda/Generic/Scopecheck.tex]{toVar}
\caption{Variable Resolution}
\end{figure}

Scope checking an entire term then amounts to lifting this action on
variables to an action on terms. The error Monad \AF{Fail} is by
definition an Applicative and by design our terms are
Traversable~(\cite{bird_paterson_1999,DBLP:journals/jfp/GibbonsO09}).
The action on term is defined mutually with the action on scopes.
As we can see in the second equation for \AF{toScope}, thanks to the
definition of \AF{WithNames}, concrete names arrive just in time to
check the subterm with newly bound variables.

\begin{figure}[h]
\ExecuteMetaData[generic-syntax.agda/Generic/Scopecheck.tex]{scopecheck}
\caption{Generic Scope Checking for Terms and Scopes}
\end{figure}
