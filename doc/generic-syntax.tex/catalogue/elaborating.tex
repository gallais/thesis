\section{An Algebraic Approach to Elaboration}\label{section:elaboration}

Instead of generating a type or checking that a candidate will do, we
can use our language of \AD{Desc}riptions to define not only an
untyped source language but also an intrinsically typed internal
language. During typechecking we simultaneously generate an
expression's type and a well scoped and well typed term of that
type. We use \AF{STLC} (defined in Section~\ref{par:intrinsicSTLC}) as
our internal language.

Before we can jump right in, we need to set the stage: a \AR{Semantics} for a
\AF{Bidi} term will involve ({\AD{Mode} \AF{─Scoped}}) notions of values and
computations but an \AF{STLC} term is ({\AD{Type} \AF{─Scoped}}). We first
introduce a \AF{Typing} associating types to each of the modes in scope,
together with an erasure function \AF{⌞\_⌟} extracting the context of types
implicitly defined by such a \AF{Typing}.
%
We will systematically distinguish contexts of modes (typically named \AB{ms})
and their associated typings (typically named \AB{Γ}).

\begin{figure}[h]
\begin{minipage}[t]{0.4\textwidth}
  \ExecuteMetaData[generic-syntax.agda/Generic/Semantics/Elaboration/Typed.tex]{typing}
\end{minipage}
\begin{minipage}[t]{0.5\textwidth}
  \ExecuteMetaData[generic-syntax.agda/Generic/Semantics/Elaboration/Typed.tex]{fromtyping}
\end{minipage}
\caption{Typing: From Contexts of \AD{Mode}s to Contexts of \AD{Type}s\label{fig:typingmodes}}
\end{figure}

We can then explain what it means for an elaboration process of type \AB{σ}
in a context of modes \AB{ms} to produce a term of the
({\AD{Type} \AF{─Scoped}}) family \AB{T}: for any typing \AB{Γ} of this
context of modes, we should get a value of type
{(\AB{T} \AB{σ} \AF{⌞} \AB{Γ} \AF{⌟})}.

\begin{figure}[h]
\ExecuteMetaData[generic-syntax.agda/Generic/Semantics/Elaboration/Typed.tex]{elab}
\caption{Elaboration of a Scoped Family}
\end{figure}

Our first example of an elaboration process is our notion of environment values.
To each variable in scope of mode \AIC{Infer} we associate an elaboration function
targeting \AD{Var}. In other words: our values are all in scope i.e. provided any
typing of the scope of modes, we can assuredly return a type together with a
variable of that type.

\begin{figure}[h!]
\ExecuteMetaData[generic-syntax.agda/Generic/Semantics/Elaboration/Typed.tex]{varmode}
\caption{Values as Elaboration Functions for Variables\label{fig:elabvalues}}
\end{figure}

We can for instance prove that we have such an inference function for a newly bound
variable of mode \AIC{Infer}: given that the context has been extended with a variable
of mode \AIC{Infer}, the \AF{Typing} must also have been extended with a type \AB{σ}.
We can return that type paired with the variable \AIC{z}.

\begin{figure}[h]
\ExecuteMetaData[generic-syntax.agda/Generic/Semantics/Elaboration/Typed.tex]{var0}
\caption{Inference Function for the 0-th Variable\label{fig:elabvar0}}
\end{figure}

The computations are a bit more tricky. On the one hand, if we are in
checking mode then we expect that for any typing of the scope of modes
and any type candidate we can \AD{Maybe} return a term at that type
in the induced context. On the other hand, in the inference mode we
expect that given any typing of the scope, we can \AD{Maybe} return
a type together with a term at that type in the induced context.

\begin{figure}[h]
\ExecuteMetaData[generic-syntax.agda/Generic/Semantics/Elaboration/Typed.tex]{elabmode}
\caption{Computations as \AD{Mode}-indexed Elaboration Functions\label{fig:elabcomputations}}
\end{figure}

Because we are now writing a typechecker which returns evidence of its
claims, we need more informative variants of the equality and
\AF{isArrow} checks.
%
In the equality checking case we want to get a proof of propositional
equality but we only care about the successful path and will happily
return \AIC{nothing} when failing. Agda's support for (dependent!)
\AK{do}-notation makes writing the check really easy.

\begin{figure}[h]
  \ExecuteMetaData[generic-syntax.agda/Generic/Semantics/Elaboration/Typed.tex]{equal}
\caption{Informative Equality Check\label{fig:informativeeqcheck}}
\end{figure}

For the arrow type, we introduce a family \AD{Arrow} constraining the
shape of its index to be an arrow type and redefine \AF{isArrow} as
a \emph{view} targeting this inductive
family~\cite{DBLP:conf/popl/Wadler87,DBLP:journals/jfp/McBrideM04}.
%
We deliberately overload the constructor of the \AD{isArrow} family by calling
it \AIC{\_`→\_}. This means that the proof that a given type has the shape
{(\AB{σ} \AIC{`→} \AB{τ})} is literally written {(\AB{σ} \AIC{`→} \AB{τ})}.
This allows us to specify \emph{in the type} whether we want to work with the
full set of values in \AD{Type} or only the subset corresponding to function
types and to then proceed to write the same programs a Haskell programmers would,
with the added confidence that ours are guaranteed to be total.

\begin{figure}[h]
\begin{minipage}[t]{0.45\textwidth}
  \ExecuteMetaData[generic-syntax.agda/Generic/Semantics/Elaboration/Typed.tex]{arrowdata}
\end{minipage}\hfill\begin{minipage}[t]{0.45\textwidth}
  \ExecuteMetaData[generic-syntax.agda/Generic/Semantics/Elaboration/Typed.tex]{arrowview}
\end{minipage}
\caption{Arrow View\label{fig:informativecheck}}
\end{figure}

We now have all the basic pieces and can start writing elaboration code. We
will use lowercase letter for terms in \AF{Bidi} and uppercase ones for their
elaborated counterparts in \AF{STLC}. We once more start by dealing with each
constructor in isolation before putting everything together to get a
\AR{Semantics}. These steps are very similar to the ones in the previous
section.

\paragraph{Application} In the application case, we start by elaborating the
function and we get its type together with its internal representation. We then
check that the inferred type is indeed an \AD{Arrow} and elaborate the argument
using the corresponding domain. We conclude by returning the codomain together
with the internal function applied to the internal argument.
\begin{agdasnippet}
  \ExecuteMetaData[generic-syntax.agda/Generic/Semantics/Elaboration/Typed.tex]{app}
\end{agdasnippet}
\paragraph{λ-abstraction} For the λ-abstraction case, we start by
checking that the type candidate \AB{arr} is an \AD{Arrow}. We can
then elaborate the body \AB{b} of the lambda in a context of modes extended
with one \AIC{Infer} variable, and the corresponding \AF{Typing} extended
with the function's domain. From this we get
an internal term \AB{B} corresponding to the body of the λ-abstraction and
conclude by returning it wrapped in a \AIC{`lam} constructor.
\begin{agdasnippet}
  \ExecuteMetaData[generic-syntax.agda/Generic/Semantics/Elaboration/Typed.tex]{lam}
\end{agdasnippet}
\paragraph{Cut: A \AD{Check} in an \AD{Infer} position} For cut, we start by
elaborating the term with the type annotation provided and return them paired
together.
\begin{agdasnippet}
  \ExecuteMetaData[generic-syntax.agda/Generic/Semantics/Elaboration/Typed.tex]{cut}
\end{agdasnippet}
\paragraph{Embedding of \AD{Infer} into \AD{Check}} For the change of direction
\AIC{Emb} we not only want to check that the inferred type and the type candidate
are equal: we need to cast the internal term labelled with the inferred type to
match the type candidate. Luckily, Agda's dependent \AK{do}-notation make our
job easy once again: when we make the pattern \AIC{refl} explicit, the equality holds
in the rest of the block.
\begin{agdasnippet}
  \ExecuteMetaData[generic-syntax.agda/Generic/Semantics/Elaboration/Typed.tex]{emb}
\end{agdasnippet}

We have almost everything we need to define elaboration as a semantics. Discharging
the \ARF{th\textasciicircum{}𝓥} constraint is a bit laborious and the proof doesn't
yield any additional insight so we leave it out here. The semantical counterpart of
variables (\ARF{var}) is fairly straightforward: provided a \AF{Typing}, we run the
inference and touch it up to return a term rather than a mere variable. Finally we
define the algebra (\ARF{alg}) by pattern-matching on the constructor and using our
previous combinators.

\begin{figure}[h]
  \ExecuteMetaData[generic-syntax.agda/Generic/Semantics/Elaboration/Typed.tex]{elaborate}
  \caption{\AF{Elaborate}, the elaboration semantics\label{defn:Elaborate}}
\end{figure}

We can once more define a specialised version of the traversal induced by this
\AR{Semantics} for closed terms: not only can we give a (trivial) initial
environment (using the \AF{closed} corollary defined in Figure~\ref{fig:evalclosed})
but we can also give a (trivial) initial \AF{Typing}. This leads to the
definitions in Figure~\ref{fig:typedelaboration}.

\begin{figure}[h]
\begin{minipage}{0.45\textwidth}
  \ExecuteMetaData[generic-syntax.agda/Generic/Semantics/Elaboration/Typed.tex]{typemode}
\end{minipage}\hfill
\begin{minipage}{0.45\textwidth}
  \ExecuteMetaData[generic-syntax.agda/Generic/Semantics/Elaboration/Typed.tex]{type-}
\end{minipage}
\caption{Evidence-producing Type (Checking / Inference) Function}
\label{fig:typedelaboration}
\end{figure}

Revisiting the example introduced in Section~\ref{section:typechecking},
we can check that elaborating the expression {(λx. x : β → β) (λx. x)}
yields the type {β} together with the term {(λx. x) (λx. x)} in internal
syntax. Type annotations have disappeared in the internal syntax as all
the type invariants are enforced intrinsically.

\begin{figure}[h]
  \ExecuteMetaData[generic-syntax.agda/Generic/Semantics/Elaboration/Typed.tex]{example}
\end{figure}
