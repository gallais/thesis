\subsection{Reference Counting and Inlining as a Semantics}\label{section:inlining}

Although useful in its own right, desugaring all let bindings can lead
to an exponential blow-up in code size. Compiler passes typically try
to maintain sharing by only inlining let-bound expressions which appear
at most one time. Unused expressions are eliminated as dead code whilst
expressions used exactly one time can be inlined: this transformation is
size preserving and opens up opportunities for additional optimisations.

As we will see shortly, we can implement reference counting and size
respecting let-inlining as a generic transformation over all syntaxes
with binding equipped with let binders. This two-pass simple transformation
takes linear time which may seem surprising given the results due to Appel and
Jim~\citeyear{DBLP:journals/jfp/AppelJ97}. Our optimisation only inlines
let-bound variables whereas theirs also encompasses the reduction of static
β-redexes of (potentially) recursive function. While we can easily count how
often a variable is used in the body of a let binder, the interaction between
inlining and β-reduction in theirs creates cascading simplification opportunities
thus making the problem much harder.

But first, we need to look at an example demonstrating that this is a
slightly subtle matter. Assuming that \AB{expensive} takes a long time
to evaluate, inlining all of the lets in the first expression is a really
good idea whilst we only want to inline the one binding \AB{y} in the
second one to avoid duplicating work. That is to say that the contribution
of the expression bound to \AB{y} in the overall count depends directly
on whether \AB{y} itself appears free in the body of the let which binds it.

\begin{figure}[h]
\begin{minipage}{0.45\textwidth}
  \ExecuteMetaData[generic-syntax.agda/Generic/Syntax/LetCounter.tex]{cheap}
\end{minipage}
\begin{minipage}{0.45\textwidth}
  \ExecuteMetaData[generic-syntax.agda/Generic/Syntax/LetCounter.tex]{expensive}
\end{minipage}
\end{figure}

Our transformation will consist of two passes: the first one will annotate
the tree with accurate count information precisely recording whether
let-bound variables are used \AIC{zero}, \AIC{one}, or \AIC{many} times.
The second one will inline precisely the let-binders whose variable is
used at most once.

During the counting phase we need to be particularly careful not to overestimate
the contribution of a let-bound expression. If the let-bound variable is not used
then we can naturally safely ignore the associated count. But if it used \AIC{many}
times then we know we will not inline this let-binding and the count should
therefore only contribute once to the running total. We define the \AF{control}
combinator in Figure~\ref{fig:combinatorscount} precisely to explicitly handle this
subtle case.

The first step is to introduce the \AD{Counter} additive monoid
(cf. Figure~\ref{fig:countermonoid}). Addition will
allow us to combine counts coming from different subterms: if any of the two
counters is \AIC{zero} then we return the other, otherwise we know we have
\AIC{many} occurences.

\begin{figure}[h]
\begin{minipage}{0.45\textwidth}
  \ExecuteMetaData[generic-syntax.agda/Generic/Syntax/LetCounter.tex]{counter}
\end{minipage}
\begin{minipage}{0.45\textwidth}
  \ExecuteMetaData[generic-syntax.agda/Generic/Syntax/LetCounter.tex]{addition}
\end{minipage}
\caption{The (\AD{Counter}, \AIC{zero}, \AF{\_+\_}) additive monoid}
\label{fig:countermonoid}
\end{figure}

The syntax extension \AF{CLet} defined in Figure~\ref{fig:cletdef} is
a variation on the \AF{Let} syntax extension of Section~\ref{section:letbinding},
attaching a \AD{Counter} to each \AF{Let} node. The annotation process
can then be described as a function computing a
{(\AB{d} \AF{`+} \AF{CLet})} term from a {(\AB{d} \AF{`+} \AF{Let})} one.

\begin{figure}[h]
  \ExecuteMetaData[generic-syntax.agda/Generic/Syntax/LetCounter.tex]{clet}
  \caption{Counted Lets}
  \label{fig:cletdef}
\end{figure}

We keep a tally of the usage information for the variables in scope. This
allows us to know which \AD{Counter} to attach to each \AF{Let} node.
Following the same strategy as in Section~\ref{section:genericscoping},
we use the standard library's \AD{All} to represent this mapping. We say
that a scoped value has been \AF{Counted} if it is paired with a \AD{Count}.

\begin{figure}[h]
\begin{minipage}{0.45\textwidth}
  \ExecuteMetaData[Generic/Syntax/LetCounter.tex]{count}
\end{minipage}
\begin{minipage}{0.45\textwidth}
  \ExecuteMetaData[Generic/Semantics/Elaboration/LetCounter.tex]{counted}
\end{minipage}
\caption{Counting i.e. Associating a \AD{Counter} to each \AD{Var} in scope.}
\end{figure}

The two most basic counts are described in Figure~\ref{fig:basiccount}: the
empty one is \AIC{zero} everywhere and the one corresponding to a single use
of a single variable \AB{v} which is \AIC{zero} everywhere except for \AB{v}
where it's \AIC{one}.

\begin{figure}[H]
\begin{minipage}{0.45\textwidth}
  \ExecuteMetaData[generic-syntax.agda/Generic/Syntax/LetCounter.tex]{zeros}
\end{minipage}
\begin{minipage}{0.45\textwidth}
  \ExecuteMetaData[generic-syntax.agda/Generic/Syntax/LetCounter.tex]{fromVar}
\end{minipage}
\caption{Zero Count and Count of One for a Specific Variable}\label{fig:basiccount}
\end{figure}

When we collect usage information from different subterms, we need to put the
various counts together. The combinators in Figure~\ref{fig:combinatorscount}
allow us to easily do so: \AF{merge} adds up two counts in a pointwise manner
while \AF{control} uses one \AD{Counter} to decide whether to erase an existing
\AD{Count}. This is particularly convenient when computing the contribution of
a let-bound expression to the total tally: the contribution of the let-bound
expression will only matter if the corresponding variable is actually used.

\begin{figure}[H]
\begin{minipage}{0.5\textwidth}
  \ExecuteMetaData[generic-syntax.agda/Generic/Syntax/LetCounter.tex]{merge}
\end{minipage}
\begin{minipage}{0.4\textwidth}
  \ExecuteMetaData[generic-syntax.agda/Generic/Syntax/LetCounter.tex]{control}
\end{minipage}
\caption{Combinators to Compute \AD{Count}s}\label{fig:combinatorscount}
\end{figure}

We can now focus on the core of the annotation phase. We define a
\AR{Semantics} whose values are variables themselves and whose computations
are the pairing of a term in {(\AB{d} \AF{`+} \AF{CLet})} together with
a \AF{Count}. The variable case is trivial: provided a variable \AB{v},
we return {(\AIC{`var} \AB{v})} together with the count {(\AF{fromVar} \AB{v})}.

The non-let case is purely structural: we reify the \AF{Kripke} function
space and obtain a scope together with the corresponding \AF{Count}. We
unceremoniously \AF{drop} the \AD{Counter}s associated to the variables
bound in this subterm and return the scope together with the tally for
the ambient context.

\begin{figure}[h]
  \ExecuteMetaData[generic-syntax.agda/Generic/Semantics/Elaboration/LetCounter.tex]{reifycount}
  \caption{Purely Structural Case}
\end{figure}

The \AF{Let}-to-\AF{CLet} case in Figure~\ref{fig:lettoclet} is the most
interesting one. We start by reifying the \AB{body} of the let binder which
gives us a tally \AB{cx} for the bound variable and \AB{ct} for the body's
contribution to the ambient environment's \AD{Count}. We annotate the node
with \AB{cx} and use it as a \AF{control} to decide whether we are going to
merge any of the let-bound's expression contribution \AB{ce} to form the
overall tally.

\begin{figure}[h]
  \ExecuteMetaData[generic-syntax.agda/Generic/Semantics/Elaboration/LetCounter.tex]{letcount}
  \caption{Annotating Let Binders}\label{fig:lettoclet}
\end{figure}

Putting all of these things together we obtain the \AR{Semantics} \AF{Annotate}.
We promptly specialise it using an environment of placeholder values to obtain
the traversal \AF{annotate} elaborating raw let-binders into counted ones.

\begin{figure}[h]
  \ExecuteMetaData[generic-syntax.agda/Generic/Semantics/Elaboration/LetCounter.tex]{annotate}
\caption{Specialising \AR{semantics} to obtain an annotation function}
\end{figure}

Using techniques similar to the ones described in Section~\ref{section:letbinding},
we can write an \AF{Inline} semantics working on {(\AB{d} \AF{`+} \AF{CLet})} terms
and producing {(\AB{d} \AF{`+} \AF{Let})} ones. We make sure to preserve all the
let-binders annotated with \AIC{many} and to inline all the other ones. By composing
\AF{Annotate} with \AF{Inline} we obtain a size-preserving generic optimisation pass.
