\section{An Algebraic Approach to Typechecking}
\label{section:typechecking}

Following Atkey~(\citeyear{atkey2015algebraic}), we can consider type
checking and type synthesis as a possible semantics for a
bidirectional~(\cite{pierce2000local}) language.
%
We reuse the syntax introduced in Section~\ref{par:bidirectional}; it
gives us a simply typed bidirectional calculus as a two-moded language using
a notion of \AD{Mode} to distinguish between terms for which we will be able
to \AIC{Synth}esise the type and the ones for which we will have to
\AIC{Check} a type candidate. We can write \AF{Type-}, the type-level
function computing from a \AD{Mode} the associated type synthesis or
checking behaviour we expect.


\begin{figure}[h]
\ExecuteMetaData[generic-syntax.agda/Generic/Semantics/TypeChecking.tex]{typemode}
\caption{\AD{Type-} Synthesis / Checking Specification\label{fig:typecheckspec}}
\end{figure}

Our goal in this section is to construct of a \AR{Semantics} whose
associated evaluator will correspond to a function implementing this
specification. In other words, we want to define a semantics
\AF{Typecheck} such that we can obtain the following \AF{type-} function
as a corrolary.

\ExecuteMetaData[generic-syntax.agda/Generic/Semantics/TypeChecking.tex]{type-}

We will first have to decide what the appropriate notions of values and
computations for this semantics should be.

\subsection{Values and Computations for Type Checking}

The values stored in the environment of the typechecking function will
attach \AD{Type} information to bound variables whose \AD{Mode} is
\AIC{Synth}, guaranteeing no variable ever uses the \AIC{Check} mode.
Hence the definition of \AD{Var-} in \cref{fig:typecheckvar}.

\begin{figure}[h]
  \ExecuteMetaData[generic-syntax.agda/Generic/Semantics/TypeChecking.tex]{varmode}
\caption{\AD{Var-} Relation indexed by \AD{Mode}\label{fig:typecheckvar}}
\end{figure}

In contrast, the generated computations will, depending on the mode,
either take a type candidate and \AIC{Check} it is valid or \AIC{Synth}
a type for their argument. This is the logic represented by the \AF{Type-}
relation defined in \cref{fig:typecheckspec}.
%
These computations are always potentially failing as terms may not be
well typed. As a consequence we use the \AD{Maybe} monad. In an actual
compiler pipeline we would naturally use a different error monad and
generate helpful error messages pointing out where the type error occurred.
The interested reader can see a fine-grained analysis of type errors
in the extended example of a typechecker in
\citet{DBLP:journals/jfp/McBrideM04}.

\subsection{Handling Type Constraints}

A change of direction from synthesising to checking will require being
able to check that the type that was synthesised and the type that was
required agree. So we introduce the function \AF{\_=?\_} that checks
two types for equality.
%
Similarly we will sometimes be handed a type that we expect to be a
function type. We will have to check that it is and so we introduce
\AF{isArrow}, a function making sure that our candidate's head
constructor is indeed an arrow, and returning the domain and codomain.

\ExecuteMetaData[generic-syntax.agda/Generic/Semantics/TypeChecking.tex]{typeeq}

\ExecuteMetaData[generic-syntax.agda/Generic/Semantics/TypeChecking.tex]{isArrow}

\subsection{Typechecking as a \AR{Semantics}}

We can now define typechecking as a \semrec{}. We describe the
algorithm constructor by constructor; in the \AR{Semantics}
definition (omitted here) the algebra will simply perform a
dispatch and pick the relevant auxiliary lemma. Note that in the
following code, \AF{\_<\$\_} is, following classic Haskell notations,
the function which takes an \AB{A} and a {\AD{Maybe} \AB{B}} and
returns a {\AD{Maybe} \AB{A}}
which has the same structure as its second argument.

\paragraph{Application} When facing an application: synthesise the type of the
function, make sure it is an arrow type, check the argument at the domain's
type and return the codomain.
\begin{agdasnippet}
\ExecuteMetaData[generic-syntax.agda/Generic/Semantics/TypeChecking.tex]{app}
\end{agdasnippet}
%
\paragraph{λ-abstraction} For a λ-abstraction: check that the input
type \AB{arr} is an arrow type and check the body \AB{b} at the
codomain type in the extended environment (using \AF{bind}) where the
newly bound variable is of mode \AIC{Synth} and has the domain's type.
\begin{agdasnippet}
\ExecuteMetaData[generic-syntax.agda/Generic/Semantics/TypeChecking.tex]{lam}
\end{agdasnippet}
%
\paragraph{Embedding of \AD{Synth} into \AD{Check}} The change of
direction from \AIC{Synth}esisable to \AIC{Check}able is successful when the
synthesised type is equal to the expected one.
\begin{agdasnippet}
\ExecuteMetaData[generic-syntax.agda/Generic/Semantics/TypeChecking.tex]{emb}
\end{agdasnippet}
%
\paragraph{Cut: A \AD{Check} in an \AD{Synth} position}
So far, our bidirectional syntax only permits the construction
  of STLC terms in \emph{canonical
    form}~(\cite{Pfenning:04,Dunfield:2004:TT:964001.964025}).
  In order to construct
  non-normal (redex) terms, whose semantics is given logically by the
  `cut' rule, we need to reverse direction.
Our final semantic operation, \AF{cut},
always comes with a type candidate against which to check the term and
to be returned in case of success.
\begin{agdasnippet}
\ExecuteMetaData[generic-syntax.agda/Generic/Semantics/TypeChecking.tex]{cut}
\end{agdasnippet}
%
We have defined a bidirectional typechecker for this simple language by
leveraging the \semrec{} framework and can now obtain the \AF{type-}
function we motivated this work with.
Defining \AF{β} to be {(\AIC{α} \AIC{`→} \AIC{α})}, we can synthesise the
type of the expression {(λx. x : β → β) (λx. x)}.

\begin{figure}[h!]
  \ExecuteMetaData[generic-syntax.agda/Generic/Semantics/TypeChecking.tex]{example}
\caption{Example of Type- Synthesis\label{defn:BidiSemantics}}
\end{figure}

The output of this function is not very informative. As we will see shortly,
there is nothing stopping us from moving away from a simple computation
returning a {(\AD{Maybe} \AD{Type})} to an evidence-producing function
elaborating a term in \AF{Bidi} to a well scoped and typed term in
\AF{STLC}.
