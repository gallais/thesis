

\chapter{Building Generic Proofs about Generic Programs}

ACMM~\citeyear{allais2017type} have
already shown that, for the simply typed $\lambda$-calculus, introducing an abstract
notion of Semantics not only reveals the shared structure of common
traversals, it also allows them to give abstract proof frameworks for
simulation or fusion lemmas. Their idea naturally extends to our generic
presentation of semantics for all syntaxes.

The most important concept in this section is (\AF{Zip} \AB{d}), a relation
transformer which characterises structurally equal layers such that their
substructures are themselves related by the relation it is passed as an
argument. It inherits a lot of its relational arguments' properties: whenever
\AB{R} is reflexive (respectively symmetric or transitive) so is {\AF{Zip} \AB{d} \AB{R}}.\label{lem:zipstable}

It is defined by induction on the description and case analysis on the two
layers which are meant to be equal:
\begin{itemize}
  \item In the stop token case \AIC{`∎} \AB{i}, the two layers are considered to
    be trivially equal (i.e. the constraint generated is the unit type)
  \item When facing a recursive position {\AIC{`X} \AB{$\Delta$} \AB{j} \AB{d}}, we
    demand that the two substructures are related by {\AB{R} \AB{$\Delta$} \AB{j}}
    and that the rest of the layers are related by \AF{Zip} \AB{d} \AB{R}
  \item Two nodes of type {\AIC{`$\sigma$} \AB{A} \AB{d}} will
    be related if they both carry the same payload \AB{a} of type \AB{A} and if
    the rest of the layers are related by {\AF{Zip} (\AB{d} \AB{a}) \AB{R}}.
\end{itemize}

\begin{figure}[h]
\ExecuteMetaData[generic-syntax.agda/Generic/Zip.tex]{ziptype}
\caption{Zip: Characterising Structurally Equal Values with Related Substructures}
\end{figure}

If we were to take a fixpoint of \AF{Zip}, we could obtain a structural
notion of equality for terms which we could prove equivalent to propositional
equality. Although interesting in its own right, this section will focus
on more advanced use-cases.
