\chapter{Building Generic Proofs about Generic Programs}

We have already shown in \cref{sec:simulationrel,sec:fusionrel} that, for the
simply typed $\lambda$-calculus, introducing an abstract notion of Semantics
not only reveals the shared structure of common traversals, it also allows
us to give abstract proof frameworks for simulation or fusion lemmas. These
ideas naturally extend to our generic presentation of semantics for all syntaxes.

\section{Additional Relation Transformers}

During our exploration of generic proofs about the behaviour of \AR{Semantics}
for a concrete object language, we have introduced a notion \AR{Rel} of
relations as well as a relation transformer for environments
(cf. \cref{sec:relation-transformers}). Working on a universe of syntaxes,
we are going to need to define additional relators.

\paragraph{Kripke relator}
The Kripke relator is a generalisation of the ad-hoc definition introduced
in \cref{fig:relationalkripke}. We assume that we have two types of values
\AB{𝓥ᴬ} and \AB{𝓥ᴮ}
as well as a relation \AB{𝓥ᴿ} for pairs of such values, and two types
of computations \AB{𝓒ᴬ} and \AB{𝓒ᴮ} whose notion of relatedness is
given by \AB{𝓒ᴿ}. We can define
\AF{Kripkeᴿ} relating Kripke functions of type
{(\AF{Kripke} \AB{𝓥ᴬ} \AB{𝓒ᴬ})} and {(\AF{Kripke} \AB{𝓥ᴮ} \AB{𝓒ᴮ})}
respectively by stating that they send related inputs
to related outputs. We use the relation transformer \AF{All} defined
in the previous paragraph.

\begin{figure}[h]
 \ExecuteMetaData[shared/Data/Var/Varlike.tex]{kripkeR}
\caption{Relational Kripke Function Spaces: From Related Inputs to Related Outputs\label{fig:Kripke-rel}}
\end{figure}

\paragraph{Desc relator}
The relator (\AF{⟦} \AB{d} \AF{⟧ᴿ}) is a relation transformer which characterises
structurally equal layers such that their substructures are themselves related
by the relation it is passed as an argument. It inherits a lot of its relational
arguments' properties: whenever \AB{R} is reflexive (respectively symmetric or
transitive) so is {(\AF{⟦} \AB{d} \AF{⟧ᴿ} \AB{R})}.\label{lem:zipstable}

It is defined by induction on the description and case analysis on the two
layers which are meant to be equal:
\begin{itemize}
  \item In the stop token case \AIC{`∎} \AB{i}, the two layers are considered to
    be trivially equal (i.e. the constraint generated is the unit type)
  \item When facing a recursive position {\AIC{`X} \AB{$\Delta$} \AB{j} \AB{d}}, we
    demand that the two substructures are related by {\AB{R} \AB{$\Delta$} \AB{j}}
    and that the rest of the layers are related by (\AF{⟦} \AB{d} \AF{⟧ᴿ} \AB{R})
  \item Two nodes of type {\AIC{`$\sigma$} \AB{A} \AB{d}} will
    be related if they both carry the same payload \AB{a} of type \AB{A} and if
    the rest of the layers are related by (\AF{⟦} \AB{d} \AB{a} \AF{⟧ᴿ} \AB{R})
\end{itemize}

\begin{figure}[h]
 \ExecuteMetaData[generic-syntax.agda/Generic/Relator.tex]{ziptype}
\caption{Relator: Characterising Structurally Equal Values with Related Substructures\label{fig:zip-rel}}
\end{figure}

If we were to take a fixpoint of \AF{⟦\_⟧ᴿ}, we could obtain a structural
notion of equality for terms which we could prove equivalent to propositional
equality. Although interesting in its own right, this section will focus
on more advanced use-cases.

%%%%%%%%%%%%%%%%%%%%%%%%%%%%%%%%%%%%%%%%%%%%%%%%%%%%%%%%%%%%%%%%%%%%%%%
%% SIMULATION

\chapter{The Simulation Relation}
\label{sec:simulationrel}

Thanks to \AF{Semantics}, we have already saved work by not reiterating the
same traversals. Moreover, this disciplined approach to building models and
defining the associated evaluation functions can help us refactor the proofs
of some of these semantics' properties.
%
Instead of using proof scripts as Benton et al.~(\citeyear{benton2012strongly})
do, we describe abstractly the constraints the logical relations
(\cite{reynolds1983types})
defined on computations (and environment values) have to respect to ensure
that evaluating a term in related environments
produces related outputs. This gives us a generic framework to
state and prove, in one go, properties about all of these semantics.

Our first example of such a framework will stay simple on purpose.
However it is no mere bureaucracy: the
result proven here will actually be useful in the next section
when considering more complex properties.

This first example is describing the relational interpretation of the terms.
It should give the reader a good introduction to the setup before we take on
more complexity. The types involved might look a bit scarily abstract but the
idea is rather simple: we have a \AR{Simulation} between two \AR{Semantics}
when evaluating a term in related environments yields related values. The bulk
of the work is to make this intuition formal.


%%%%%%%%%%%%%%%%%%%%%%%%%%%%%%%%%%%%%%%%%%%%%%%%%%%%%%%%%%%%%%%%%%%%%%%
%% RELATORS

\section{Relations and Environment Relation Transformer}
\label{sec:relation-transformers}

In our exploration of generic proofs about the behaviour of
various \AR{Semantics}, we are going to need to manipulate
relations between distinct notions of values or computations.
In this section, we introduce the notion of relation we are going to
use as well as these two key relation transformers.

In Section~\ref{sec:genenvironment} we introduced a generic notion
of well typed and scoped environment as a function from variables
to values. Its formal definition is given in Figure~\ref{fig:environment}
as a record type. This record wrapper helps Agda's type inference
reconstruct the type family of values whenever it is passed an
environment.

For the same reason, we will use a record wrapper for the concrete
implementation of our notion of relation over (I \AF{─Scoped})
families. A \AR{Rel}ation between two such families \AB{T} and \AB{U}
is a function which to any \AB{σ} and \AB{Γ} associates a relation
between (\AB{T} \AB{σ} \AB{Γ}) and (\AB{U} \AB{σ} \AB{Γ}). Our first
example of such a relation is \AF{Eqᴿ} the equality relation between
an (\AB{I}\AF{─Scoped}) family \AB{T} and itself.

\begin{figure}[h]
\begin{minipage}[t]{0.65\textwidth}
  \ExecuteMetaData[shared/Data/Relation.tex]{rel}
\end{minipage}
\begin{minipage}[t]{0.25\textwidth}
  \ExecuteMetaData[shared/Data/Relation.tex]{eqR}
\end{minipage}
\caption{Relation Between \AB{I}\AF{─Scoped}~Families and Equality Example\label{ex:fam-eq}}
\label{fig:reldef}
\end{figure}

Once we know what relations are, we are going to have to lift relations on values
and computations to relations on environments, \AF{Kripke} function spaces or
on \AB{d}-shaped terms whose subterms have been evaluated already. These relation
transformers will be instances of what Kawahara calls `relators'.
This is what the rest of this section focuses on.


\paragraph{Environment relator}
Provided a relation \AB{𝓥ᴿ} for notions of values \AB{𝓥ᴬ} and \AB{𝓥ᴮ}, by
pointwise lifting we can define a relation {(\AR{All} \AB{𝓥ᴿ} \AB{Γ})} on
\AB{Γ}-environments of values \AB{𝓥ᴬ} and \AB{𝓥ᴮ} respectively. We once more
use a record wrapper simply to facilitate Agda's job when reconstructing
implicit arguments.

\begin{figure}[h]
  \ExecuteMetaData[shared/Data/Relation.tex]{all}
  \caption{Relating \AB{Γ}-Environments in a Pointwise Manner
    \label{defn:Env-rel}}
\end{figure}

The first example of two environment being related is \AF{reflᴿ} that, to any
environment \AB{ρ} associates a trivial proof of the statement
{(\AR{All} \AF{Eqᴿ} \AB{Γ} \AB{ρ} \AB{ρ})}.

\begin{figure}[h]
  \ExecuteMetaData[shared/Data/Relation.tex]{reflR}
\caption{Pointwise Lifting of \AIC{refl}\label{defn:reflR}}
\end{figure}

The combinators we introduced in Figure~\ref{fig:baseenv} to build environments
(\AF{ε}, \AF{\_∙\_}, etc.) have natural relational counterparts.
We reuse the same names for them, simply appending an \AF{ᴿ} suffix.


Once we have all of these definitions, we can spell out what it means to
simulate a semantics with another.

\section{Simulation Constraints}
\label{sec:simconstraints}

Given two Semantics \AB{𝓢ᴬ} and \AB{𝓢ᴮ} in simulation with respect to
relations \AB{𝓥ᴿ} for values and \AB{𝓒ᴿ} for computations, we have that
for any term \AB{t} and environments \AB{ρᴬ} and \AB{ρᴮ}, if the two
environments are \AB{𝓥ᴿ}-related in a pointwise manner then the semantics
associated to \AB{t} by \AB{𝓢ᴬ} using \AB{ρᴬ} is \AB{𝓒ᴿ}-related to the
one associated to \AB{t} by \AB{𝓢ᴮ} using \AB{ρᴮ}.

The evidence that we have a \AR{Simulation} between two \AR{Semantics}
(𝓢ᴬ and 𝓢ᴮ) is packaged in a record indexed by the semantics as well
as two relations. The first one (\AB{𝓥ᴿ}) relates values and the second
one (\AB{𝓒ᴿ}) describes simulation for computations. Our goal is to prove
that given that the simulation constraints are satisfied, evaluation in
semantics 𝓢ᴬ and evaluation in semantics 𝓢ᴮ yield related computations
provided related environments of values. We start by making formal this
idea of related evaluations by introducing \AF{𝓡}, a specialisation of
𝓒ᴿ:

\ExecuteMetaData[type-scope-semantics.agda/Properties/Simulation/Specification.tex]{crel}

We can now spell out what the simulation constraints are. Following the
same presentation we used for \AR{Semantics}
(cf. \cref{section:generic-semantics}), we will study the constraints field
by field and accompany them with comments as well as showing the use we make
of these constraints in the proof of the fundamental lemma of \AR{Simulation}.

We start by giving the two types at hand: the \AR{Simulation} record
parameterised by the two semantics and corresponding relations and
the statement of \AF{simulation}, the fundamental lemma of simulations
taking proofs that evaluation environments are related and returning
proofs that the respective evaluation functions yield related results.

\begin{AgdaSuppressSpace}
\ExecuteMetaData[type-scope-semantics.agda/Properties/Simulation/Specification.tex]{simulation}
\ExecuteMetaData[type-scope-semantics.agda/Properties/Simulation/Specification.tex]{fundamental:type}
\end{AgdaSuppressSpace}

The set of simulation constraints is in one-to-one correspondence with that of
semantical constructs. We start with the relational counterpart of \AIC{`var}
and \ARF{var}. Provided that \AB{ρᴬ} and \AB{ρᴮ} carry values related by
\AB{𝓥ᴿ}, the result of evaluating the variable \AB{v} in each respectively
should yield computations related by \AB{𝓒ᴿ}. That is to say that the
respective \ARF{var} turn related values into related computations.

\begin{AgdaSuppressSpace}
\ExecuteMetaData[type-scope-semantics.agda/Properties/Simulation/Specification.tex]{var}
\ExecuteMetaData[type-scope-semantics.agda/Properties/Simulation/Specification.tex]{fundamental:var}
\end{AgdaSuppressSpace}

Value constructors in the language follow a similar pattern: provided that
the evaluation environments are related, we expect the computations to be
related too.

\noindent\begin{minipage}{0.6\textwidth}
  \ExecuteMetaData[type-scope-semantics.agda/Properties/Simulation/Specification.tex]{base}
\end{minipage}\hfill\begin{minipage}{0.4\textwidth}
  \ExecuteMetaData[type-scope-semantics.agda/Properties/Simulation/Specification.tex]{fundamental:base}
\end{minipage}

Then come the structural cases: for language constructs like \AIC{`app}
and \AIC{`ifte} whose subterms live in the same context as the overall
term, the constraints are purely structural. Provided that the evaluation
environments are related, and that the evaluation of the subterms in
each environment respectively are related then the evaluations of the
overall terms should also yield related results.

\begin{AgdaSuppressSpace}
\ExecuteMetaData[type-scope-semantics.agda/Properties/Simulation/Specification.tex]{struct}
\ExecuteMetaData[type-scope-semantics.agda/Properties/Simulation/Specification.tex]{fundamental:struct}
\end{AgdaSuppressSpace}


Finally, we reach the most interesting case. The semantics attached to the
body of a \AIC{`lam} is expressed in terms of a Kripke function space. As
a consequence, the relational semantics will need a relational notion of
Kripke function space (\AF{Kripkeᴿ}) to spell out the appropriate simulation
constraint. This relational Kripke function space states that in any
thinning of the evaluation context and provided two related inputs, the
evaluation of the body in each thinned environment extended with the
appropriate value should yield related computations.

\begin{figure}[h]
\ExecuteMetaData[type-scope-semantics.agda/Properties/Simulation/Specification.tex]{rkripke}
\caption{Relational Kripke Function Spaces: From Related Inputs to Related Outputs\label{fig:relationalkripke}}
\end{figure}

This allows us to describe the constraint for \AIC{`lam}: provided related
environments of values, if we have a relational Kripke function space for
the body of the \AIC{`lam} then both evaluations should yield related
results. Together with a constraint guaranteeing that value-relatedness
is stable under thinnings, this is enough to discharge the λ case and thus
conclude the proof.

\begin{AgdaSuppressSpace}
\ExecuteMetaData[type-scope-semantics.agda/Properties/Simulation/Specification.tex]{lam}
\ExecuteMetaData[type-scope-semantics.agda/Properties/Simulation/Specification.tex]{thV}
\ExecuteMetaData[type-scope-semantics.agda/Properties/Simulation/Specification.tex]{fundamental:lam}
\end{AgdaSuppressSpace}

Now that we have our set of constraints together with a proof that they
entail the theorem we expected to hold, we can start looking for interesting
instances of a \AR{Simulation}.

\section{Syntactic Traversals are Extensional}

A first corollary of the fundamental lemma of simulations is the fact that semantics
arising from a \AR{Syntactic} (cf.~\cref{fig:syntactic}) definition are extensional. We
can demonstrate this by proving that every syntactic semantics is in simulation with
itself. That is to say that the evaluation function yields propositionally equal
values provided extensionally equal environments of values.

Under the assumption that \AB{Syn} is a \AR{Syntactic} instance, we can define the
corresponding \AR{Semantics} \AB{𝓢} by setting
\ExecuteMetaData[type-scope-semantics.agda/Properties/Simulation/Instances.tex]{synsem}.
Using \AF{Eqᴿ} the \AR{Rel} defined as the pointwise lifting of propositional equality,
we can make our earlier claim formal and prove it. All the constraints are discharged
either by reflexivity or by using congruence to combine various hypotheses.

\begin{figure}[h]
\ExecuteMetaData[type-scope-semantics.agda/Properties/Simulation/Instances.tex]{syn-ext}
\caption{\AR{Syntactic}~Traversals are in \AR{Simulation}~ with Themselves\label{fig:synselfsim}}
\end{figure}

Because the \AR{Simulation} statement is not necessarily extremely illuminating, we spell
out the type of the corollary to clarify what we just proved: whenever two environments
agree on each variable, evaluating a term with either of them produces equal results.

\begin{figure}[h]
\ExecuteMetaData[type-scope-semantics.agda/Properties/Simulation/Instances.tex]{synext}
\caption{\AR{Syntactic}~Traversals are Extensional\label{fig:synextensional}}
\end{figure}

This may look like a trivial result however we have found multiple use cases
for it during the development of our solution to the POPLMark Reloaded
challenge (\citeyear{poplmark2}): when proceeding by equational reasoning,
it is often the case that we can make progress on each side of the equation
only to meet in the middle with the same traversals using two environments
manufactured in slightly different ways but ultimately equal.
This lemma allows us to bridge that last gap.

More generally when working with higher-order functions in intensional
type theory, it is extremely useful to know that from \emph{extensionally}
equal inputs we can guarantee that we will obtain \emph{intensionally}
equal outputs.

\section{Renaming is a Substitution}

Similarly, it is sometimes the case that after a bit of rewriting we end
up with an equality between one renaming and one substitution. But it turns
out that as long as the substitution is only made up of variables, it is
indeed equal to the corresponding renaming. We can make this idea formal
by introducing the \AF{VarTermᴿ} relation stating that a variable and a
term are morally equal like so:

\begin{figure}[h]
\ExecuteMetaData[type-scope-semantics.agda/Properties/Simulation/Instances.tex]{varterm}
\caption{Characterising Equal Variables and Terms\label{fig:vartermR}}
\end{figure}

We can then state our result: we can prove a simulation lemma between \AF{Renaming}
and \AF{Substitution} where values (i.e. variables in the cases of renaming and terms
in terms of substitution) are related by \AF{VarTermᴿ} and computations (i.e. terms)
are related by \AF{Eqᴿ}. Once again we proceed by reflexivity and congruence.

\begin{figure}[h]
\ExecuteMetaData[type-scope-semantics.agda/Properties/Simulation/Instances.tex]{renissub}
\caption{\AF{Renaming}~is in \AR{Simulation}~with \AF{Substitution}\label{fig:renissub}}
\end{figure}

Rather than showing one more time the type of the corollary, we show a specialized
version where we pick the substitution to be precisely the thinning used on which we
have mapped the \AIC{`var} constructor.

\begin{figure}[h]
\ExecuteMetaData[type-scope-semantics.agda/Properties/Simulation/Instances.tex]{renassub}
\caption{Renaming as a Substitution\label{fig:renassub}}
\end{figure}

\section{The PER for βιξη-Values is Closed under Evaluation}

Now that we are a bit more used to the simulation framework and simulation
lemmas, we can look at a more complex example: the simulation lemma relating
normalisation by evaluation's \AF{eval} function to itself.

This may seem bureaucratic but it is crucial: the model definition uses
the host language's function space which contains more functions than
just the ones obtained by evaluating a simply-typed $λ$-term. Some
properties that may not be provable for arbitrary Agda programs can
be proven to hold for the ones obtained by evaluation. In particular,
functional extensionality is not provable in intensional type theory
but we can absolutely prove that the programs we obtain by evaluating
terms with a function type take extensionally equal inputs to
extensionally equal outputs.

This strong property will in particular imply that evaluation in
environment consisting of \emph{extensionally} equal values will
yield \emph{intensionally} equal normal forms as shown in
\cref{fig:normequals}.

%
%% A value at type {(\AIC{`Bool} \AIC{`→} \AIC{`Bool})} may for instance
%% inspect the extended context it is called in and behave differently
%% depending on its content. The \AF{exotic} term we define behaves like
%% the identity function \emph{except} if it is called in an extended
%% environment whose second variable is of type \AIC{`Bool} in which case
%% it returns it.

%% \begin{figure}[h]
%% \ExecuteMetaData[type-scope-semantics.agda/Semantics/NormalisationByEvaluation/BetaIotaXiEta.tex]{exotic}
%% \caption{Exotic value, not quite the identity\label{fig:nbeexotic}}
%% \end{figure}

%% This exotic term is not the result of evaluating a term in the source
%% language and breaks a desirable property: the fact that we can commute
%% thinning and reification. Indeed, thinning the \AF{exotic} value may
%% lead to it being reified in a context that happens to have exactly the
%% right form to trigger the special case as demonstrated by the two
%% conflicting test cases in \cref{fig:nbeexotictest}
%% (writing \AF{2⇒2} for the type of boolean functions).

%% \begin{figure}[h]
%% \ExecuteMetaData[type-scope-semantics.agda/Semantics/NormalisationByEvaluation/BetaIotaXiEta.tex]{exotictest}
%% \caption{Thinning and Reification do not commute for \AF{exotic}\label{fig:nbeexotictest}}
%% \end{figure}

%% Clearly, these exotic functions have undesirable behaviours and need
%% to be ruled out if we want to be able to prove that normalisation has
%% good properties.

This notion of extensional equality for values in the model is formalised
by defining a Partial Equivalence Relation (PER)
(\cite{mitchell1996foundations}) which is to say a
relation which is symmetric and transitive but may not be reflexive for
all elements in its domain. The elements equal to themselves will be
guaranteed to be well behaved. We will show that given an environment
of values PER-related to themselves, the evaluation of a $λ$-term
produces a computation equal to itself too.

We start by defining the PER for the model. It is constructed by
induction on the type and ensures that terms which behave the same
extensionally are declared equal. Values at base types are concrete
data: either trivial for values of type \AIC{`Unit} or normal forms
for values of type \AIC{`Bool}. They are considered equal when they
are effectively syntactically the same, i.e. propositionally equal.
Functions on the other hand are declared equal whenever equal inputs
map to equal outputs.

\begin{figure}[h]
\ExecuteMetaData[type-scope-semantics.agda/Properties/Simulation/Instances.tex]{per}
\caption{Partial Equivalence Relation for Model Values\label{fig:nbeper}}
\end{figure}

On top of being a PER (i.e. symmetric and transitive), we can prove by a
simple case analysis on the type that this relation is also stable under
thinning for \AF{Model} values defined in \cref{fig:thnbemodel}.

\begin{figure}[h]
\ExecuteMetaData[type-scope-semantics.agda/Properties/Simulation/Instances.tex]{thPER}
\caption{Stability of the \AF{PER}~under \AF{Thinning}\label{fig:nbeperth}}
\end{figure}

The interplay of reflect and reify with this notion of equality has to be described
in one go because of their mutual definition. It confirms that \AF{PER} is an appropriate
notion of semantic equality: \AF{PER}-related values are reified to propositionally
equal normal forms whilst propositionally equal neutral terms are reflected
to \AF{PER}-related values.

\begin{figure}[h]
\ExecuteMetaData[type-scope-semantics.agda/Properties/Simulation/Instances.tex]{reifyreflect}
\caption{Relational Versions of Reify and Reflect\label{fig:nbeperreifyreflect}}
\end{figure}

Just like in the definition of the evaluation function, conditional branching is the
interesting case. Provided a pair of boolean values (i.e. normal forms of type
\AIC{`Bool}) which are PER-equal (i.e. syntactically equal) and two pairs of PER-equal
\AB{σ}-values corresponding respectively to the left and right branches of the two
if-then-elses, we can prove that the two semantical if-then-else produce PER-equal values.
Because of the equality constraint on the booleans, Agda allows us to only write the
three cases we are interested in: all the other ones are trivially impossible.

In case the booleans are either \AIC{`tt} or \AIC{`ff}, we can immediately conclude
by invoking one of the hypotheses. Otherwise we remember from \cref{fig:nbeifte}
that the evaluation function
produces a value by reflecting the neutral term obtained after reifying both branches.
We can play the same game but at the relational level this time and we obtain precisely
the proof we wanted.

\begin{figure}[h]
\ExecuteMetaData[type-scope-semantics.agda/Properties/Simulation/Instances.tex]{ifte}
\caption{Relational If-Then-Else\label{fig:nbeiftenelser}}
\end{figure}

This provides us with all the pieces necessary to prove our simulation lemma. The relational
counterpart of \AIC{`lam} is trivial as the induction hypothesis corresponds precisely to
the PER-notion of equality on functions. Similarly the case for \AIC{`app} is easily discharged:
the PER-notion of equality for functions is precisely the strong induction hypothesis we need
to be able to make use of the assumption that the respective function's arguments are PER-equal.

\begin{figure}[h]
\ExecuteMetaData[type-scope-semantics.agda/Properties/Simulation/Instances.tex]{nbe}
\caption{Normalisation by \AF{Eval}uation is in \AF{PER}-\AR{Simulation}~ with Itself\label{fig:nbeselfsim}}
\end{figure}

As a corollary, we can deduce that evaluating a term in two environments related pointwise
by \AF{PER} yields two semantic objects themselves related by \AF{PER}. Which, once reified,
give us two equal terms.

\begin{figure}[h]
\ExecuteMetaData[type-scope-semantics.agda/Properties/Simulation/Instances.tex]{normR}
\caption{Normalisation in \AF{PER}-related Environments Yields Equal Normal Forms}\label{fig:normequals}
\end{figure}

We can now move on to the more complex example of a proof framework built generically
over our notion of \AF{Semantics}.

\chapter{Fusions of Evaluations}

When studying the meta-theory of a calculus, one systematically
needs to prove fusion lemmas for various semantics. For instance,
Benton et al.~(\citeyear{benton2012strongly}) prove six such lemmas
relating renaming, substitution and a typeful semantics embedding
their calculus into Coq. This observation naturally led us to
defining a fusion framework describing how to relate three semantics:
the pair we sequence and their sequential composition. The fundamental
lemma we prove can then be instantiated six times
to derive the corresponding corollaries.


