\chapter{Building Generic Proofs about Generic Programs}

We have already shown in \cref{sec:simulationrel,sec:fusionrel} that, for the
simply typed $\lambda$-calculus, introducing an abstract notion of Semantics
not only reveals the shared structure of common traversals, it also allows
us to give abstract proof frameworks for simulation or fusion lemmas. These
ideas naturally extend to our generic presentation of semantics for all syntaxes.

\section{Additional Relation Transformers}

During our exploration of generic proofs about the behaviour of \AR{Semantics}
for a concrete object language, we have introduced a notion \AR{Rel} of
relations as well as a relation transformer for environments
(cf. \cref{sec:relation-transformers}). Working on a universe of syntaxes,
we are going to need to define additional relators.

\paragraph{Kripke relator}
The Kripke relator is a generalisation of the ad-hoc definition introduced
in \cref{fig:relationalkripke}. We assume that we have two types of values
\AB{𝓥ᴬ} and \AB{𝓥ᴮ}
as well as a relation \AB{𝓥ᴿ} for pairs of such values, and two types
of computations \AB{𝓒ᴬ} and \AB{𝓒ᴮ} whose notion of relatedness is
given by \AB{𝓒ᴿ}. We can define
\AF{Kripkeᴿ} relating Kripke functions of type
{(\AF{Kripke} \AB{𝓥ᴬ} \AB{𝓒ᴬ})} and {(\AF{Kripke} \AB{𝓥ᴮ} \AB{𝓒ᴮ})}
respectively by stating that they send related inputs
to related outputs. We use the relation transformer \AF{All} defined
in the previous paragraph.

\begin{figure}[h]
 \ExecuteMetaData[shared/Data/Var/Varlike.tex]{kripkeR}
\caption{Relational Kripke Function Spaces: From Related Inputs to Related Outputs\label{fig:Kripke-rel}}
\end{figure}

\paragraph{Desc relator}
The relator (\AF{⟦} \AB{d} \AF{⟧ᴿ}) is a relation transformer which characterises
structurally equal layers such that their substructures are themselves related
by the relation it is passed as an argument. It inherits a lot of its relational
arguments' properties: whenever \AB{R} is reflexive (respectively symmetric or
transitive) so is {(\AF{⟦} \AB{d} \AF{⟧ᴿ} \AB{R})}.\label{lem:zipstable}

It is defined by induction on the description and case analysis on the two
layers which are meant to be equal:
\begin{itemize}
\item In the stop token case {(\AIC{`∎} \AB{i})}, the two layers are two proofs
    that the branche's respective indices match the specified one. We consider
    these two proofs to be trivially equal (i.e. the constraint generated is the
    unit type).
    %
    This would not hold true in Homotopy Type Theory~\cite{hottbook}
    but if we were to generalise the work to that setting, we could either explicitly
    restrict our setup to language whose indices are equipped with a decidable equality
    or insist on studying the ways in which these proofs can be equal.
  \item When facing a recursive position {(\AIC{`X} \AB{Δ} \AB{j} \AB{d})}, we
    demand that the two substructures are related by {\AB{R} \AB{Δ} \AB{j}}
    and that the rest of the layers are related by (\AF{⟦} \AB{d} \AF{⟧ᴿ} \AB{R})
  \item Two nodes of type {(\AIC{`σ} \AB{A} \AB{d})} will
    be related if they both carry the same payload \AB{a} of type \AB{A} and if
    the rest of the layers are related by (\AF{⟦} \AB{d} \AB{a} \AF{⟧ᴿ} \AB{R})
\end{itemize}

\begin{figure}[h]
 \ExecuteMetaData[generic-syntax.agda/Generic/Relator.tex]{ziptype}
\caption{Relator: Characterising Structurally Equal Values with Related Substructures\label{fig:zip-rel}}
\end{figure}

If we were to take a fixpoint of \AF{⟦\_⟧ᴿ}, we could obtain a structural
notion of equality for terms which we could prove equivalent to propositional
equality. Although interesting in its own right, this section will focus
on more advanced use-cases.

%%%%%%%%%%%%%%%%%%%%%%%%%%%%%%%%%%%%%%%%%%%%%%%%%%%%%%%%%%%%%%%%%%%%%%%
%% SIMULATION

\chapter{The Simulation Relation}
\label{sec:simulationrel}

Thanks to \AF{Semantics}, we have already saved work by not reiterating the
same traversals. Moreover, this disciplined approach to building models and
defining the associated evaluation functions can help us refactor the proofs
of some properties of these semantics.

Instead of using proof scripts as Benton et al.~(\citeyear{benton2012strongly})
do, we describe abstractly the constraints the logical relations (\cite{reynolds1983types})
defined on computations (and environment values) have to respect to ensure
that evaluating a term in related environments
produces related outputs. This gives us a generic framework to
state and prove, in one go, properties about all of these semantics.

Our first example of such a framework will stay simple on purpose.
However it is no mere bureaucracy: the
result proven here will actually be useful in the next section
when considering more complex properties.

This first example is describing the relational interpretation of the terms.
It should give the reader a good introduction to the setup before we take on
more complexity. The types involved might look a bit scarily abstract but the
idea is rather simple: we have a \AR{Simulation} between two \AR{Semantics}
when evaluating a term in related environments yields related values. The bulk
of the work is to make this intuition formal.

\section{Relations Between Scoped Families}

We start by defining what it means to be a relation between two \scoped{\AB{I}}
families \AB{T} and \AB{U}: at every type \AB{σ} and every context \AB{Γ}, we
expect to have a relation between (\AB{T}~\AB{σ}~\AB{Γ}) and (\AB{U}~\AB{σ}~\AB{Γ}).
We use a \AK{record} wrapper for two reasons. First, we define the relations we
are interested in by copattern-matching thus preventing their eager unfolding by
Agda; this makes the goals much more readable during interactive development.
Second, it is easier for Agda to recover \AB{T} and \AB{U} by unification when
they appear as explicit parameters of a record rather than as applied families
in the body of the definition.

\begin{figure}[h]
\ExecuteMetaData[shared/Data/Relation.tex]{rel}
\caption{Relation Between \scoped{\AB{I}} Families}
\end{figure}

If we have a relation for values, we can lift it in a pointwise manner to a relation
on environment of values. We call this relation transformer \AR{All}. We also define
it using a \AK{record} wrapper, for the same reasons.

\begin{figure}[h]
\ExecuteMetaData[shared/Data/Relation.tex]{all}
\caption{Relation Between Environments of Values}
\end{figure}

For virtually every combinator on environments, we have a corresponding combinator
for \AR{All}: the empty environment \AF{ε} is associated to \AF{εᴿ} the proof that
two empty environments are always related, to the environment extension  \AF{\_∙\_}
corresponds the relation on environment extension \AF{\_∙ᴿ\_} which provided takes a
proof that two environments are related and that two values are related and returns
the proof that the environments each extended with the appropriate value are both
related, etc.

Once we have all of these definitions, we can spell out what it means to simulate
a semantics with another.

\section{Simulation Constraints}

The evidence that we have a \AR{Simulation} between two \AR{Semantics} is
packaged in a record indexed by the semantics as well as two relations.
The first one (\AB{𝓥ᴿ}) relates values and the second one (\AB{𝓒ᴿ})
describes simulation for computations.

\ExecuteMetaData[type-scope-semantics.agda/Properties/Simulation/Specification.tex]{simulation}

The set of simulation constraints is in one-to-one correspondance with that of
semantical constructs. We start with value thinnings: provided two values are
related, their respective thinnings should still be related.

\ExecuteMetaData[type-scope-semantics.agda/Properties/Simulation/Specification.tex]{thV}

Our other constraints are going to heavily feature \AB{𝓒ᴿ} applied to one term evaluated
twice: once by \AB{𝓢ᴬ} with the environment of values \AB{ρᴬ} and once by \AB{𝓢ᴮ} with
\AB{ρᴮ}. To make the types more readable, we introduce an intermediate definition \AF{𝓡}
making this pattern explicit.

\ExecuteMetaData[type-scope-semantics.agda/Properties/Simulation/Specification.tex]{crel}

The relational counterpart of \AIC{`var} and \ARF{var} is the first field to make use
of \AB{𝓡}: provided that \AB{ρᴬ} and \AB{ρᴮ} carry values related by \AB{𝓥ᴿ}, the result
of evaluating the variable \AB{v} in each respectively should yield computations related
by \AB{𝓒ᴿ}.

\ExecuteMetaData[type-scope-semantics.agda/Properties/Simulation/Specification.tex]{var}

Value constructors in the language follow a similar pattern: provided that the evaluation
environment are related, we expect the computations to be related too.

\ExecuteMetaData[type-scope-semantics.agda/Properties/Simulation/Specification.tex]{base}

Then come the structural cases: for language constructs like \AIC{`app} and \AIC{`ifte}
whose subterms live in the same context as the overall term, the constraints are purely
structural. Provided that the evaluation environments are related, and that the evaluation
of the subterms in each environment respectively are related then the evaluations of the
overall terms should also yield related results.

\ExecuteMetaData[type-scope-semantics.agda/Properties/Simulation/Specification.tex]{struct}

Finally, we reach the most interesting case. The semantics attached to the body of a
\AIC{`lam} is expressed in terms of a Kripke function space. As a consequence, the
relational semantics will need a relational notion of Kripke function space (\AF{Kripkeᴿ})
to spell out the appropriate simulation constraint. This relational Kripke function space
states that in any thinning of the evaluation context and provided two related inputs,
the evaluation of the body in each thinned environment extended with the appropriate
value should yield related computations.

\begin{figure}[h]
\ExecuteMetaData[type-scope-semantics.agda/Properties/Simulation/Specification.tex]{rkripke}
\caption{Relational Kripke Function Spaces: From Related Inputs to Related Outputs\label{fig:relationalkripke}}
\end{figure}

This allows us to describe the constraint for \AIC{`lam}: provided related environments
of values, if we have a relational Kripke function space for the body of the \AIC{`lam}
then both evaluations should yield related results.

\ExecuteMetaData[type-scope-semantics.agda/Properties/Simulation/Specification.tex]{lam}

This specification is only useful if it is accompanied by a a fundamental lemma of
simulations stating that the evaluation of a term on related inputs yields related
outputs.

\section{Fundamental Lemma of Simulations}

Given two Semantics \AB{𝓢ᴬ} and \AB{𝓢ᴮ} in simulation with respect to relations
\AB{𝓥ᴿ} for values and \AB{𝓒ᴿ} for computations, we have that for any term \AB{t}
and environments \AB{ρᴬ} and \AB{ρᴮ}, if the two environments are \AB{𝓥ᴿ}-related
in a pointwise manner then the semantics associated to \AB{t} by \AB{𝓢ᴬ} using \AB{ρᴬ}
is \AB{𝓒ᴿ}-related to the one associated to \AB{t} by \AB{𝓢ᴮ} using \AB{ρᴮ}.

In a manner reminiscent of our proof of the fundamental lemma of \AR{Semantics}, we
introduce a \AM{Fundamental} module parametrised by a record packing the evidence
that two semantics are in \AR{Simulation}. This allows us to bring all of the
corresponding relational counterpart of term constructors into scope by \AK{open}ing
the record. The traversal then uses them to combine the induction hypotheses arising
structurally.

\begin{figure}[h]
\ExecuteMetaData[type-scope-semantics.agda/Properties/Simulation/Specification.tex]{fundamental}
\caption{Fundamental Lemma of \AR{Simulation}s\label{fig:fundsim}}
\end{figure}

We can now consider the second criterion for usefulness: the existence of interesting
instances of a \AR{Simulation}.

\section{Syntactic Traversals are Extensional}

A first corollary of the fundamental lemma of simulations is the fact that semantics
arising from a \AR{Syntactic} (cf.~\cref{fig:syntactic}) definition are extensional. We
can demonstrate this by proving that every syntactic semantics is in simulation with
itself. That is to say that the evaluation function yields propositionally equal
values provided extensionally equal environments of values.

Under the assumption that \AB{Syn} is a \AR{Syntactic} instance, we can define the
corresponding \AR{Semantics} \AB{𝓢} by setting
\ExecuteMetaData[type-scope-semantics.agda/Properties/Simulation/Instances.tex]{synsem}.
Using \AF{Eqᴿ} the \AR{Rel} defined as the pontwise lifting of propositional equality,
we can make our earlier claim formal and prove it. All the constraints are discharged
either by reflexivity or by using congruence to combine various hypotheses.

\begin{figure}[h]
\ExecuteMetaData[type-scope-semantics.agda/Properties/Simulation/Instances.tex]{syn-ext}
\caption{\AR{Syntactic}~ Traversals are in \AR{Simulation}~ with Themselves\label{fig:synselfsim}}
\end{figure}

Because the \AR{Simulation} statement is not necessarily extremely illuminating, we spell
out the type of the corollary to clarify what we just proved: whenever two environments
agree on each variable, evaluating a term with either of them produces equal results.

\begin{figure}[h]
\ExecuteMetaData[type-scope-semantics.agda/Properties/Simulation/Instances.tex]{synext}
\caption{\AR{Syntactic}~ Traversals are Extensional\label{fig:synextensional}}
\end{figure}

This may look like a trivial result however we have found multiple use cases for it
during the development of our solution to the POPLMark Reloaded challenge
(\citeyear{poplmark2}): when proceeding by equational reasoning, it is often the case
that we can make progress on each side of the equation only to meet in the middle with
the same traversals using two environments manufactured in slightly different ways but
ultimately equal. This lemma allows us to bridge that last gap.

\section{Renaming is a Substitution}

Similarly, it is sometimes the case that after a bit of rewriting we end up with an
equality between one renaming and one substitution. But it turns out that as long as
the substitution is only made up variables, it is indeed equal to the corresponding
renaming. We can make this idea formal by introducting the \AF{VarTermᴿ} relation
stating that a variable and a term are morally equal like so:

\begin{figure}[h]
\ExecuteMetaData[type-scope-semantics.agda/Properties/Simulation/Instances.tex]{varterm}
\caption{Characterising Equal Variables and Terms\label{fig:vartermR}}
\end{figure}

We can then state our result: we can prove a simulation lemma between \AF{Renaming}
and \AF{Substitution} where values (i.e. variables in the cases of renaming and terms
in terms of substitution) are related by \AF{VarTermᴿ} and computations (i.e. terms)
are related by \AF{Eqᴿ}. Once again we proceed by reflexivity and congruence.

\begin{figure}[h]
\ExecuteMetaData[type-scope-semantics.agda/Properties/Simulation/Instances.tex]{renissub}
\caption{\AF{Renaming}~ is in \AR{Simulation}~ with \AF{Substitution}\label{fig:renissub}}
\end{figure}

Rather than showing one more time the type of the corollary, we show a specialized
version where we pick the substitution to be precisely the thinning used on which we
have mapped the \AIC{`var} constructor.

\begin{figure}[h]
\ExecuteMetaData[type-scope-semantics.agda/Properties/Simulation/Instances.tex]{renassub}
\caption{Renaming as a Substitution\label{fig:renassub}}
\end{figure}

\section{The PER for βιξη-Values is Closed under Evaluation}

Now that we are a bit more used to the simulation framework and simulation lemmas,
we can look at a more complex example: the simulation lemma relating normalisation
by evaluation's \AF{eval} function to itself. This may seem bureaucratic but it is
crucial: the model definition uses the host language's function space which contains
more functions than just the ones obtained by evaluating a simply-typed $λ$-term.
A value at type {(\AIC{`Bool} \AIC{`→} \AIC{`Bool})} may for instance behave like
boolean negation on canonical terms but be the constant \AIC{`tt} function on neutral
value. It does not correspond to any term in the source language: any candidate term
would allow us to write expressions not stable under substitution!

\begin{figure}[h]
\ExecuteMetaData[type-scope-semantics.agda/Semantics/NormalisationByEvaluation/BetaIotaXiEta.tex]{exotic}
\caption{Exotic Value, Not Quite Equal to Negation\label{fig:nbeexotic}}
\end{figure}

Clearly, these exotic functions have undesirable behaviours and need to be ruled out
if we want to be able to prove that normalisation has good properties. This is done
by defining a Partial Equivalence Relation (PER) (\cite{mitchell1996foundations})
on the model which is to say a relation which is symmetric and transitive but may
not be reflexive for all elements in its domain. The elements equal to themselves will be
guaranteed to be well behaved. We will show that given an environment of values
PER-related to themselves, the evaluation of a $λ$-term produces a computation
equal to itself too.


We start by defining the PER for the model. It is constructed by induction on the type
and ensures that terms which behave the same extensionally are declared equal. Values at
base types are concrete data: either trivial for values of type \AIC{`Unit} or normal
forms for values of type \AIC{`Bool}. They are considered equal when they are effectively
syntactically the same, i.e. propositionally equal. Functions on the other hand are
declared equal whenever equal inputs map to equal outputs.

\begin{figure}[h]
\ExecuteMetaData[type-scope-semantics.agda/Properties/Simulation/Instances.tex]{per}
\caption{Partial Equivalence Relation for Model Values\label{fig:nbeper}}
\end{figure}

On top of being a PER (i.e. symmetric and transitive), we can prove by a simple case
analysis on the type that this relation is also stable under thinning for \AF{Model}
values defined in \cref{fig:thnbemodel}.

\begin{figure}[h]
\ExecuteMetaData[type-scope-semantics.agda/Properties/Simulation/Instances.tex]{thPER}
\caption{Stability of the \AF{PER}~ under \AF{Thinning}\label{fig:nbeperth}}
\end{figure}

The interplay of reflect and reify with this notion of equality has to be described
in one go because of their mutual definition. It confirms that \AF{PER} is an appropriate
notion of semantic equality: \AF{PER}-related values are reified to propositionally
equal normal forms whilst propositionally equal neutral terms are reflected
to \AF{PER}-related values.

\begin{figure}[h]
\ExecuteMetaData[type-scope-semantics.agda/Properties/Simulation/Instances.tex]{reifyreflect}
\caption{Relational Versions of Reify and Reflect\label{fig:nbeperreifyreflect}}
\end{figure}

Just like in the definition of the evaluation function, conditional branching is the
interesting case. Provided a pair of boolean values (i.e. normal forms of type
\AIC{`Bool}) which are PER-equal (i.e. syntactically equal) and two pairs of PER-equal
\AB{σ}-values corresponding respectively to the left and right branches of the two
if-then-elses, we can prove that the two semantical if-then-else produce PER-equal values.
Because of the equality constraint on the booleans, Agda allows us to only write the
three cases we are interested in: all the other ones are trivially impossible.

In case the booleans are either \AIC{`tt} or \AIC{`ff}, we can immediately conclude
by invoking one of the hypotheses. Otherwise we remember from \cref{fig:nbeifte}
that the evaluation function
produces a value by reflecting the neutral term obtained after reifying both branches.
We can play the same game but at the relational level this time and we obtain precisely
the proof we wanted.

\begin{figure}[h]
\ExecuteMetaData[type-scope-semantics.agda/Properties/Simulation/Instances.tex]{ifte}
\caption{Relational If-Then-Else\label{fig:nbeiftenelser}}
\end{figure}

This provides us with all the pieces necessary to prove our simulation lemma. The relational
counterpart of \AIC{`lam} is trivial as the induction hypothesis corresponds precisely to
the PER-notion of equality on functions. Similarly the case for \AIC{`app} is easily discharged:
the PER-notion of equality for functions is precisely the strong induction hypothesis we need
to be able to make use of the assumption that the respective function's arguments are PER-equal.

\begin{figure}[h]
\ExecuteMetaData[type-scope-semantics.agda/Properties/Simulation/Instances.tex]{nbe}
\caption{Normalisation by \AF{Eval}uation is in \AF{PER}-\AR{Simulation}~ with Itself\label{fig:nbeselfsim}}
\end{figure}

As a corollary, we can deduce that evaluating a term in two environments related pointwise
by \AF{PER} yields two semantic objects themselves related by \AF{PER}. Which, once reified,
give us two equal terms.

\begin{figure}[h]
\ExecuteMetaData[type-scope-semantics.agda/Properties/Simulation/Instances.tex]{normR}
\caption{Normalisation in \AF{PER}-related Environments Yields Equal Normal Forms}
\end{figure}

We can now move on to the more complex example of a proof framework built generically
over our notion of \AF{Semantics}.

\chapter{The Fusion Relation}
\label{sec:fusionrel}

When studying the meta-theory of a calculus, one systematically needs to prove
fusion lemmas for various semantics. For instance, Benton et al.~(\citeyear{benton2012strongly})
prove six such lemmas relating renaming, substitution and a typeful semantics
embedding their calculus into Coq. This observation naturally lead us to
defining a fusion framework describing how to relate three semantics: the pair
we sequence and their sequential composition. The fundamental lemma we prove
can then be instantiated six times to derive the corresponding corollaries.

\section{Fusion Constraints}

The evidence that \AB{𝓢ᴬ}, \AB{𝓢ᴮ} and \AB{𝓢ᴬᴮ} are such that \AB{𝓢ᴬ} followed
by \AB{𝓢ᴮ} is equivalent to \AB{𝓢ᴬᴮ} (e.g. \AF{Substitution} followed by \AF{Renaming}
can be reduced to \AF{Substitution}) is packed in a record \AR{Fusion} indexed by the
three semantics but also three relations. The first one (\AB{𝓔ᴿ}) characterises the
triples of environments (one for each one of the semantics) which are compatible.
The second one (\AB{𝓥ᴿ}) states what it means for two environment values of \AB{𝓢ᴮ}
and \AB{𝓢ᴬᴮ} respectively to be related. The last one (\AB{𝓒ᴿ}) relates computations
obtained as results of running \AB{𝓢ᴮ} and \AB{𝓢ᴬᴮ} respectively.

\ExecuteMetaData[type-scope-semantics.agda/Properties/Fusion/Specification.tex]{fusion}

As before, most of the fields of this record describe what structure these relations
need to have. However, we start with something slightly different: given that we are
planing to run the \AR{Semantics} \AB{𝓢ᴮ} \emph{after} having run \AB{𝓢ᴬ}, we need
two components: a way to extract a term from an \AB{𝓢ᴬ} and a way to manufacture a
placeholder \AB{𝓢ᴬ} value when going under a binder. Our first two fields are therefore:

\ExecuteMetaData[type-scope-semantics.agda/Properties/Fusion/Specification.tex]{reifyvar0}

Then come two constraints dealing with the relations talking about evaluation environments.
\ARF{\_∙ᴿ\_} tells us how to extend related environments: one should be able to push related
values onto the environments for \AB{𝓢ᴮ} and \AB{𝓢ᴬᴮ} whilst merely extending the one
for \AB{𝓢ᴬ} with the token value \ARF{var0ᴬ}.

\ARF{th\textasciicircum{}𝓔ᴿ} guarantees that it is always possible to thin the environments
for \AB{𝓢ᴮ} and \AB{𝓢ᴬᴮ} in a \AB{𝓔ᴿ} preserving manner.

\ExecuteMetaData[type-scope-semantics.agda/Properties/Fusion/Specification.tex]{thV}

Then we have the relational counterpart of the various term constructors. We can once
again introduce an extra definition \AF{𝓡} which will make the type of the combinators
defined later on clearer. \AF{𝓡} relates at a given type a term and three environments
by stating that the computation one gets by sequentially evaluating the term in the first
and then the second environment is related to the one obtained by directly evaluating
the term in the third environment. Note the use of \ARF{reifyᴬ} to recover a \AD{Term}
from a computation in \AB{𝓒ᴬ} before using the second evaluation function \AF{evalᴮ}.

\ExecuteMetaData[type-scope-semantics.agda/Properties/Fusion/Specification.tex]{crel}

As with the previous section, only a handful of these combinators are out
of the ordinary. We will start with the \AIC{`var} case. It states that
fusion indeed happens when evaluating a variable using related environments.

\ExecuteMetaData[type-scope-semantics.agda/Properties/Fusion/Specification.tex]{var}

Just like for the simulation relation, the relational counterpart of value constructors
in the language state that provided that the evaluation environment are related,
we expect the computations to be related too.

\ExecuteMetaData[type-scope-semantics.agda/Properties/Fusion/Specification.tex]{base}

Similarly, we have purely structural constraints for term constructs which have purely
structural semantical counterparts. For \AIC{`app} and \AIC{`ifte}, provided that the
evaluation environments are related and that the evaluation of the subterms in each
environment respectively are related then the evaluations of the overall terms should
also yield related results.

\ExecuteMetaData[type-scope-semantics.agda/Properties/Fusion/Specification.tex]{struct}

Finally, the \AIC{`lam}-case puts some strong restrictions on the way the
$λ$-abstraction's body may be used by \AB{𝓢ᴬ}: we assume it is evaluated in an
environment thinned by one variable and extended using \ARF{var0ᴬ}. But it is
quite natural to have these restrictions: given that \ARF{reifyᴬ} quotes the
result back, we are expecting this type of evaluation in an extended context
(i.e. under one lambda). And it turns out that this is indeed enough for all of
our examples. The evaluation environments used by the semantics \AB{𝓢ᴮ} and
\AB{𝓢ᴬᴮ} on the other hand can be arbitrarily thinned before being extended with
related values to be substituted for the variable bound by the \AIC{`lam}.

\ExecuteMetaData[type-scope-semantics.agda/Properties/Fusion/Specification.tex]{lam}

\section{Fundamental Lemma of Fusions}

As with simulation, we measure the usefulness of this framework by the way we can
prove its fundamental lemma and then obtain useful corollaries. Once again,
having carefully identified what the constraints should be, proving the fundamental
lemma is not a problem.

\ExecuteMetaData[type-scope-semantics.agda/Properties/Fusion/Specification.tex]{fundamental}

\section{The Special Case of Syntactic Semantics}

The translation from \AR{Syntactic} to \AR{Semantics} uses a lot of constructors
as their own semantic counterpart, it is hence possible to generate evidence of
\AR{Syntactic} triplets being fusable with much fewer assumptions. We isolate
them and prove the result generically to avoid repetition. A \AR{SynFusion}
record packs the evidence for \AR{Syntactic} semantics \AB{Synᴬ}, \AB{Synᴮ} and
\AB{Synᴬᴮ}. It is indexed by these three \AR{Syntactic}s as well as two relations
(\AB{𝓣ᴿ} and \AB{𝓔ᴿ}) corresponding to the \AB{𝓥ᴿ} and \AB{𝓔ᴿ} ones of the
\AR{Fusion} framework; \AB{𝓒ᴿ} will always be \AF{Eqᴿ} as we are talking about terms.

\ExecuteMetaData[type-scope-semantics.agda/Properties/Fusion/Syntactic/Specification.tex]{synfusion}

The first two constraints \ARF{\_∙ᴿ\_} and \ARF{th\textasciicircum{}𝓔ᴿ} are directly taken
from the \AR{Fusion} specification: we still need to be able to extend existing related
environment with related values, and to thin environments in a relatedness-preserving manner.

\ExecuteMetaData[type-scope-semantics.agda/Properties/Fusion/Syntactic/Specification.tex]{thV}

We once again define \AF{𝓡}, a specialised version of its \AR{Fusion} counterpart
stating that the results of the two evaluations are propositionally equal.

\ExecuteMetaData[type-scope-semantics.agda/Properties/Fusion/Syntactic/Specification.tex]{crel}

Once we have \AF{𝓡}, we can concisely write down the constraint \ARF{varᴿ} which
is also already present in the definition of \AR{Fusion}.

\ExecuteMetaData[type-scope-semantics.agda/Properties/Fusion/Syntactic/Specification.tex]{var}

Finally, we have a fourth constraint (\ARF{zroᴿ}) saying that \AB{Synᴮ} and
\AB{Synᴬᴮ}'s respective \ARF{zro}s are producing related values. This will provide
us with just the right pair of related values to use in \AR{Fusion}'s \ARF{lamᴿ}.

\ExecuteMetaData[type-scope-semantics.agda/Properties/Fusion/Syntactic/Specification.tex]{zro}

Everything else is a direct consequence of the fact we are only considering
syntactic semantics. Given a \AR{SynFusion} relating three \AR{Syntactic}
semantics, we get a \AR{Fusion} relating the corresponding \AR{Semantics} where
\AB{𝓒ᴿ} is \AF{Eqᴿ}, the pointwise lifting of propositional equality. The proof
relies on the way the translation from \AR{Syntactic} to \AR{Semantics} is formulated
in \cref{sec:syntactic}.

\begin{figure}
\ExecuteMetaData[type-scope-semantics.agda/Properties/Fusion/Syntactic/Specification.tex]{fundamental}
\caption{Fundamental Lemma of Syntactic Fusions\label{fig:fundamentalsynfus}}
\end{figure}

We are now ready to give our first examples of Fusions. They are the first results one
typically needs to prove when studying the meta-theory of a language.

\section{Interactions of Renaming and Substitution}

Renaming and Substitution can interact in four ways: all but one of these
combinations is equivalent to a single substitution (the sequential execution
of two renamings is equivalent to a single renaming). These four lemmas are
usually proven in painful separation. Here we discharge them by rapid successive
instantiation of our framework, using the earlier results to satisfy the later
constraints. We only present the first instance in full details and then only
spell out the \AR{SynFusion} type signature which makes explicit the relations
used to constraint the input environments.

First, we have the fusion of two sequential renaming traversals into a single
renaming. Environments are related as follows: the composition of the two
environments used in the sequential traversals should be pointwise equal to
the third one. The composition operator \AF{select} is defined in \cref{fig:extendth}.

\begin{figure}[h]
\ExecuteMetaData[type-scope-semantics.agda/Properties/Fusion/Syntactic/Instances.tex]{renrenfus}
\caption{Syntactic Fusion of Two Renamings\label{fig:renrenfus}}
\end{figure}

Using the fundamental lemma of syntactic fusions, we get a proper \AR{Fusion} record
on which we can then use the fundamental lemma of fusions to get the renaming fusion
law we expect.

\begin{figure}[h]
\ExecuteMetaData[type-scope-semantics.agda/Properties/Fusion/Syntactic/Instances.tex]{renren}
\caption{Corollary: Renaming Fusion Law\label{fig:renren}}
\end{figure}

A similar proof gives us the fact that a renaming followed by a substitution is equivalent
to a substitution. Environments are once more related by composition.

\begin{figure}[h]
\ExecuteMetaData[type-scope-semantics.agda/Properties/Fusion/Syntactic/Instances.tex]{rensubfus}
\ExecuteMetaData[type-scope-semantics.agda/Properties/Fusion/Syntactic/Instances.tex]{rensub}
\caption{Renaming - Substitution Fusion Law\label{fig:rensub}}
\end{figure}

For the proof that a substitution followed by a renaming is equivalent to a
substitution, we need to relate the environments in a different manner:
composition now amounts to applying the renaming to every single term in the
substitution. We also depart from the use of \AF{Eqᴿ} as the relation for values:
indeed we now compare variables and terms. The relation \AF{VarTermᴿ} defined in
\cref{fig:vartermR} relates variables and terms by wrapping the variable in a
\AIC{`var} constructor and demanding it is equal to the term.

\begin{figure}[h]
\ExecuteMetaData[type-scope-semantics.agda/Properties/Fusion/Syntactic/Instances.tex]{subrenfus}
\ExecuteMetaData[type-scope-semantics.agda/Properties/Fusion/Syntactic/Instances.tex]{subren}
\caption{Substitution - Renaming Fusion Law\label{fig:subren}}
\end{figure}

Finally, the fusion of two sequential substitutions into a single one uses a
similar notion of composition. Here the second substitution is applied to each
term of the first and we expect the result to be pointwise equal to the third.
Values are once more considered related whenever they are propositionally equal.

\begin{figure}[h]
\ExecuteMetaData[type-scope-semantics.agda/Properties/Fusion/Syntactic/Instances.tex]{subsubfus}
\ExecuteMetaData[type-scope-semantics.agda/Properties/Fusion/Syntactic/Instances.tex]{subsub}
\caption{Substitution Fusion Law\label{fig:subsub}}
\end{figure}

As we are going to see in the following section, we are not limited
to \AR{Syntactic} statements.

\section{Other Examples of Fusions}

The most simple example of fusion of two \AR{Semantics} involving a non \AR{Syntactic}
one is probably the proof that \AR{Renaming} followed by normalization by evaluation's
\AR{Eval} is equivalent to \AR{Eval} with an adjusted environment.

\subsection{Fusion of Renaming Followed by Evaluation}
\label{sec:fusionrennbe}

As is now customary, we start with an auxiliary definition which will make our
type signatures a lot lighter. It is a specialised version of the relation \AF{𝓡}
introduced when spelling out the \AR{Fusion} constraints. Here the relation is \AF{PER}
and the three environments carry respectively \AD{Var} (i.e. it is a \AF{Thinning}) for
the first one, and \AF{Model} values for the two other ones.

\ExecuteMetaData[type-scope-semantics.agda/Properties/Fusion/Instances.tex]{crel}

We start with the most straigtforward of the non-trivial cases: the relational
counterpart of \AIC{`app}. The \AF{Kripkeᴿ} structure of the induction hypothesis
for the function has precisely the strength we need to make use of the hypothesis
for its argument.

\begin{figure}[h]
\ExecuteMetaData[type-scope-semantics.agda/Properties/Fusion/Instances.tex]{appR}
\caption{Relational Application}
\end{figure}

The relational counterpart of \AIC{`ifte} is reminiscent of the one we used when
proving that normalisation by evaluation is in simulation with itself in
\cref{fig:nbeiftenelser}: we have two arbitrary boolean values resulting from the
evaluation of \AB{b} in two distinct manners but we know them to be the same thanks
to them being \AF{PER}-related. The canonical cases are trivially solved by
using one of the assumptions whilst the neutral case can be proven to hold thanks
to the relational versions of \AF{reify} and \AF{reflect}.

\begin{figure}[h]
\ExecuteMetaData[type-scope-semantics.agda/Properties/Fusion/Instances.tex]{ifteR}
\caption{Relational If-Then-Else}
\end{figure}

The rest of the constraints can be discharged fairly easily; either by using a
constructor, combining some of the provided hypotheses or using general results
such as the stability of \AF{PER}-relatedness under thinning of the \AF{Model}
values.

\begin{figure}[h]
\ExecuteMetaData[type-scope-semantics.agda/Properties/Fusion/Instances.tex]{reneval}
\caption{Renaming Followed by Evaluation is an Evaluation}
\end{figure}


By the fundamental lemma of \AR{Fusion}, we get the result we are looking for:
a renaming followed by an evaluation is equivalent to an evaluation in a touched
up environment.

\begin{figure}[h]
\ExecuteMetaData[type-scope-semantics.agda/Properties/Fusion/Instances.tex]{renevalfun}
\caption{Corollary: Fusion Principle for Renaming followed by Evaluation\label{fig:renevalfun}}
\end{figure}

This gives us the tools to prove the substitution lemma for evaluation.

\subsection{Substitution Lemma for Evaluation}

Given any semantics, the substitution lemma (see for instance \cite{mitchell1991kripke})
states that evaluating a term after performing a substitution is equivalent to evaluating
the term with an environment obtained by evaluating each term in the substitution.
Formally (\AB{t} is a term, \AB{γ} a substitution, \AB{ρ} an evaluation environment,
\AF{\_[\_]} denotes substitution, and \AF{⟦\_⟧\_} evaluation):

\[
\AF{⟦}~\AB{t}~\AF{[}\AB{γ}\AF{]}~\AF{⟧}~\AB{ρ}~≡~\AF{⟦}~\AB{t}~\AF{⟧}~(\AF{⟦}~\AB{γ}~\AF{⟧}~\AB{ρ})
\]

This is a key lemma in the study of a language's meta-theory and it fits our \AR{Fusion}
framework perfectly. We start by describing the constraints imposed on the environments.
They may seem quite restrictive but they are actually similar to the Uniformity condition
described by C. Coquand~(\citeyear{coquand2002formalised}) in her detailed account of NBE
for a ST$λ$C with explicit substitution and help root out exotic term
(cf. \cref{fig:nbeexotic}).

First we expect the two evaluation environments to only contain \AF{Model} values
which are \AF{PER}-related to themselves. Second, we demand that the evaluation of
the substitution in a \emph{thinned} version of the first evaluation environment
is \AF{PER}-related in a pointwise manner to the \emph{similarly thinned}
second evaluation environment. This constraint amounts to a weak commutation lemma
between evaluation and thinning; a stronger version would be to demand that thinning
of the result is equivalent to evaluation in a thinned environment.

\begin{figure}[h]
\ExecuteMetaData[type-scope-semantics.agda/Properties/Fusion/Instances.tex]{subr}
\caption{Constraints on Triples of Environments for the Substitution Lemma}
\end{figure}

We can then state and prove the substitution lemma using \AF{Subᴿ} as the constraint
on environments and \AF{PER} as the relation for both values and computations.

\begin{figure}[h]
\ExecuteMetaData[type-scope-semantics.agda/Properties/Fusion/Instances.tex]{subeval}
\caption{Substitution Followed by Evaluation is an Evaluation}
\end{figure}

The proof is similar to that of fusion of renaming with evaluation in
\cref{sec:fusionrennbe}: we start by defining a notation \AF{𝓡} to lighten the
types, then combinators \AF{APPᴿ} and \AF{IFTEᴿ}. The cases for
\ARF{th\textasciicircum{}𝓔ᴿ}, \ARF{\_∙ᴿ\_}, and \ARF{varᴿ} are a bit more tedious:
they rely crucially on the fact that we can prove a fusion principle and an identity
lemma for \AF{th\textasciicircum{}Model} as well as an appeal to \AF{reneval}
(\cref{fig:renevalfun}) and multiple uses of \AF{Eval\textasciicircum{}Sim}
(\cref{fig:nbeselfsim}). Because the technical details do not give any additional
hindsight, we do not include the proof here.

