\chapter{Other Programs}

All the programs operating on syntax we have seen so far have been
instances of our notion of \AR{Semantics}. This shows that our
framework covers a lot of useful cases.

An interesting question raised by this observation is whether there
are any interesting traversals that are not captured by this setting.
At least two types of programs are clearly not instances of \AR{Semantics}:
those that do not assign a meaning to a term in terms of the meaning of
its direct subterms, and those whose return type depends on the input
term.

In this chapter we study two such programs. We also demonstrate that
although they may not fit the exact pattern we have managed to abstract,
it is still sometimes possible to take advantage of our data-generic
setting and to implement them once and for all syntaxes with binding.

\section{Big Step Evaluators}

Chapman's thesis~(\citeyear{chapman2009type}) provides us with a good
example of a function that does not assign a meaning to a term by only
relying on the meaning of its direct subterms: a big step evaluator.

Setting aside the issue of proving such a function terminating, a big
step evaluator computes the normal form of a term by first recursively
computing the normal forms of its subterms and then either succeeding
or, having uncovered a new redex, firing it and then normalising the
reduct. This is embodied by the case for application
(cf. ~\cref{figure:bigstepapp}).

\begin{figure}
\begin{mathpar}
\inferrule
  {f ⇓ λx.b \and t ⇓ v \and b [x ↦ v] ⇓ nf}
  {f\,t ⇓ nf}
\end{mathpar}
\caption{Big step evaluation of an application}\label{figure:bigstepapp}
\end{figure}

In contrast, the \AR{Semantics} constraints enforce that the meaning
of each term constructor is directly expressed in terms of the meaning
of its arguments. There is no way to ``restart'' the evaluator once the
redex has been fired. This means we will not be able to make it an
instance of our framework and, consequently, we will not be able to use
the tools introduced in the next chapters to reason generically over
such functions.

\input{generic-syntax.tex/catalogue/equality.tex}
