\chapter{Other Programs}

All the programs operating on syntax we have seen so far have been
instances of our notion of \AR{Semantics}. This shows that our
framework covers a lot of useful cases.

An interesting question raised by this observation is whether there
are any interesting traversals that are not captured by this setting.
At least two types of programs are clearly not instances of \AR{Semantics}:
those that do not assign a meaning to a term in terms of the meaning of
its direct subterms, and those whose return type depends on the input
term.

In this chapter we study two such programs. We also demonstrate that
although they may not fit the exact pattern we have managed to abstract,
it is still sometimes possible to take advantage of our data-generic
setting and to implement them once and for all syntaxes with binding.

\section{Big Step Evaluators}

Chapman's thesis~(\citeyear{chapman2009type}) provides us with a good
example of a function that does not assign a meaning to a term by only
relying on the meaning of its direct subterms: a big step evaluator.

Setting aside the issue of proving such a function terminating, a big
step evaluator computes the normal form of a term by first recursively
computing the normal forms of its subterms and then either succeeding
or, having uncovered a new redex, firing it and then normalising the
reduct. This is embodied by the case for application
(cf. ~\cref{figure:bigstepapp}).

\begin{figure}
\begin{mathpar}
\inferrule
  {f ⇓ λx.b \and t ⇓ v \and b [x ↦ v] ⇓ nf}
  {f\,t ⇓ nf}
\end{mathpar}
\caption{Big step evaluation of an application}\label{figure:bigstepapp}
\end{figure}

In contrast, the \AR{Semantics} constraints enforce that the meaning
of each term constructor is directly expressed in terms of the meaning
of its arguments. There is no way to ``restart'' the evaluator once the
redex has been fired. This means we will not be able to make it an
instance of our framework and, consequently, we will not be able to use
the tools introduced in the next chapters to reason generically over
such functions.

\section{Generic Decidable Equality for Terms}

A program deciding whether two terms are equal is clearly another good
example of a function that cannot be written as an instance of our notion
of \AR{Semantics}. In type theory we can do better than an uninformative
boolean function claiming that two terms are equal: we can implement a
decision procedure for propositional equality~(\cite{DBLP:conf/icfp/LohM11})
which either returns a proof that its two inputs are equal or a proof that
they cannot possibly be. Such a decidability proof necessarily mentions
its inputs in the type of its output while our framework does not allow
these type of dependencies.

This kind of boilerplate function is however something users really do
not want to write themselves. Haskell programmers for instance are used
to receiving help from the `deriving'
mechanism~(\cite{DBLP:journals/entcs/HinzeJ00,DBLP:conf/haskell/MagalhaesDJL10})
to automatically generate common traversals for every inductive type
they define.

Recalling that generic programming is normal programming over a universe
in a dependently typed language~(\cite{DBLP:conf/ifip2-1/AltenkirchM02}),
we ought to be able to deliver similar functionalities for syntaxes with
binding. The techniques used in this concrete example are general enough
that they also apply to the definition of an ordering test, a \texttt{Show}
instance, etc.

The notion of decidability can be neatly formalised by an inductive family
with two constructors: a \AF{Set} \AB{P} is decidable if we can either say
\AIC{yes} and return a proof of \AB{P} or \AIC{no} and provide a proof of
the negation of \AB{P} (here, a proof that \AB{P} implies the empty type
\AD{⊥}).

\begin{figure}[h]
\begin{minipage}[t]{0.45\textwidth}
  \ExecuteMetaData[shared/Stdlib.tex]{bottom}
\end{minipage}
\begin{minipage}[t]{0.45\textwidth}
  \ExecuteMetaData[shared/Stdlib.tex]{dec}
\end{minipage}
\caption{Empty Type and Decidability as an Inductive Family}
\end{figure}

To get acquainted with these new notions we can start by proving that equality
of variables is decidable.

\subsection{Deciding Variable Equality}

The type of the decision procedure for equality of variables is as follows:
given any two variables (of the same type, in the same context), the set of
equality proofs between them is \AD{Dec}idable.

\begin{agdasnippet}
\ExecuteMetaData[generic-syntax.agda/Generic/Equality.tex]{eqVarType}
\end{agdasnippet}

We can easily dismiss two trivial cases: if the two variables have distinct
head constructors then they cannot possibly be equal. Agda allows us to
dismiss the impossible premise of the function stored in the \AIC{no}
constructor by using an absurd pattern \AS{()}.

\begin{agdasnippet}
\ExecuteMetaData[generic-syntax.agda/Generic/Equality.tex]{eqVarNo}
\end{agdasnippet}

Otherwise if the two head constructors agree we can be in one of two
situations. If they are both \AIC{z} then we can conclude that the two
variables are indeed equal to each other.

\begin{agdasnippet}
\ExecuteMetaData[generic-syntax.agda/Generic/Equality.tex]{eqVarYesZ}
\end{agdasnippet}

Finally if the two variables are {(\AIC{s} \AB{v})} and {(\AIC{s} \AB{w})}
respectively then we need to check recursively whether \AB{v} is equal
to \AB{w}. If it is the case we can conclude by invoking the congruence
rule for \AIC{s}. If \AB{v} and \AB{w} are not equal then a proof that
{(\AIC{s} \AB{v})} and {(\AIC{s} \AB{w})} are will lead to a direct
contradiction by injectivity of the constructor \AIC{s}.

\begin{agdasnippet}
\ExecuteMetaData[generic-syntax.agda/Generic/Equality.tex]{eqVarYesS}
\end{agdasnippet}

\subsection{Deciding Term Equality}

The constructor \AIC{`σ} for descriptions gives us the ability to store
values of any \AF{Set} in terms. For some of these \AF{Set}s (e.g.
{(\AD{ℕ} → \AD{ℕ})}), equality is not decidable. As a consequence
our decision procedure will be conditioned to the satisfaction of a
certain set of \AF{Constraints} which we can compute from the \AD{Desc}
itself, as show in Figure~\ref{fig:eqconstraints}. We demand that we are
able to decide equality for all of the \AF{Set}s mentioned in a description.

\begin{figure}[h]
\ExecuteMetaData[generic-syntax.agda/Generic/Equality.tex]{constraints}
\caption{Constraints Necessary for Decidable Equality}\label{fig:eqconstraints}
\end{figure}

Remembering that our descriptions are given a semantics as a big right-nested
product terminated by an equality constraint, we realise that proving decidable
equality will entail proving equality between proofs of equality. We are happy
to assume Streicher's axiom K~(\cite{DBLP:conf/lics/HofmannS94}) to easily
dismiss this case. A more conservative approach would be to demand that equality
is decidable on the index type \AB{I} and to then use the classic Hedberg
construction~(\citeyear{DBLP:journals/jfp/Hedberg98}) to recover uniqueness of
identity proofs for \AB{I}.

Assuming that the constraints computed by {(\AF{Constraints} \AB{d})} are
satisfied, we define the decision procedure for equality of terms together
with its equivalent for bodies. The function \AF{eq\textasciicircum{}Tm}
is a straightforward case analysis dismissing trivially impossible cases
where terms have distinct head constructors (\AIC{`var} vs. \AIC{`con})
and using either \AF{eq\textasciicircum{}Var} or \AF{eq\textasciicircum{}⟦⟧}
otherwise. The latter is defined by induction over \AB{e}. The somewhat
verbose definitions are not enlightening so we leave them out here.

\begin{figure}[h]
\ExecuteMetaData[generic-syntax.agda/Generic/Equality.tex]{eqTmType}
\caption{Type of Decidable Equality for Terms and Bodies}\label{fig:eqtype}
\end{figure}

We now have an informative decision procedure for equality between terms
provided that the syntax they belong to satisfies a set of constraints.
Other generic functions and decision procedures can be defined
following the same approach: implement a similar function for variables
first, compute a set of constraints, and demonstrate that they are
sufficient to handle any input term.

