\section{Fusion Lemma}\label{section:fusion}

There are abundant result that can be reformulated as the ability to fuse
two traversals obtained as \AR{Semantics} into one.
%
When claiming that \AF{Tm} is
a Functor, we have to prove that two successive renamings can be fused into
a single renaming where the \AF{Thinning}s have been composed.
%
Similarly,
demonstrating that \AF{Tm} is a relative Monad (\cite{JFR4389}) implies proving
that two consecutive substitutions can be merged into a single one whose
environment is the first one, where the second one has been applied in a
pointwise manner.
%
The \emph{Substitution Lemma} central
to most model constructions (see for instance~\cite{mitchell1991kripke}) states
that a syntactic substitution followed by the evaluation of the resulting term
into the model is equivalent to the evaluation of the original term with an
environment corresponding to the evaluated substitution.

A direct application of these results is our entry to the
POPLMark Reloaded challenge (\citeyear{poplmarkreloaded}). By using a
\AD{Desc}-based representation of intrinsically well typed and well scoped
terms we directly inherit not only renaming and substitution but also all
four fusion lemmas as corollaries
of our generic results. This allows us to remove the usual boilerplate
and go straight to the point.
As all of these statements have precisely the same structure, we can
once more devise a framework which will, provided that its constraints are
satisfied, prove a generic fusion lemma.


Our fundamental lemma of \AR{Fusion} states that from a triple of related
environments (the exact meaning of ``related'' will be defined in the next
section), one gets a pair of related computations (again the meaning of
``related'' will be made formal soon):

\begin{agdasnippet}
  \ExecuteMetaData[generic-syntax.agda/Generic/Fusion.tex]{fusiontype}
\end{agdasnippet}

Fusion is more involved than simulation so we will step through each one of
the constraints individually, trying to give the reader an intuition
for why they are shaped the way they are.


\subsection{The Fusion Constraints}

The notion of fusion is defined for a triple of \AR{Semantics}; each \AB{𝓢ⁱ}
being defined for values in \AB{𝓥ⁱ} and computations in \AB{𝓒ⁱ}. The
fundamental lemma associated to such a set of constraints will state that
running \AB{𝓢ᴮ} after \AB{𝓢ᴬ} is equivalent to running \AB{𝓢ᴬᴮ} only.

The definition of fusion is parametrised by three relations: \AB{𝓔ᴿ} relates
triples of environments of values in {(\AB{Γ} \AR{─Env}) \AB{𝓥ᴬ} \AB{Δ}},
{(\AB{Δ} \AR{─Env}) \AB{𝓥ᴮ} \AB{Θ}} and {(\AB{Γ} \AR{─Env}) \AB{𝓥ᴬᴮ} \AB{Θ}}
respectively; \AB{𝓥ᴿ} relates pairs of values \AB{𝓥ᴮ} and \AB{𝓥ᴬᴮ};
and \AB{𝓒ᴿ}, our notion of equivalence for evaluation results, relates pairs
of computation in \AB{𝓒ᴮ} and \AB{𝓒ᴬᴮ}.

\ExecuteMetaData[generic-syntax.agda/Generic/Fusion.tex]{fusionrec}

The first obstacle we face is the formal definition of ``running \AB{𝓢ᴮ}
after \AB{𝓢ᴬ}'': for this statement to make sense, the result of running
\AB{𝓢ᴬ} ought to be a term. Or rather, we ought to be able to extract a
term from a \AB{𝓒ᴬ}. Hence the first constraint: the existence of a
\ARF{reifyᴬ} function, which we supply as a field of the record \AR{Fusion}.
When dealing with syntactic semantics such as renaming or substitution
this function will be the identity. Nothing prevents proofs, such as the
idempotence of NbE, which use a bona fide reification function that extracts
terms from model values.

\ExecuteMetaData[generic-syntax.agda/Generic/Fusion.tex]{reify}

Then, we have to think about what happens when going under a binder: \AB{𝓢ᴬ}
will produce a \AF{Kripke} function space where a syntactic value is required.
Provided that \AB{𝓥ᴬ} is \AR{VarLike}, we can make use of \AF{reify}
(defined in \cref{fig:kripkereify}) to get a
\AF{Scope} back which will be more amenable to running the semantics \AB{𝓢ᴮ}.
Hence the second constraint.

\ExecuteMetaData[generic-syntax.agda/Generic/Fusion.tex]{vlV}

We can combine these two functions to define the reification procedure we will
use in practice when facing Kripke function spaces: \AF{quoteᴬ} which takes such
a function and returns a term by first feeding placeholder values to the Kripke
function space and getting a \AB{𝓒ᴬ} back and then reifying it thanks to \ARF{reifyᴬ}.

\ExecuteMetaData[generic-syntax.agda/Generic/Fusion.tex]{quote}

Still thinking about going under binders: if three evaluation environments
\AB{ρᴬ} of type
{(\AB{Γ} \AR{─Env}) \AB{𝓥ᴬ} \AB{Δ}}, \AB{ρᴮ}
in {(\AB{Δ} \AR{─Env}) \AB{𝓥ᴮ} \AB{Θ}},
and \AB{ρᴬᴮ} in {(\AB{Γ} \AR{─Env}) \AB{𝓥ᴬᴮ} \AB{Θ}}
are related by \AB{𝓔ᴿ} and we are
given a thinning \AB{ρ} from \AB{Θ} to \AB{Ω} then \AB{ρᴬ},
the thinned \AB{𝓥ᴮ} and the
thinned \AB{ρᴬᴮ} should still be related.

\ExecuteMetaData[generic-syntax.agda/Generic/Fusion.tex]{thV}

Remembering that \AF{\_++\textasciicircum{}Env\_} is used in the definition
of \AF{body} (cf. \cref{fig:genbody})
to combine two disjoint environments {(\AB{Γ} \AR{─Env}) \AB{𝓥} \AB{Θ}} and
{(\AB{Δ} \AR{─Env}) \AB{𝓥} \AB{Θ}} into one of type
{((\AB{Γ} \AF{++} \AB{Δ}) \AR{─Env}) \AB{𝓥} \AB{Θ})}, we mechanically need a
constraint stating that \AF{\_++\textasciicircum{}Env\_} is compatible with
\AB{𝓔ᴿ}. We demand
as an extra precondition that the values \AB{ρᴮ} and \AB{ρᴬᴮ} are extended
with are related in a pointwise manner according to \AB{𝓥ᴿ}. Lastly, for all
the types to match up, \AB{ρᴬ} has to be extended with placeholder variables
which we can do thanks to the \AR{VarLike} constraint \ARF{vl\textasciicircum{}𝓥ᴬ}.

\ExecuteMetaData[generic-syntax.agda/Generic/Fusion.tex]{appendR}

We finally arrive at the constraints focusing on the semantical counterparts
of the terms' constructors. In order to have readable types we introduce an
auxiliary definition \AF{𝓡}. Just like in \cref{sec:fusionrel}, it relates at
a given type a term and three environments by stating that sequentially evaluating
the term in the first and then the second environment on the one hand and directly
evaluating the term in the third environment on the other yields related computations.

\ExecuteMetaData[generic-syntax.agda/Generic/Fusion.tex]{crel}

As one would expect, the \ARF{varᴿ} constraint states that from related environments
the two evaluation processes described by \AF{𝓡} yield related outputs.

\ExecuteMetaData[generic-syntax.agda/Generic/Fusion.tex]{varR}

The case of the algebra follows a similar idea albeit being more complex.
It states that we should be able to prove that a \AIC{`con}-headed term's
evaluations are related according to \AF{𝓡} provided that the evaluation
of the constructor's body one way or the other yields structurally similar
results (hence the use of the ({\AF{⟦} \AB{d} \AF{⟧ᴿ}})
relation transformer defined in \cref{def:zipd}) where the relational
Kripke function space relates the semantical objects one can find in place
of the subterms.

\ExecuteMetaData[generic-syntax.agda/Generic/Fusion.tex]{algR}

\subsection{Fundamental Lemma of Fusion}

This set of constraints is enough to prove a fundamental lemma
of \AR{Fusion} stating that from a triple of related environments,
one gets a pair of related computations: the composition of \AB{𝓢ᴬ}
and \AB{𝓢ᴮ} on one hand and \AB{𝓢ᴬᴮ} on the other.

\begin{figure}[h]
\ExecuteMetaData[generic-syntax.agda/Generic/Fusion.tex]{fusiontype}
\caption{Statement of the Fundamental Lemma of Fusion\label{fig:fusiontype}}
\end{figure}

This lemma is once again proven mutually with its counterpart for \AR{Semantics}'
\AF{body}'s action on \AR{Scope}s: given related environments and a scope, the
evaluation of the recursive positions using \AB{𝓢ᴬ} followed by their reification
and their evaluation in \AB{𝓢ᴮ} should yield a piece of data \emph{structurally}
equal to the one obtained by using \AB{𝓢ᴬᴮ} instead where the values replacing
the recursive substructures are \AF{Kripkeᴿ}-related.

\begin{figure}[h]
\ExecuteMetaData[generic-syntax.agda/Generic/Fusion.tex]{bodytype}
\caption{Statement of the Fundamental Lemma of Fusion for Bodies\label{fig:fusiontype}\label{defn:Fusion}}
\end{figure}

The proofs involve two functions we have not mentioned before:
\AF{liftᴿ} maps a proof that a property holds for any recursive substructure
over the arguments of said constructor to obtain a \AF{⟦ d ⟧ᴿ} object. The proof
we obtain does not exactly match the premise in \ARF{algᴿ}; we need to
adjust it by rewriting an \AF{fmap}-fusion equality called \AF{fmap²}.

\begin{figure}[h]
\ExecuteMetaData[generic-syntax.agda/Generic/Fusion.tex]{fusioncode}
\ExecuteMetaData[generic-syntax.agda/Generic/Fusion.tex]{bodycode}
\caption{Proof of the Fundamental Lemma of Fusion}
\end{figure}

\subsection{Applications}

A direct consequence of this result is the four lemmas collectively stating
that any pair of renamings and / or substitutions can be fused together to
produce either a renaming (in the renaming-renaming interaction case) or a
substitution (in all the other cases).

One such example is the fusion of substitution followed by renaming into a
single substitution where the renaming has been applied to the environment.

\begin{figure}[h]
\ExecuteMetaData[generic-syntax.agda/Generic/Fusion/Syntactic.tex]{subren}
\caption{Generic Substitution-Renaming Fusion Principle\label{defn:SubRen-Fusion}}
\end{figure}

All four lemmas are proved in rapid succession by instantiating the \AR{Fusion}
framework four times, using the first results to discharge constraints in the
later ones. The last such result is the generic fusion result for substitution
with itself.

\begin{figure}[h]
\ExecuteMetaData[generic-syntax.agda/Generic/Fusion/Syntactic.tex]{subsub}
\caption{Generic Substitution-Substitution Fusion Principle}
\end{figure}

\subsection{Variations on these results}

Another corollary of the fundamental lemma of fusion is the observation that
Kaiser, Schäfer, and Stark (\citeyear{Kaiser-wsdebr}) make: \emph{assuming
functional extensionality}, all of our kind-and-scope safe traversals are
compatible with variable renaming.
%
We reproduced this result generically for all syntaxes (see accompanying code).
The need for functional extensionality arises in the proof when dealing with
subterms which have extra bound variables. These terms are interpreted as
Kripke functional spaces in the host language and we can only prove that they
take equal inputs to equal outputs. An intensional notion of equality will
simply not do here.
%
As a consequence, we refrain from using the generic result in practice when
an axiom-free alternative is provable.


Kaiser, Schäfer and Stark's observation naturally raises the question of whether
the same semantics are also stable under substitution.
%
Our semantics implementing printing with names is a clear counter-example: a
fresh name is generated each time we go under a binder, meaning that the same
term will be printed differently dependening on whether it is located under
one or two binders.
%
Consequently, printing $u$ and substituting the result for $t$ during the
printing of $(λx.t)(λx.λy.t)$ will lead to a different result than printing
the term $(λx.u)(λx.λy.u)$ where the substitution has already been performed.
