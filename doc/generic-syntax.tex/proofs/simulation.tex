\section{Simulation Lemma}\label{section:simulation}

A constraint mentioning all of these relation transformers appears naturally when
we want to say that a semantics can simulate another one. For instance, renaming
is simulated by substitution: we simply have to restrict ourselves to environments
mapping variables to terms which happen to be variables.
More generally, given a semantics \AB{𝓢ᴬ} with values \AB{𝓥ᴬ} and computations
\AB{𝓒ᴬ} and a semantics \AB{𝓢ᴮ} with values \AB{𝓥ᴮ} and computations \AB{𝓒ᴮ},
we want to establish the constraints under which these two semantics yield
related computations provided they were called with environments of related values.
That is to say we want to prove the following generic result:

\begin{agdasnippet}
  \ExecuteMetaData[generic-syntax.agda/Generic/Simulation.tex]{simtype}
\end{agdasnippet}

These constraints are packaged in a record type called \AR{Simulation} and
parametrised over the semantics as well as the notion of relatedness used
for values (given by a relation \AB{𝓥ᴿ}) and computations
(given by a relation \AB{𝓒ᴿ}).

\begin{agdasnippet}
  \ExecuteMetaData[generic-syntax.agda/Generic/Simulation.tex]{recsim}
\end{agdasnippet}

The two first constraints are self-explanatory: the operations
\ARF{th\textasciicircum{}𝓥} and \ARF{var} defined by each semantics
should be compatible with the notions of relatedness used for values and computations.

\begin{agdasnippet}
\addtolength{\leftskip}{\parindent}
  \ExecuteMetaData[generic-syntax.agda/Generic/Simulation.tex]{thR}
\end{agdasnippet}
\begin{agdasnippet}
\addtolength{\leftskip}{\parindent}
  \ExecuteMetaData[generic-syntax.agda/Generic/Simulation.tex]{varR}
\end{agdasnippet}

The third constraint is similarly simple: the algebras (\ARF{alg}) should take
related recursively evaluated subterms of respective types
\AF{⟦} \AB{d} \AF{⟧} (\AF{Kripke} \AB{𝓥ᴬ} \AB{𝓒ᴬ}) and
\AF{⟦} \AB{d} \AF{⟧} (\AF{Kripke} \AB{𝓥ᴮ} \AB{𝓒ᴮ}) to related computations.
The difficuly is in defining an appropriate notion of relatedness \AF{bodyᴿ}
for these recursively evaluated subterms.

\begin{agdasnippet}
\addtolength{\leftskip}{\parindent}
  \ExecuteMetaData[generic-syntax.agda/Generic/Simulation.tex]{algR}
\end{agdasnippet}

We can combine \AF{⟦\_⟧ᴿ} and \AF{Kripkeᴿ} to express the idea that two recursively
evaluated subterms are related whenever they have an equal shape (which means their
Kripke functions can be grouped in pairs) and that all the pairs of Kripke function
spaces take related inputs to related outputs.

\begin{agdasnippet}
\addtolength{\leftskip}{\parindent}
  \ExecuteMetaData[generic-syntax.agda/Generic/Simulation.tex]{bodyR}
\end{agdasnippet}

The fundamental lemma of simulations is a generic theorem showing that for
each pair of \semrec{} respecting the \AR{Simulation} constraint, we
get related computations given environments of related input values.
In Figure~\ref{defn:Simulation}, this theorem is once more mutually
proven with a statement about \AF{Scope}s,
and \AD{Size}s play a crucial role in ensuring that the function is indeed total.

\begin{figure}[h]
  \ExecuteMetaData[generic-syntax.agda/Generic/Simulation.tex]{simbody}
\caption{Fundamental Lemma of \AR{Simulation}s\label{defn:Simulation}}
\end{figure}

Instantiating this generic simulation lemma, we can for instance prove
that renaming is a special case of substitution, or that renaming and
substitution are extensional i.e. that given environments equal in
a pointwise manner they produce syntactically equal terms. Of course these
results are not new but having them generically over all syntaxes with
binding is convenient. We experienced this first hand when tackling the
POPLMark Reloaded challenge~(\citeyear{poplmarkreloaded}) where
\AF{rensub} (defined in Figure~\ref{defn:RenSub}) was actually needed.

\begin{figure}[h]
\ExecuteMetaData[generic-syntax.agda/Generic/Simulation/Syntactic.tex]{rensub}
\ExecuteMetaData[generic-syntax.agda/Generic/Simulation/Syntactic.tex]{rensubfun}
\caption{Renaming as a Substitution via Simulation\label{defn:RenSub}}
\end{figure}

When studying specific languages, new opportunities to deploy the
fundamental lemma of simulations arise. Our solution to the POPLMark
Reloaded challenge for instance describes the fact that
{(\AF{sub} \AB{$\rho$} \AB{t})}
reduces to {(\AF{sub} \AB{$\rho$'} \AB{t})} whenever for all \AB{v},
\AB{$\rho$}(\AB{v}) reduces to \AB{$\rho$'}(\AB{v}) as a \AR{Simulation}.
The main theorem (strong normalisation of STLC via a logical relation)
is itself an instance of (the unary version of) the simulation lemma.

The \AR{Simulation} proof framework is the simplest example of the abstract
proof frameworks introduced in ACMM~(\citeyear{allais2017type}). We also
explain how a similar framework can be defined
for fusion lemmas and deploy it for the renaming-substitution interactions
but also their respective interactions with normalisation by evaluation.
Now that we are familiarised with the techniques at hand, we can tackle
this more complex example for all syntaxes definable in our framework.
