\chapter{Conclusion}

\section{Summary}

In the first half of this thesis, we have expanded on the work published in
\cite{allais2017type}. Starting from McBride's Kit (\citeyear{mcbride2005type})
making explicit the common structure of renaming and substitution, we observed
that normalisation by evaluation had a similar shape. This led us to
defining a notion of type-and-scope preserving \AR{Semantics} where, crucially,
$\lambda$-abstraction is interpreted as a \AF{Kripke} function space. This
pattern was general enough to encompass not only renaming, substitution and
normalisation by evaluation but also printing with names, continuation passing
style transformations and as we have seen later on even let-inlining, typechecking
and elaboration to a typed core language.

Once this shared structure was highlighted, we took advantage of it and designed
proof frameworks to prove simulation lemmas and fusion principles for the traversals
defined as instances of \AR{Semantics}. These allowed us to prove, among other
things, that syntactic traversals are extensional, that multiple renamings and
substitutions can be fused in a single pass and that the substitution lemma holds
for NBE's evaluation. Almost systematically, previous results where used to discharge
the goals arising in the later proofs.

In the second half, we have built on the work published in \cite{generic-syntax}.
By extending Chapman, Dagand, McBride, and Morris' universe of datatype descriptions
(\citeyear{Chapman:2010:GAL:1863543.1863547}) to support a notion of binding,
we have given a generic presentation of syntaxes with binding. We then defined a
generic notion of type-and-scope preserving \AR{Semantics} for these syntaxes with
binding. It captures a large class of scope-and-type safe generic programs: from
renaming and substitution, to normalisation by evaluation, the desugaring of new
constructors added by a language transformer, printing with names or typechecking.

We have seen how to construct generic proofs about these generic programs. We
first introduced a simulation relation showing what it means for two semantics
to yield related outputs whenever they are fed related inputs. We then built on
our experience to tackle a more involved case: identifying a set of constraints
guaranteeing that two semantics run consecutively can be subsumed by a single pass
of a third one.

We have put all of these results into practice by using them to solve the
POPLMark Reloaded challenge which consists in formalising strong
normalisation for the simply typed $\lambda$-calculus via a logical-relation
argument. This also gave us the opportunity to try our framework on larger
languages by tackling the challenge's extensions to sum types and G\"{o}del's
System T. Compared to the Coq solution contributed by our co-authors, we could
not rely on tactics and had to write all proof terms by hand. However the
expressiveness of dependently-typed pattern-matching, the power of size-based
termination checking and the consequent library we could rely on thanks to the
work presented in this thesis meant that our proofs were just as short as
the tactics-based ones.

\section{Related Work}


\subsection{Variable Binding} The representation of variable binding
in formal systems has been a hot topic for decades. Part of the purpose
of the first POPLMark challenge~(\citeyear{poplmark}) was to explore and
compare various methods.

Having based our work on a de Bruijn encoding of variables, and thus a
canonical treatment of \(\alpha\)-equivalence classes, our work has no
direct comparison with permutation-based treatments such as those of
Pitts' and Gabbay's nominal syntax~(\citeyear{gabbay:newaas-jv}).

Our generic universe of syntax is based on
scoped and typed de Bruijn indices~(\citeyear{de1972lambda}) but it is not
a necessity. It is for instance possible to give an interpretation
of \AD{Desc}riptions corresponding to Chlipala's Parametric Higher-Order
Abstract Syntax~(\citeyear{DBLP:conf/icfp/Chlipala08}) and we would be interested
to see what the appropriate notion of \AD{Semantics} is for this representation.

\subsection{Alternative Binding Structures} The binding structure we
present here is based on a flat, lexical scoping strategy. There are
other strategies and it would be interesting to see whether
our approach could be reused in these cases.

\subsubsection{Groups of binders}

Weirich, Yorgey, and Sheard's work~(\citeyear{DBLP:conf/icfp/WeirichYS11})
encompasses a large array of patterns (n-ary, nested, recursive, and telescopic)
and can inform our design. They do not enforce scoping invariants internally
which forces them to introduce separate constructors for a simple binder, a
recursive one, or a telescopic pattern. They recover guarantees by giving
their syntaxes a nominal semantics thus bolting down the precise meaning of
each combinator and then proving that users may only generate well formed
terms. We discuss below these various kind of binders and how they relate to
our framework.

\paragraph{N-ary binders} bring multiple variables into scope in one fell
swoop, for instance an n-ary λ-abstraction (λ x₁, ⋯, xₙ → b).
Our setting readily supports n-ary binders as
the \AIC{`X} constructor indeed takes a list of sorts that are bound in
the subterm.

\paragraph{Nested binders} bring multiple variables into scope where
latter binders may have an associated subterm that can refer to earlier
binders, for instance a let-binding
operator (let x₁ = e₁, ⋯, xₙ = eₙ in b) where each eᵢ can refer to any
of the xⱼ for $j < i$.
%
We can easily model these by using the \AIC{`σ} constructor to store a list
of sorts and then compute the rest of the description
by induction over it.

\paragraph{Recursive binders}
bring multiple variables into scope where each binder may have an associated
subterm that can refer to any of the binders, for instance a mutually recursive
let-binding operator  (let x₁ = e₁, ⋯, xₙ = eₙ in b) where each eᵢ can refer to
any of the xⱼ.
%
These binders are not special either: syntactically they are simply nested
n-ary binders, and their recursive nature only becomes apparent in the
user-provided interpretation of the constructors.

\paragraph{Telescopes}
Last but not least, telescopes bring multiple variables into scope where
the \emph{sort} of latter variables can depend on earlier ones, for instance
dependent lambda abstractions.
%
These are much harder to handle because they equip contexts with their
own internal binding structure. Chapman, Kireev, Nester, and Wadler's work on
an intrinsic formulation of System F~(\citeyear{DBLP:conf/mpc/ChapmanKNW19})
offers a worked out example of such a system. Extracting a general framework
from their specific approach would be a really valuable contribution but we
expect it to be quite challenging.

\subsubsection{Binding in imperative languages}

Bach Poulsen, Rouvoet, Tolmach, Krebbers and Visser~(\citeyear{BachPoulsen})
introduce notions of scope graphs and frames to scale the techniques typical
of well scoped and typed deep embeddings to imperative languages.
They showcase the core ideas of their work using STLC extended with references
and then demonstrate that they can already handle a large subset of Middleweight
Java.

We have demonstrated that our framework could be used to define effectful
semantics by choosing an appropriate monad
stack~(\cite{DBLP:journals/iandc/Moggi91}).
This suggests we should be able to model STLC+Ref. It is however clear that
the scoping structures handled by scope graphs and frames are, in their full
generality, out of reach for our framework. In contrast, our work shines by
its generality: we define an entire universe of syntaxes and provide users
with traversals and lemmas implemented \emph{once and for all}.

\subsubsection{Pattern-matching binders}

Many other opportunities to enrich the notion of binder in our library are
highlighted by Cheney~(\citeyear{DBLP:conf/icfp/Cheney05a}).
%
Our framework has nothing to say about global scoping or namespaces but
could be extended to handle pattern bindings.

Currently, our terms are always relative monads with respect to the built-in
\AD{Var} functor of scoped de Bruijn variables. Its type is
{(\AB{I} → \AD{List} \AB{I} → \AD{Set})} which forces term sorts and
scope sorts to be identical.
%
By decoupling the sorts of terms and context entries, and letting users
define their own notion of variable, we could handle pattern binders by
declaring for instance that each pattern binder is pushing a list of
term sorts in the context (the variables bound by the pattern) and
defining a type family of `nested' de Bruijn indices of type
{(\AB{I} → \AD{List} (\AD{List} \AB{I}) → \AD{Set})} that selects a
pattern binder first, and then a specific variable bound by that pattern.

We do not expect such an extension to represent a lot of work.

\subsubsection{Substructural logics}

The ability to constrain the ways in which variables may be duplicated, used,
or discarded is a natural extension of our work.
%
Wood and Atkey already offer a very convincing generalisation of our framework
describing substructural type systems~(\citeyear{DBLP:conf/esop/WoodA22}).

In their presentation, substitutions have an associated matrix describing its
action on the quantities associated to the variables in scope.

%% We have demonstrated how to write g eneric programs over the potentially
%% cyclic structures of Ghani, Hamana, Uustalu and Vene~\citeyear{ghani2006representing}.
%% Further work by Hamana~\citeyear{Hamana2009} yielded a different presentation
%% of cyclic structures which preserves sharing: pointers can not only refer
%% to nodes above them but also across from them in the cyclic tree. Capturing
%% this class of inductive types as a set of syntaxes with binding and writing
%% generic programs over them is still
%% an open problem.

\subsection{Semantics of Syntaxes with Binding} An early foundational study
of a general \emph{semantic} framework for signatures with binding, algebras
for such signatures, and initiality of the term algebra, giving rise to a
categorical `program' for substitution and proofs of its properties, was given
by Fiore, Plotkin and Turi~(\citeyear{FiorePlotkinTuri99}).%
They worked in the category of presheaves
over renamings, (a skeleton of) the category of finite sets. The presheaf
condition corresponds to our notion of being \AF{Thinnable}. Exhibiting
algebras based on both de Bruijn \emph{level} and \emph{index} encodings,
their approach isolates the usual (abstract) arithmetic required of such
encodings.

By contrast, we are working in an \emph{implemented} type theory where the
encoding can be understood as its own foundation without appeal to an external
mathematical semantics. We are able to go further in developing machine-checked
such implementations and proofs, themselves generic with respect to an abstract syntax
\AD{Desc} of syntaxes-with-binding. Moreover, the usual source of implementation
anxiety, namely concrete arithmetic on de Bruijn indices, has been successfully
encapsulated via the \AF{□} coalgebra structure. It is perhaps noteworthy that
our type-theoretic constructions, by contrast with their categorical ones,
appear to make fewer commitments as to functoriality, thinnability, etc. in our
specification of semantics, with such properties typically being \emph{provable}
as a further instance of our framework.

\subsection{Meta-Theory Automation via Tactics and Code Generation} The
tediousness of repeatedly
proving similar statements has unsurprisingly led to various attempts at
automating the pain away via either code generation or the definition of
tactics. These solutions can be seen as untrusted oracles driving the
interactive theorem prover.

Polonowski's DBGen~(\citeyear{polonowski:db}) takes as input a raw syntax with
comments annotating binding sites. It generates a module defining lifting,
substitution as well as a raw syntax using names and a validation function
transforming named terms into de Bruijn ones; we refrain from calling it a
scopechecker as terms are not statically proven to be well scoped.

Kaiser, Schäfer, and Stark~(\citeyear{Kaiser-wsdebr}) build on our previous paper
to draft possible theoretical foundations for Autosubst, a so-far untrusted
set of tactics. The paper is based on a specific syntax: well scoped call-by-value
System F. In contrast, our effort has been here to carve out
a precise universe of syntaxes with binding and give a systematic account
of these syntaxes' semantics and proofs.

Keuchel, Weirich, and Schrijvers' Needle~(\citeyear{needleandknot}) is a code
generator written in Haskell producing syntax-specific Coq modules
implementing common traversals and lemmas about them.

\subsection{Universes of Syntaxes with Binding} Keeping in mind Altenkirch
and McBride's observation that generic programming is everyday programming
in dependently-typed languages~(\citeyear{DBLP:conf/ifip2-1/AltenkirchM02}),
we can naturally
expect generic, provably sound, treatments of these notions in tools such as
Agda or Coq.

Keuchel~(\citeyear{Keuchel:Thesis:2011}) together with
Jeuring~(\citeyear{DBLP:conf/icfp/KeuchelJ12})
define a universe of syntaxes with binding with a rich notion of binding
patterns closed under products but also sums as long as the disjoint
patterns bind the same variables.
%
They give their universe two distinct semantics: a first one based on well
scoped de Bruijn indices and a second one based on Parametric Higher-Order
Abstract Syntax (PHOAS)~(\cite{DBLP:conf/icfp/Chlipala08}) together with
a generic conversion function from the de Bruijn syntax to the PHOAS one.
%
Following McBride~(\citeyear{mcbride2005type}), they implement both renaming
and substitution in one fell swoop. They leave other
opportunities for generic programming and proving to future work.

Keuchel, Weirich, and Schrijvers' Knot~(\citeyear{needleandknot}) implements
as a set of generic programs the traversals and lemmas generated in specialised
forms by their Needle program. They see Needle as a pragmatic choice: working
directly with the free monadic terms over finitary containers would be too
cumbersome. In
our experience solving the POPLMark Reloaded challenge, Agda's pattern
synonyms make working with an encoded definition almost
seamless.

The GMeta generic framework~(\citeyear{gmeta}) provides a universe of syntaxes
and offers various binding conventions (locally nameless~\cite{Chargueraud2012}
or de Bruijn indices).
%
It also generically implements common traversals (e.g. computing
the sets of free variables,
% measuring the size of a term,
shifting
de Bruijn indices or substituting terms for parameters) as well as common
predicates (e.g. being a closed term) and provides generic lemmas proving that
they are well behaved. It does not offer a generic framework
for defining new well scoped-and-typed semantics and proving their properties.

Érdi~(\citeyear{gergodraft}) defines a universe inspired by a first draft of
this paper and gives three different interpretations (raw, scoped and
typed syntax) related via erasure.
%
He provides scope- and type- preserving renaming and
substitution as well as various generic proofs that they are well behaved but
offers neither a generic notion of semantics, nor generic proof frameworks.

Copello~(\citeyear{copello2017}) works with \emph{named} binders and
defines nominal techniques (e.g. name swapping) and ultimately
$\alpha$-equivalence over a universe of regular trees with binders
inspired by Morris'~(\citeyear{morris-regulartt}).

\subsection{Fusion of Successive Traversals}

The careful characterisation of the successive recursive traversals which can be
fused together into a single pass in a semantics-preserving way is not new. This
transformation is a much needed optimisation principle in a high-level
functional language.

Through the careful study of the recursion operator associated to each
strictly positive datatype,
Malcolm~(\citeyear{DBLP:journals/scp/Malcolm90}) defined optimising
fusion proof principles.
%
Other optimisations such as deforestation~(\cite{DBLP:journals/tcs/Wadler90})
or the compilation of a recursive definition into an equivalent abstract
machine-based tail-recursive program~(\cite{DBLP:conf/icfp/CortinasS18})
rely on similar generic proofs that these transformations are
meaning-preserving.


\section{Limitations of the Current Framework}


Although quite versatile already our current framework has some limitations
which suggest avenues for future work. We list these limitations from easiest
to hardest to resolve. Remember that each modification to the universe of
syntaxes needs to be given an appropriate semantics.

\paragraph{Inefficient Environment Weakening} Our fundamental lemma of
semantics systematically traverses its environment of values whenever it
needs to push it under a binder. This means that if we need to push an
environment under $n$ successive binders, we will thin each of the values
it carries $n$ times.
%
Preliminary work demonstrates that it is possible to avoid these repeated
traversals. The key idea is to use an inductive notion of environments in
which the thin-and-extend operation used to go under a binder is reified
as a constructor. At variable-lookup time, we merge the accumulated
thinnings and actually apply them to the original value.
%
Intuitively, going under a binder amounts to pushing a frame consisting of
a thinning and an environment of values for the newly bound variables onto
the evaluation stack (which is essentially a type-aligned list of frames).


\paragraph{Unrestricted Variables} Our current notion of variable can be used
to form a term of any kind. We remarked in Sections~\ref{section:typechecking}
and \ref{section:elaboration} that in some languages we want to restrict this
ability to one kind in particular. In that case, we wanted users to only be able
to use variables at the kind \AIC{Infer} of our bidirectional language. For the
time being we made do by restricting the environment values our \AR{Semantics}
use to a subset of the kinds: terms with variables of the wrong kind will not be
given a semantics.

\paragraph{Flat Binding Structure} Our current setup limits us to flat binding
structures: variable and binder share the same kinds. This prevents us from
representing languages with binding patterns, for instance pattern-matching
let-binders which can have arbitrarily nested patterns taking pairs apart.

\paragraph{Closure under Derivation} One-hole contexts play a major role in the
theory of programming languages. Just like the one-hole context of a datatype is
a datatype~(\cite{DBLP:journals/fuin/AbbottAMG05}), we would like our universe
to be closed under derivatives so that the formalisation of e.g. evaluation
contexts could benefit directly from the existing machinery.

\paragraph{Closure under Closures} Jander's work on formalising and certifying
continuation passing style transformations~(\cite{Jander:Thesis:2019})
highlighted the need for a notion of syntaxes with closures. Recalling
that our notion of Semantics is always compatible with precomposition
with a renaming~(\cite{Kaiser-wsdebr}) but not necessarily
precomposition with a substitution (printing is for instance not
stable under substitution), accommodating terms with suspended
substitutions is a real challenge. Preliminary experiments show that a
drastic modification of the type of the fundamental lemma of
\AR{Semantics} makes dealing with such closures possible. Whether the
resulting traversal has good properties that can be proven generically
is still an open problem.

\section{Future Work}

The diverse influences leading to this work suggest many opportunities
for future research.

\subsection{Total Compilers with Typed Intermediate Representations}

Some of our core examples of generic semantics correspond to compiler passes:
desugaring, elaboration to a typed core, type-directed partial evaluation,
or CPS transformation. This raises the question of how many such common
compilation passes can be implemented generically.

Other semantics such as printing with names or a generic notion of raw terms
together with a generic scope checker (not shown here but available in
\cite{repo2018})
are infrastructure a compiler would have to rely on. Together with our library
of \emph{total} parser combinators (\cite{allais2018agdarsec}) and our
declarative
syntax for defining hierarchical command line interfaces (\cite{allaisagdargs}),
this suggests we are close to being able to define an entire (toy) compiler with
strong invariants enforced in the successive intermediate representations.

To tackle modern languages with support for implicit arguments, a total account
of (higher-order) unification is needed. It would ideally be defined generically
over our notion of syntax thus allowing us to progressively extend our language
as we see fit without having to revisit that part of the compiler.

\subsection{Generic Meta-Theory}

If we cannot use our descriptions to define an intrinsically-typed syntax for
a dependently-typed theory, we can however give a well-scoped version and then
define typing judgments. When doing so we have a lot of freedom in how we
structure them and a natural question to ask is whether we can identify a
process which will always give us judgments with good properties e.g.
stability under substitution or decidable typechecking.

We can in fact guarantee such results by carefully managing the flow of
information in the judgments and imposing that no information ever comes
out of nowhere. This calls for the definition of a universe of typing
judgments and the careful analysis of its properties.

\subsection{A Theory of Ornaments for Syntaxes}

The research program introduced by McBride's unpublished paper introducing
ornaments for inductive families (\citeyear{mcbride2010ornamental}) allows users
to make explicit the fact that some inductive families are refinements of
others.
Once their shared structure is known, the machine can assist the user in
transporting an existing codebase from the weakly-typed version of the datatype
to its strongly typed variant (\cite{dagand_mcbride_2014}). These ideas can be
useful even in ML-style settings (\cite{Williams:2014:OP:2633628.2633631}).

Working out a similar notion of ornaments for syntaxes would yield similar
benefits but for manipulating binding-aware families. This is particularly
evident when considering the elaboration semantics which given a scoped term
produces a scoped-and-typed term by type-checking or type-inference.

If the proofs we developed in this thesis would still be out of reach for
ML-style languages, the programming part can be replicated using the usual
Generalised Algebraic Data Types (GADTs) based encodings
(\cite{danvytagless,lindley2014hasochism}) and could thus still benefit from
such ornaments being made first order.

\subsection{Derivatives of Syntaxes}

Our work on the POPLMark Reloaded challenge highlighted a need for a generic
definition of evaluation contexts (i.e. terms with holes), congruence closures
and the systematic study of their properties. This would build on the work of
Huet (\citeyear{huet_1997}) and Abbott, Altenkirch, McBride and Ghani
(\citeyear{abbott2005data}) and would allow us to revisit previous work
based on concrete instances of our \AF{Semantics}-based approach to
formalising syntaxes with binding such as McLaughlin, McKinna and
Stark (\citeyear{craig2018triangle}).
