\chapter{Conclusion}

\section{Summary}

In the first half of this thesis, we have expanded on the work published in
\cite{allais2017type}. Starting from McBride's Kit (\citeyear{mcbride2005type})
making explicit the common structure of renaming and substitution, we observed
that normalisation by evaluation had a similar shape. This led us to
defining a notion of type-and-scope preserving \AR{Semantics} where, crucially,
$\lambda$-abstraction is interpreted as a \AF{Kripke} function space. This
pattern was general enough to encompass not only renaming, substitution and
normalisation by evaluation but also printing with names, continuation passing
style transformations and as we have seen later on even let-inlining, typechecking
and elaboration to a typed core language.

Once this shared structure was highlighted, we took advantage of it and designed
proof frameworks to prove simulation lemmas and fusion principles for the traversals
defined as instances of \AR{Semantics}. These allowed us to prove, among other
things, that syntactic traversals are extensional, that multiple renamings and
substitutions can be fused in a single pass and that the substitution lemma holds
for NBE's evaluation. Almost systematically, previous results where used to discharge
the goals arising in the later proofs.

In the second half, we have built on the work published in \cite{generic-syntax}.
By extending Chapman, Dagand, McBride, and Morris' universe of datatype descriptions
(\citeyear{Chapman:2010:GAL:1863543.1863547}) to support a notion of binding,
we have given a generic presentation of syntaxes with binding. We then defined a
generic notion of type-and-scope preserving \AR{Semantics} for these syntaxes with
binding. It captures a large class of scope-and-type safe generic programs: from
renaming and substitution, to normalisation by evaluation, the desugaring of new
constructors added by a language transformer, printing with names or typechecking.

We have seen how to construct generic proofs about these generic programs. We
first introduced a simulation relation showing what it means for two semantics
to yield related outputs whenever they are fed related inputs. We then built on
our experience to tackle a more involved case: identifying a set of constraints
guaranteeing that two semantics run consecutively can be subsumed by a single pass
of a third one.

We have put all of these results into practice by using them to solve the (to be
published) POPLMark Reloaded challenge which consists in formalising strong
normalisation for the simply typed $\lambda$-calculus via a logical-relation
argument. This also gave us the opportunity to try our framework on larger
languages by tackling the challenge's extensions to sum types and G\"{o}del's
System T. Compared to the Coq solution contributed by our co-authors, we could
not rely on tactics and had to write all proof terms by hand. However the
expressiveness of dependently-typed pattern-matching, the power of size-based
termination checking and the consequent library we could rely on thanks to the
work presented in this thesis meant that our proofs were just as short as
the tactics-based ones.

\section{Further Work}

The diverse influences leading to this work suggest many opportunities
for future research.

\subsection{Total Compilers with Typed Intermediate Representations}

Some of our core examples of generic semantics correspond to compiler passes:
desugaring, elaboration to a typed core, type-directed partial evaluation,
or CPS transformation. This raises the question of how many such common
compilation passes can be implemented generically.

Other semantics such as printing with names or a generic notion of raw terms
together with a generic scope checker (not shown here but available in \cite{repo2018})
are infrastructure a compiler would have to rely on. Together with our library of
\emph{total} parser combinators (\cite{allais2018agdarsec}) and our declarative
syntax for defining hierarchical command line interfaces (\cite{allaisagdargs}),
this suggests we are close to being able to define an entire (toy) compiler with
strong invariants enforced in the successive intermediate representations.

To tackle modern languages with support for implicit arguments, a total account of
(higher-order) unification is needed. It would ideally be defined generically over
our notion of syntax thus allowing us to progressively extend our language as we
see fit without having to revisit that part of the compiler.

\subsection{Generic Meta-Theory}

If we cannot use our descriptions to define an intrinsically-typed syntax for
a dependently-typed theory, we can however give a well-scoped version and then
define typing judgments. When doing so we have a lot of freedom in how we
structure them and a natural question to ask is whether we can identify a process
which will always give us judgments with good properties e.g. stability under
substitution or decidable typechecking.

We can in fact guarantee such results by carefully managing the flow of information
in the judgments and imposing that no information ever comes out of nowhere. This
calls for the definition of a universe of typing judgments and the careful analysis
of its properties.

\subsection{A Theory of Ornaments for Syntaxes}

The reseach programme introduced by McBride's unpublished paper introducing
ornaments for inductive families (\citeyear{mcbride2010ornamental}) allows users
to make explicit the fact that some inductive families are refinements of others.
Once their shared structure is known, the machine can assist the user in
transporting an existing codebase from the weakly-typed version of the datatype
to its strongly typed variant (\cite{dagand_mcbride_2014}). These ideas can be
useful even in ML-style settings (\cite{Williams:2014:OP:2633628.2633631}).

Working out a similar notion of ornaments for syntaxes would yield similar
benefits but for manipulating binding-aware families. This is particularly
evident when considering the elaboration semantics which given a scoped term
produces a scoped-and-typed term by type-checking or type-inference.

If the proofs we developped in this thesis would still be out of reach for
ML-style languages, the programming part can be replicated using the usual
Generalised Algebraic Data Types (GADTs) based encodings
(\cite{danvytagless,lindley2014hasochism}) and could thus still benefit from
such ornaments being made first order.

\subsection{Derivatives of Syntaxes}

Our work on the POPLMark Reloaded challenge highlighted a need for a generic
definition of evaluation contexts (i.e. terms with holes), congruence closures
and the systematic study of their properties. This would build on the work of
Huet (\citeyear{huet_1997}) and Abbott, Altenkirch, McBride and Ghani
(\citeyear{abbott2005data}) and would allow us to revisit previous work
based on concrete instances of our \AF{Semantics}-based approach to formalising
syntaxes with binding such as McLaughlin, McKinna and Stark (\citeyear{craig2018triangle}).
