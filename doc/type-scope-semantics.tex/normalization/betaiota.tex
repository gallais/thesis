
\section{Normalisation by Evaluation for βι}

The decision to apply the η-rule lazily can be pushed even further: one may
forgo using the ξ-rule too and simply perform weak-head normalisation. This
drives computation only when absolutely necessary, e.g.
when two terms compared for equality have matching head constructors
and one needs to inspect these constructors' arguments to conclude.

For that purpose, we introduce an inductive family describing terms in weak-head
normal forms.

\subsection{Weak-Head Normal Forms}

A weak-head normal form (respectively a weak-head neutral form) is a term which has
been evaluated just enough to reveal a head constructor (respectively to reach a
stuck elimination). There are no additional constraints on the subterms: a λ-headed
term is in weak-head normal form no matter the shape of its body. Similarly an
application composed of a variable as the function and a term as the argument is in
weak-head neutral form no matter what the argument looks like. This means in particular
that unlike with \AD{Ne} and \AD{Nf} there is no mutual dependency between the definitions
of \AD{WHNE} (defined first) and \AD{WHNF}.

\begin{figure}[h]
\ExecuteMetaData[type-scope-semantics.agda/Syntax/WeakHead.tex]{weakhead}
\caption{Weak-Head Normal and Neutral Forms\label{fig:weakhead}}
\end{figure}

Naturally, it is possible to define the thinnings
\AF{th\textasciicircum{}WHNE} and \AF{th\textasciicircum{}WHNF} as well as erasure
functions \AF{erase\textasciicircum{}WHNE} and \AF{erase\textasciicircum{}WHNF}
with codomain \AD{Term}. We omit their simple definitions here.

\subsection{Model Construction}

The model construction is much like the previous one except
that source terms are now stored in the model too. This means that
from an element of the model, one can pick either the reduced version
of the input term (i.e. a stuck term or the term's computational
content) or the original. We exploit this ability most
notably in reification where once we have obtained either a
head constructor or a head variable, no subterm needs to
be evaluated.

\begin{figure}[h]
\ExecuteMetaData[type-scope-semantics.agda/Semantics/NormalisationByEvaluation/BetaIota.tex]{model}
\caption{Model Definition for Computing Weak-Head Normal Forms\label{fig:betaiotamodel}}
\end{figure}

\AF{Thinnable} can be defined rather straightforwardly based on the template provided
in the previous section: once more all the notions used in the model definition
are \AF{Thinnable} themselves. Reflection and reification also follow the same recipe
as in the previous section.


The application and conditional branching rules are more
interesting. One important difference with respect to the previous
section is that we do not grow the spine of a stuck term using
reified versions of its arguments but rather the corresponding
\emph{source} term. Thus staying true to the idea that we only head
reduce enough to expose either a constructor or a variable and let
the other subterms untouched.

\begin{figure}[h]
\ExecuteMetaData[type-scope-semantics.agda/Semantics/NormalisationByEvaluation/BetaIota.tex]{app}
\ExecuteMetaData[type-scope-semantics.agda/Semantics/NormalisationByEvaluation/BetaIota.tex]{ifte}
\caption{Semantical Counterparts of \AIC{`app} and \AIC{`ifte}\label{fig:betaiotaappifte}}
\end{figure}

The semantical counterpart of \AIC{`lam} is also slightly trickier than before. Indeed, we
need to recover the source term the value corresponds to. Luckily we know it has to be
λ-headed and once we have introduced a fresh variable with \AIC{`lam}, we can project
out the source term of the body evaluated using this fresh variable as a placeholder
value.

\begin{figure}[h]
\ExecuteMetaData[type-scope-semantics.agda/Semantics/NormalisationByEvaluation/BetaIota.tex]{lam}
\caption{Semantical Counterparts of \AIC{`lam}\label{fig:betaiotalam}}
\end{figure}

We can finally put together all of these semantic counterparts to
obtain a \AR{Semantics} corresponding to weak-head normalisation.
We omit the now self-evident definition of \AF{norm\textasciicircum{}βι} as the
composition of evaluation and reification.
