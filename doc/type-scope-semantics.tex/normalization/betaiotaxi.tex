
\section{Normalisation by Evaluation for βιξ}

As mentioned above, actual proof systems such as Coq rely on evaluation
strategies that avoid applying η-rules: unsurprisingly, it is a rather
bad idea to η-expand proof terms which are already large when typechecking
complex developments.

In these systems, the η-rule is never deployed except when comparing a
neutral and a constructor-headed term for equality. Instead of declaring
them distinct, the algorithm does one step of η-expansion on the
neutral term and compares their subterms structurally. The conversion test
fails only when confronted with neutral terms with distinct head
variables or normal forms with different head constructors.
This leads us to using a predicate \AB{NoEta} which holds for all types
thus allowing us to embed all neutrals into normal forms.

\subsection{Model Construction}

This equational theory can be decided with what happens to be the most
straightforward model construction described in \cref{nbestuckmodel}:
in our \AF{Model} everything is either a (non expanded) stuck computation
(i.e. a neutral term \AD{Ne}) or a \AF{Value} arising from a constructor-headed
term and whose computational behaviour is described by an Agda value of the
appropriate type.
%
Values of the \AIC{`Unit} type are modelled by the Agda's unit type, values
of the \AIC{`Bool} type are Agda booleans and values of functions from \AB{σ}
to \AB{τ} are modelled by Kripke function spaces from the type of elements of
the \AF{Model} at type \AB{σ} and ones at type \AB{τ}.
%
It is important to note that the functional values have the \AF{Model} as both
domain and codomain: there is no reason to exclude the fact that either the
argument or the body may or may not be stuck.

%% It is possible to alter the model definition described earlier so that it
%% avoids unnecessary η-expansions. We proceed by enriching the traditional
%% model with extra syntactical artefacts in a manner reminiscent of Coquand
%% and Dybjer's~(\citeyear{CoqDybSK}) approach to defining an NBE procedure
% for the SK combinator calculus. Their resorting to gluing
%% terms to elements of the model was dictated by the sheer impossibility to write
%% a sensible reification procedure but, in hindsight, it provides us with a
%% powerful technique to build models internalizing alternative equational
%% theories.

\begin{figure}[h]
\ExecuteMetaData[type-scope-semantics.agda/Semantics/NormalisationByEvaluation/BetaIotaXi.tex]{model}
\caption{Model Definition for βιξ\label{nbestuckmodel}}
\end{figure}

We can observe that we have only used families constant in their scope
index, neutral forms or \AF{□}-closed families. All of these are
\AF{Thinnable} hence \AF{Value} and \AF{Model} also are. Because the
two families are defined mutually, so are the proofs that they are both
thinnable.

\begin{figure}[h]
\ExecuteMetaData[type-scope-semantics.agda/Semantics/NormalisationByEvaluation/BetaIotaXi.tex]{thmodel}
\caption{The \AF{Model} is \AF{Thinnable}}
\end{figure}

\subsection{Reify and Reflect}

During the definition of our \AR{Semantics} acting on elements of
type \AF{Model}, we are inevitably going to be faced with a situation
where we are eliminating what happens to be a stuck computation
(e.g. applying a stuck function to its argument, or branching over
a stuck boolean).
%
In this case we should return a stuck computation. By definition
that means we ought to be able to take the eliminator's semantic
arguments and turn them into syntactic objects i.e. to \emph{reify}
them. For reasons that will become obvious in the definition of
\AF{reify} in \cref{fig:bix:reifyreflect}, we will first need to
define \AF{reflect}, the function that reflects neutral terms as
model values.

By definition we can trivially embed neutral terms into the model
using the first injection into the disjoin sum type.

Reification is quite straightforward too: semantic unit value are
turned into the trivial syntactic object of type \AIC{`Unit},
semantic booleans are reified as their syntactic counterpart and
semantic function give rise to lambdas, they then get applied to
the variable thus introduced and the resulting term gets reified too.

\begin{figure}[h]
\ExecuteMetaData[type-scope-semantics.agda/Semantics/NormalisationByEvaluation/BetaIotaXi.tex]{reifyreflect}
\caption{Reflect, Reify and Interpretation for Fresh Variables}\label{fig:bix:reifyreflect}
\end{figure}

\subsection{A \AR{Semantics} Targetting our \AF{Model}}

Now that we have defined the \AF{Model} we are interested in and that
we have proved that we can both embed stuck computations into it and
obtain normal forms from it, it is time to define a \AR{Semantics}
targetting it. We will study the most striking semantic combinators
one by one and then put everything together.

Semantic application is perhaps the most interesting of the combinators
needed to define our value of type (\AR{Semantics} \AF{Model} \AF{Model}).
It follows the case distinction we mentioned earlier: in case the function
is a stuck term, we grow its spine by reifying its argument; otherwise we
have an Agda function ready to be applied. We use the \AF{extract} operation
of the \AB{□} comonad (cf. \cref{fig:boxcomonad}) to say that we are using
the function in the ambient context, not an extended one.

\begin{figure}[h]
\ExecuteMetaData[type-scope-semantics.agda/Semantics/NormalisationByEvaluation/BetaIotaXi.tex]{app}
\caption{Semantical Counterpart of \AIC{`app}}
\end{figure}

When defining the semantical counterpart of \AIC{`ifte}, we follow a similar
case distinction.
%
The value case is straightforward: depending on the boolean value we
pick either the left or the right branch which are precisely of the right
type already.
%
If the boolean evaluates to a stuck term, we follow the same strategy we
used for semantic application: we reify the two branches and assemble a
neutral term.

\begin{figure}[h]
\ExecuteMetaData[type-scope-semantics.agda/Semantics/NormalisationByEvaluation/BetaIotaXi.tex]{ifte}
\caption{Semantical Counterpart of \AIC{`ifte}}
\end{figure}

Finally, we have all the necessary components to show that evaluating
the term whilst is a perfectly valid \AR{Semantics}. As showcased earlier,
normalisation is obtained as a direct corollary of \AR{NBE} by the
composition of reification and evaluation in an environment of placeholder
values.

\ExecuteMetaData[type-scope-semantics.agda/Semantics/NormalisationByEvaluation/BetaIotaXi.tex]{norm}

Now that we have our definition of \AF{NBE} for the βιξ rules, we
can run the \AF{test} defined in \cref{fig:nbetest} and, obtaining
(λb.λu. \texttt{if~} b \texttt{~then~()~else~}u),
observe that
we have indeed reduced all of the βι redexes, even if they were
hidden under a λ-abstraction. Note however that we still have a stuck
if-then-else conditionals even though the return type is a record type
with only one inhabitant: we are not performing η-expansion so we cannot
expect this type of knowledge to be internalised!

\begin{figure}[h]
\ExecuteMetaData[type-scope-semantics.agda/Semantics/NormalisationByEvaluation/BetaIotaXi.tex]{test}
\caption{Running example: the βιξ case}\label{fig:betaiotaxitest}
\end{figure}
