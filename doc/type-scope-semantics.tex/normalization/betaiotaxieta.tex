\section{Normalisation by Evaluation for βιξη}
\label{normbye}

As we have just seen, we can leverage the host language's evaluation
machinery to write a normaliser for our own embedded language. We opted
for a normalisation procedure deciding the equational theory corresponding
to βιξ rules. We can develop more ambitious model constructions that will
extend the supported equational theory.
%
In this section we will focus on η-rules but it is possible to go even
beyond and build models that decide equational theories incorporating
for instance various functor and fusion laws. The interested reader can
direct its attention to our previous work formalising such a construction
(\citeyear{DBLP:conf/icfp/AllaisMB13}).

In order to decide βιξη rules, we are going to build the \AF{Model} by
induction on its type argument. This time we do not need to distinguish
neutral terms from terms that compute: each type may or may not have
neutral forms depending on whether it has a set of accompanying η-rules.
%
Functions (respectively inhabitants of \AIC{`Unit}) in the source language
will be represented as function spaces (respectively values of type \AR{⊤})
no matter whether they are stuck or not. Evaluating a \AIC{`Bool} may however
yield a stuck term so we cannot expect the model to give us anything more than
an open term in normal form. Note that there are advanced constructions
adding η-rules for sum types
(\cite{DBLP:conf/tlca/Ghani95,DBLP:conf/flops/AltenkirchU04,DBLP:conf/tlca/Lindley07})
but they are unwieldy and thus outside the scope of this thesis.


The model construction then follows the established pattern pioneered by
Berger~(\citeyear{berger1993program}) and formally analysed and thoroughly
explained by Catarina Coquand~(\citeyear{coquand2002formalised}). We work
by induction on the type and describe η-expanded values: all inhabitants
of (\AF{Model} \AIC{`Unit} \AB{Γ}) are equal and all elements
of (\AF{Model} (\AB{σ} \AIC{`→} \AB{τ}) \AB{Γ}) are functions in Agda.

\begin{figure}[h]
\ExecuteMetaData[type-scope-semantics.agda/Semantics/NormalisationByEvaluation/BetaIotaXiEta.tex]{model}
\caption{Model for Normalisation by Evaluation\label{fig:nbemodel}}
\end{figure}

This model is defined by induction on the type in terms either of
syntactic objects (\AD{Nf}) or using the \AF{□}-operator which is
a closure operator for thinnings. As such, it is trivial to prove
that for all type \AB{σ}, (\AF{Model} \AB{σ}) is \AF{Thinnable}.

\begin{figure}[h]
\ExecuteMetaData[type-scope-semantics.agda/Semantics/NormalisationByEvaluation/BetaIotaXiEta.tex]{thmodel}
\caption{Values in the Model are Thinnable\label{fig:thnbemodel}}
\end{figure}

Application's semantic counterpart is easy to define: given that \AB{𝓥}
and \AB{𝓒} are equal in this instance definition we can feed the argument
directly to the function, with the identity renaming. This corresponds to
\AF{extract} for the comonad \AF{□}.

\begin{figure}[h]
\ExecuteMetaData[type-scope-semantics.agda/Semantics/NormalisationByEvaluation/BetaIotaXiEta.tex]{app}
\caption{Semantic Counterpart of \AIC{`app}\label{fig:nbeapp}}
\end{figure}

Conditional branching however is more subtle: the boolean value \AIC{`ifte}
branches on may be a neutral term in which case the whole elimination form
is stuck. This forces us to define \AF{reify} and \AF{reflect} first.
These functions, also known as quote and unquote respectively, give the
interplay between neutral terms, model values and normal forms.
\AF{reflect} performs a form of semantic η-expansion: all stuck \AIC{`Unit}
terms are equated and all functions are λ-headed. It allows us to define
\AF{var0}, the semantic counterpart of (\AIC{`var} \AIC{z}).

\begin{figure}[h]
\ExecuteMetaData[type-scope-semantics.agda/Semantics/NormalisationByEvaluation/BetaIotaXiEta.tex]{reifyreflect}
\caption{Reify and Reflect\label{fig:reifyreflectnbe}}
\end{figure}

We can then give the semantics of \AIC{`ifte}: if the boolean is a value, the
appropriate branch is picked; if it is stuck then the whole expression is stuck.
It is then turned into a neutral form by reifying the two branches and then reflected
in the model.

\begin{figure}[h]
\ExecuteMetaData[type-scope-semantics.agda/Semantics/NormalisationByEvaluation/BetaIotaXiEta.tex]{ifte}
\caption{Semantic Counterpart of \AIC{`ifte}\label{fig:nbeifte}}
\end{figure}

We can then combine these components. The semantics of a λ-abstraction is
simply the identity function: the structure of the functional case in the
definition of the model matches precisely the shape expected in a
\AF{Semantics}. Because the environment carries model values, the
variable case is trivial.

\begin{figure}[h]
\ExecuteMetaData[type-scope-semantics.agda/Semantics/NormalisationByEvaluation/BetaIotaXiEta.tex]{eval}
\caption{Evaluation is a \AR{Semantics}\label{fig:evalnbe}}
\end{figure}

With these definitions in hand, we can instantiate the \AR{NBE} interface we
introduced in \cref{fig:nbeinterface}. This gives us access to a normaliser
by kick-starting the evaluation with an environment of placeholder values
obtained by reflecting the term's free variables and then reifying the
result.

\begin{figure}[h]
\ExecuteMetaData[type-scope-semantics.agda/Semantics/NormalisationByEvaluation/BetaIotaXiEta.tex]{norm}
\caption{Normalisation as Reification of an Evaluated Term\label{fig:normnbe}}
\end{figure}

We can now observe the effect of this normaliser implementing a stronger
equational theory by running our \AF{test} defined \cref{fig:nbetest} and
obtaining {(\lam{b}{\lam{u}{\uni}})}. In the previous section, the normaliser
yielded a term still containing a stuck if-then-else conditional that has
here been replaced with the only normal form of the unit type.

\begin{figure}[h]
\ExecuteMetaData[type-scope-semantics.agda/Semantics/NormalisationByEvaluation/BetaIotaXiEta.tex]{test}
\caption{Running example: the βιξη case}\label{fig:betaiotaxietatest}
\end{figure}
