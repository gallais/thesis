\chapter{The Simulation Relation}

Thanks to \AF{Semantics}, we have already saved work by not reiterating the
same traversals. Moreover, this disciplined approach to building models and
defining the associated evaluation functions can help us refactor the proofs
of some properties of these semantics.

Instead of using proof scripts as Benton et al.~(\citeyear{benton2012strongly})
do, we describe abstractly the constraints the logical relations~\cite{reynolds1983types}
defined on computations (and environment values) have to respect to ensure
that evaluating a term in related environments
produces related outputs. This gives us a generic framework to
state and prove, in one go, properties about all of these semantics.

Our first example of such a framework will stay simple on purpose.
However it is no mere bureaucracy: the
result proven here will actually be useful in the next section
when considering more complex properties.

This first example is describing the relational interpretation of the terms.
It should give the reader a good introduction to the setup before we take on
more complexity. The types involved might look a bit scarily abstract but the
idea is rather simple: we have a \AR{Simulation} between two \AR{Semantics}
when evaluating a term in related environments yields related values. The bulk
of the work is to make this intuition formal.

\section{Relations Between Scoped Families}

We start by defining what it means to be a relation between two \scoped{\AB{I}}
families \AB{T} and \AB{U}: at every type \AB{σ} and every context \AB{Γ}, we
expect to have a relation between (\AB{T}~\AB{σ}~\AB{Γ}) and (\AB{U}~\AB{σ}~\AB{Γ}).
We use a \AK{record} wrapper for two reasons. First, we define the relations we
are interested in by copattern-matching thus preventing their eager unfolding by
Agda; this makes the goals much more readable during interactive development.
Second, it is easier for Agda to recover \AB{T} and \AB{U} by unification when
they appear as explicit parameters of a record rather than as applied families
in the body of the definition.

\begin{figure}[h]
\ExecuteMetaData[shared/Data/Relation.tex]{rel}
\caption{Relation Between \scoped{\AB{I}} Families}
\end{figure}

If we have a relation for values, we can lift it in a pointwise manner to a relation
on environment of values. We call this relation transformer \AR{All}. We also define
it using a \AK{record} wrapper, for the same reasons.

\begin{figure}[h]
\ExecuteMetaData[shared/Data/Relation.tex]{all}
\caption{Relation Between Environments of Values}
\end{figure}

For virtually every combinator on environments, we have a corresponding combinator
for \AR{All}: the empty environment \AF{ε} is associated to \AF{εᴿ} the proof that
two empty environments are always related, to the environment extension  \AF{\_∙\_}
corresponds the relation on environment extension \AF{\_∙ᴿ\_} which provided takes a
proof that two environments are related and that two values are related and returns
the proof that the environments each extended with the appropriate value are both
related.

Once we have all of these definitions, we can spell out what it means to simulate
a semantics with another.

\section{Simulation Constraints}

The evidence that we have a \AR{Simulation} between two \AR{Semantics} is
packaged in a record indexed by the semantics as well as two relations.
We call \AF{Rel} the type of these relations; the first one (\AB{𝓥ᴿ})
relates values in the respective environments and the second one (\AB{𝓒ᴿ})
describes simulation for computations.




\section{Fundamental Lemma of Simulations}
