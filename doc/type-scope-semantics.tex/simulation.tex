\chapter{The Simulation Relation}

Thanks to \AF{Semantics}, we have already saved work by not reiterating the same traversals.
Moreover, this disciplined approach to building models and
defining the associated evaluation functions can help us refactor
the proofs of some properties of these semantics.

Instead of using proof scripts as Benton et al.~(\citeyear{benton2012strongly})
do, we describe abstractly the constraints the logical relations~\cite{reynolds1983types}
defined on computations (and environment values) have to respect to ensure
that evaluating a term in related environments
produces related outputs. This gives us a generic framework to
state and prove, in one go, properties about all of these semantics.

Our first example of such a framework will stay simple on purpose.
However it is no mere bureaucracy: the
result proven here will actually be useful in the next section
when considering more complex properties.

\section{}

This first example is describing the relational interpretation
of the terms. It should give the reader a good introduction to
the setup before we take on more complexity. The types
involved might look a bit scarily abstract but the idea is rather simple:
we have a \AR{Simulation} between
two \AR{Semantics} when evaluating a term in related environments yields
related values. The bulk of the work is to make this intuition formal.

The evidence that we have a \AR{Simulation} between two \AR{Semantics} is
packaged in a record indexed by the semantics as well as two relations.
We call \AF{RModel} (for \emph{R}elational \emph{Model}) the type of these
relations; the first one (\AB{𝓥\textasciicircum{}R}) relates values in the respective environments
and the second one (\AB{𝓒\textasciicircum{}R}) describes simulation for computations.
