\chapter{The Fusion Relation}

When studying the meta-theory of a calculus, one systematically needs to prove
fusion lemmas for various semantics. For instance, Benton et al.~(\citeyear{benton2012strongly})
prove six such lemmas relating renaming, substitution and a typeful semantics
embedding their calculus into Coq. This observation naturally lead us to
defining a fusion framework describing how to relate three semantics: the pair
we sequence and their sequential composition. The fundamental lemma we prove
can then be instantiated six times to derive the corresponding corollaries.

\section{Fusion Constraints}

The evidence that \AB{𝓢ᴬ}, \AB{𝓢ᴮ} and \AB{𝓢ᴬᴮ} are such that \AB{𝓢ᴬ} followed
by \AB{𝓢ᴮ} is equivalent to \AB{𝓢ᴬᴮ} (e.g. \AF{Substitution} followed by \AF{Renaming}
can be reduced to \AF{Substitution}) is packed in a record \AR{Fusion} indexed by the
three semantics but also three relations. The first one (\AB{𝓥ᴿ}) states what it means
for two environment values of \AB{𝓢ᴮ} and \AB{𝓢ᴬᴮ} respectively to be related. The
second one (\AB{𝓔ᴿ}) characterises the triples of environments (one for each one of
the semantics) which are compatible. The last one (\AB{𝓒ᴿ}) relates computations
obtained as results of running \AB{𝓢ᴮ} and \AB{𝓢ᴬᴮ} respectively.

\ExecuteMetaData[type-scope-semantics.agda/Properties/Fusion/Specification.tex]{fusion}

As before, most of the fields of this record describe what structure these relations
ceed to have. However, we start with something slightly different: given that we are
planing to run the \AR{Semantics} \AB{𝓢ᴮ} \emph{after} having run \AB{𝓢ᴬ}, we need
two components: a way to extract a term from an \AB{𝓢ᴬ} and a way to manufacture a
dummy \AB{𝓢ᴬ} value when going under a binder. Our first two fields are therefore:

\ExecuteMetaData[type-scope-semantics.agda/Properties/Fusion/Specification.tex]{reifyvar0}

Then come two constraints dealing with the relations talking about evaluation environments.
\ARF{\_∙ᴿ\_} tells us how to extend related environments: one should be able to push related
values onto the environments for \AB{𝓢ᴮ} and \AB{𝓢ᴬᴮ} whilst merely extending the one
for \AB{𝓢ᴬ} with the token value \ARF{var0ᴬ}.

\ARF{th\textasciicircum{}𝓔ᴿ} guarantees that it is always possible to thin the environments
for \AB{𝓢ᴮ} and \AB{𝓢ᴬᴮ} in a \AB{𝓔ᴿ} preserving manner.

\ExecuteMetaData[type-scope-semantics.agda/Properties/Fusion/Specification.tex]{thV}

Then we have the relational counterpart of the various term constructors. We can once
more introduce an extra definition \AF{𝓡} which will make the type of the combinators
defined later on clearer. \AF{𝓡} relates at a given type a term and three environments
by stating that the computation one gets by sequentially evaluating the term in the first
and then the second environment is related to the one obtained by directly evaluating
the term in the third environment. Note the use of \ARF{reifyᴬ} to recover a \AD{Term}
from a computation in \AB{𝓒ᴬ} before using the second evaluation function \AF{evalᴮ}.

\ExecuteMetaData[type-scope-semantics.agda/Properties/Fusion/Specification.tex]{crel}

As with the previous section, only a handful of these combinators are out
of the ordinary. We will start with the \AIC{`var} case. It states that
fusion indeed happens when evaluating a variable using related environments.

\ExecuteMetaData[type-scope-semantics.agda/Properties/Fusion/Specification.tex]{var}

Just like for the simulation relation, the relational counterpart of value constructors
in the language state that provided that the evaluation environment are related,
we expect the computations to be related too.

\ExecuteMetaData[type-scope-semantics.agda/Properties/Fusion/Specification.tex]{base}

Similarly, we have purely structural constraints for term constructs which have purely
structural semantical counterparts. For \AIC{`app} and \AIC{`ifte}, provided that the
evaluation environments are related and that the evaluation of the subterms in each
environment respectively are related then the evaluations of the overall terms should
also yield related results.

\ExecuteMetaData[type-scope-semantics.agda/Properties/Fusion/Specification.tex]{struct}

Lastly, the \AIC{`lam}-case puts some rather strong restrictions on the way the
$λ$-abstraction's body may be used by \AB{𝓢ᴬ}: we assume it is evaluated in an
environment thinned by one variable) and extended using \ARF{var0ᴬ}. But it is
quite natural to have these restrictions: given that \ARF{reifyᴬ} quotes the
result back, we are expecting this type of evaluation in an extended context
(i.e. under one lambda). And it turns out that this is indeed enough for all of
our examples. The evaluation environments used by the semantics \AB{𝓢ᴮ} and
\AB{𝓢ᴬᴮ} on the other hand can be arbitrarily thinned before being extended with
related values to be substituted for the variable bound by the \AIC{`lam}.

\ExecuteMetaData[type-scope-semantics.agda/Properties/Fusion/Specification.tex]{lam}

\section{Fundamental Lemma of Fusions}

As with simulation, we measure the utility of this framework by the way we can
prove its fundamental lemma and then obtain useful corollaries. Once again,
having carefully identified what the constraints should be, proving the fundamental
lemma is not a problem.

\ExecuteMetaData[type-scope-semantics.agda/Properties/Fusion/Specification.tex]{fundamental}

\section{The Special Case of Syntactic Semantics}

The translation from \AR{Syntactic} to \AR{Semantics} uses a lot of constructors
as their own semantic counterpart, it is hence possible to generate evidence of
\AR{Syntactic} triplets being fusable with much fewer assumptions. We isolate
them and prove the result generically to avoid repetition. A \AR{SynFusion}
record packs the evidence for \AR{Syntactic} semantics \AB{Synᴬ}, \AB{Synᴮ} and
\AB{Synᴬᴮ}. It is indexed by these three \AR{Syntactic}s as well as two relations
corresponding to the \AB{𝓥ᴿ} and \AB{𝓔ᴿ} ones of the \AR{Fusion} framework;
\AB{𝓒ᴿ} will always be \AF{Eqᴿ} as we are talking about terms. It contains the
same \ARF{∙ᴿ}, \ARF{th\textasciicircum{}𝓥ᴿ} and \ARF{varᴿ} fields as a \AR{Fusion}
as well as a fourth one (\ARF{var0ᴿ}) saying that \AB{Synᴮ} and \AB{Synᴬᴮ}'s
respective \ARF{var0}s are producing related values.

\ExecuteMetaData[type-scope-semantics.agda/Properties/Fusion/Syntactic/Specification.tex]{synfusion}
\ExecuteMetaData[type-scope-semantics.agda/Properties/Fusion/Syntactic/Specification.tex]{crel}
\ExecuteMetaData[type-scope-semantics.agda/Properties/Fusion/Syntactic/Specification.tex]{thV}
\ExecuteMetaData[type-scope-semantics.agda/Properties/Fusion/Syntactic/Specification.tex]{var}
\ExecuteMetaData[type-scope-semantics.agda/Properties/Fusion/Syntactic/Specification.tex]{zro}


Given a \AR{SynFusion} relating three \AR{Syntactic} semantics, we get a
\AR{Fusion} relating the corresponding \AR{Semantics} where \AB{𝓒ᴿ} is the
propositional equality. The proof relies on the way the translation from
\AR{Syntactic} to \AR{Semantics} is formulated in \cref{syntactic}.

\ExecuteMetaData[type-scope-semantics.agda/Properties/Fusion/Syntactic/Specification.tex]{fundamental}

\section{Interactions of Renaming and Substitution}

\todo{4 fusion lemmas}


These four lemmas are usually proven in painful separation. Here we discharged
them by rapid successive instantiation of our framework, using the earlier
results to satisfy the later constraints. As we are going to see in the following
section, we are not limited to \AR{Syntactic} statements.

\section{Examples of Fusable Semantics}

The most simple example of fusable \AR{Semantics} involving a non \AR{Syntactic}
one is probably the proof that \AR{Renaming} followed by normalization by evaluation's
\AR{Eval} is equivalent to \AR{Eval} with an adjusted environment.

\todo{insert}

Then, we use the framework to prove that \AR{Eval} after a \AR{Substitution}
amounts to normalising the original term where the substitution has been
evaluated first. The constraints imposed on the environments might seem
quite restrictive but they are actually similar to the Uniformity condition
described by C. Coquand~(\citeyear{coquand2002formalised}) in her detailed
account of NBE for a ST$λ$C with explicit substitution.
