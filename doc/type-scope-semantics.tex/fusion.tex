\chapter{The Fusion Relation}
\label{sec:fusionrel}

When studying the meta-theory of a calculus, one systematically needs to prove
fusion lemmas for various semantics. For instance, Benton et al.~(\citeyear{benton2012strongly})
prove six such lemmas relating renaming, substitution and a typeful semantics
embedding their calculus into Coq. This observation naturally lead us to
defining a fusion framework describing how to relate three semantics: the pair
we sequence and their sequential composition. The fundamental lemma we prove
can then be instantiated six times to derive the corresponding corollaries.

\section{Fusion Constraints}

The evidence that \AB{𝓢ᴬ}, \AB{𝓢ᴮ} and \AB{𝓢ᴬᴮ} are such that \AB{𝓢ᴬ} followed
by \AB{𝓢ᴮ} is equivalent to \AB{𝓢ᴬᴮ} (e.g. \AF{Substitution} followed by \AF{Renaming}
can be reduced to \AF{Substitution}) is packed in a record \AR{Fusion} indexed by the
three semantics but also three relations. The first one (\AB{𝓔ᴿ}) characterises the
triples of environments (one for each one of the semantics) which are compatible.
The second one (\AB{𝓥ᴿ}) states what it means for two environment values of \AB{𝓢ᴮ}
and \AB{𝓢ᴬᴮ} respectively to be related. The last one (\AB{𝓒ᴿ}) relates computations
obtained as results of running \AB{𝓢ᴮ} and \AB{𝓢ᴬᴮ} respectively.
%
As always, provided that these constraints are satisfied we should be able
to prove \AF{fusion} the fundamental lemma of \AR{Fusion} stating that given
related environments we will get related outputs. We will interleave the
definition of the record's fields together with the proof of \AF{fusion}.
Our first goal will be to define this notion of relatedness formally.

\begin{AgdaSuppressSpace}
\ExecuteMetaData[type-scope-semantics.agda/Properties/Fusion/Specification.tex]{fusion}
\ExecuteMetaData[type-scope-semantics.agda/Properties/Fusion/Specification.tex]{fundamental:type}
\end{AgdaSuppressSpace}

As before, most of the fields of this record describe what structure these
relations need to have. However, we start with something slightly different:
given that we are planing to run the \AR{Semantics} \AB{𝓢ᴮ} \emph{after}
having run \AB{𝓢ᴬ}, we need a way to extract a term from a computation
returned by \AB{𝓢ᴬ}. Hence our first field \ARF{reifyᴬ}. Typical examples
include the identity function (whenever the first semantics is a \AR{Syntactic}
one, cf. \cref{sec:fusion-syntactic}) or one of the \AF{reify} function
followed by an erasure from \AF{Nf} to \AF{Term} (whenever the first semantics
corresponds to normalisation by evaluation)

\ExecuteMetaData[type-scope-semantics.agda/Properties/Fusion/Specification.tex]{reify}

Once we have this key component we can introduce the relation \AF{𝓡} which
will make the type of the combinators defined later on clearer. \AF{𝓡} relates
at a given type a term and three environments by stating that the computation
one gets by sequentially evaluating the term in the first and then the second
environment is related to the one obtained by directly evaluating the term in
the third environment. Note the use of \ARF{reifyᴬ} to recover a \AD{Term}
from a computation in \AB{𝓒ᴬ} before using the second evaluation function
{\AF{semantics} \AB{𝓢ᴮ}}.

\ExecuteMetaData[type-scope-semantics.agda/Properties/Fusion/Specification.tex]{crel}

As with the previous section, only a handful of the relational counterpart
to the term constructors and their associated semantic counterpart are out
of the ordinary. We will start with the \AIC{`var} case. It states that
fusion indeed happens when evaluating a variable using related environments.

\begin{AgdaSuppressSpace}
\ExecuteMetaData[type-scope-semantics.agda/Properties/Fusion/Specification.tex]{var}
\ExecuteMetaData[type-scope-semantics.agda/Properties/Fusion/Specification.tex]{fundamental:var}
\end{AgdaSuppressSpace}

Just like for the simulation relation, the relational counterpart of value
constructors in the language state that provided that the evaluation environment
are related, we expect the computations to be related too.

\noindent\begin{minipage}{0.65\textwidth}
  \ExecuteMetaData[type-scope-semantics.agda/Properties/Fusion/Specification.tex]{base}
\end{minipage}\begin{minipage}{0.35\textwidth}
  \ExecuteMetaData[type-scope-semantics.agda/Properties/Fusion/Specification.tex]{fundamental:base}
\end{minipage}

Similarly, we have purely structural constraints for term constructs which
have purely structural semantic counterparts. For \AIC{`app} and \AIC{`ifte},
provided that the evaluation environments are related and that the evaluation
of the subterms in each environment respectively are related then the evaluations
of the overall terms should also yield related results.

\begin{AgdaSuppressSpace}
\ExecuteMetaData[type-scope-semantics.agda/Properties/Fusion/Specification.tex]{struct}
\ExecuteMetaData[type-scope-semantics.agda/Properties/Fusion/Specification.tex]{fundamental:struct}
\end{AgdaSuppressSpace}

Finally, the \AIC{`lam}-case is as always the most complex of all. The
constraint we pick puts some strong restrictions on the way the $λ$-abstraction's
body may be used by \AB{𝓢ᴬ}: we assume it is evaluated in an environment
thinned by one variable and extended using \ARF{var0ᴬ}, a placeholder value
which is another parameter of this record.
%
It is quite natural to have these restrictions: given that \ARF{reifyᴬ}
quotes the result back, we are expecting this type of evaluation in an
extended context (i.e. under one lambda). And it turns out that this is
indeed enough for all of our examples. The evaluation environments used
by the semantics \AB{𝓢ᴮ} and \AB{𝓢ᴬᴮ} on the other hand can be arbitrarily
thinned before being extended with related values to be substituted for
the variable bound by the \AIC{`lam}.

\begin{AgdaSuppressSpace}
\ExecuteMetaData[type-scope-semantics.agda/Properties/Fusion/Specification.tex]{var0}
\ExecuteMetaData[type-scope-semantics.agda/Properties/Fusion/Specification.tex]{lam}
\end{AgdaSuppressSpace}

This last observation reveals the need for additional combinators stating
that the environment-manipulating operations we have used respect the
notion of relatedness at hand. We introduce two constraints dealing with
the relations talking about evaluation environments. \ARF{\_∙ᴿ\_} tells
us how to extend related environments: one should be able to push related
values onto the environments for \AB{𝓢ᴮ} and \AB{𝓢ᴬᴮ} whilst merely
extending the one for \AB{𝓢ᴬ} with the token value \ARF{var0ᴬ}.
\ARF{th\textasciicircum{}𝓔ᴿ} guarantees that it is always possible to thin
the environments for \AB{𝓢ᴮ} and \AB{𝓢ᴬᴮ} in a \AB{𝓔ᴿ} preserving manner.

\ExecuteMetaData[type-scope-semantics.agda/Properties/Fusion/Specification.tex]{thV}

Using these last constraint, we can finally write the case of the \AF{fusion}
proof dealing with λ-abstraction.

\ExecuteMetaData[type-scope-semantics.agda/Properties/Fusion/Specification.tex]{fundamental:lam}

As with simulation, we measure the usefulness of this framework not only
by our ability to prove its fundamental lemma but also to then obtain
useful corollaries. We will start with the special case of syntactic
semantics.

\section{The Special Case of Syntactic Semantics}
\label{sec:fusion-syntactic}

The translation from \AR{Syntactic} to \AR{Semantics} uses a lot of constructors
as their own semantic counterpart, it is hence possible to generate evidence of
\AR{Syntactic} triplets being fusible with much fewer assumptions. We isolate
them and prove the result generically to avoid repetition. A \AR{SynFusion}
record packs the evidence for \AR{Syntactic} semantics \AB{Synᴬ}, \AB{Synᴮ} and
\AB{Synᴬᴮ}. It is indexed by these three \AR{Syntactic}s as well as two relations
(\AB{𝓣ᴿ} and \AB{𝓔ᴿ}) corresponding to the \AB{𝓥ᴿ} and \AB{𝓔ᴿ} ones of the
\AR{Fusion} framework; \AB{𝓒ᴿ} will always be \AF{Eqᴿ} as we are talking about terms.

\ExecuteMetaData[type-scope-semantics.agda/Properties/Fusion/Syntactic/Specification.tex]{synfusion}

The first two constraints \ARF{\_∙ᴿ\_} and \ARF{th\textasciicircum{}𝓔ᴿ} are directly taken
from the \AR{Fusion} specification: we still need to be able to extend existing related
environment with related values, and to thin environments in a relatedness-preserving manner.

\ExecuteMetaData[type-scope-semantics.agda/Properties/Fusion/Syntactic/Specification.tex]{thV}

We once again define \AF{𝓡}, a specialised version of its \AR{Fusion} counterpart
stating that the results of the two evaluations are propositionally equal.

\ExecuteMetaData[type-scope-semantics.agda/Properties/Fusion/Syntactic/Specification.tex]{crel}

Once we have \AF{𝓡}, we can concisely write down the constraint \ARF{varᴿ} which
is also already present in the definition of \AR{Fusion}.

\ExecuteMetaData[type-scope-semantics.agda/Properties/Fusion/Syntactic/Specification.tex]{var}

Finally, we have a fourth constraint (\ARF{zroᴿ}) saying that \AB{Synᴮ} and
\AB{Synᴬᴮ}'s respective \ARF{zro}s are producing related values. This will provide
us with just the right pair of related values to use in \AR{Fusion}'s \ARF{lamᴿ}.

\ExecuteMetaData[type-scope-semantics.agda/Properties/Fusion/Syntactic/Specification.tex]{zro}

Everything else is a direct consequence of the fact we are only considering
syntactic semantics. Given a \AR{SynFusion} relating three \AR{Syntactic}
semantics, we get a \AR{Fusion} relating the corresponding \AR{Semantics} where
\AB{𝓒ᴿ} is \AF{Eqᴿ}, the pointwise lifting of propositional equality. The proof
relies on the way the translation from \AR{Syntactic} to \AR{Semantics} is formulated
in \cref{sec:syntactic}.

\begin{figure}
\ExecuteMetaData[type-scope-semantics.agda/Properties/Fusion/Syntactic/Specification.tex]{fundamental}
\caption{Fundamental Lemma of Syntactic Fusions\label{fig:fundamentalsynfus}}
\end{figure}

We are now ready to give our first examples of Fusions. They are the first results one
typically needs to prove when studying the meta-theory of a language.

\section{Interactions of Renaming and Substitution}

Renaming and Substitution can interact in four ways: all but one of these
combinations is equivalent to a single substitution (the sequential execution
of two renamings is equivalent to a single renaming). These four lemmas are
usually proven in painful separation. Here we discharge them by rapid successive
instantiation of our framework, using the earlier results to satisfy the later
constraints. We only present the first instance in full details and then only
spell out the \AR{SynFusion} type signature which makes explicit the relations
used to constraint the input environments.

First, we have the fusion of two sequential renaming traversals into a single
renaming. Environments are related as follows: the composition of the two
environments used in the sequential traversals should be pointwise equal to
the third one. The composition operator \AF{select} is defined in \cref{fig:extendth}.

\begin{figure}[h]
\ExecuteMetaData[type-scope-semantics.agda/Properties/Fusion/Syntactic/Instances.tex]{renrenfus}
\caption{Syntactic Fusion of Two Renamings\label{fig:renrenfus}}
\end{figure}

Using the fundamental lemma of syntactic fusions, we get a proper \AR{Fusion} record
on which we can then use the fundamental lemma of fusions to get the renaming fusion
law we expect.

\begin{figure}[h]
\ExecuteMetaData[type-scope-semantics.agda/Properties/Fusion/Syntactic/Instances.tex]{renren}
\caption{Corollary: Renaming Fusion Law\label{fig:renren}}
\end{figure}

A similar proof gives us the fact that a renaming followed by a substitution is equivalent
to a substitution. Environments are once more related by composition.

\begin{figure}[h]
\ExecuteMetaData[type-scope-semantics.agda/Properties/Fusion/Syntactic/Instances.tex]{rensubfus}
\ExecuteMetaData[type-scope-semantics.agda/Properties/Fusion/Syntactic/Instances.tex]{rensub}
\caption{Renaming - Substitution Fusion Law\label{fig:rensub}}
\end{figure}

For the proof that a substitution followed by a renaming is equivalent to a
substitution, we need to relate the environments in a different manner:
composition now amounts to applying the renaming to every single term in the
substitution. We also depart from the use of \AF{Eqᴿ} as the relation for values:
indeed we now compare variables and terms. The relation \AF{VarTermᴿ} defined in
\cref{fig:vartermR} relates variables and terms by wrapping the variable in a
\AIC{`var} constructor and demanding it is equal to the term.

\begin{figure}[h]
\ExecuteMetaData[type-scope-semantics.agda/Properties/Fusion/Syntactic/Instances.tex]{subrenfus}
\ExecuteMetaData[type-scope-semantics.agda/Properties/Fusion/Syntactic/Instances.tex]{subren}
\caption{Substitution - Renaming Fusion Law\label{fig:subren}}
\end{figure}

Finally, the fusion of two sequential substitutions into a single one uses a
similar notion of composition. Here the second substitution is applied to each
term of the first and we expect the result to be pointwise equal to the third.
Values are once more considered related whenever they are propositionally equal.

\begin{figure}[h]
\ExecuteMetaData[type-scope-semantics.agda/Properties/Fusion/Syntactic/Instances.tex]{subsubfus}
\ExecuteMetaData[type-scope-semantics.agda/Properties/Fusion/Syntactic/Instances.tex]{subsub}
\caption{Substitution Fusion Law\label{fig:subsub}}
\end{figure}

As we are going to see in the following section, we are not limited
to \AR{Syntactic} statements.

\section{Other Examples of Fusions}

The most simple example of fusion of two \AR{Semantics} involving a non \AR{Syntactic}
one is probably the proof that \AR{Renaming} followed by normalization by evaluation's
\AR{Eval} is equivalent to \AR{Eval} with an adjusted environment.

\subsection{Fusion of Renaming Followed by Evaluation}
\label{sec:fusionrennbe}

As is now customary, we start with an auxiliary definition which will make our
type signatures a lot lighter. It is a specialised version of the relation \AF{𝓡}
introduced when spelling out the \AR{Fusion} constraints. Here the relation is \AF{PER}
and the three environments carry respectively \AD{Var} (i.e. it is a \AF{Thinning}) for
the first one, and \AF{Model} values for the two other ones.

\ExecuteMetaData[type-scope-semantics.agda/Properties/Fusion/Instances.tex]{crel}

We start with the most straightforward of the non-trivial cases: the relational
counterpart of \AIC{`app}. The \AF{Kripkeᴿ} structure of the induction hypothesis
for the function has precisely the strength we need to make use of the hypothesis
for its argument.

\begin{figure}[h]
\ExecuteMetaData[type-scope-semantics.agda/Properties/Fusion/Instances.tex]{appR}
\caption{Relational Application}
\end{figure}

The relational counterpart of \AIC{`ifte} is reminiscent of the one we used when
proving that normalisation by evaluation is in simulation with itself in
\cref{fig:nbeiftenelser}: we have two arbitrary boolean values resulting from the
evaluation of \AB{b} in two distinct manners but we know them to be the same thanks
to them being \AF{PER}-related. The canonical cases are trivially solved by
using one of the assumptions whilst the neutral case can be proven to hold thanks
to the relational versions of \AF{reify} and \AF{reflect}.

\begin{figure}[h]
\ExecuteMetaData[type-scope-semantics.agda/Properties/Fusion/Instances.tex]{ifteR}
\caption{Relational If-Then-Else}
\end{figure}

The rest of the constraints can be discharged fairly easily; either by using a
constructor, combining some of the provided hypotheses or using general results
such as the stability of \AF{PER}-relatedness under thinning of the \AF{Model}
values.

\begin{figure}[h]
\ExecuteMetaData[type-scope-semantics.agda/Properties/Fusion/Instances.tex]{reneval}
\caption{Renaming Followed by Evaluation is an Evaluation}
\end{figure}


By the fundamental lemma of \AR{Fusion}, we get the result we are looking for:
a renaming followed by an evaluation is equivalent to an evaluation in a touched
up environment.

\begin{figure}[h]
\ExecuteMetaData[type-scope-semantics.agda/Properties/Fusion/Instances.tex]{renevalfun}
\caption{Corollary: Fusion Principle for Renaming followed by Evaluation\label{fig:renevalfun}}
\end{figure}

This gives us the tools to prove the substitution lemma for evaluation.

\subsection{Substitution Lemma for Evaluation}

Given any semantics, the substitution lemma (see for instance \cite{mitchell1991kripke})
states that evaluating a term after performing a substitution is equivalent to evaluating
the term with an environment obtained by evaluating each term in the substitution.
Formally (\AB{t} is a term, \AB{γ} a substitution, \AB{ρ} an evaluation environment,
\AF{\_[\_]} denotes substitution, and \AF{⟦\_⟧\_} evaluation):

\[
\AF{⟦}~\AB{t}~\AF{[}\AB{γ}\AF{]}~\AF{⟧}~\AB{ρ}~≡~\AF{⟦}~\AB{t}~\AF{⟧}~(\AF{⟦}~\AB{γ}~\AF{⟧}~\AB{ρ})
\]

This is a key lemma in the study of a language's meta-theory and it fits our \AR{Fusion}
framework perfectly. We start by describing the constraints imposed on the environments.
They may seem quite restrictive but they are actually similar to the Uniformity condition
described by C. Coquand~(\citeyear{coquand2002formalised}) in her detailed account of NBE
for a ST$λ$C with explicit substitution and help root out exotic term
(cf. \cref{fig:nbeexotic}).

First we expect the two evaluation environments to only contain \AF{Model} values
which are \AF{PER}-related to themselves. Second, we demand that the evaluation of
the substitution in a \emph{thinned} version of the first evaluation environment
is \AF{PER}-related in a pointwise manner to the \emph{similarly thinned}
second evaluation environment. This constraint amounts to a weak commutation lemma
between evaluation and thinning; a stronger version would be to demand that thinning
of the result is equivalent to evaluation in a thinned environment.

\begin{figure}[h]
\ExecuteMetaData[type-scope-semantics.agda/Properties/Fusion/Instances.tex]{subr}
\caption{Constraints on Triples of Environments for the Substitution Lemma}
\end{figure}

We can then state and prove the substitution lemma using \AF{Subᴿ} as the constraint
on environments and \AF{PER} as the relation for both values and computations.

\begin{figure}[h]
\ExecuteMetaData[type-scope-semantics.agda/Properties/Fusion/Instances.tex]{subeval}
\caption{Substitution Followed by Evaluation is an Evaluation}
\end{figure}

The proof is similar to that of fusion of renaming with evaluation in
\cref{sec:fusionrennbe}: we start by defining a notation \AF{𝓡} to lighten the
types, then combinators \AF{APPᴿ} and \AF{IFTEᴿ}. The cases for
\ARF{th\textasciicircum{}𝓔ᴿ}, \ARF{\_∙ᴿ\_}, and \ARF{varᴿ} are a bit more tedious:
they rely crucially on the fact that we can prove a fusion principle and an identity
lemma for \AF{th\textasciicircum{}Model} as well as an appeal to \AF{reneval}
(\cref{fig:renevalfun}) and multiple uses of \AF{Eval\textasciicircum{}Sim}
(\cref{fig:nbeselfsim}). Because the technical details do not give any additional
hindsight, we do not include the proof here.
