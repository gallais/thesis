\chapter{Variations on Normalisation by Evaluation}

Normalisation by Evaluation (NBE) is a technique leveraging the computational
power of a host language in order to normalise expressions of a deeply
embedded one. The process is based on a model construction describing a
family of types by induction on its \AF{Type} index. Two
procedures are then defined: the first (\AF{eval}) constructs an element
of \AB{𝓒} \AB{σ} \AB{Γ} provided a well typed term of the corresponding
\AD{Term} \AB{σ} \AB{Γ} type whilst the second (\AF{reify}) extracts, in
a type-directed manner, normal forms \AD{Nf} \AB{σ} \AB{Γ} from elements
of the model \AB{𝓒} \AB{σ} \AB{Γ}. NBE composes the two procedures. The
definition of this \AF{eval} function is a natural candidate for our
\AF{Semantics} framework. NBE is always defined \emph{for} a
given equational theory; we start by recalling the various
rules a theory may satisfy.

\subsection{Reduction Rules}

Thanks to \AF{Renaming} and \AF{Substitution} respectively, we can formally
define η-expansion and β-reduction. The η-rules say that for some types,
terms have a canonical form: functions will all be λ-headed whilst records will
collect their fields -- here this makes all elements of \AIC{\unit{}} equal to \AIC{`one}.

\todo{insert definition of eta}

\begin{figure}[h]
\begin{mathpar}
\inferrule{\AB{t} ~\AS{:}~ \AD{Term}~ (\AB{σ} ~\AIC{`→}~ \AB{τ})~ \AB{Γ}
  }{\AB{t} \leadsto{} \AF{eta}~\AB{t}
  }{η_1}
\and \inferrule{\AB{t} ~\AS{:}~ \AD{Term}~ \AIC{`Unit}~ \AB{Γ}
  }{\AB{t} \leadsto{} \AIC{`one}
  }{η_2}
\and
\inferrule{
  }{\AIC{`app} ~(\AIC{`lam}~\AB{t})~ \AB{u} \leadsto \AB{t}~ \AF{⟨}~ \AB{u} ~\AF{/0⟩}
  }{β}
\end{mathpar}
\caption{βη Rules for our Calculus\label{fig:betaetarules}}
\end{figure}

The β-rule is the main driver for actual computation,
but the presence of an inductive data type (\AIC{`Bool}) and its eliminator
(\AIC{`ifte}) means we have further redexes: whenever the
boolean the eliminator branches on is in canonical form, we may apply
a ι-rule. Finally, the ξ-rule lets us reduce under
λ-abstractions --- the distinction between weak-head normalisation and
strong normalisation.

\begin{figure}[h]
\begin{mathpar}
\inferrule{
  }{\AIC{`ifte}~\AIC{`tt}~\AB{l}~\AB{r} ~\leadsto{}~ \AB{l}
  }{ι_1}
\and
\inferrule{
  }{\AIC{`ifte}~\AIC{`ff}~\AB{l}~\AB{r} ~\leadsto{}~ \AB{r}
  }{ι_2}
\and
\inferrule{\AB{t} ~\leadsto{}~ \AB{u}
  }{\AIC{`lam}~ \AB{t} ~\leadsto{}~ \AIC{`lam}~\AB{u}
  }{ξ}
\end{mathpar}
\caption{ιξ Rules for our Calculus\label{fig:betaetarules}}
\end{figure}

Now that we have recalled all these rules, we can talk precisely about the
sort of equational theory decided by the model construction we choose to
perform. We start with the usual definition of NBE
which goes under λs and produces η-long βι-short normal forms.

\subsection{Normal and Neutral Forms}

We parametrise the mutually defined inductive families \AD{Ne} and \AD{Nf}
by a predicate \AB{R} constraining the types at which one may embed a neutral
as a normal form. This make it possible to control the way NBE η-expands all
terms at certain types.

\begin{figure}[h]
\ExecuteMetaData[type-scope-semantics.agda/Syntax/Normal.tex]{normal}
\caption{Neutral and Normal Forms}
\end{figure}

Once more, the expected notions of thinning \AF{th\textasciicircum{}Ne} and \AF{th\textasciicircum{}Nf}
are induced as \AD{Ne} and \AD{Nf} are syntaxes. We omit their purely
structural implementation here and wish we could do so in source code,
too: our constructions so far have been syntax-directed and could
surely be leveraged by a generic account of syntaxes with binding.
We will tackle this in~\cref{a-universe}.

\section{Normalisation by Evaluation for βιξη}
\label{normbye}

In the case of NBE, the environment values and the computations in the model
will both use the same type family \AF{Model}, defined by induction on the
\AD{Type} argument.
The η-rules allow us to represent functions (resp. inhabitants
of \AIC{`Unit}) in the source language as function spaces (resp. \AR{⊤}).
In Agda, there are no such rules for boolean values. We thus need
a notion of syntactic normal forms.
