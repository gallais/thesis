\chapter{Variations on Normalisation by Evaluation}
\label{sec:variationsnormalisation}

Normalisation by Evaluation (NBE) is a technique leveraging the computational
power of a host language in order to normalise expressions of a deeply
embedded one (\cite{berger1991inverse,berger1993program,CoqDybSK,coquand2002formalised}).
A classic evaluation function manipulates a term in a purely syntactic manner,
firing redexes one after the other until none is left. In comparison,
normalisation by evaluation is a semantic technique that translates a term
into a program in the host language, relying on the host language's evaluation
machinery to produce a value. The challenging aspect of this technique is twofold.
First, we need to pick a translation that retains enough information that we
may extract a normal form from the returned value. Second, we need to make sure
that our leveraging of the host language's evaluation machinery does decide the
equational theory we are interested in. In this chapter, we will consider different
equational theories and as a consequence different translations.

\subsection{Interface of a NBE Procedure}

The normalisation by evaluation process is based on a model construction
inspired by the logical predicates of normalisation proofs. It is essentially
the computational part of such proofs.

Such a construction starts by describing a family of types \AB{Model} by
induction on its \AF{Type} index. Two procedures are then defined: the
first (\AF{eval}) constructs an element of (\AB{Model} \AB{σ} \AB{Γ})
provided a well typed term of the corresponding (\AD{Term} \AB{σ} \AB{Γ})
type whilst the second (\AF{reify}) extracts, in a type-directed manner,
normal forms (\AD{Nf} \AB{σ} \AB{Γ}) from elements of the model
(\AB{Model} \AB{σ} \AB{Γ}). NBE composes the two procedures.

The definition of this \AF{eval} function is a natural candidate for our
\AF{Semantics} framework. We introduce in \cref{fig:nbeinterface}) an abstract
interface for NBE formalising this observation.
%
We will see in this chapter that the various variations on normalisation by
evaluation that we will consider can all be described as instances of this
\AR{NBE} interface.

The \AR{NBE} interface is parametrised by two (\AD{Type} \AR{─Scoped})
families: the notion of model values (\AB{Model}) and normal forms
(\AB{Nf}) specific to this procedure. The interface packages a \AR{Semantics}
working on the \AB{Model}, an embedding of variables into model values
and a reification function from model values to normal forms.

\begin{figure}[h]
\ExecuteMetaData[type-scope-semantics.agda/Semantics/NormalisationByEvaluation/Specification.tex]{recnbe}
\caption{\AR{NBE} interface}\label{fig:nbeinterface}
\end{figure}

From each \AR{NBE} instance we can derive a normalisation function turning
terms into normal forms. The \ARF{embed} constraint guarantees that we can
manufacture an environment of placeholder values in which to run our
\AR{Semantics} to obtain an \AF{eval} function. Composing this evaluation
function with the reification procedure yields the normalisation procedure.

\begin{figure}[h]
\begin{minipage}[t]{0.55\textwidth}
\ExecuteMetaData[type-scope-semantics.agda/Semantics/NormalisationByEvaluation/Specification.tex]{eval}
\end{minipage}\begin{minipage}[t]{0.35\textwidth}
\ExecuteMetaData[type-scope-semantics.agda/Semantics/NormalisationByEvaluation/Specification.tex]{nbe}
\end{minipage}
\caption{Evaluation and normalisation functions derived from \AR{NBE}}\label{fig:nbeinterface}
\end{figure}

As we have explained earlier, NBE is always defined \emph{for} a given
equational theory. We start by recalling the various rules a theory may
satisfy.

\subsection{Reduction Rules}

We characterise an equational theory by a set of reduction rules. The
equational theory is obtained by taking the congruence closure of these
reduction rules under all term constructors except for λ-abstraction.
Indeed, we will consider systems with and without reductions under
λ-abstractions and can only make this distinction if using the congruence
rule for λ (usually called ξ) is made explicit.

\paragraph{Computation rules} Our first set of rules describes the kind
of computations we may expect. The β rule states that functions applied
to their argument may fire. It is the main driver for actual computation,
but the presence of an inductive data type and its eliminator means we
have further redexes: the ι rules specify that if-then-else conditionals
applied to concrete booleans may return the appropriate branch.

\begin{figure}[h]
\begin{mathpar}
  \inferrule
      { }
      {(λx. t) u ↝ t [u / x]}
      {β}
\end{mathpar}
\begin{mathpar}
  \inferrule
      { }
      {\mathtt{if ~true~ then~} l \mathtt{~else~} r \leadsto l}
      {ι_1}
  \and
  \inferrule
      { }
      {\mathtt{if~ false~ then~} l \mathtt{~else~} r \leadsto r}
      {ι_2}
\end{mathpar}
\caption{Computation rules: β and ι reductions}\label{fig:betaiotarules}
\end{figure}

\paragraph{Canonicity rules} The η-rules say that for some types, terms
have a canonical form: functions will all be λ-headed whilst records will
collect their fields -- here this makes all elements of the unit type equal
to \texttt{one}.

\begin{figure}[h]
\begin{mathpar}
\inferrule{Γ ⊢ t : σ → τ
  }{t \leadsto{} λx.t\,x
  }{η_1}
\and \inferrule{Γ ⊢ t : \mathtt{Unit}
  }{t \leadsto{} \mathtt{one}
  }{η_2}
\end{mathpar}
\caption{Canonicity rules: η rules for function and unit types\label{fig:etarules}}
\end{figure}

\paragraph{Congruence rule} Congruence rules are necessary if we do not
want to be limited to only computing whenever the root of the term happens
to already be a reducible expression. By deciding which ones are included,
we can however control the evaluation strategy of our calculus. We will
study the impact of the ξ-rule lets us reduce under λ-abstractions --- the
distinction between weak-head normalisation and strong normalisation.

\begin{figure}[h]
\begin{mathpar}
\inferrule{t ~\leadsto{}~ u
  }{λx.t ~\leadsto{}~ λx.u
  }{ξ}
\end{mathpar}
\caption{Congruence rule: ξ for strong normalisation\label{fig:xirules}}
\end{figure}

Now that we have recalled all these rules, we can talk precisely about the
sort of equational theory decided by the model construction we choose to
perform. We start with the usual definition of NBE
which goes under λs and produces η-long βι-short normal forms.

\subsection{Normal and Neutral Forms}

We parametrise the mutually defined inductive families \AD{Ne} and \AD{Nf}
by a predicate \AB{NoEta} constraining the types at which one may embed a neutral
as a normal form. This constraint shows up in the type of \AIC{`neu}; it makes
it possible to control whether the NBE should η-expands all terms at certain
types by prohibiting the existence of neutral terms at said type.

\begin{figure}[h]
\ExecuteMetaData[type-scope-semantics.agda/Syntax/Normal.tex]{normal}
\caption{Neutral and Normal Forms}
\end{figure}

Once more, the expected notions of thinning \AF{th\textasciicircum{}Ne} and
\AF{th\textasciicircum{}Nf} are induced as \AD{Ne} and \AD{Nf} are syntaxes.
We omit their purely
structural implementation here and wish we could do so in source code,
too: our constructions so far have been syntax-directed and could
surely be leveraged by a generic account of syntaxes with binding.
We will tackle this problem in~\cref{a-universe}.


\section{Normalisation by Evaluation for βιξη}
\label{normbye}

In the case of NBE, the environment values and the computations in the model
will both use the same type family \AF{Model}, defined by induction on the
\AD{Type} argument. The η-rules allow us to represent functions (respectively
inhabitants of \AIC{`Unit}) in the source language as function spaces
(respectively values of type \AR{⊤}). Evaluating a \AIC{`Bool} may however
yield a stuck term so we can't expect the model to give us anything more than
an open term in normal form.

The model construction then follows the usual pattern pioneered by
Berger~(\citeyear{berger1993program}) and formally analysed and thoroughly
explained by Catarina Coquand~(\citeyear{coquand2002formalised}). We work
by induction on the type and describe η-expanded values: all inhabitants
of (\AF{Model} \AIC{`Unit} \AB{Γ}) are equal and all elements
of (\AF{Model} (\AB{σ} \AIC{`→} \AB{τ}) \AB{Γ}) are functions in Agda.

\begin{figure}[h]
\ExecuteMetaData[type-scope-semantics.agda/Semantics/NormalisationByEvaluation/BetaIotaXiEta.tex]{model}
\caption{Model for Normalisation by Evaluation\label{fig:nbemodel}}
\end{figure}

This model is defined by induction on the type in terms either of
syntactic objects (\AD{Nf}) or using the \AF{□}-operator which is
a closure operator for thinnings. As such, it is trivial to prove
that for all type \AB{σ}, (\AF{Model} \AB{σ}) is \AF{Thinnable}.

\begin{figure}[h]
\ExecuteMetaData[type-scope-semantics.agda/Semantics/NormalisationByEvaluation/BetaIotaXiEta.tex]{thmodel}
\caption{Values in the Model are Thinnable\label{fig:thnbemodel}}
\end{figure}

Application's semantic counterpart is easy to define: given that \AB{𝓥}
and \AB{𝓒} are equal in this instance definition we can feed the argument
directly to the function, with the identity renaming. This corresponds to
\AF{extract} for the comonad \AF{□}.

\begin{figure}[h]
\ExecuteMetaData[type-scope-semantics.agda/Semantics/NormalisationByEvaluation/BetaIotaXiEta.tex]{app}
\caption{Semantic Counterpart of \AIC{`app}\label{fig:nbeapp}}
\end{figure}

Conditional branching however is more subtle: the boolean value \AIC{`if} branches on
may be a neutral term in which case the whole elimination form is stuck. This forces
us to define \AF{reify} and \AF{reflect} first. These functions, also known as quote
and unquote respectively, give the interplay between neutral terms, model values and
normal forms. \AF{reflect} performs a form of semantic η-expansion: all stuck \AIC{`Unit}
terms are equated and all functions are λ-headed. It allows us to define \AF{var0}, the
semantic counterpart of (\AIC{`var} \AIC{z}).

\begin{figure}[h]
\ExecuteMetaData[type-scope-semantics.agda/Semantics/NormalisationByEvaluation/BetaIotaXiEta.tex]{reifyreflect}
\caption{Reify and Reflect\label{fig:reifyreflectnbe}}
\end{figure}

We can then give the semantics of \AIC{`ifte}: if the boolean is a value, the
appropriate branch is picked; if it is stuck then the whole expression is stuck.
It is then turned into a neutral form by reifying the two branches and then reflected
in the model.

\begin{figure}[h]
\ExecuteMetaData[type-scope-semantics.agda/Semantics/NormalisationByEvaluation/BetaIotaXiEta.tex]{ifte}
\caption{Semantic Counterpart of \AIC{`ifte}\label{fig:nbeifte}}
\end{figure}

We can then combine these components. The semantics of a λ-abstraction is simply the
identity function: the structure of the functional case in the definition of the model
matches precisely the shape expected in a \AF{Semantics}. Because the environment
carries model values, the variable case is trivial.

\begin{figure}[h]
\ExecuteMetaData[type-scope-semantics.agda/Semantics/NormalisationByEvaluation/BetaIotaXiEta.tex]{eval}
\caption{Evaluation is a \AR{Semantics}\label{fig:evalnbe}}
\end{figure}

We can define a normaliser by kickstarting the evaluation with an environment of
placeholder values obtained by reflecting the term's free variables and then reifying
the result.

\begin{figure}[h]
\ExecuteMetaData[type-scope-semantics.agda/Semantics/NormalisationByEvaluation/BetaIotaXiEta.tex]{norm}
\caption{Normalisation as Reification of an Evaluated Term\label{fig:normnbe}}
\end{figure}


\section{Normalisation by Evaluation for βιξ}

As mentioned above, actual proof systems such as Coq rely on evaluation
strategies that avoid applying η-rules: unsurprisingly, it is a rather
bad idea to η-expand proof terms which are already large when typechecking
complex developments.

In these systems, the η-rule is never deployed except when comparing a
neutral and a constructor-headed term for equality. Instead of declaring
them distinct, the algorithm does one step of η-expansion on the
neutral term and compares their subterms structurally. The conversion test
fails only when confronted with neutral terms with distinct head
variables or normal forms with different head constructors.
This leads us to using a predicate \AB{NoEta} which holds for all types
thus allowing us to embed all neutrals into normal forms.

Now that this is established, we can focus on the model construction. As
noted in the definition of the \AR{NBE} interface (cf. \cref{fig:nbeinterface}),
the \AR{Semantics} underlying normalisation by evaluation will use the
same type family for environment values and the computations in the model.

\subsection{Model Construction}

This equational theory can be decided with what happens to be the most
straightforward model construction described in \cref{nbestuckmodel}:
in our \AF{Model} everything is either a (non expanded) stuck computation
(i.e. a neutral term \AD{Ne}) or a \AF{Value} arising from a constructor-headed
term and whose computational behaviour is described by an Agda value of the
appropriate type.
%
Values of the \AIC{`Unit} type are modelled by the Agda's unit type, values
of the \AIC{`Bool} type are Agda booleans and values of functions from \AB{σ}
to \AB{τ} are modelled by Kripke function spaces from the type of elements of
the \AF{Model} at type \AB{σ} and ones at type \AB{τ}.
%
It is important to note that the functional values have the \AF{Model} as both
domain and codomain: there is no reason to exclude the fact that either the
argument or the body may or may not be stuck.

%% It is possible to alter the model definition described earlier so that it
%% avoids unnecessary η-expansions. We proceed by enriching the traditional
%% model with extra syntactical artefacts in a manner reminiscent of Coquand
%% and Dybjer's~(\citeyear{CoqDybSK}) approach to defining an NBE procedure
% for the SK combinator calculus. Their resorting to gluing
%% terms to elements of the model was dictated by the sheer impossibility to write
%% a sensible reification procedure but, in hindsight, it provides us with a
%% powerful technique to build models internalizing alternative equational
%% theories.

\begin{figure}[h]
\ExecuteMetaData[type-scope-semantics.agda/Semantics/NormalisationByEvaluation/BetaIotaXi.tex]{model}
\caption{Model Definition for βιξ\label{nbestuckmodel}}
\end{figure}

We can observe that we have only used families constant in their scope
index, neutral forms or \AF{□}-closed families. All of these are
\AF{Thinnable} hence \AF{Value} and \AF{Model} also are. We give these
proofs in \cref{fig:thbixmodel}.

\begin{figure}[h]
\ExecuteMetaData[type-scope-semantics.agda/Semantics/NormalisationByEvaluation/BetaIotaXi.tex]{thmodel}
\caption{The \AF{Model}~is \AF{Thinnable}}\label{fig:thbixmodel}
\end{figure}

\subsection{Reify and Reflect}

During the definition of our \AR{Semantics} acting on elements of
type \AF{Model}, we are inevitably going to be faced with a situation
where we are eliminating what happens to be a stuck computation
(e.g. applying a stuck function to its argument, or branching over
a stuck boolean).
%
In this case we should return a stuck computation. By definition
that means we ought to be able to take the eliminator's semantic
arguments and turn them into syntactic objects i.e. to \emph{reify}
them. For reasons that will become obvious in the definition of
\AF{reify} in \cref{fig:bix:reify}, we will first need to
define \AF{reflect}, the function that reflects neutral terms as
model values.

By definition we can trivially embed neutral terms into the model
using the first injection into the disjoint sum type. From \AF{reflect}
we can derive \AF{var0}, the semantic version of the first variable
in scope that we will use to reify the body of a λ-abstraction.

\begin{figure}[h]
\ExecuteMetaData[type-scope-semantics.agda/Semantics/NormalisationByEvaluation/BetaIotaXi.tex]{reflect}
\caption{Reflect and Fresh Semantic Variables}\label{fig:bix:reflect}
\end{figure}

Reification is quite straightforward too as demonstrated in \cref{fig:bix:reify}.
A \AF{Model} value is
either a neutral term that can be trivially turned into a normal
form or a \AF{Value}. Reification of \AF{Value}s proceeds by induction
on their type. Unit values are turned into \AIC{`one}, the trivial
syntactic object of type \AIC{`Unit}. Semantic booleans are reified
as their syntactic counterpart. Finally semantic functions give rise
to lambdas. In the context thus extended we may craft \AF{var0}, the
semantic counterpart of the fresh variable, and apply the semantic
function to it before reifying the resulting semantic body to one in
normal form.

\begin{figure}[h]
\ExecuteMetaData[type-scope-semantics.agda/Semantics/NormalisationByEvaluation/BetaIotaXi.tex]{reify}
\caption{Reify}\label{fig:bix:reify}
\end{figure}

\subsection{A \AR{Semantics} Targetting our \AF{Model}}

Now that we have defined the \AF{Model} we are interested in and that
we have proved that we can both embed stuck computations into it and
obtain normal forms from it, it is time to define a \AR{Semantics}
targetting it. We will study the most striking semantic combinators
one by one and then put everything together.

Semantic application is perhaps the most interesting of the combinators
needed to define our value of type (\AR{Semantics} \AF{Model} \AF{Model}).
It follows the case distinction we mentioned earlier: in case the function
is a stuck term, we grow its spine by reifying its argument; otherwise we
have an Agda function ready to be applied. We use the \AF{extract} operation
of the \AB{□} comonad (cf. \cref{fig:boxcomonad}) to say that we are using
the function in the ambient context, not an extended one.

\begin{figure}[h]
\ExecuteMetaData[type-scope-semantics.agda/Semantics/NormalisationByEvaluation/BetaIotaXi.tex]{app}
\caption{Semantical Counterpart of \AIC{`app}}
\end{figure}

When defining the semantical counterpart of \AIC{`ifte}, we follow a similar
case distinction.
%
The value case is straightforward: depending on the boolean value we
pick either the left or the right branch which are precisely of the right
type already.
%
If the boolean evaluates to a stuck term, we follow the same strategy we
used for semantic application: we reify the two branches and assemble a
neutral term.

\begin{figure}[h]
\ExecuteMetaData[type-scope-semantics.agda/Semantics/NormalisationByEvaluation/BetaIotaXi.tex]{ifte}
\caption{Semantical Counterpart of \AIC{`ifte}}
\end{figure}

Finally, we have all the necessary components to show that evaluating
a term in our \AF{Model} is a perfectly valid \AR{Semantics} (we call
the corresponding \AF{Semantics} record \AF{Eval} but leave it out here).
%
As showcased earlier, normalisation is obtained as a direct corollary of
\AR{NBE} by the composition of reification and evaluation in an environment
of placeholder values.

\ExecuteMetaData[type-scope-semantics.agda/Semantics/NormalisationByEvaluation/BetaIotaXi.tex]{norm}

Now that we have our definition of \AF{NBE} for the βιξ rules, we
can run the \AF{test} defined in \cref{fig:nbetest} and, obtaining
(\lam{b}{\lam{u}{\ifte{b}{\uni}{u}}}),
observe that
we have indeed reduced all of the βι redexes, even if they were
hidden under a λ-abstraction. Note however that we still have a stuck
if-then-else conditional even though the return type is a record type
with only one inhabitant: we are not performing η-expansion so we cannot
expect this type of knowledge to be internalised!

\begin{figure}[h]
\ExecuteMetaData[type-scope-semantics.agda/Semantics/NormalisationByEvaluation/BetaIotaXi.tex]{test}
\caption{Running example: the βιξ case}\label{fig:betaiotaxitest}
\end{figure}

\section{Normalisation by Evaluation for βι}

The decision to apply the η-rule lazily as we have done at the beginning of
this chapter can be pushed even further: one may forgo using the ξ-rule too
and simply perform weak-head normalisation. This drives computation only when
absolutely necessary, e.g. when two terms compared for equality have matching
head constructors and one needs to inspect these constructors' arguments to
conclude.

For that purpose, we introduce an inductive family describing terms in
weak-head normal forms.

\subsection{Weak-Head Normal Forms}

A weak-head normal form (respectively a weak-head neutral form) is a term
which has been evaluated just enough to reveal a head constructor
(respectively to reach a stuck elimination). There are no additional
constraints on the subterms: a λ-headed term is in weak-head normal form
no matter the shape of its body. Similarly an application composed of a
variable as the function and a term as the arguments in weak-head neutral
form no matter what the argument looks like. This means in particular
that unlike with \AD{Ne} and \AD{Nf} there is no mutual dependency between
the definitions of \AD{WHNE} (defined first) and \AD{WHNF}.

\begin{figure}[h]
\ExecuteMetaData[type-scope-semantics.agda/Syntax/WeakHead.tex]{weakhead}
\caption{Weak-Head Normal and Neutral Forms\label{fig:weakhead}}
\end{figure}

Naturally, it is possible to define the thinnings
\AF{th\textasciicircum{}WHNE} and \AF{th\textasciicircum{}WHNF}
as well as erasure
functions \AF{erase\textasciicircum{}WHNE} and \AF{erase\textasciicircum{}WHNF}
with codomain \AD{Term}. We omit their simple definitions here.

\subsection{Model Construction}

The model construction is much like the previous one except
that source terms are now stored in the model too. This means that
from an element of the model, one can pick either the reduced version
of the input term (i.e. a stuck term or the term's computational
content) or the original. We exploit this ability most
notably in reification where once we have obtained either a
head constructor or a head variable, no subterm needs to
be evaluated.

\begin{figure}[h]
\ExecuteMetaData[type-scope-semantics.agda/Semantics/NormalisationByEvaluation/BetaIota.tex]{model}
\caption{Model Definition for Computing Weak-Head Normal Forms\label{fig:betaiotamodel}}
\end{figure}

\AF{Thinnable} can be defined rather straightforwardly based on the template
provided in the previous sections: once more all the notions used in the model
definition are \AF{Thinnable} themselves. Reflection and reification also
follow the same recipe as in the previous section.

The application and conditional branching rules are more
interesting. One important difference with respect to the previous
section is that we do not grow the spine of a stuck term using
reified versions of its arguments but rather the corresponding
\emph{source} term. Thus staying true to the idea that we only head
reduce enough to expose either a constructor or a variable and let
the other subterms untouched.

\begin{figure}[h]
\ExecuteMetaData[type-scope-semantics.agda/Semantics/NormalisationByEvaluation/BetaIota.tex]{app}
\ExecuteMetaData[type-scope-semantics.agda/Semantics/NormalisationByEvaluation/BetaIota.tex]{ifte}
\caption{Semantical Counterparts of \AIC{`app} and \AIC{`ifte}\label{fig:betaiotaappifte}}
\end{figure}

The semantical counterpart of \AIC{`lam} is also slightly trickier than
before. Indeed, we need to recover the source term the value corresponds
to. Luckily we know it has to be λ-headed and once we have introduced a
fresh variable with \AIC{`lam}, we can project out the source term of
the body evaluated using this fresh variable as a placeholder value.

\begin{figure}[h]
\ExecuteMetaData[type-scope-semantics.agda/Semantics/NormalisationByEvaluation/BetaIota.tex]{lam}
\caption{Semantical Counterparts of \AIC{`lam}\label{fig:betaiotalam}}
\end{figure}

We can finally put together all of these semantic counterparts to
obtain a \AR{Semantics} corresponding to weak-head normalisation.
We omit the now self-evident definition of \AF{NBE} where \ARF{embed}
is obtained by using \AF{reflect}.

We can once more run our test and observe that it simply outputs the
term it was fed. Indeed our example is λ-headed, meaning that it is
already in weak-head normal form and that the normaliser does not need
to do any work.

\begin{figure}[h]
\ExecuteMetaData[type-scope-semantics.agda/Semantics/NormalisationByEvaluation/BetaIota.tex]{test}
\caption{Running example: the βι case}\label{fig:betaiotaxitest}
\end{figure}

