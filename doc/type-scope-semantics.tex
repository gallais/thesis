\chapter{Refactoring Common Traversals}

A programmer implementing an embedded language with bindings has a
wealth of possibilities. However, should she want to be able to inspect
the terms produced by her users in order to optimise or even compile
them, she will have to work with a deep embedding. Which means that she
will have to (re)implement a great number of traversals doing such
mundane things as renaming, substitution, or partial evaluation.
Should she want to get help from the typechecker in order to fend
off common bugs, she can opt for inductive families~\cite{dybjer1991inductive}
to enforce precise invariants. But the traversals now have to be
invariant preserving too!

\section{McBride's Kit}

In an unpublished manuscript, McBride~(\citeyear{mcbride2005type})
observes the similarity between the types and implementations of
renaming and substitution for the simply typed $λ$-calculus (ST$λ$C) in a
dependently typed language as shown in \cref{ren}. There are three
differences between the implemenations of renaming and substitution:
\begin{enumerate}
  \item
    in the variable case, after renaming a variable we must wrap it
    in a \AIC{`var} constructor whereas a substitution directly
    produces a term;
  \item
    when weakening a renaming to push it under a $λ$ we need only
    post-compose the remaning with the De Bruijn variable successor
    constructor \AIC{s} (which is essentially weakening for variables)
    whereas for a substitution we need a weakening operation for terms.
    It can be given by renaming via the successor constructor
    (\AF{ren} (\AIC{pack} \AIC{s}));
  \item
    also in the $λ$ case, when pushing a renaming or a substitution under
    a binder we must extend it to ensure that the variable bound by the
    $λ$ is mapped to itself. For renaming this involves its extension by
    the zeroth variable \AIC{z} whereas for subsitutions we must extend by
    the zeroth variable seen as a term (\AIC{`var} \AIC{z}).
\end{enumerate}

\begin{figure}[h]
\ExecuteMetaData[type-scope-semantics/StateOfTheArt/McBride05.tex]{ren}
\ExecuteMetaData[type-scope-semantics/StateOfTheArt/McBride05.tex]{sub}
\caption{Renaming\label{ren} and Substitution\label{sub} for the ST$λ$C}
\end{figure}

McBride then defines a notion of ``Kit'' abstracting these differences.


\begin{figure}[h]
\ExecuteMetaData[type-scope-semantics/StateOfTheArt/McBride05.tex]{kitdef}
\ExecuteMetaData[type-scope-semantics/StateOfTheArt/McBride05.tex]{kitsem}
\caption{Kit traversal for the ST$λ$C\label{kit}, for $K$ of type \AR{Kit} ⧫}
\end{figure}

\begin{figure}[h]
\ExecuteMetaData[type-scope-semantics/StateOfTheArt/McBride05.tex]{nbe}
\caption{Normalisation by Evaluation for the ST$λ$C\label{nbe}}
\end{figure}

The uses of \ARF{Kit.─} operations in the generalising
the traversal function \AF{kit} are shown (in pink) in \cref{kit}.

The contributions of the present paper are twofold:
\begin{itemize}
\item{} We generalise the ``Kit'' approach from syntax to semantics
bringing operations like normalisation (cf.~\cref{nbe}) and printing
with a name supply into our framework.

\item{} We  prove
generic results about simulations between and fusions of semantics
given by, and enabled by, Kit.
\end{itemize}

\paragraph{Outline} We start by defining the simple calculus we will
use as a running example. We then introduce a notion of environments
and one well known instance: the category of renamings. This leads us
to defining a generic notion of type and scope-preserving Semantics
together with a generic evaluation function. We then showcase the
ground covered by these Semantics: from the syntactic ones
corresponding to renaming and substitution to printing with names,
variations of Normalisation by Evaluation or CPS transformations.
Finally, given the generic
definition of Semantics, we can prove fundamental lemmas about these
evaluation functions: we characterise the semantics which can simulate
one another and give an abstract treatment of composition yielding
compaction and reuse of proofs compared to Benton et
al.~(\citeyear{benton2012strongly}).

\section{The Calculus and Its Embedding}


\[
\begin{array}{lcl}
⟨\type{}⟩ & ::=    & \unit{} \\
          & \mid{} & \bool{} \\
          & \mid{} & \arrow{⟨\type{}⟩}{⟨\type{}⟩}
\end{array}
\]
