\chapter{Conclusion}

\section{Summary}

We have demonstrated that we can exploit the shared structure highlighted by the introduction
of \AR{Semantics} to further alleviate the implementer's pain by tackling the properties of
these \AR{Semantics} in a similarly abstract approach. We characterised, using a first logical
relation, the traversals which were producing related outputs provided they were fed related
inputs. A more involved second logical relation gave us a general description of fusion of
traversals where we study triples of semantics such that composing the two first ones would
yield an instance of the third one.

\section{Related Work}
As we have already explained in \cref{chapter:program-conclusion}, this construction
really shines in a simply typed setting but it is not limited to it. Just like we can
build an well-scoped but untyped version of \AR{Semantics}, \AR{Simulation} and \AR{Fusion}
fundamental theorems akin to the ones proven in this part also hold true. The common
structure across all these variations suggests a possible generic scope safe treatment
of syntaxes with binding.

Benton, Hur, Kennedy and McBride's joint work~(\citeyear{benton2012strongly}) was not
limited to defining traversals. They proved fusion lemmas describing the interactions
of renaming and substituion using tactics rather than defining a generic proof framework
like we do. They have also proven the evaluation function of their denotational semantics
correct; however they chose to use propositional equality and to assume function extensionality
rather than resorting to the traditional Partial Equivalence Relation approach we use.

\todo{Mention fold fusion; deforestation}

\section{Further work}

