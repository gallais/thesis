\chapter{CPS Transformations}
\label{cps-transformation}

In their generic account of continuation passing style (CPS) transformations,
Hatcliff and Danvy (\citeyear{hatcliff1994generic}) decompose both call by name
and call by value CPS transformations in two phases. The first one, an
embedding of the source language into Moggi's Meta Language
(\citeyear{moggi1991notions}),
picks an evaluation strategy whilst the second one is a generic erasure
from Moggi's ML back to the original language. Looking closely at the
structure of the first pass, we can see that it is an instance of our
Semantics framework.

Let us start with the definition of Moggi's Meta Language (ML). Its types
are fairly straightforward, we simply have an extra constructor \AIC{\#\_}
for computations and the arrow has been turned into a \emph{computational}
arrow meaning that its codomain is always considered to be a computational type.

\begin{figure}[h]
\ExecuteMetaData[type-scope-semantics.agda/Syntax/MoggiML/Type.tex]{ctype}
\caption{Inductive Definition of Types for Moggi's ML}
\end{figure}

Then comes the Meta-Language itself (cf.~\cref{fig:moggiml}). It incorporates
\AD{Term} constructors and eliminators with slightly different types: \emph{value}
constructors are associated to \emph{value} types whilst eliminators (and their
branches) have \emph{computational} types. Two new term constructors have been
added: \AIC{`ret} and \AIC{`bind} make \AIC{\#\_} a monad. They can be used to
explicitly schedule the evaluation order of various subterms.

\begin{figure}[h]
\ExecuteMetaData[type-scope-semantics.agda/Syntax/MoggiML/Calculus.tex]{calculus}
\caption{Definition of Moggi's Meta Language\label{fig:moggiml}}
\end{figure}

As explained in Hatcliff and Danvy's paper, the translation from \AD{Type} to
\AD{CType} fixes the calling convention the CPS translation will have. Both call
by name (\AF{CBV}) and call by value (\AF{CBV}) can be encoded. They behave the
same way on base types but differ on the way they translate function spaces.
In \AF{CBN} the argument of a function is a computation (i.e. it is wrapped in
a \AIC{\#\_} type constructor) whilst it is expected to have been fully evaluated
in CBV. Let us look more closely at these two translations.

\section{Translation into Moggi's Meta-Language}

\subsection{Call by Name}

We define the translation \AF{CBN} of \AF{Type} in a call by name style together
with a shorthand for the computational version of the translation \AF{\#CBN}. As
explained earlier, base types are kept identical whilst function spaces are turned
into function spaces whose domains and codomains are computational.

\begin{figure}[h]
\ExecuteMetaData[type-scope-semantics.agda/Semantics/CPS/CBN.tex]{cbn}
\caption{Translation of \AF{Type} in a Call by Name style\label{fig:moggicbn}}
\end{figure}

Once we know how to translate types, we can start thinking about the way terms are
handled. The term's type will \emph{have} to be computational as there is no guarantee
that the input term is in normal form. In a call by name strategy, variables in context
are also assigned a computational type.

By definition of \AF{Semantics}, our notions of environment values and computations
will \emph{have} to be of type (\AF{Type} \AF{─Scoped}). This analysis leads us to
define the generic transformation \AF{\_\textasciicircum{}CBN} in \cref{fig:cbntransformer}.

\begin{figure}[h]
\ExecuteMetaData[type-scope-semantics.agda/Semantics/CPS/CBN.tex]{cbntransformer}
\caption{\AF{\ensuremath{\cdot{}}─Scoped} Transformer for Call by Name\label{fig:cbntransformer}}
\end{figure}

Our notion of environment values are then (\AD{Var} \AF{\textasciicircum{}CBN})
whilst computations will be (\AD{ML} \AF{\textasciicircum{}CBN}). Once these
design decisions are made, we can start drafting the semantical counterpart of
common combinators.

As usual, we define combinators corresponding to the two eliminators first.
In these cases, we need to evaluate the subterm the redex is potentially
stuck on first. This means evaluating the function first in an application
node (which will then happily consume the thunked argument) and the boolean
in the case of boolean branching.

\begin{figure}[h]
\ExecuteMetaData[type-scope-semantics.agda/Semantics/CPS/CBN.tex]{app}
\ExecuteMetaData[type-scope-semantics.agda/Semantics/CPS/CBN.tex]{ifte}
\caption{Semantical Counterparts for \AIC{`app} and \AIC{`ifte}\label{fig:cbnelim}}
\end{figure}

Values have a straightforward interpretation: they are already fully evaluated
and can thus simply be returned as trivial computations using \AIC{`ret}. This
gives us everything we need to define the embedding of STLC into Moggi's ML in
call by name style.

\subsection{Call by Value}

Call by value follows a similar pattern. As the name suggests, in call by value
function arguments are expected to be values already. In the definition of \AF{CBV}
this translates to function spaces being turned into functions spaces where only
the \emph{codomain} is made computational.

\begin{figure}[h]
\ExecuteMetaData[type-scope-semantics.agda/Semantics/CPS/CBV.tex]{cbv}
\caption{Translation of \AF{Type} in a Call by Value style\label{fig:moggicbv}}
\end{figure}

We can then move on to the notion of values and computations for our call by value
\AF{Semantics}. All the variables in scope should refer to values hence the choice
to translate \AB{Γ} by mapping \AF{CBV} over it in both cases. As with the call by
name translation, we need our target type to be computational: the input terms are
not guaranteed to be in normal form.

\begin{figure}[h]
\ExecuteMetaData[type-scope-semantics.agda/Semantics/CPS/CBV.tex]{cbvalue}
\caption{Values and Computations for the \AF{CBN} CPS \AR{Semantics}\label{fig:moggicbvalue}}
\end{figure}

Albeit being defined at different types, the semantical counterparts of value
constructors and \AIC{`ifte} are the same as in the call by name case. The
interpretation of \AIC{`app} is where we can see a clear difference: we need
to evaluate the function \emph{and} its argument before applying one to the
other. We pick a left-to-right evaluation order but that is arbitrary: another
decision would lead to a different but equally valid translation.

\begin{figure}[h]
\ExecuteMetaData[type-scope-semantics.agda/Semantics/CPS/CBV.tex]{app}
\caption{Semantical Counterparts for \AIC{`app}\label{fig:cbvelim}}
\end{figure}

Finally, the corresponding \AF{Semantics} can be defined (code omitted here).

\section{Translation Back from Moggi's Meta-Language}

Once we have picked an embedding from STLC to Moggi's ML, we can kickstart it by using
an environment of placeholder values just like we did for normalisation by evaluation.
The last thing missing to get the full CPS translation is to have the generic function
elaborating terms in \AD{ML} into \AD{STLC} ones.

We first need to define a translation of types in Moggi's Meta-Language to types in
the simply-typed lambda-calculus. The translation is parametrised by \AB{r}, the
\emph{return} type. Type constructors common to both languages are translated to their
direct counterpart and the computational type constructor is translated as double
\AB{r}-negation (i.e. (\AB{\ensuremath{\cdot{}}} \AIC{`→} \AB{r}) \AIC{`→} \AB{r}).

\begin{figure}[h]
\ExecuteMetaData[type-scope-semantics.agda/Syntax/MoggiML/CPS.tex]{cps}
\caption{Translating Moggi's ML's Types to STLC Types\label{fig:cpstypeml}}
\end{figure}

Once these translations have been defined, we give a generic elaboration function
getting rid of the additional language constructs available in Moggi's ML. It
takes any term in Moggi's ML and returns a term in STLC where both the type and
the context have been translated using (\AF{CPS[} \AB{r} \AF{]}) for an abstract
parameter \AB{r}.

\ExecuteMetaData[type-scope-semantics.agda/Syntax/MoggiML/CPS.tex]{elabtype}

All the constructors which appear both in Moggi's ML and STLC are translated
in a purely structural manner. The only two interesting cases are \AIC{`ret}
and \AIC{`bind} which correspond to the \AIC{\#} monad in Moggi's ML and are
interpreted as return and bind for the continuation monad in STLC.

First, (\AIC{`ret} \AB{t}) is interpreted as the function which takes the current
continuation and applies it to its translated argument (\AF{cps} \AB{t}).

\ExecuteMetaData[type-scope-semantics.agda/Syntax/MoggiML/CPS.tex]{elabret}

Then, (\AIC{`bind} \AB{m} \AB{f}) gets translated as the function grabbing the
current continuation \AB{k}, and running the translation of \AB{m} with the
continuation which, provided the value \AB{v} obtained by running \AB{m},
runs \AB{f} applied to \AB{v} and \AB{k}. Because the translations of \AB{m}
and \AB{f} end up being used in extended contexts, we need to make use of the
fact \AD{Term}s are thinnable.

\ExecuteMetaData[type-scope-semantics.agda/Syntax/MoggiML/CPS.tex]{elabbind}

By highlighting the shared structure, of the call by name and call by value
translations we were able to focus on the interesting part: the ways in which
they differ. The formal treatment of the type translations underlines the
fact that in both cases the translation of a function's domain is uniform.
This remark opens the door to alternative definitions; we could for instance
consider a mixed strategy which is strict in machine-representable types thus
allowing an unboxed representation (\cite{10.1007/3540543961_30}) but lazy in
every other type.
