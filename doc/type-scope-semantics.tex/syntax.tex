\chapter{Intrinsically Scoped and Typed Syntax}
\label{sec:scopedtypedterms}

A programmer implementing an embedded language with bindings has a wealth of
possibilities. However, should they want to be able to inspect the terms produced
by their users in order to optimise or even compile them, they will have to work
with a deep embedding.

\section{A Primer on Scope And Type Safe Terms}\label{section:primer-term}

Scope safe terms follow the discipline that every variable is either bound by
some binder or is explicitly accounted for in a context. Bellegarde and Hook
(\citeyear{BELLEGARDE1994287}), Bird and Patterson (\citeyear{bird_paterson_1999}),
and Altenkirch and Reus (\citeyear{altenkirch1999monadic}) introduced the classic
presentation of scope safety using inductive \emph{families}
(\cite{dybjer1994inductive}) instead of inductive types to represent abstract
syntax. Indeed, using a family indexed by a \AD{Set}, we can track scoping
information at the type level. The empty \AD{Set} represents the empty scope.
The functor $1 + (\_)$ extends the running scope with an extra variable.

An inductive type is the fixpoint of an endofunctor on \AD{Set}. Similarly, an
inductive family is the fixpoint of an endofunctor on ({\AD{Set} \AS{→} \AD{Set}}).
Using inductive families to enforce scope safety, we get the following definition
of the untyped $\lambda$-calculus:
$T(F) = \lambda X \!\in\! \AD{Set}.\; X + (F(X) \times F(X)) + F(1 + X)$.
This endofunctor offers a choice of three constructors.  The first one corresponds
to the variable case; it packages an inhabitant of \AB{X}, the index \AD{Set}.
The second corresponds to an application node; both the function and its argument
live in the same scope as the overall expression. The third corresponds to a
$\lambda$-abstraction; it extends the current scope with a fresh variable.
The language is obtained as the fixpoint of $T$:

\begin{figure}[h]
\[
   \mathit{UTLC} = \mu F \in \AD{Set}^{\AD{Set}}.
   \lambda X \!\in\! \AD{Set}.\; X + (F(X) \times F(X)) + F(1 + X)
\]
\caption{Well-Scoped Untyped Lambda Calculus as the Fixpoint of a Functor}
\end{figure}

Since `UTLC' is an endofunction on \AD{Set}, it makes sense to ask whether it is
also a functor and a monad. Indeed it is, as Altenkirch and Reus have shown. The
functorial action corresponds to renaming, the monadic `return' corresponds to
the use of variables, and the monadic `join' corresponds to substitution. The
functor and monad laws correspond to well known properties from the equational
theories of renaming and substitution. We will revisit these properties
below in \cref{sec:fusionrel}.

There is no reason to restrict this technique to fixpoints of endofunctors on
$\AD{Set}^{\AD{Set}}$. The more general case of fixpoints of (strictly positive)
endofunctors on $\AD{Set}^{\AB{J}}$ can be endowed with similar operations by
using Altenkirch, Chapman and Uustalu's relative monads
(\citeyear{Altenkirch2010, JFR4389}).

We pick as our \AB{J} the category whose objects are inhabitants of
({\AD{List} \AD{I}}) (\AB{I} is a parameter of the construction) and whose morphisms
are thinnings (see \cref{sec:categoryrenamings}).  This ({\AD{List} \AB{I}}) is
intended to represent the list of the sort (/ kind / types depending on the
application) of the de Bruijn variables in scope. We can recover an untyped
approach by picking $I$ to be the unit type. Given this typed setting, our functors
take an extra $I$ argument corresponding to the type of the expression being built.
This is summed up by the large type ({\AB{I} \AF{─Scoped}}) defined in
\cref{fig:scoped}.

\section{The Calculus and Its Embedding}
\label{sec:stlccalculus}

We work with a deeply embedded simply typed λ-calculus (STλC). It has \unit{} and
\bool{} as base types and serves as a minimal example of a system with a record
type equipped with an η-rule and a sum type. We describe the types both as a
grammar and the corresponding inductive type in Agda in~\cref{fig:type}.

\begin{figure}[h]
\begin{minipage}{0.5\textwidth}
\[
\begin{array}{lcl}
\nonterminal{\type{}}
  & ::=    & \unit{}
  ~ \mid{} ~ \bool{} \\
  & \mid{} & \arrow{\nonterminal{\type{}}}{\nonterminal{\type{}}}
\end{array}
\]
\end{minipage}
\begin{minipage}{0.5\textwidth}
\ExecuteMetaData[type-scope-semantics.agda/Syntax/Type.tex]{type}
\end{minipage}
\caption{Types used in our Running Example\label{fig:type}}
\end{figure}

The language's constructs are those one expects from a λ-calculus: variables,
application and λ-abstraction. We then have a constructor for values of the unit
type but no eliminator (a term of unit type carries no information). Finally,
we have two constructors for boolean values and the expected if-then-else eliminator.

\begin{figure}[h]
\[
\begin{array}{lcl}
\nonterminal{\term{}}
  & ::=    & x
  ~ \mid{} ~ \app{\nonterminal{\term{}}}{\nonterminal{\term{}}}
  ~ \mid{} ~ \lam{x}{\nonterminal{\term{}}} \\
  & \mid{} & \uni{} \\
  & \mid{} & \tru{}
  ~ \mid{} ~ \fls{}
  ~ \mid{} ~ \ifte{\nonterminal{\term{}}}{\nonterminal{\term{}}}{\nonterminal{\term{}}}
\end{array}
\]
\caption{Grammar of our Language\label{fig:grammar:term}}
\end{figure}

\subsection{Well Scoped and Typed by Construction}

To talk about the variables in scope and their type, we need \emph{contexts}. We
choose to represent them as lists of types; \AIC{[]} denotes the empty list and
(\AB{σ} \AIC{∷} \AB{Γ}) the list \AB{Γ} extended with a fresh variable of type \AB{σ}.
Because we are going to work with a lot of well typed and well scoped families,
we defined (\AB{I} \AR{−Scoped}) as the set of type and scope indexed families.

\begin{figure}[h]
\ExecuteMetaData[shared/Data/Var.tex]{scoped}
\caption{Typed and Scoped Definitions\label{fig:scoped}}
\end{figure}

Our first example of a type and scope indexed family is \AD{Var}, the type of Variables.
A variable is a position in a typing context, represented as a typed
de Bruijn~(\citeyear{de1972lambda}) index. This amounts to an inductive definition of
context membership. We use the combinators defined in \cref{sec:indexed-combinators}
to show only local changes to the context.

\begin{figure}[h]
\ExecuteMetaData[shared/Data/Var.tex]{var}
\caption{Well Scoped and Typed de Bruijn indices\label{fig:variable}}
\end{figure}

The \AIC{z} (for zero) constructor refers to the nearest binder in a non-empty scope.
The \AIC{s} (for successor) constructor lifts an existing variable in a given scope
to the extended scope where an extra variable has been bound. The constructors' types
respectively normalise to:

\begin{center}
  \AIC{z} : {\AS{∀} $\lbrace$\AB{σ} \AB{Γ}$\rbrace$ \AS{→}
            \AD{Var} \AB{σ} (\AB{σ} \AIC{::} \AB{Γ})}
  \qquad
  \AIC{s} : {\AS{∀} $\lbrace$\AB{σ} \AB{τ} \AB{Γ}$\rbrace$ \AS{→}
            \AD{Var} \AB{σ} \AB{Γ} \AS{→} \AD{Var} \AB{σ} (\AB{τ} \AIC{::} \AB{Γ})}
\end{center}

The syntax for this calculus guarantees that terms are well scoped-and-typed
by construction. This presentation due to
Altenkirch and Reus~(\citeyear{altenkirch1999monadic}) relies heavily on
Dybjer's~(\citeyear{dybjer1991inductive}) inductive families. Rather than
having untyped pre-terms and a typing relation assigning a type to
them, the typing rules are here enforced in the syntax. Notice that
the only use of \AF{\_⊢\_} to extend the context is for the body of
a \AIC{`lam}.

\begin{figure}[h]
\ExecuteMetaData[type-scope-semantics.agda/Syntax/Calculus.tex]{term}
\caption{Well Scoped and Typed Calculus\label{fig:term}}
\end{figure}
