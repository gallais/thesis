\begin{titlingpage}
    \begin{center}
        \vspace*{1cm}

        \Huge
        \textbf{Syntaxes with Binding, \\ Their Programs, and Proofs}

        \vspace{1.5cm}

        \LARGE

        \textbf{Guillaume Xavier ALLAIS}

        \vfill

        A thesis presented for the degree of\\
        Doctor of Philosophy

        \vspace{0.8cm}

        \includegraphics[width=0.4\textwidth]{strath_coat}

        \vspace{0.8cm}

        \Large
        Computer and Information Sciences\\
        University of Strathclyde\\
        United Kingdom\\
        2021

    \end{center}
\end{titlingpage}


\pagebreak
{
\Large
\hspace{0pt}
\vfill{}
This thesis is the result of the author’s original research. It has been
composed by the author and has not been previously submitted for
examination which has led to the award of a degree.

The copyright of this thesis belongs to the author under the terms of the
United Kingdom Copyright Acts as qualified by University of Strathclyde
Regulation 3.50. Due acknowledgement must always be made of the use of
any material contained in, or derived from, this thesis.

\vspace{1.5cm}

\hfill
\begin{minipage}{0.4\textwidth}
Signed: Guillaume ALLAIS \\
Date: \today{}
\end{minipage}
\vfill
\hspace{0pt}
}
\pagebreak

\hspace{0pt}
\vfill{}
\begin{abstract}
  Almost every programming language's syntax includes a notion of binder and
  corresponding bound occurrences, along with the accompanying notions of
  α-equivalence, capture-avoiding substitution, typing contexts,
  runtime environments, and so on.

  In the past, implementing and reasoning about programming languages required
  careful handling to maintain the correct behaviour of bound variables. Modern
  programming languages include features that enable constraints like scope
  safety to be expressed in types.
  %
  Nevertheless, the programmer is still forced to write the same boilerplate
  over again for each new implementation of a scope safe operation (e.g.,
  renaming, substitution, desugaring, printing, etc.), and then again
  for correctness proofs.

  In a case study focusing on the simply typed lambda calculus, we study these
  well scoped traversals and observe that they all share the same structure.
  %
  This enables us to formulate them as instances of a more general program whose
  properties can be established generically.
  %
  Alas, the programmer is still forced to duplicate this effort for every new
  language they implement.

  This leads us to defining an expressive universe of syntaxes with binding
  and to demonstrate how to implement scope safe traversals once and for all by
  generic programming and how to derive properties of these traversals by
  generic proving.
  %
  Our universe description, generic traversals and proofs, and our examples have
  all been formalised in Agda.
\end{abstract}
\vfill
\hspace{0pt}
\pagebreak
