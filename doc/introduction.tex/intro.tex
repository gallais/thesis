\chapter{Introduction}
\label{introduction}

In modern typed programming languages, programmers writing embedded Domain
Specific Languages (DSLs) (\cite{hudak1996building}) and researchers
formalising them can now get help from the host language's type system.
Their language's deep embeddings as inductive types can enforce strong
invariants, their programs can be proven to be invariant-respecting,
and they can even prove theorems about these programs.

\paragraph{Data} Using Generalised Algebraic Data Types (GADTs) or the more
general indexed families of type theory (\cite{dybjer1994inductive}) for
representing their syntax, programmers can \emph{statically} enforce some of
the invariants in their languages.

Managing variable scope is a popular use case
(\cite{BELLEGARDE1994287,bird_paterson_1999,altenkirch1999monadic}) as directly
manipulating raw de Bruijn indices (\cite{de1972lambda}) is error-prone.


Solutions range from enforcing well scopedness to ensuring full type and scope
correctness. In short, we use types to ensure that ``illegal states are unrepresentable'',
where illegal states are ill scoped or ill typed terms.

\paragraph{Programs} The definition of \scopeandtypesafe{} representations is naturally only a start:
once the programmer has access to a good representation of the language they are
interested in, they will then want to (re)implement standard traversals
manipulating terms. Renaming and substitution are the two most typical examples
of such traversals. Other common examples include an evaluator, a printer for
debugging purposes and eventually various compilation passes such as a continuation
passing style transformation or some form of inlining. Now that well-typedness
and well-scopedness are enforced statically, all of these traversals have to be
implemented in a \scopeandtypesafe{} manner.

\paragraph{Proofs} This last problem is only faced by those that want really high assurance or whose
work is to study a language's meta-theory: they need to prove theorems about these
traversals. The most common statements are simulation lemmas stating that two
semantics transport related inputs to related outputs, or fusion lemmas demonstrating
that the sequential execution of two semantics is equivalent to a single traversal
by a third one. These proofs often involve a wealth of auxiliary lemmas which are
known to be true for all syntaxes but have to be re-proven for every new language
e.g. identity and extensionality lemmas for renaming and substitution.

Despite the large body of knowledge in how to use types to define well formed
syntax (see the related work in \cref{chapter:program-conclusion}), it is still
necessary for the working DSL designer or formaliser to redefine essential
functions like renaming and substitution for each new syntax, and then to
reprove essential lemmas about those functions.

\section{Our Contributions}

In this thesis we address all three challenges by applying the methodology of
datatype-genericity to programming and proving with syntaxes with binding.
We ultimately give a binding-aware syntax for well typed and scoped DSLs; we
spell out a set of sufficient constraints entailing the existence of a fold-like
type-and-scope preserving traversal; and we provide proof frameworks
to either demonstrate that two traversals are in simulation or that they can be
fused together into a third one.

The first part of this work focuses on the simply-typed lambda calculus. We
identify a large catalogue of traversals which can be refactored into a common
scope aware fold-like function. Once this shared structure has been made
visible, we can design proof frameworks allowing us to show that some traversals
are related.

The second part builds on the observation that the shape of the constraints we
see both in the definition of the generic traversal and the proof frameworks
are in direct correspondence with the language's constructors. This pushes us
to define a description language for syntaxes with binding and to \emph{compute}
the constraints from such a description. We can then write generic programs for
\emph{all} syntaxes with binding and state and prove generic theorems characterising
these programs.

\section{Source Material}

This thesis is based on the content of two fully-formalised published papers. They
correspond roughly to part one and two on the one hand
(``Type-and-scope Safe Programs and Their Proofs'' \cite{allais2017type, repo2017})
and part three on the other
(``A Type and Scope Safe Universe of Syntaxes with Binding: Their Semantics and Proofs''
\cite{generic-syntax, repo2018}).

The study of variations on normalisation by evaluation in \cref{sec:variationsnormalisation}
originated from the work on a third paper ``New equations for neutral terms''
(\cite{new-equations}) which, combined with McBride's Kit for renaming and substitution
(\citeyear{mcbride2005type}), led to the identification of a shared structure between renaming,
substitution and the model construction in normalisation by evaluation.

The first consequent application of this work is our solution to the
POPLMark Reloaded challenge currently under review for publication
(\cite{poplmarkreloaded, poplmark2}) for which we formalised a proof of
strong normalisation for the simply-typed lambda-calculus (and its extension
with sum types).



\section{Plan}

This work is fully formalised in Agda so we provide an introduction
to the language and the conventions we adopted in Section~\ref{introduction-agda}.
We also use categorical terms to concisely describe some of the
properties of our constructions so the reader is provided with an
introduction to category theory in Section~\ref{introduction-category}.



\todo{TODO}
