\chapter{Introduction to Agda}
\label{introduction-agda}

The techniques and abstractions defined in this thesis are language-independent:
all the results can be replicated in any Martin L\"of Type Theory
(\citeyear{martin1982constructive}) equipped with inductive
families~(\cite{dybjer1994inductive}). In practice, all of the content of this
thesis has been formalised in Agda~(\cite{norell2009dependently}) so we provide
a (brutal) introduction to dependently-typed programming in Agda.

Agda is a dependently typed programming language based on Martin-L\"{o}f Type Theory
with inductive families, induction-recursion (\cite{Dybjer1999}), copattern-matching
(\cite{abelPientkaThibodeauSetzer:popl13}) and sized types
(\cite{DBLP:journals/corr/abs-1012-4896}).

\section{Set and Universe Levels}

Agda is both a dependently typed programming language and a proof assistant.
As such, its type system needs to be sound. A direct consequence is that the
type of all types cannot itself be a type. This is solved by using a stratified
tower of universes.

\begin{feature}[Universe Levels]
Agda avoids Russell-style paradoxes by introducing a tower of universes
\AF{Set₀} (usually written \AF{Set}), \AF{Set₁}, \AF{Set₂}, etc. Each
\AF{Setₙ} does not itself have type \AF{Setₙ} but rather \AF{Setₙ₊₁} thus
preventing circularity.
\end{feature}

In practice most of our constructions will stay at the lowest level \AF{Set}
(commonly called the ``type of small sets''), and we will only occasionally
mention \AF{Set₁}. The barest of examples involving both notions is the
following definition of \AF{ID}, the (large) type of the identity function on
small sets, together with \AF{id} the corresponding identity function.

\begin{minipage}{0.5\textwidth}
  \ExecuteMetaData[introduction.agda/introduction.tex]{idType}
\end{minipage}\begin{minipage}{0.5\textwidth}
  \ExecuteMetaData[introduction.agda/introduction.tex]{idTerm}
\end{minipage}

In the type \AF{ID} we wrapped the \AF{Set} argument using curly braces.
This is our way of telling Agda that this argument should be easy to figure
out and that we do not want to have to write it explicitly every time we
call \AF{id}.

\begin{feature}[Implicit Arguments]
Programmers can mark some of a function's arguments as implicit by
wrapping them in curly braces. The values of these implicit arguments
can be left out at the function's call sites and Agda will reconstruct
them by unification~(\cite{DBLP:conf/tlca/AbelP11}).
\end{feature}

\section{Data and (co)pattern matching}

\paragraph{Sum Types} Agda supports inductive families
(\cite{dybjer1991inductive}), a generalisation of the algebraic data types
one can find in most functional programming languages.

The plainest of sum types is the empty type \AD{⊥}, a type with no constructor.
We define it together with \AF{⊥-elim} also known as the principle of explosion:
from something absurd, we can deduce anything. The absurd pattern \AS{()} is
used to communicate the fact that there can be no value of type \AD{⊥}.

\begin{minipage}[t]{0.5\textwidth}
  \ExecuteMetaData[introduction.agda/introduction.tex]{bottom}
\end{minipage}\begin{minipage}[t]{0.5\textwidth}
  \ExecuteMetaData[introduction.agda/introduction.tex]{bottomelim}
\end{minipage}

\begin{feature}[Syntax Highlighting] We rely on Agda's \LaTeX{} backend to produce
syntax highlighting for all the code fragments in this thesis. The convention
is as follows: keywords are highlighted in \AK{orange}, data constructors in
\AIC{green}, record fields in \ARF{pink}, types and functions in \AD{blue}
while bound variables are \AB{black}.
\end{feature}

The next classic example of a sum type is the type of boolean values. We define
it as the datatype with two constructors \AIC{true} and \AIC{false}. These
constructors are \emph{distinct} and Agda understands that fact: an argument
of type {\AIC{true} \AD{≡} \AIC{false}} can be dismissed with the same absurd
pattern one uses for values of type \AD{⊥}.

\begin{minipage}[t]{0.5\textwidth}
  \ExecuteMetaData[introduction.agda/introduction.tex]{bool}
\end{minipage}\begin{minipage}[t]{0.5\textwidth}
  \ExecuteMetaData[introduction.agda/introduction.tex]{boolis2}
\end{minipage}

All definitions in Agda need to be total. In particular this means that a
function is only considered defined once Agda is convinced that all the
possible input values have been covered by the pattern-matching equations
the user has written. The coverage checker performs this analysis.

\begin{feature}[Coverage Checking] During typechecking the coverage checker
elaborates the given pattern-matching equations into a case tree
(\cite{DBLP:journals/jfp/CockxA20}). It makes
sure that all the branches are either assigned a right-hand side or obviously
impossible. This allows users to focus on the cases of interest, letting the
machine check the other ones.
\end{feature}

The function \AF{if\_then\_else\_} is defined by pattern-matching on its
boolean argument. If it is \AIC{true} then the second argument is returned,
otherwise we get back the third one.

\ExecuteMetaData[introduction.agda/introduction.tex]{ifte}

\begin{feature}[Mixfix Identifiers] We use Agda's mixfix operator notation
(\cite{danielsson2011parsing}) where underscores denote argument positions.
This empowers us to define the \AF{if\_then\_else\_} construct rather than
relying on it being built in like in so many other languages. For another
example, at the type-level this time, see e.g. the notation \AR{\_×\_} for
the type of pairs defined in \cref{par:recordtypes}.
\end{feature}

\paragraph{Inductive Types}

As is customary in introductions to dependently typed programming, our first
inductive type will be the Peano-style natural numbers.%
They are defined as an inductive
type with two constructors: \AIC{zero} and \AIC{suc}cessor. Our first program
manipulating them is addition; it is defined by recursion on the first argument.

\begin{minipage}{0.5\textwidth}
  \ExecuteMetaData[introduction.agda/introduction.tex]{nat}
\end{minipage}\begin{minipage}{0.5\textwidth}
  \ExecuteMetaData[introduction.agda/introduction.tex]{add}
\end{minipage}

For recursive definitions ensuring that all definitions are total entails
an additional check on top of coverage checking: all calls to that function
should be terminating.

\begin{feature}[Termination Checking] Agda is equipped with a sophisticated
termination checking algorithm (\cite{abel1998foetus}). For each function it
attempts to produce a well-founded lexicographic ordering of its inputs such
that each recursive call is performed on smaller inputs according to this
order.
\end{feature}

\paragraph{Record Types}\label{par:recordtypes}

Agda also supports record types; they are defined by their list of fields.
Unlike inductive types they enjoy η-rules. That is to say that any two
values of the same record type are judgmentally equal whenever all of their
fields' values are.

We define the unit type (\AR{⊤}) which has one constructor (\AIC{tt}) but no
field. As a consequence, the η-rule for \AR{⊤} states that any two values of
the unit type are equal to each other. This is demonstrated by the equality
proof we provide.

\begin{minipage}[t]{0.5\textwidth}
  \ExecuteMetaData[introduction.agda/introduction.tex]{unit}
\end{minipage}
\begin{minipage}[t]{0.5\textwidth}
\ExecuteMetaData[introduction.agda/introduction.tex]{uniteq}
\end{minipage}

\begin{feature}[Underscore as Placeholder Name]
\label{feature:placeholder}
An underscore used in place of an identifier in a binder means that the binding
should be discarded. For instance {(\AS{λ} \_ \AS{→} a)} defines a constant function.
Toplevel bindings can similarly be discarded which is a convenient way of
writing unit tests (in type theory programs can be run at typechecking time)
without polluting the namespace.
\end{feature}

The second classic example of a record type is the pair type \AR{\_×\_} which
has an infix constructor \AIC{\_,\_} and two fields: its first (\ARF{fst}) and
second (\ARF{snd}) projections. The η-rule for pairs states that any value is
equal to the pairing of its first and second projection. This is demonstrated
by the equality proof we provide.


\begin{minipage}[t]{0.5\textwidth}
  \ExecuteMetaData[introduction.agda/introduction.tex]{pair}
\end{minipage}
\begin{minipage}[t]{0.5\textwidth}
\ExecuteMetaData[introduction.agda/introduction.tex]{paireq}
\end{minipage}

%% Defining a record type \AR{R} also results in the definition of a module \AB{R}
%% parametrised by a value of type \AR{R} and containing a projection function
%% for each field. We can \AK{open} it to make these projections available to the
%% outside world, or use \AB{R}-qualified names.

Agda gives us a lot of different ways to construct a value of a record type
and to project the content of a field out of a record value. We will now
consider the same function \AF{swap} implemented in many different fashion
to highlight all the possibilities offered to us.

If the record declaration introduced a \AK{constructor} we may use it to
pattern-match on values of that record type on the left or construct
values of that record type on the right. We can always also simply use
the \AK{record} keyword and list the record's fields. The two following
equivalent definitions are demonstrating these possibilities.

\begin{minipage}[t]{0.5\textwidth}
  \ExecuteMetaData[introduction.agda/introduction.tex]{swaprecord}
\end{minipage}\begin{minipage}[t]{0.5\textwidth}
  \ExecuteMetaData[introduction.agda/introduction.tex]{swaprecordlhs}
\end{minipage}

From now on, we will only use the constructor \AIC{\_,\_} rather than the
more verbose definition involving the \AK{record} keyword. Instead of
pattern-matching on a record, we may want to project out the values of
its fields. For each field, Agda defines a projection of the same name.
They may be used prefix or suffix in which case one needs to prefix the
projection's name with a dot. This is demonstrated in the two following
equivalent definitions.

\begin{minipage}[t]{0.5\textwidth}
  \ExecuteMetaData[introduction.agda/introduction.tex]{swapprefix}
\end{minipage}\begin{minipage}[t]{0.5\textwidth}
\ExecuteMetaData[introduction.agda/introduction.tex]{swapsuffix}
\end{minipage}

Instead of using the record constructor to build a value of a record type,
we may define it by explaining what ought to happen when a user attempts
to project out the value of a field. This definition style dubbed
copattern-matching focuses on the \emph{observations} one may make of the
value. It was introduced by Abel, Pientka, Thibodeau,
and Setzer~(\citeyear{abelPientkaThibodeauSetzer:popl13}).
It can be used both in prefix and suffix style.

\begin{minipage}[t]{0.5\textwidth}
  \ExecuteMetaData[introduction.agda/introduction.tex]{swapcoprefix}
\end{minipage}\begin{minipage}[t]{0.5\textwidth}
  \ExecuteMetaData[introduction.agda/introduction.tex]{swapcosuffix}
\end{minipage}

Finally, we may use an anonymous copattern-matching function to define
the record value by the observation one may make of it. Just like any
other anonymous (co)pattern-matching function, it is introduced by the
construct {(λ \AK{where})} followed by a block of clauses.

\ExecuteMetaData[introduction.agda/introduction.tex]{swapcoanonymous}


\begin{feature}[Implicit Generalisation] The latest version of Agda supports ML-style
implicit prenex polymorphism and we make heavy use of it: every unbound variable
should be considered implicitly bound at the beginning of the telescope. In the above
example, \AB{A} and \AB{B} are introduced using this mechanism.
\end{feature}

\paragraph{Recursive Functions}

Definitions taking values of an inductive type as argument are constructed by
pattern matching in a style familiar to Haskell programmers: one lists
clauses assuming a first-match semantics. If the patterns on the left hand
side are covering all possible cases and the recursive calls are structurally
smaller, the function is total. All definitions have to be total in Agda.

The main difference with Haskell is that in Agda, we can perform so-called
``large elimination'': we can define a \AF{Set} by pattern-matching on a
piece of data. Here we use our unit and pair records to define a tuple of
size \AB{n} by recursion over \AB{n}: a tuple of size \AIC{zero} is empty
whilst a (\AIC{suc} \AB{n}) tuple is a value paired with an \AB{n} tuple.

\ExecuteMetaData[introduction.agda/introduction.tex]{ntuple}

\begin{technique}[Intrinsically Enforced Properties]
In the definition of an \AB{n}-ary tuple, the length of the tuple is part
of the type (indeed the definition proceeds by induction over this natural
number). We say that the property that the tuple has length \AB{n} is
enforced \emph{intrinsically}. Conversely some properties are only proven
after the fact, they are called \emph{extrinsic}.

The choice of whether a property should be enforced intrinsically or
extrinsically is for the programmer to make, trying to find a sweet
spot for the datastructure and the task at hand. We typically build
basic hygiene intrinsically and prove more complex properties later.
\end{technique}


Types can now depend on values, as a consequence in a definition by pattern
matching each clause has a type \emph{refined} based on the patterns which appear
on the left hand side. This will be familiar to Haskell programmers used to
manipulating Generalized Algebraic Data Types (GADTs). Let us see two examples of
a type being refined based on the pattern appearing in a clause.

First, we introduce \AF{replicate} which takes a natural number \AB{n} and a value
\AB{a} of type \AB{A} and returns a tuple of \AB{A}s of size \AB{n} by duplicating
\AB{a}. The return type of \AF{replicate} reduces to \AR{⊤} when the natural number
is \AIC{zero} and (\AB{A} \AR{×} (\AB{n} \AF{-Tuple} \AB{A})) when it is
(\AIC{suc} \AB{n}) thus making the following definition valid.

\ExecuteMetaData[introduction.agda/introduction.tex]{replicate}

Second, we define \AF{map\textasciicircum{}-Tuple} which takes a function and
applies it to each one of the elements in a \AF{-Tuple}.

\begin{convention}[Caret-based naming]
We use a caret to generate a mnemonic name: \AF{map} refers to the fact
that we can map a function over the content of a data structure and
\AF{-Tuple} clarifies that the data structure in question is the family
of \AB{n}-ary tuples.

We use this convention consistently throughout this thesis, using names
such as \AF{map\textasciicircum{}Rose} for the proof that we can map a
function over the content of a rose tree, \AF{th\textasciicircum{}Var}
for the proof that \emph{var}iables are \emph{th}innable,
or \AF{vl\textasciicircum{}Tm} for the proof that terms are var-like.
\end{convention}

Both the type of the \AF{-Tuple} argument and the \AF{-Tuple} return type are
refined based on the pattern the natural number matches. In the second clause,
this tells us the \AF{-Tuple} argument is a pair, allowing us to match on it
with the pair constructor \AIC{\_,\_}.

\ExecuteMetaData[introduction.agda/introduction.tex]{mapntuple}

\begin{remark}[Functor]
In Section~\ref{sec:intro-proving} we will prove that applying
\AF{map\textasciicircum{}-Tuple} to the identity yields the identity
and that \AF{map\textasciicircum{}-Tuple} commutes with function
composition. That is to say we will prove that {(\AB{n} \AF{-Tuple})}
is an endofunctor (see \cref{def:endofunctor}) on \Set.
\end{remark}

\paragraph{Dependent Record Types}

Record types can be dependent, i.e. the type of later fields can depend on that of
former ones. We define a \AR{Tuple} as a natural number (its \ARF{length}) together
with a (\ARF{length} \AF{-Tuple}) of values (its \ARF{content}).

\ExecuteMetaData[introduction.agda/introduction.tex]{tuple}

In Agda, as in all functional programming languages, we can define anonymous functions
by using a \AS{λ}-abstraction. Additionally, we can define anonymous (co)pattern-matching
functions by using (\AS{λ} \AK{where}) followed by an indented block of clauses.
We use here copattern-matching, that is to say that we define a \AR{Tuple} in terms
of the observations that one can make about it: we specify its length first, and
then its content. We use postfix projections (hence the dot preceding the field's name).

\ExecuteMetaData[introduction.agda/introduction.tex]{maptuple}

\begin{technique}[Lazily unfolded definitions]
  When record values are going to appear in types, it is often a good idea to define
  them by copattern-matching: this prevents the definition from being unfolded eagerly
  thus making the goal more readable during interactive development.
\end{technique}

\paragraph{Strict Positivity} In order to rule out definitions leading to
inconsistencies, all datatype definitions need to be strictly positive.
Although a syntactic criterion originally (its precise definition is beyond
the scope of this discussion), Agda goes beyond by recording internally
whether functions use their arguments in a strictly positive manner.
This allows us to define types like rose trees where the subtrees of a
\AIC{node} are stored in a \AR{Tuple}, a function using its \AF{Set}
argument in a strictly positive manner.

\ExecuteMetaData[introduction.agda/introduction.tex]{rose}

\section{Sized Types and Termination Checking}
\label{sec:sizetermination}

If we naïvely define rose trees like above then we quickly realise that we cannot
re-use higher order functions on \AR{Tuple} to define recursive functions on \AD{Rose}.
As an example, let us consider \AF{map\textasciicircum{}Rose}. In the \AIC{node} case,
the termination checker does not realise that the partially applied recursive call
(\AF{map\textasciicircum{}Rose} \AB{f}) passed to \AF{map\textasciicircum{}Tuple}
will only ever be used on subterms. We need to use an unsafe \AK{TERMINATING} pragma
to force Agda to accept the definition.

\ExecuteMetaData[introduction.agda/introduction.tex]{maprose}

This is not at all satisfactory: we do not want to give up safety to write such a
simple traversal. The usual solution to this issue is to remove the level of
indirection introduced by the call to \AF{map\textasciicircum{}Tuple} by mutually
defining with \AF{map\textasciicircum{}Rose} an inlined version of
(\AF{map\textasciicircum{}Tuple} (\AF{map\textasciicircum{}Rose} \AB{f})).
In other words, we have to effectively perform supercompilation
(\cite{mendelgleason2012}) by hand.

\ExecuteMetaData[introduction.agda/introduction.tex]{inlinedmaprose}

This is, of course, still unsatisfactory: we need to duplicate code every
time we want to write a traversal! By using sized types, we can have a more
compositional notion of termination checking: the size of a term is reflected
in its type. No matter how many levels of indirection there are between the
location where we are peeling off a constructor and the place where the function
is actually called recursively, as long as the intermediate operations are
size-preserving we know that the recursive call will be legitimate.

Writing down the sizes explicitly, we get the following implementation. Note
that in (\AF{map\textasciicircum{}Tuple} (\AF{map\textasciicircum{}Rose} \AB{j} \AB{f})),
\AB{j} (bound in \AIC{node}) is smaller than \AB{i} and therefore the recursive
call is justified.

\ExecuteMetaData[introduction.agda/introduction.tex]{erose}
\ExecuteMetaData[introduction.agda/introduction.tex]{maperose}

In practice we make the size arguments explicit in the types but implicit in the
terms. This leads to programs that look just like our ideal implementation, with
the added bonus that we have now \emph{proven} the function to be total.

\ExecuteMetaData[introduction.agda/introduction.tex]{irose}
\ExecuteMetaData[introduction.agda/introduction.tex]{mapirose}

\section{Proving the Properties of Programs}
\label{sec:intro-proving}
We make explicit some of our programs' invariants directly in their
type thus guaranteeing that the implementation respects them. The
type of \AF{map\textasciicircum{}-Tuple} for instance tells us that
this is a length preserving operation. We cannot possibly encode all
of a function's properties directly into its type but we can use
Agda's type system to state and prove theorems of interest.

In Agda, programming \emph{is} proving. The main distinction between
a proof and a program is a social one: we tend to call programs the
objects whose computational behaviour matters to us while the other
results correspond to proofs.

We can for instance prove that mapping the identity function over a
tuple amounts to not doing anything.

\begin{lemma}[Identity lemma for \AF{map}]
Given a function \AB{f} extensionally equal to the identity,
{(\AF{map} \AB{f})} is itself extensionally
equal to the identity.
\end{lemma}

\begin{technique}[Extensional Statements] Note that we have stated
the theorem not in terms of the identity function but rather in
terms of a function which behaves like it. Agda's notion of equality
is intensional, that is to say that we cannot prove that two functions
yielding the same results when applied to the same inputs are necessarily
equal. In practice this means that stating a theorem in terms of the
specific implementation of a function rather than its behaviour limits
vastly our ability to use this lemma. Additionally, we can always
obtain the more specific statement as a corollary.
\end{technique}

The proof of this statement is a program proceeding by induction
on \AB{n} followed by a case analysis on the \AB{n}-ary tuple.
In the base case, the equality is trivial and in the second case
\AF{map} computes just enough to reveal
the constructor pairing {(\AB{f} \AB{a})} together with the result
obtained recursively. We can combine the proof that {(\AB{f} \AB{a})}
is equal to \AB{a} and the induction hypothesis by an appeal to
congruence for the pair constructor \AIC{\_,\_}.

\ExecuteMetaData[introduction.agda/introduction.tex]{mapidentity}

\begin{technique}[Functional Induction] Whenever we need to prove
a theorem about a function's behaviour, it is best to proceed by
functional induction. That is to say that the proof and the function
will follow the same pattern-matching strategy.
\end{technique}

\begin{lemma}[Fusion lemma for \AF{map}]
Provided a function \AB{f} from \AB{A} to \AB{B},
a function \AB{g} from \AB{B} to \AB{C}, and
a function \AB{h} from \AB{A} to \AB{C} extensionally equal to the
composition of \AB{f} followed by \AB{g},
mapping \AB{f} and then \AB{g} over an \AB{n}-ary
tuple \AB{as} is the same as merely mapping \AB{h}.
\end{lemma}

\ExecuteMetaData[introduction.agda/introduction.tex]{mapfusion}

\begin{lemma}[Fusion Lemma for \AF{replicate}]
Given a function \AB{f} from \AB{A} to \AB{B}, a value \AB{a} of type \AB{A},
and a natural number \AB{n}, mapping \AB{f} over the \AB{n}-ary tuple obtained
by replicating \AB{a} is the same as replicating (\AB{f} \AB{a}).
\end{lemma}

\ExecuteMetaData[introduction.agda/introduction.tex]{mapreplicate}



\section{Working with Indexed Families}
\label{sec:indexed-combinators}

On top of the constructs provided by the language itself, we can define various
domain specific languages (DSL) which give us the means to express ourselves
concisely. We are going to manipulate a lot of indexed families representing
scoped languages so we give ourselves a few combinators corresponding to the
typical operations we want to perform on them.

These combinators will allow us to write the types for index-preserving
operations without having to mention the indices. This will empower us
to use very precise indexed types whilst having normal looking type
declarations. We have implemented a generalisation of these combinators
to n-ary relations (\cite{DBLP:conf/icfp/Allais19}) but this work is out
of the scope of this thesis.



First, noticing that most of the time we silently thread the current scope, we lift
the function space pointwise with \AF{\_⇒\_}.

\ExecuteMetaData[shared/Stdlib.tex]{arrow}

Second, the \AF{\_⊢\_} combinator makes explicit the \emph{adjustment} made to the
index by a function. Conforming to the convention (see e.g. \cite{martin1982constructive})
of mentioning only context \emph{extensions} when presenting judgements, we write
({\AB{f} \AF{⊢} \AB{P}}) where \AB{f} is the modification and \AB{P} the indexed
Set it operates on. This is the reindexing functor we saw in
\cref{example:reindexingfunctor}.

\ExecuteMetaData[shared/Stdlib.tex]{adjust}

Although it may seem surprising at first to define binary infix operators as having
arity three, they are meant to be used partially applied, surrounded by \AF{∀[\_]}
or \AF{∃⟨\_⟩}. These combinators turn an indexed Set into a Set by quantifying over
the index. The function \AF{∀[\_]} universally quantifies over an implicit value
of the right type while \AF{∃⟨\_⟩}, defined using a dependent record, corresponds
to an existential quantifier.

\ExecuteMetaData[shared/Stdlib.tex]{forall}

\begin{minipage}[t]{0.6\textwidth}
  \ExecuteMetaData[shared/Stdlib.tex]{sigma}
\end{minipage}
\begin{minipage}[t]{0.4\textwidth}
  \ExecuteMetaData[shared/Stdlib.tex]{exists}
\end{minipage}

We make \AF{\_⇒\_} associate to the right as one would expect and give it the
highest precedence level as it is the most used combinator. These combinators
lead to more readable type declarations.  For instance, the compact expression
\AF{∀[} \AF{suc} \AF{⊢} (\AB{P} \AF{⇒} \AB{Q}) \AF{⇒} \AB{R} \AF{]}
desugars to the more verbose type
\AS{∀} \{\AB{i}\} \AS{→} (\AB{P} (\AF{suc} \AB{i}) \AS{→} \AB{Q} (\AF{suc} \AB{i})) \AS{→} \AB{R} \AB{i}.

As the combinators act on the \emph{last} argument of any indexed family, this inform our
design: our notions of variables, languages, etc. will be indexed by their kind first and
scope second. This will be made explicit in the definition of \AF{─Scoped} in
\cref{fig:scoped}.
