\chapter{Introduction to Category Theory}
\label{introduction-category}

Although this thesis is not about category theory, some categorical
notions do show up. They help us expose the structure of the problem
and the solutions we designed. Hence this (short) introduction to
category theory focused on the notions that will be of use to us.
Computer scientists looking for a more substantial introduction
to category theory can refer to
Pierce's little book~(\citeyear{DBLP:books/daglib/0069193}) for general
notions and Altenkirch, Chapman and Uustalu's papers
(\citeyear{Altenkirch2010,JFR4389}) if they are
specifically interested in relative monads.

\section{What even is a Category?}

A category is a set of morphisms closed under a well-behaved notion of
composition. This is made formal by the following definition.

\begin{definition}\label{def:category}
A \textbf{category} \AB{𝓒} is defined by a set of objects (written Obj(\AB{𝓒}))
and a family of morphisms (written \AB{𝓒}(\AB{A},\AB{B})) between any two
such objects \AB{A} and \AB{B} such that:
\begin{enumerate}
  \item For any object \AB{A}, we have an \emph{identity morphism} $id_{\AB{A}}$ in {\AB{𝓒}(\AB{A},\AB{A})}
  \item Given three objects \AB{A}, \AB{B}, and \AB{C}, and two morphisms
    \AB{f} in {\AB{𝓒}(\AB{A},\AB{B})} and \AB{g} in {\AB{𝓒}(\AB{B},\AB{C})},
    we have a unique morphism {(\AB{g} $∘_{\AB{𝓒}}$ \AB{f})} in
    {\AB{𝓒}(\AB{A},\AB{C})} corresponding to the composition
    of \AB{f} and \AB{g}.
  \item Composition is associative
  \item The appropriate identity morphisms are left and right neutral elements for composition
\end{enumerate}
\end{definition}

\begin{convention} When the reader should be able to reconstruct the
information from the context, we may write $id$ instead of $id_{\AB{A}}$
and $(\_∘\_)$ instead of $(\_∘_{\AB{𝓒}}\_)$ respectively.
\end{convention}

\begin{example}[Discrete category]
\label{example:discretecat}
From any type \AB{A} we can create the discrete category corresponding to
\AB{A}. Its objects are values of type \AB{A} and the only morphisms it has
are the identity functions from each element to itself. It derives its name
from the total lack of morphisms between distinct objects.
\end{example}

\begin{example}[Sets and functions]
\label{example:setsfunctionscat}
In this category called \Set, objects are sets and morphisms are total
functions between them. Identities and compositions are the usual notions
of identity and composition for functions.
\end{example}

\begin{example}[Families and index-respecting functions]
\label{example:familiescat}
Given a set \AB{I}, we define $\Set^I$ as the category whose objects
are families of type {(\AB{I} → Set)} and whose morphisms between
two families \AB{P} and \AB{Q} are index preserving functions i.e.
functions of type {(∀\AB{i} → \AB{P} \AB{i} → \AB{Q} \AB{i})}.
\end{example}

\section{Functors}

A functor is a morphism between two categories that is compatible with their
respective notions of composition.

\begin{definition}[Functor]\label{def:functor}
A functor \AB{F} between two categories
\AB{𝓒} and \AB{𝓓} is defined by its action on the objects and
morphisms of \AB{𝓒}. It takes objects in \AB{𝓒} to objects in
\AB{𝓓}, and morphisms in {\AB{𝓒}(\AB{A},\AB{B})} to morphisms in
{\AB{𝓓}(\AB{F} \AB{A}, \AB{F} \AB{B})} such that the structure of
\AB{𝓒} is preserved. In other words:
\begin{enumerate}
  \item {\AB{F} $id_{\AB{A}}$} is equal to $id_{\AB{F} \AB{A}}$
  \item {\AB{F} (\AB{g} $∘_{\AB{𝓒}}$ \AB{f})} is equal to
      {(\AB{F} \AB{g} $∘_{\AB{𝓓}}$ \AB{F} \AB{f})}
\end{enumerate}
\end{definition}

\begin{example}[Reindexing]
\label{example:reindexingfunctor}
Given two sets \AB{I} and \AB{J}, every function \AB{f}
from \AB{I} to \AB{J} induces a functor from the category $\Set^J$ to
$\Set^I$ (cf. \cref{example:familiescat}).
We write {(\AB{f} ⊢\_)} for this functor (we explain this notation in
Section~\ref{sec:indexed-combinators}) and define its action on objects
and morphisms as follows:

\begin{itemize}
  \item Given an object \AB{P} i.e. a predicate of type (\AB{J} → Set),
    we define {(\AB{f} ⊢ \AB{P})} to be the
    family {(λ\AB{i} → \AB{P} (\AB{f} \AB{i}))} of type (\AB{I} → Set)
  \item Given a morphism \AB{prf} from \AB{P} to \AB{Q} (i.e. a function
    of type {(∀\AB{j} → \AB{P} \AB{j} → \AB{Q} \AB{j})}),
    we define {(\AB{f} ⊢ \AB{prf})} to be
    the morphism {(λ \AB{i} \AB{p} → \AB{prf} (\AB{f} \AB{i}) \AB{p})}
\end{itemize}

The constraints spelt out in the definition of a functor are trivially verified.
\end{example}

Programmers may be more familiar with endofunctors: functors whose domain
and codomain are the same category.

\begin{definition}[Endofunctor]\label{def:endofunctor}
An endofunctor on a category \AB{𝓒} is a functor
from \AB{𝓒} to \AB{𝓒}.
\end{definition}

\begin{example}[List]
\label{example:listfunctor}
The inductive type associating to each type \AB{A}, the type of finite
lists of elements of \AB{A} is an endofunctor on the category of Sets
and functions (cf. \cref{example:setsfunctionscat}).

Its action on morphism is simply to map the function over all of the lists'
element. The equality constraints spelt out by the definition of Functor
correspond to the classic identity and fusion lemmas for the map function.
We see detailed proofs for a similar result in Section~\ref{sec:intro-proving}
in the context of proving programs correct in Agda.
\end{example}

\section{Monads}

A monad is an endofunctor whose action on objects is well-behaved under
iteration. For any amount of nesting of this action $\AB{T}^n \AB{A}$,
it has a canonical operation computing a value of type {\AB{T} \AB{A}}
in a way that is compatible with the functor structure.

\begin{definition}[Monad]
\label{def:monad}
A monad is an endofunctor \AB{T} on \AB{𝓒} equipped with:
\begin{enumerate}
  \item For any object \AB{A}, a function $η_{\AB{A}}$ of type
    {\AB{𝓒}(\AB{A}, \AB{T} \AB{A})} (the unit)
  \item For any objects \AB{A} and \AB{B}, a mapping $\_^*$ from
    {\AB{𝓒}(\AB{A}, \AB{T} \AB{B})} to {\AB{𝓒}(\AB{T} \AB{A}, \AB{T} \AB{B})}
    (the Kleisli extension)
\end{enumerate}

such that:

\begin{enumerate}
  \item For any object \AB{A}, $η_{\AB{A}}\,^*$ is equal to $id_{\AB{T} \AB{A}}$
  \item For any objects \AB{A} and \AB{B} and \AB{f} a morphism of type
    {\AB{𝓒}(\AB{A}, \AB{T} \AB{B})}, $\AB{f}\,^* ∘ η_{\AB{A}}$ is equal to \AB{f}
  \item For any objects \AB{A}, \AB{B}, \AB{C} and two morphisms
    \AB{f} in {\AB{𝓒}(\AB{A}, \AB{T} \AB{B})} and \AB{g} in {\AB{𝓒}(\AB{B}, \AB{T} \AB{C})},
    $(\AB{g}^* ∘ \AB{f})^*$ is equal to $(\AB{g}^* ∘ \AB{f}^*)$
\end{enumerate}
\end{definition}

\begin{example}[List]We have already seen in \cref{example:listfunctor}
that the \AB{List} type constructor is a functor. It is additionally
a monad. $η_{\AB{A}}$ takes an element an returns a singleton list, while
$\_^*$ maps its input function on every value in the input list before
flattening the result. The equality constraints are easily verified.
\end{example}

We will refine the following example in the next section and revisit it
in spirit throughout this thesis as expressions-with-variables as monads
is part of the very foundations of our work.

\begin{example}[Arithmetic expressions with free variables]
\label{example:arithsetexample}
The type of abstract syntax trees of very simple closed arithmetic expressions
involving natural numbers ($\underline{n}$ stands for the embedding of a natural
number $n$) and addition of two expressions can be described by following
grammar:

$$t ::= \underline{n} \mid t + t$$

We can freely add variables to this little language by adding an additional
constructor for values in a type of variables \AB{I} (we write \AB{i} for any
value of type \AB{I}).

$$t ::= i \mid \underline{n} \mid t + t$$

This new type is an endofunctor on the category of sets and functions: a
mapping from {\AB{I} to \AB{J}} can be seen as a variable renaming. It is,
additionally, a monad: $η$ is the var constructor itself, $\_^*$ corresponds
to parallel substitution.

\end{example}

\section{Monads need not be Endofunctors}\label{sec:relativemonad}

Although monads are defined as endofunctors, we can relax this
constraint by studying Altenkirch, Chapman and Uustalu's
\emph{relative} monads instead~(\citeyear{Altenkirch2010,JFR4389}).

\begin{definition}[Relative Monad]
\label{def:relative-monad}
A monad relative to a functor \AB{V} from \AB{𝓒} to \AB{𝓓}
is a functor \AB{T} from \AB{𝓒} to \AB{𝓓} equipped with:
\begin{enumerate}
  \item For any object \AB{A}, a mapping \AB{η}$_{\AB{A}}$ of type
    {\AB{𝓓}(\AB{V} \AB{A}, \AB{T} \AB{A})} (the unit)
  \item For any objects \AB{A} and \AB{B}, a mapping $\_^*$
    from {\AB{𝓓}(\AB{V} \AB{A}, \AB{T} \AB{B})}
    to {\AB{𝓓}(\AB{T} \AB{A}, \AB{T} \AB{B})} (the Kleisli extension)
\end{enumerate}

such that they verify similar constraints to the ones spelt out in the
definition of a monad (cf. definition \ref{def:monad}).
\end{definition}

\begin{example}[Arithmetic expressions with finitely many free variables]
\label{example:arithnatexample}

This example is a slight modification of \cref{example:arithsetexample}.
Instead of allowing any type to represent the variables free in a term,
we restrict ourselves to using a natural number that represents the number
of variables that are in scope.

A variable in a tree indexed by $n$ is now a natural number $m$ together
with a proof that it is smaller than $n$. This new type is a functor between
the discrete category of natural numbers%
\footnote{We could pick a source category with more structure and
obtain a more interesting result but this is beyond the scope of this
introduction}%
(cf. \cref{example:discretecat}) and the category of sets and functions.

It is also a relative monad with respect to the trivial functor taking
a natural number and mapping it to the set of natural numbers that are
smaller (our notion of variable). Once again mapping is renaming, $η$
is the var constructor itself, and $\_^*$ corresponds to parallel
substitution.
\end{example}
